% !Tex program = xelatex
\documentclass[12pt]{article}
\usepackage[dvipsnames]{xcolor} 
\usepackage[paperwidth=135mm,paperheight=180mm,left=4mm,right=4mm,top=8mm,bottom=8mm]{geometry}
\usepackage[xetex]{graphicx}
\usepackage{array}
\usepackage{setspace}
\usepackage{multirow}
\usepackage{pxfonts}
\usepackage{bbding}
\usepackage{wasysym} 
\usepackage{fontspec}
\usepackage{multicol}
\usepackage{supertabular}
\usepackage{fancyhdr}
\pagestyle{fancy}
\fancyhf{}
\rhead{}
\lhead{}
\cfoot{}
\usepackage{../templates/listofitems}
\newcommand{\yearname}{2015}
%%%%%%%%%%%%%%%%%%%%%%%%%%%%%%%%%%%%%%%%%%%%%%%%%%%%%%%%%%%%%%%%%%%%%%%%%%%%%%%
%% MOONPHASE CODE 
%%%%%%%%%%%%%%%%%%%%%%%%%%%%%%%%%%%%%%%%%%%%%%%%%%%%%%%%%%%%%%%%%%%%%%%%%%%%%%%
%Credits: http://tex.stackexchange.com/questions/34785/how-to-typeset-moon-phase-symbols (Jake!)
\usepackage{tikz}
\usetikzlibrary{calendar,fpu}

\tikzset{
    moon colour/.style={
        moon fill/.style={
            fill=#1
        }
    },
    sky colour/.style={
        sky draw/.style={
            draw=#1
        },
        sky fill/.style={
            fill=#1
        }
    },
    southern hemisphere/.style={
        rotate=180
    }
}

\makeatletter
\pgfcalendardatetojulian{2010-01-15}{\c@pgf@counta} % 2010-01-15 07:11 UTC -- http://aa.usno.navy.mil/cgi-bin/aa_moonphases.pl?year=2010&ZZZ=END
\def\synodicmonth{29.530588853}
\newcommand{\moon}[2][]{%
    \edef\checkfordate{\noexpand\in@{-}{#2}}%
    \checkfordate%
    \ifin@%
        \pgfcalendardatetojulian{#2}{\c@pgf@countb}%
        \pgfkeys{/pgf/fpu=true,/pgf/fpu/output format=fixed}%
        \pgfmathsetmacro\dayssincenewmoon{\the\c@pgf@countb-\the\c@pgf@counta-(7/24+11/(24*60))}%
        \pgfmathsetmacro\lunarage{mod(\dayssincenewmoon,\synodicmonth)}
        \pgfkeys{/pgf/fpu=false}%%
    \else%
        \def\lunarage{#2}%
    \fi%
    \pgfmathsetmacro\leftside{ifthenelse(\lunarage<=\synodicmonth/2,cos(360*(\lunarage/\synodicmonth)),1)}%
    \pgfmathsetmacro\rightside{ifthenelse(\lunarage<=\synodicmonth/2,-1,-cos(360*(\lunarage/\synodicmonth))}%
    \tikz [moon colour=white,sky colour=black,#1]{
        \draw [moon fill, sky draw] (0,0) circle [radius=1ex];
        \draw [sky draw, sky fill] (0,1ex)
            arc (90:-90:\rightside ex and 1ex)
            arc (-90:90:\leftside ex and 1ex)
            -- cycle;
    }%
}
%%%%%%%%%%%%%%%%%%%%%%%%%%%%%%%%%%%%%%%%%%%%%%%%%%%%%%%%%%%%%%%%%%%%%%%%%%%%%%%
%% END MOONPHASE CODE
%%%%%%%%%%%%%%%%%%%%%%%%%%%%%%%%%%%%%%%%%%%%%%%%%%%%%%%%%%%%%%%%%%%%%%%%%%%%%%%
% \setlength{\footskip}{2mm}
% PDF SETUP
% ---- FILL IN HERE THE DOC TITLE AND AUTHOR
\defaultfontfeatures{Scale=MatchLowercase,Mapping=tex-text}
\setmainfont{siddhanta.ttf}[Path=../fonts/,Script=Devanagari] 
\setsansfont[Path=../fonts/,Scale=0.95,Numbers=Lining]{AlegreyaSans-Regular.ttf}
% \newfontfamily\noto[Path=../fonts/, Ligatures=TeX]{NotoSansUI-Regular}
%%%%%%% Numbers and counters %%%%%%%
\newcount\num
\newcount\tempone \newcount\temptwo
\newcommand{\devanumber}[1]{%
\num=#1\devanumberrecurse}
\setlength{\textheight}{150mm}
\newcommand{\devanumberrecurse}{%
{\tempone=\num
%  \showthe\tempone\ %
\ifnum\num > 0 
    \divide \num by 10%
    \temptwo=\num \multiply\temptwo by -10%
    \devanumberrecurse%
%   \\stage 2\ %
%   \showthe\temptwo\ %
%   temp1=\number\tempone\ %
%   num=\number\num\ %
    \advance\tempone by \temptwo%
    \devadigit
\fi
}}
\newcommand{\devadigit}{%
\ifcase\tempone०\or१\or२\or३\or४\or५\or६\or७\or८\or९\fi%\number\tempone%
}
\newcommand{\eventsep}{~\raisebox{1pt}{\scriptsize$\Diamondblack$} }
\newcommand{\TO}{\hspace{1pt}\raisebox{1pt}{\footnotesize\RIGHTarrow}\hspace{1pt}}
\newcommand{\To}{\hspace{1pt}\raisebox{1pt}{\footnotesize\RIGHTarrow}\hspace{1pt}}
\newcommand{\Too}{\hspace{1pt}\raisebox{1pt}{\footnotesize\RIGHTarrow\hspace{-5pt}\RIGHTarrow}\hspace{1pt}}
%%%%%%%% Calendar display stuff %%%%%%%%%%%
\newcommand{\samvatsaraName}{}
\newcommand{\solarMonthName}{}
\newcommand{\solarMonthEndTime}{}
\newcommand{\lunarMonthName}{}
\newcommand{\lunarRtu}{}
\newcommand{\solarMonthDate}{}
\newcommand{\vaaraName}{}
\newcommand{\rtuName}{}
\newcommand{\ayanamName}{}

\newcommand{\sunmonth}[9]{%
\renewcommand{\solarMonthName}{#1}
\renewcommand{\solarMonthDate}{#2}
\renewcommand{\solarMonthEndTime}{#3}
\renewcommand{\lunarMonthName}{#4}
\renewcommand{\lunarRtu}{#5}
\renewcommand{\vaaraName}{#6}
\renewcommand{\samvatsaraName}{#7}
\renewcommand{\ayanamName}{#8}
\renewcommand{\rtuName}{#9}
}
\newcommand{\tamil}[1]{%
{\fontspec[Scale=0.8,FakeStretch=0.9,Path=../fonts/]{NotoSansTamil-Regular.ttf} \footnotesize #1}%
}
\newcommand{\kalas}[1]{%
\setsepchar{ }
\readlist\arg{#1}
{\small{\mbox{ब्राह्म\,\textsf{\arg[1]}{\scriptsize\RIGHTarrow}\,सङ्गव\,\textsf{\arg[4]}{\scriptsize\RIGHTarrow}\,मध्याह्न\,\textsf{\arg[7]}{\scriptsize\RIGHTarrow}\,अपराह्ण\,\textsf{\arg[8]}{\scriptsize\RIGHTarrow}\,सायाह्न\,\textsf{\arg[9]}{\scriptsize\RIGHTarrow}}\hfill {दिनान्तः{\scriptsize\RIGHTarrow}\textsf{\arg[14]}}}}\\[-1ex]
{\small{\mbox{प्रातः सन्ध्या \textsf{\arg[2]}{\scriptsize\RIGHTarrow}\textsf{\arg[3]} माध्याह्निक \textsf{\arg[5]}{\scriptsize\RIGHTarrow}\textsf{\arg[6]} 
सायं \textsf{\arg[10]}{\scriptsize\RIGHTarrow}\textsf{\arg[11]}}\hfill शयन \textsf{\arg[12]}{\scriptsize\RIGHTarrow}\textsf{\arg[13]}}\\[-4.5ex]}
}
\newcommand{\sunmoonrsdata}[5]{%
\mbox{\large\sun{\small\UParrow}\textsf{#1}~\sun{\small\DOWNarrow}\textsf{#2}}\hfill
\mbox{\large\rightmoon{\small\UParrow}\textsf{#3}~\rightmoon{\small\DOWNarrow}\textsf{#4}}\\
#5
 }
\newcommand{\sunmoonsrdata}[5]{%
\mbox{\large\sun{\small\UParrow}\textsf{#1}~\sun{\small\DOWNarrow}\textsf{#2}}\hfill
\mbox{\large\rightmoon{\small\DOWNarrow}\textsf{#4}~\rightmoon{\small\UParrow}\textsf{#3}}\\
#5
 }
\newcommand{\ahorAtram}{अहोरात्रम्}
\newcommand{\tithi}[2]{\raisebox{-1pt}{\moon[scale=0.8]{#1}}\hspace{2pt}#2}
\newcommand{\tnykdata}[6]{\large%\fontsize{13pt}{16pt}\selectfont
{#1}\\%Tithi
{नक्षत्रम्–#2 (#3)}\\%Nakshatram and Rashi
{\setstretch{0.55}
\begin{tabular}{@{}r@{}p{108mm}@{}}
योगः–&#4\\[2pt]%Yogam
करणम्–&#5\\%Karanam
\end{tabular}}\mbox{}\\[3pt]
\parbox[c][2ex][c]{0.9\linewidth}{\footnotesize #6}%Lagna, if required
}
\newcommand{\avamA}{
    \raisebox{1.5pt}{\fcolorbox{white}{gray!40}{\scriptsize अवमा}}
}
\newcommand{\tridina}{
    \raisebox{1.5pt}{\fcolorbox{white}{gray!40}{\scriptsize त्रिदिनस्पृक्}}
}
\renewcommand{\time}[2]{#1 (#2)}
\newcommand{\anga}[2]{\mbox{#1\To{}\textsf{#2}}}
\newcommand{\fullanga}[1]{\mbox{#1\To{}\ahorAtram}}
\newcommand{\fulltithi}[1]{\mbox{#1\To{}\ahorAtram\tridina}}
\newcommand{\lagna}[2]{\mbox{#1\RIGHTarrow\textsf{#2}}}
\newcommand{\uanga}[1]{\mbox{#1\Too}}
\newcommand{\rygdata}[3]{%
\begin{minipage}{\linewidth}
\centering
\rule[-1ex]{0.7\textwidth}{.4pt}
\small राहु॰~\textsf{#1}~~यम॰~\textsf{#2}~~गुलिक॰~\textsf{#3}%Rahu Yama Gulika
\end{minipage}
}
\newcommand{\caldata}[7]{%
\clearpage
\begin{minipage}{\linewidth}
#3% Calls \sunmonth
\large% Fixes font size
{\centering\begin{tabular}{c|c}
\large \textsf{\yearname} & {\large\samvatsaraName}\\[-1ex]%YYYY
& {\footnotesize \ayanamName \hspace{6pt} \rtuName}\\[0.2ex]
\mbox{\sffamily\fontsize{20}{25}\selectfont {\uppercase{#1}}} & \parbox[c][14pt][c]{3cm}{\centering\LARGE\solarMonthName}\\[-4pt]%mmm
& {\mbox{\small \solarMonthEndTime}}\\[-2pt]
& {\parbox[c][15pt]{52mm}{\centering\lunarMonthName}}\\[-6pt]
& {\parbox[c][10pt]{52mm}{\centering\scriptsize(\lunarRtu)}}\\[-5pt]
\hspace{0.465\linewidth} & \hspace{0.465\linewidth} \\[-6pt]
\mbox{\sffamily\fontsize{96}{115}\selectfont #2} & \mbox{\fontsize{90}{24}\selectfont \devanumber{\solarMonthDate}}\\[1.6ex]%DD
\mbox{\sffamily\fontsize{24}{28}\selectfont\uppercase{#7}} & \parbox[c][24pt][t]{1cm}{\centering\LARGE\vaaraName}\\[1.2ex]%Day of the week
\hline
\end{tabular}
}\mbox{}\\[-4pt]
#4\\[0.5em]%Sun rise, kalas etc
#5\mbox{}\\[1em]%Tithi, Nakshatram, Varam, Yogam
% \vspace{\fill}
{\parbox[b]{0.95\linewidth}{\centering\normalsize\textcolor{RoyalBlue}{#6}}}%Festivals
\end{minipage}
}

\addtolength{\headsep}{-3ex}
\setlength\parindent{0pt}
\pagestyle{empty}
\begin{document}
\mbox{}
\renewcommand{\yearname}{2018}
\begin{center}
{\sffamily \fontsize{80}{80}\selectfont  2018\\[0.5cm]}
\mbox{\fontsize{48}{48}\selectfont हेमलम्बः–विलम्बः}\\
\mbox{\fontsize{32}{32}\selectfont कलि } %
{\sffamily \fontsize{43}{43}\selectfont  5118–5119\\[0.5cm]}
\hrule
\vspace{0.2cm}
{\sffamily \fontsize{50}{50}\selectfont  \uppercase{Chennai}\\[0.2cm]}
{\sffamily \fontsize{23}{23}\selectfont  {13.090°N, 80.270°E}\\[0.2cm]}
\hrule
\end{center}
\clearpage\pagestyle{fancy}
\caldata{JANUARY}{1}{\sunmonth{धनुः}{17}{}{पौषः}{हेमन्तऋतुः}{सोमः}{हेमलम्बः}{दक्षिणायनम्}{हेमन्तऋतुः}}
{\sunmoonrsdata{06:34}{17:49}{17:11}{06:31*}{12:12}
{\kalas{04:52 05:44 09:34 08:49 10:19 16:19 11:04 13:19 15:34 17:04 18:40 21:01 22:36 01:48*}}}
{\tnykdata{\anga{\tithi{14}{शुक्ल-चतुर्दशी}}{\time{12-54}{11:44}}\hspace{1ex}}%
{\anga{मृगशीर्षम्}{\time{20-40}{14:51}}\hspace{1ex}}{चन्द्रराशिः—\mbox{मिथुनम्}}%
{\anga{शुक्लः}{\time{7-51}{09:43}}\hspace{1ex}\anga{ब्रह्म}{\time{56-52}{05:19*}}\hspace{1ex}\uanga{इन्द्रः}}%
{\anga{वणिजः}{\time{12-54}{11:44}}\hspace{1ex}\anga{विष्टिः}{\time{38-6}{21:49}}\hspace{1ex}\uanga{बवः}}{}
}
{अन्धकासुर-वधः\eventsep देवी-पर्व-१०\eventsep पार्वणव्रतम् पूर्णिमायाम्\eventsep पञ्च-पर्व-पूजा (पूर्णिमा)\eventsep सोममृगशीर्ष-पुण्यकालः\RIGHTarrow{}14:51\eventsep वेङ्कटाचले पूर्णिमा-गरुड-सेवा}
{Mon} 
\cfoot{\rygdata{07:59--09:23}{10:48--12:12}{13:36--15:01}}
\caldata{JANUARY}{2}{\sunmonth{धनुः}{18}{}{पौषः}{हेमन्तऋतुः}{मङ्गलः}{हेमलम्बः}{दक्षिणायनम्}{हेमन्तऋतुः}}
{\sunmoonrsdata{06:35}{17:50}{18:17}{---}{12:12}
{\kalas{04:53 05:44 09:35 08:50 10:20 16:20 11:05 13:20 15:35 17:05 18:41 21:01 22:37 01:48*}}}
{\tnykdata{\anga{\tithi{15}{पौर्णमासी}}{\time{3-16}{07:54}}\hspace{1ex}\anga{\tithi{16}{कृष्ण-प्रथमा}}{\time{53-52}{04:08*}}\hspace{1ex}\avamA{}}%
{\anga{आर्द्रा}{\time{12-55}{11:45}}\hspace{1ex}}{चन्द्रराशिः—\mbox{मिथुनम्\RIGHTarrow{03:30*}}}%
{\anga{इन्द्रः}{\time{45-58}{00:58*}}\hspace{1ex}\uanga{वैधृतिः}}%
{\anga{बवः}{\time{3-16}{07:54}}\hspace{1ex}\anga{बालवः}{\time{28-31}{18:00}}\hspace{1ex}\anga{कौलवः}{\time{53-52}{04:08*}}\hspace{1ex}\uanga{तैतिलः}}{}
}
{आर्द्रादर्शनम्\eventsep बदरी ज्योतिर्मठ-प्रतिष्ठापन-जयन्ती~\#{२५०३}\eventsep \tamil{சடைய நாயனார் (61) குருபூஜை}\eventsep पार्वण-प्रायश्चित्तावकाशः दर्शे\eventsep पूर्णमासेष्टिः\eventsep पूर्णिमा-व्रतम्\eventsep रमण-महर्षि-जयन्ती~\#{१३९}\eventsep स्थालीपाकः\eventsep तैत्तिरीय-उत्सर्गः पौर्णमास्याम्\eventsep तुङ्गभद्रा~शृङ्गगिरि शारदामठ-प्रतिष्ठापन-जयन्ती~\#{२५०१}\eventsep \tamil{உந்து~மதக்களிற்றன்}\eventsep शाकम्भरी-जयन्ती}
{Tue} 
\cfoot{\rygdata{15:01--16:26}{09:24--10:48}{12:12--13:37}}
\caldata{JANUARY}{3}{\sunmonth{धनुः}{19}{}{पौषः}{हेमन्तऋतुः}{बुधः}{हेमलम्बः}{दक्षिणायनम्}{हेमन्तऋतुः}}
{\sunmoonsrdata{06:35}{17:51}{19:23}{07:33}{12:13}
{\kalas{04:53 05:44 09:35 08:50 10:20 16:20 11:05 13:20 15:36 17:06 18:42 21:02 22:37 01:49*}}}
{\tnykdata{\anga{\tithi{17}{कृष्ण-द्वितीया}}{\time{45-4}{00:37*}}\hspace{1ex}}%
{\anga{पुनर्वसुः}{\time{5-27}{08:46}}\hspace{1ex}\anga{पुष्यः}{\time{58-44}{06:05*}}\hspace{1ex}}{चन्द्रराशिः—\mbox{कर्कटः}}%
{\anga{वैधृतिः}{\time{35-33}{20:49}}\hspace{1ex}\uanga{विष्कम्भः}}%
{\anga{तैतिलः}{\time{19-21}{14:20}}\hspace{1ex}\anga{गरः}{\time{45-4}{00:37*}}\hspace{1ex}\uanga{वणिजः}}{}
}
{काञ्ची ३७ जगद्गुरु श्री-विद्याघनेन्द्र सरस्वती ३ आराधना~\#{१२३०}\eventsep काञ्ची ६२ जगद्गुरु श्री-चन्द्रशेखरेन्द्र सरस्वती ४ आराधना~\#{२३५}\eventsep वैधृति-श्राद्धम्}
{Wed} 
\cfoot{\rygdata{12:13--13:37}{08:00--09:24}{10:49--12:13}}
\caldata{JANUARY}{4}{\sunmonth{धनुः}{20}{}{पौषः}{हेमन्तऋतुः}{गुरुः}{हेमलम्बः}{दक्षिणायनम्}{हेमन्तऋतुः}}
{\sunmoonsrdata{06:36}{17:51}{20:26}{08:29}{12:13}
{\kalas{04:54 05:45 09:36 08:51 10:21 16:21 11:06 13:21 15:36 17:06 18:42 21:02 22:38 01:49*}}}
{\tnykdata{\anga{\tithi{18}{कृष्ण-तृतीया}}{\time{37-19}{21:31}}\hspace{1ex}}%
{\anga{आश्रेषा}{\time{53-9}{03:51*}}\hspace{1ex}}{चन्द्रराशिः—\mbox{कर्कटः\RIGHTarrow{03:51*}}}%
{\anga{विष्कम्भः}{\time{25-55}{16:58}}\hspace{1ex}\uanga{प्रीतिः}}%
{\anga{वणिजः}{\time{11-2}{11:01}}\hspace{1ex}\anga{विष्टिः}{\time{37-19}{21:31}}\hspace{1ex}\uanga{बवः}}{}
}
{}
{Thu} 
\cfoot{\rygdata{13:38--15:02}{06:36--08:00}{09:24--10:49}}
\caldata{JANUARY}{5}{\sunmonth{धनुः}{21}{}{पौषः}{हेमन्तऋतुः}{शुक्रः}{हेमलम्बः}{दक्षिणायनम्}{हेमन्तऋतुः}}
{\sunmoonsrdata{06:36}{17:52}{21:24}{09:20}{12:14}
{\kalas{04:54 05:45 09:36 08:51 10:21 16:22 11:06 13:21 15:37 17:07 18:43 21:03 22:38 01:50*}}}
{\tnykdata{\anga{\tithi{19}{कृष्ण-चतुर्थी}}{\time{30-59}{19:00}}\hspace{1ex}}%
{\anga{मघा}{\time{49-4}{02:14*}}\hspace{1ex}}{चन्द्रराशिः—\mbox{सिंहः}}%
{\anga{प्रीतिः}{\time{17-23}{13:33}}\hspace{1ex}\uanga{आयुष्मान्}}%
{\anga{बवः}{\time{3-57}{08:11}}\hspace{1ex}\anga{बालवः}{\time{30-59}{19:00}}\hspace{1ex}\anga{कौलवः}{\time{58-27}{05:59*}}\hspace{1ex}\uanga{तैतिलः}}{}
}
{लम्बोदर-महागणपति सङ्कटहर-चतुर्थी-व्रतम्}
{Fri} 
\cfoot{\rygdata{10:49--12:14}{15:03--16:27}{08:00--09:25}}
\caldata{JANUARY}{6}{\sunmonth{धनुः}{22}{}{पौषः}{हेमन्तऋतुः}{शनिः}{हेमलम्बः}{दक्षिणायनम्}{हेमन्तऋतुः}}
{\sunmoonsrdata{06:36}{17:52}{22:19}{10:06}{12:14}
{\kalas{04:54 05:45 09:37 08:51 10:22 16:22 11:07 13:22 15:37 17:07 18:43 21:03 22:39 01:50*}}}
{\tnykdata{\anga{\tithi{20}{कृष्ण-पञ्चमी}}{\time{26-22}{17:09}}\hspace{1ex}}%
{\anga{पूर्वफल्गुनी}{\time{46-43}{01:18*}}\hspace{1ex}}{चन्द्रराशिः—\mbox{सिंहः}}%
{\anga{आयुष्मान्}{\time{10-10}{10:40}}\hspace{1ex}\uanga{सौभाग्यः}}%
{\anga{तैतिलः}{\time{26-22}{17:09}}\hspace{1ex}\anga{गरः}{\time{54-45}{04:31*}}\hspace{1ex}\uanga{वणिजः}}{}
}
{त्यागराज-आराधना/बहुल-पञ्चमी~\#{१७१}}
{Sat} 
\cfoot{\rygdata{09:25--10:50}{13:39--15:03}{06:36--08:01}}
\caldata{JANUARY}{7}{\sunmonth{धनुः}{23}{}{पौषः}{हेमन्तऋतुः}{भानुः}{हेमलम्बः}{दक्षिणायनम्}{हेमन्तऋतुः}}
{\sunmoonsrdata{06:37}{17:53}{23:12}{10:49}{12:15}
{\kalas{04:55 05:46 09:37 08:52 10:22 16:23 11:07 13:22 15:38 17:08 18:44 21:04 22:39 01:50*}}}
{\tnykdata{\anga{\tithi{21}{कृष्ण-षष्ठी}}{\time{23-38}{16:04}}\hspace{1ex}}%
{\anga{उत्तरफल्गुनी}{\time{46-17}{01:07*}}\hspace{1ex}}{चन्द्रराशिः—\mbox{सिंहः\RIGHTarrow{07:11}}}%
{\anga{सौभाग्यः}{\time{4-26}{08:23}}\hspace{1ex}\uanga{शोभनः}}%
{\anga{वणिजः}{\time{23-38}{16:04}}\hspace{1ex}\anga{विष्टिः}{\time{53-1}{03:49*}}\hspace{1ex}\uanga{बवः}}{}
}
{\tamil{இயற்பகை நாயனார் (2) குருபூஜை}}
{Sun} 
\cfoot{\rygdata{16:28--17:53}{12:15--13:39}{15:04--16:28}}
\caldata{JANUARY}{8}{\sunmonth{धनुः}{24}{}{पौषः}{हेमन्तऋतुः}{सोमः}{हेमलम्बः}{दक्षिणायनम्}{हेमन्तऋतुः}}
{\sunmoonsrdata{06:37}{17:53}{00:02*}{11:30}{12:15}
{\kalas{04:55 05:46 09:37 08:52 10:22 16:23 11:07 13:23 15:38 17:08 18:44 21:04 22:40 01:51*}}}
{\tnykdata{\anga{\tithi{22}{कृष्ण-सप्तमी}}{\time{22-53}{15:46}}\hspace{1ex}}%
{\anga{हस्तः}{\time{47-45}{01:43*}}\hspace{1ex}}{चन्द्रराशिः—\mbox{कन्या}}%
{\anga{शोभनः}{\time{0-14}{06:43}}\hspace{1ex}\anga{अतिगण्डः}{\time{57-34}{05:39*}}\hspace{1ex}\uanga{सुकर्म}}%
{\anga{बवः}{\time{22-53}{15:46}}\hspace{1ex}\anga{बालवः}{\time{53-15}{03:55*}}\hspace{1ex}\uanga{कौलवः}}{}
}
{पञ्च-पर्व-पूजा (अष्टमी)\eventsep पौष-अष्टका-पूर्वेद्युः\eventsep विवेकानन्द-जयन्ती~\#{१५६}}
{Mon} 
\cfoot{\rygdata{08:01--09:26}{10:51--12:15}{13:40--15:04}}
\caldata{JANUARY}{9}{\sunmonth{धनुः}{25}{}{पौषः}{हेमन्तऋतुः}{मङ्गलः}{हेमलम्बः}{दक्षिणायनम्}{हेमन्तऋतुः}}
{\sunmoonsrdata{06:37}{17:54}{00:52*}{12:10}{12:15}
{\kalas{04:55 05:46 09:38 08:52 10:23 16:24 11:08 13:23 15:39 17:09 18:45 21:05 22:40 01:51*}}}
{\tnykdata{\anga{\tithi{23}{कृष्ण-अष्टमी}}{\time{24-3}{16:15}}\hspace{1ex}}%
{\anga{चित्रा}{\time{51-2}{03:02*}}\hspace{1ex}}{चन्द्रराशिः—\mbox{कन्या\RIGHTarrow{14:18}}}%
{\anga{सुकर्म}{\time{56-17}{05:08*}}\hspace{1ex}\uanga{धृतिः}}%
{\anga{कौलवः}{\time{24-3}{16:15}}\hspace{1ex}\anga{तैतिलः}{\time{55-19}{04:45*}}\hspace{1ex}\uanga{गरः}}{}
}
{पौष-अष्टका-श्राद्धम्\eventsep प्रवासि-भारतीय-दिवसम्~\#{१५}}
{Tue} 
\cfoot{\rygdata{15:05--16:29}{09:26--10:51}{12:15--13:40}}
\caldata{JANUARY}{10}{\sunmonth{धनुः}{26}{}{पौषः}{हेमन्तऋतुः}{बुधः}{हेमलम्बः}{दक्षिणायनम्}{हेमन्तऋतुः}}
{\sunmoonsrdata{06:37}{17:54}{01:42*}{12:51}{12:16}
{\kalas{04:56 05:46 09:38 08:53 10:23 16:24 11:08 13:24 15:39 17:09 18:45 21:05 22:41 01:51*}}}
{\tnykdata{\anga{\tithi{24}{कृष्ण-नवमी}}{\time{26-58}{17:25}}\hspace{1ex}}%
{\anga{स्वाती}{\time{55-52}{04:58*}}\hspace{1ex}}{चन्द्रराशिः—\mbox{तुला}}%
{\anga{धृतिः}{\time{56-14}{05:07*}}\hspace{1ex}\uanga{शूलः}}%
{\anga{गरः}{\time{26-58}{17:25}}\hspace{1ex}\anga{वणिजः}{\time{59-1}{06:14*}}\hspace{1ex}\uanga{विष्टिः}}{}
}
{भीष्म-जयन्ती\eventsep \tamil{மானக்கஞ்சாற நாயனார் (11) குருபூஜை}\eventsep पौष-अन्वष्टका-श्राद्धम्}
{Wed} 
\cfoot{\rygdata{12:16--13:41}{08:02--09:27}{10:51--12:16}}
\caldata{JANUARY}{11}{\sunmonth{धनुः}{27}{}{पौषः}{हेमन्तऋतुः}{गुरुः}{हेमलम्बः}{दक्षिणायनम्}{हेमन्तऋतुः}}
{\sunmoonsrdata{06:38}{17:55}{02:32*}{13:33}{12:16}
{\kalas{04:56 05:47 09:38 08:53 10:23 16:25 11:09 13:24 15:40 17:10 18:46 21:06 22:41 01:52*}}}
{\tnykdata{\anga{\tithi{25}{कृष्ण-दशमी}}{\time{31-22}{19:10}}\hspace{1ex}}%
{\fullanga{विशाखा}}{चन्द्रराशिः—\mbox{तुला\RIGHTarrow{00:46*}}}%
{\anga{शूलः}{\time{57-10}{05:30*}}\hspace{1ex}\uanga{गण्डः}}%
{\anga{विष्टिः}{\time{31-22}{19:10}}\hspace{1ex}\uanga{बवः}}{}
}
{\tamil{கூடாரவல்லீ}\eventsep त्रैलोक्य-गौरी-व्रतम्}
{Thu} 
\cfoot{\rygdata{13:41--15:06}{06:38--08:02}{09:27--10:52}}
\caldata{JANUARY}{12}{\sunmonth{धनुः}{28}{}{पौषः}{हेमन्तऋतुः}{शुक्रः}{हेमलम्बः}{दक्षिणायनम्}{हेमन्तऋतुः}}
{\sunmoonsrdata{06:38}{17:56}{03:22*}{14:16}{12:17}
{\kalas{04:56 05:47 09:39 08:53 10:24 16:25 11:09 13:24 15:40 17:10 18:46 21:06 22:42 01:52*}}}
{\tnykdata{\anga{\tithi{26}{कृष्ण-एकादशी}}{\time{36-51}{21:22}}\hspace{1ex}}%
{\anga{विशाखा}{\time{1-56}{07:24}}\hspace{1ex}}{चन्द्रराशिः—\mbox{वृश्चिकः}}%
{\anga{गण्डः}{\time{58-47}{06:09*}}\hspace{1ex}\uanga{वृद्धिः}}%
{\anga{बवः}{\time{3-59}{08:14}}\hspace{1ex}\anga{बालवः}{\time{36-51}{21:22}}\hspace{1ex}\uanga{कौलवः}}{}
}
{\tamil{கறவைகள் பின்சென்று}\eventsep सर्व-षट्तिला-एकादशी}
{Fri} 
\cfoot{\rygdata{10:52--12:17}{15:06--16:31}{08:02--09:27}}
\caldata{JANUARY}{13}{\sunmonth{धनुः}{29}{}{पौषः}{हेमन्तऋतुः}{शनिः}{हेमलम्बः}{दक्षिणायनम्}{हेमन्तऋतुः}}
{\sunmoonsrdata{06:38}{17:56}{04:13*}{15:02}{12:17}
{\kalas{04:56 05:47 09:39 08:54 10:24 16:26 11:09 13:25 15:41 17:11 18:47 21:07 22:42 01:52*}}}
{\tnykdata{\anga{\tithi{27}{कृष्ण-द्वादशी}}{\time{43-4}{23:52}}\hspace{1ex}}%
{\anga{अनूराधा}{\time{8-54}{10:12}}\hspace{1ex}}{चन्द्रराशिः—\mbox{वृश्चिकः}}%
{\fullanga{वृद्धिः}}%
{\anga{कौलवः}{\time{9-54}{10:36}}\hspace{1ex}\anga{तैतिलः}{\time{43-4}{23:52}}\hspace{1ex}\uanga{गरः}}{}
}
{\tamil{போகி}\eventsep पक्षवर्धिनी-महाद्वादशी}
{Sat} 
\cfoot{\rygdata{09:28--10:52}{13:42--15:07}{06:38--08:03}}
\caldata{JANUARY}{14}{\sunmonth{मकरः}{1}{\mbox{धनुः{\tiny\RIGHTarrow}{13:24}}}{पौषः}{हेमन्तऋतुः}{भानुः}{हेमलम्बः}{उत्तरायणम्}{हेमन्तऋतुः}}
{\sunmoonsrdata{06:38}{17:57}{05:02*}{15:50}{12:18}
{\kalas{04:57 05:47 09:39 08:54 10:24 16:26 11:10 13:25 15:41 17:12 18:48 21:07 22:42 01:53*}}}
{\tnykdata{\anga{\tithi{28}{कृष्ण-त्रयोदशी}}{\time{49-41}{02:31*}}\hspace{1ex}}%
{\anga{ज्येष्ठा}{\time{16-23}{13:12}}\hspace{1ex}}{चन्द्रराशिः—\mbox{वृश्चिकः\RIGHTarrow{13:12}}}%
{\anga{वृद्धिः}{\time{0-52}{06:59}}\hspace{1ex}\uanga{ध्रुवः}}%
{\anga{गरः}{\time{16-21}{13:11}}\hspace{1ex}\anga{वणिजः}{\time{49-41}{02:31*}}\hspace{1ex}\uanga{विष्टिः}}{}
}
{\tamil{மதுரை மீனாக்ஷீ கோயிலில் கல் யானைக்கு கரும்பு கோடுத்த லீலை}\eventsep मकर-ज्योतिः\eventsep मकर-सङ्क्रान्तिः\eventsep मकर-सङ्क्रमण-पुण्यकालः~13:24\RIGHTarrow{}17:57\eventsep प्रदोष-व्रतम्~17:57\RIGHTarrow{}19:32\eventsep रवि-सङ्क्रमण-पुण्यकालः~07:00\RIGHTarrow{}17:57\eventsep सङ्क्रमण-दिन-अपराह्ण-पुण्यकालः~12:17\RIGHTarrow{}17:57}
{Sun} 
\cfoot{\rygdata{16:32--17:57}{12:18--13:42}{15:07--16:32}}
\caldata{JANUARY}{15}{\sunmonth{मकरः}{2}{}{पौषः}{हेमन्तऋतुः}{सोमः}{हेमलम्बः}{उत्तरायणम्}{हेमन्तऋतुः}}
{\sunmoonsrdata{06:38}{17:57}{05:50*}{16:39}{12:18}
{\kalas{04:57 05:48 09:39 08:54 10:25 16:27 11:10 13:26 15:42 17:12 18:48 21:08 22:43 01:53*}}}
{\tnykdata{\anga{\tithi{29}{कृष्ण-चतुर्दशी}}{\time{56-21}{05:11*}}\hspace{1ex}}%
{\anga{मूला}{\time{24-5}{16:17}}\hspace{1ex}}{चन्द्रराशिः—\mbox{धनुः}}%
{\anga{ध्रुवः}{\time{3-10}{07:55}}\hspace{1ex}\uanga{व्याघातः}}%
{\anga{विष्टिः}{\time{23-1}{15:51}}\hspace{1ex}\anga{शकुनिः}{\time{56-21}{05:11*}}\hspace{1ex}\uanga{चतुष्पात्}}{}
}
{इन्द्र-पूजा/गो-पूजा\eventsep \tamil{கனுப்~பொங்கல்}\eventsep मासशिवरात्रिः\eventsep पञ्च-पर्व-पूजा (चतुर्दशी)}
{Mon} 
\cfoot{\rygdata{08:03--09:28}{10:53--12:18}{13:43--15:08}}
\caldata{JANUARY}{16}{\sunmonth{मकरः}{3}{}{पौषः}{हेमन्तऋतुः}{मङ्गलः}{हेमलम्बः}{उत्तरायणम्}{हेमन्तऋतुः}}
{\sunmoonsrdata{06:39}{17:58}{06:37*}{17:29}{12:18}
{\kalas{04:57 05:48 09:40 08:54 10:25 16:27 11:10 13:26 15:42 17:13 18:49 21:08 22:43 01:53*}}}
{\tnykdata{\fulltithi{\tithi{30}{अमावास्या}}}%
{\anga{पूर्वाषाढा}{\time{31-43}{19:20}}\hspace{1ex}}{चन्द्रराशिः—\mbox{धनुः\RIGHTarrow{02:05*}}}%
{\anga{व्याघातः}{\time{5-29}{08:50}}\hspace{1ex}\uanga{हर्षणः}}%
{\anga{चतुष्पात्}{\time{29-38}{18:30}}\hspace{1ex}\uanga{नाग}}{}
}
{पार्वणव्रतम् अमावास्यायाम्\eventsep पञ्च-पर्व-पूजा (अमावास्या)\eventsep सर्व-मौनि (पौष/मकर) अमावास्या (अलभ्यम्–पुष्कला)\eventsep \tamil{திருநெல்வேலி நெல்லையப்பர் பத்ர தீப திருவிழா}}
{Tue} 
\cfoot{\rygdata{15:08--16:33}{09:28--10:53}{12:18--13:43}}
\caldata{JANUARY}{17}{\sunmonth{मकरः}{4}{}{पौषः}{हेमन्तऋतुः}{बुधः}{हेमलम्बः}{उत्तरायणम्}{हेमन्तऋतुः}}
{\sunmoonsrdata{06:39}{17:58}{---}{18:19}{12:19}
{\kalas{04:57 05:48 09:40 08:55 10:25 16:28 11:11 13:26 15:42 17:13 18:49 21:08 22:44 01:54*}}}
{\tnykdata{\anga{\tithi{30}{अमावास्या}}{\time{2-50}{07:47}}\hspace{1ex}}%
{\anga{उत्तराषाढा}{\time{39-3}{22:16}}\hspace{1ex}}{चन्द्रराशिः—\mbox{मकरः}}%
{\anga{हर्षणः}{\time{7-38}{09:42}}\hspace{1ex}\uanga{वज्रम्}}%
{\anga{नाग}{\time{2-50}{07:47}}\hspace{1ex}\anga{किंस्तुघ्नः}{\time{35-56}{21:01}}\hspace{1ex}\uanga{बवः}}{}
}
{दर्शेष्टिः\eventsep पार्वण-प्रायश्चित्तावकाशः पौर्णमास्याम्\eventsep स्थालीपाकः}
{Wed} 
\cfoot{\rygdata{12:19--13:43}{08:04--09:29}{10:54--12:19}}
\caldata{JANUARY}{18}{\sunmonth{मकरः}{5}{}{माघः}{शिशिरऋतुः}{गुरुः}{हेमलम्बः}{उत्तरायणम्}{हेमन्तऋतुः}}
{\sunmoonrsdata{06:39}{17:59}{07:20}{19:09}{12:19}
{\kalas{04:57 05:48 09:40 08:55 10:25 16:28 11:11 13:27 15:43 17:14 18:50 21:09 22:44 01:54*}}}
{\tnykdata{\anga{\tithi{1}{शुक्ल-प्रथमा}}{\time{8-53}{10:12}}\hspace{1ex}}%
{\anga{श्रवणः}{\time{45-51}{01:00*}}\hspace{1ex}}{चन्द्रराशिः—\mbox{मकरः}}%
{\anga{वज्रम्}{\time{9-27}{10:26}}\hspace{1ex}\uanga{सिद्धिः}}%
{\anga{बवः}{\time{8-53}{10:12}}\hspace{1ex}\anga{बालवः}{\time{41-41}{23:19}}\hspace{1ex}\uanga{कौलवः}}{}
}
{चन्द्र-दर्शनम्~17:59\RIGHTarrow{}19:09\eventsep श्रवण-व्रतम्\eventsep श्यामळानवरात्र-आरम्भः}
{Thu} 
\cfoot{\rygdata{13:44--15:09}{06:39--08:04}{09:29--10:54}}
\caldata{JANUARY}{19}{\sunmonth{मकरः}{6}{}{माघः}{शिशिरऋतुः}{शुक्रः}{हेमलम्बः}{उत्तरायणम्}{हेमन्तऋतुः}}
{\sunmoonrsdata{06:39}{18:00}{08:02}{19:58}{12:19}
{\kalas{04:58 05:48 09:40 08:55 10:26 16:29 11:11 13:27 15:43 17:14 18:50 21:09 22:44 01:54*}}}
{\tnykdata{\anga{\tithi{2}{शुक्ल-द्वितीया}}{\time{14-17}{12:22}}\hspace{1ex}}%
{\anga{श्रविष्ठा}{\time{51-56}{03:25*}}\hspace{1ex}}{चन्द्रराशिः—\mbox{मकरः\RIGHTarrow{14:15}}}%
{\anga{सिद्धिः}{\time{10-47}{10:58}}\hspace{1ex}\uanga{व्यतीपातः}}%
{\anga{कौलवः}{\time{14-17}{12:22}}\hspace{1ex}\anga{तैतिलः}{\time{46-40}{01:19*}}\hspace{1ex}\uanga{गरः}}{}
}
{\tamil{தை~வெள்ளிக்கிழமை}\eventsep व्यतीपात-श्राद्धम्}
{Fri} 
\cfoot{\rygdata{10:54--12:19}{15:09--16:34}{08:04--09:29}}
\caldata{JANUARY}{20}{\sunmonth{मकरः}{7}{}{माघः}{शिशिरऋतुः}{शनिः}{हेमलम्बः}{उत्तरायणम्}{हेमन्तऋतुः}}
{\sunmoonrsdata{06:39}{18:00}{08:42}{20:46}{12:20}
{\kalas{04:58 05:48 09:41 08:55 10:26 16:29 11:11 13:28 15:44 17:15 18:51 21:10 22:45 01:54*}}}
{\tnykdata{\anga{\tithi{3}{शुक्ल-तृतीया}}{\time{18-49}{14:11}}\hspace{1ex}}%
{\anga{शतभिषक्}{\time{57-3}{05:29*}}\hspace{1ex}}{चन्द्रराशिः—\mbox{कुम्भः}}%
{\anga{व्यतीपातः}{\time{11-29}{11:15}}\hspace{1ex}\uanga{वरीयान्}}%
{\anga{गरः}{\time{18-49}{14:11}}\hspace{1ex}\anga{वणिजः}{\time{50-41}{02:55*}}\hspace{1ex}\uanga{विष्टिः}}{}
}
{(सायन) विष्णुपदी-पुण्यकालः~06:39\RIGHTarrow{}15:03\eventsep \tamil{அப்பூதியடிகள் நாயனார் (24) குருபூஜை}\eventsep सायन-सङ्क्रमण-दिन-पूर्वाह्ण-पुण्यकालः~06:39\RIGHTarrow{}12:19\eventsep तपो-मासः/शिशिरऋतुः~08:39\RIGHTarrow{}\eventsep शान्ता~वरकुन्द-चतुर्थी}
{Sat} 
\cfoot{\rygdata{09:29--10:54}{13:45--15:10}{06:39--08:04}}
\caldata{JANUARY}{21}{\sunmonth{मकरः}{8}{}{माघः}{शिशिरऋतुः}{भानुः}{हेमलम्बः}{उत्तरायणम्}{हेमन्तऋतुः}}
{\sunmoonrsdata{06:39}{18:01}{09:21}{21:34}{12:20}
{\kalas{04:58 05:48 09:41 08:55 10:26 16:30 11:12 13:28 15:44 17:15 18:51 21:10 22:45 01:55*}}}
{\tnykdata{\anga{\tithi{4}{शुक्ल-चतुर्थी}}{\time{22-15}{15:33}}\hspace{1ex}}%
{\fullanga{पूर्वप्रोष्ठपदा}}{चन्द्रराशिः—\mbox{कुम्भः\RIGHTarrow{00:43*}}}%
{\anga{वरीयान्}{\time{11-23}{11:12}}\hspace{1ex}\uanga{परिघः}}%
{\anga{विष्टिः}{\time{22-15}{15:33}}\hspace{1ex}\anga{बवः}{\time{53-30}{04:03*}}\hspace{1ex}\uanga{बालवः}}{}
}
{मार्कण्डेय-जयन्ती\eventsep शुक्ल-चतुर्थी-व्रतम्}
{Sun} 
\cfoot{\rygdata{16:35--18:01}{12:20--13:45}{15:10--16:35}}
\caldata{JANUARY}{22}{\sunmonth{मकरः}{9}{}{माघः}{शिशिरऋतुः}{सोमः}{हेमलम्बः}{उत्तरायणम्}{हेमन्तऋतुः}}
{\sunmoonrsdata{06:39}{18:01}{09:59}{22:23}{12:20}
{\kalas{04:58 05:48 09:41 08:55 10:26 16:30 11:12 13:28 15:45 17:16 18:52 21:11 22:45 01:55*}}}
{\tnykdata{\anga{\tithi{5}{शुक्ल-पञ्चमी}}{\time{24-23}{16:24}}\hspace{1ex}}%
{\anga{पूर्वप्रोष्ठपदा}{\time{1-1}{07:04}}\hspace{1ex}}{चन्द्रराशिः—\mbox{मीनः}}%
{\anga{परिघः}{\time{10-20}{10:47}}\hspace{1ex}\uanga{शिवः}}%
{\anga{बालवः}{\time{24-23}{16:24}}\hspace{1ex}\anga{कौलवः}{\time{54-54}{04:37*}}\hspace{1ex}\uanga{तैतिलः}}{}
}
{माघी-सरस्वती-पूजा\eventsep सर्प-पूजा\eventsep वसन्त-श्री-पञ्चमी\eventsep श्रीराम-वनवास-गमनम्}
{Mon} 
\cfoot{\rygdata{08:04--09:30}{10:55--12:20}{13:45--15:11}}
\caldata{JANUARY}{23}{\sunmonth{मकरः}{10}{}{माघः}{शिशिरऋतुः}{मङ्गलः}{हेमलम्बः}{उत्तरायणम्}{हेमन्तऋतुः}}
{\sunmoonrsdata{06:39}{18:02}{10:39}{23:13}{12:20}
{\kalas{04:58 05:49 09:41 08:56 10:27 16:31 11:12 13:28 15:45 17:16 18:52 21:11 22:46 01:55*}}}
{\tnykdata{\anga{\tithi{6}{शुक्ल-षष्ठी}}{\time{25-1}{16:40}}\hspace{1ex}}%
{\anga{उत्तरप्रोष्ठपदा}{\time{3-37}{08:06}}\hspace{1ex}}{चन्द्रराशिः—\mbox{मीनः}}%
{\anga{शिवः}{\time{8-10}{09:55}}\hspace{1ex}\uanga{सिद्धः}}%
{\anga{तैतिलः}{\time{25-1}{16:40}}\hspace{1ex}\anga{गरः}{\time{54-45}{04:33*}}\hspace{1ex}\uanga{वणिजः}}{}
}
{षष्ठी-व्रतम्\eventsep \tamil{கலிக்கம்ப நாயனார் (42) குருபூஜை}}
{Tue} 
\cfoot{\rygdata{15:11--16:36}{09:30--10:55}{12:20--13:46}}
\caldata{JANUARY}{24}{\sunmonth{मकरः}{11}{}{माघः}{शिशिरऋतुः}{बुधः}{हेमलम्बः}{उत्तरायणम्}{हेमन्तऋतुः}}
{\sunmoonrsdata{06:39}{18:02}{11:20}{00:06*}{12:21}
{\kalas{04:58 05:49 09:41 08:56 10:27 16:31 11:12 13:29 15:45 17:16 18:52 21:11 22:46 01:55*}}}
{\tnykdata{\anga{\tithi{7}{शुक्ल-सप्तमी}}{\time{24-3}{16:17}}\hspace{1ex}}%
{\anga{रेवती}{\time{4-41}{08:32}}\hspace{1ex}}{चन्द्रराशिः—\mbox{मीनः\RIGHTarrow{08:32}}}%
{\anga{सिद्धः}{\time{4-48}{08:34}}\hspace{1ex}\uanga{साध्यः}}%
{\anga{वणिजः}{\time{24-3}{16:17}}\hspace{1ex}\anga{विष्टिः}{\time{52-57}{03:50*}}\hspace{1ex}\uanga{बवः}}{}
}
{अचला-सप्तमी-व्रतम्\eventsep द्वारका-मठ-प्रतिष्ठापन-जयन्ती~\#{२५०८}\eventsep मन्वादिः-(सावर्णिः-[८])\eventsep नर्मदा-जयन्ती\eventsep रथ-सप्तमी}
{Wed} 
\cfoot{\rygdata{12:21--13:46}{08:04--09:30}{10:55--12:21}}
\caldata{JANUARY}{25}{\sunmonth{मकरः}{12}{}{माघः}{शिशिरऋतुः}{गुरुः}{हेमलम्बः}{उत्तरायणम्}{हेमन्तऋतुः}}
{\sunmoonrsdata{06:39}{18:03}{12:05}{01:03*}{12:21}
{\kalas{04:58 05:49 09:41 08:56 10:27 16:31 11:12 13:29 15:46 17:17 18:53 21:12 22:46 01:55*}}}
{\tnykdata{\anga{\tithi{8}{शुक्ल-अष्टमी}}{\time{21-25}{15:13}}\hspace{1ex}}%
{\anga{अश्विनी}{\time{4-9}{08:19}}\hspace{1ex}}{चन्द्रराशिः—\mbox{मेषः}}%
{\anga{साध्यः}{\time{0-8}{06:42}}\hspace{1ex}\anga{शुभः}{\time{54-10}{04:19*}}\hspace{1ex}\uanga{शुक्लः}}%
{\anga{बवः}{\time{21-25}{15:13}}\hspace{1ex}\anga{बालवः}{\time{49-30}{02:27*}}\hspace{1ex}\uanga{कौलवः}}{}
}
{भीष्माष्टमी\eventsep खोडियार-माता-जयन्ती\eventsep \tamil{திருநெல்வேலி நெல்லையப்பர் நெல்லுக்கு வேலி கட்டிய லீலை}}
{Thu} 
\cfoot{\rygdata{13:46--15:12}{06:39--08:04}{09:30--10:55}}
\caldata{JANUARY}{26}{\sunmonth{मकरः}{13}{}{माघः}{शिशिरऋतुः}{शुक्रः}{हेमलम्बः}{उत्तरायणम्}{हेमन्तऋतुः}}
{\sunmoonrsdata{06:39}{18:03}{12:55}{02:03*}{12:21}
{\kalas{04:58 05:49 09:41 08:56 10:27 16:32 11:13 13:29 15:46 17:17 18:53 21:12 22:46 01:55*}}}
{\tnykdata{\anga{\tithi{9}{शुक्ल-नवमी}}{\time{17-11}{13:32}}\hspace{1ex}}%
{\anga{अपभरणी}{\time{2-1}{07:28}}\hspace{1ex}\anga{कृत्तिका}{\time{58-24}{06:01*}}\hspace{1ex}}{चन्द्रराशिः—\mbox{मेषः\RIGHTarrow{13:09}}}%
{\anga{शुक्लः}{\time{46-59}{01:27*}}\hspace{1ex}\uanga{ब्रह्म}}%
{\anga{कौलवः}{\time{17-11}{13:32}}\hspace{1ex}\anga{तैतिलः}{\time{44-30}{00:27*}}\hspace{1ex}\uanga{गरः}}{}
}
{काञ्ची ११ जगद्गुरु श्री-शिवानन्द चिद्घनेन्द्र सरस्वती आराधना~\#{१८४६}\eventsep कृत्तिका-व्रतम्\eventsep मध्व-नवमी\eventsep मकर-श्रवण-कार्त्तिकोत्सवः\eventsep \tamil{தை~வெள்ளிக்கிழமை}\eventsep श्यामळानवरात्र-समापनम्}
{Fri} 
\cfoot{\rygdata{10:56--12:21}{15:12--16:38}{08:04--09:30}}
\caldata{JANUARY}{27}{\sunmonth{मकरः}{14}{}{माघः}{शिशिरऋतुः}{शनिः}{हेमलम्बः}{उत्तरायणम्}{हेमन्तऋतुः}}
{\sunmoonrsdata{06:39}{18:04}{13:49}{03:05*}{12:21}
{\kalas{04:58 05:49 09:42 08:56 10:27 16:32 11:13 13:30 15:47 17:18 18:54 21:12 22:47 01:56*}}}
{\tnykdata{\anga{\tithi{10}{शुक्ल-दशमी}}{\time{11-29}{11:15}}\hspace{1ex}}%
{\anga{रोहिणी}{\time{53-29}{04:03*}}\hspace{1ex}}{चन्द्रराशिः—\mbox{वृषभः}}%
{\anga{ब्रह्म}{\time{38-43}{22:08}}\hspace{1ex}\uanga{इन्द्रः}}%
{\anga{गरः}{\time{11-29}{11:15}}\hspace{1ex}\anga{वणिजः}{\time{38-8}{21:54}}\hspace{1ex}\uanga{विष्टिः}}{}
}
{साम्ब-दशमी (सूर्यपूजा)\eventsep स्मार्त-जया/भैमी-एकादशी (गृहस्थ)\eventsep शनिरोहिणी-पुण्यकालः}
{Sat} 
\cfoot{\rygdata{09:30--10:56}{13:47--15:12}{06:39--08:04}}
\caldata{JANUARY}{28}{\sunmonth{मकरः}{15}{}{माघः}{शिशिरऋतुः}{भानुः}{हेमलम्बः}{उत्तरायणम्}{हेमन्तऋतुः}}
{\sunmoonrsdata{06:39}{18:04}{14:49}{04:09*}{12:21}
{\kalas{04:58 05:49 09:42 08:56 10:27 16:33 11:13 13:30 15:47 17:18 18:54 21:13 22:47 01:56*}}}
{\tnykdata{\anga{\tithi{11}{शुक्ल-एकादशी}}{\time{4-31}{08:28}}\hspace{1ex}\anga{\tithi{12}{शुक्ल-द्वादशी}}{\time{56-36}{05:18*}}\hspace{1ex}\avamA{}}%
{\anga{मृगशीर्षम्}{\time{47-33}{01:40*}}\hspace{1ex}}{चन्द्रराशिः—\mbox{वृषभः\RIGHTarrow{14:54}}}%
{\anga{इन्द्रः}{\time{29-33}{18:28}}\hspace{1ex}\uanga{वैधृतिः}}%
{\anga{विष्टिः}{\time{4-31}{08:28}}\hspace{1ex}\anga{बवः}{\time{30-39}{18:55}}\hspace{1ex}\anga{बालवः}{\time{56-36}{05:18*}}\hspace{1ex}\uanga{कौलवः}}{}
}
{भीष्म-द्वादशी\eventsep हरिवासरः\RIGHTarrow{}13:42\eventsep \tamil{கண்ணப்ப நாயனார் (9) குருபூஜை}\eventsep स्मार्त-जया/भैमी-एकादशी (सन्न्यस्त)\eventsep तिलपद्म-द्वादशी/तिलोत्पत्ति\eventsep त्रिस्पर्शा-महाद्वादशी\eventsep वैष्णव-जया/भैमी-एकादशी\eventsep वराह-द्वादशी}
{Sun} 
\cfoot{\rygdata{16:38--18:04}{12:21--13:47}{15:13--16:38}}
\caldata{JANUARY}{29}{\sunmonth{मकरः}{16}{}{माघः}{शिशिरऋतुः}{सोमः}{हेमलम्बः}{उत्तरायणम्}{हेमन्तऋतुः}}
{\sunmoonrsdata{06:39}{18:04}{15:53}{05:11*}{12:22}
{\kalas{04:58 05:49 09:42 08:56 10:27 16:33 11:13 13:30 15:47 17:19 18:55 21:13 22:47 01:56*}}}
{\tnykdata{\anga{\tithi{13}{शुक्ल-त्रयोदशी}}{\time{48-5}{01:53*}}\hspace{1ex}}%
{\anga{आर्द्रा}{\time{40-57}{23:02}}\hspace{1ex}}{चन्द्रराशिः—\mbox{मिथुनम्}}%
{\anga{वैधृतिः}{\time{19-45}{14:33}}\hspace{1ex}\uanga{विष्कम्भः}}%
{\anga{कौलवः}{\time{22-24}{15:37}}\hspace{1ex}\anga{तैतिलः}{\time{48-5}{01:53*}}\hspace{1ex}\uanga{गरः}}{}
}
{\tamil{அரிவாட்டாய நாயனார் (12) குருபூஜை}\eventsep \tamil{கபாலீ தெப்போத்ஸவம்}\eventsep सोम-प्रदोष-व्रतम्~18:04\RIGHTarrow{}19:39\eventsep वैधृति-श्राद्धम्\eventsep वराह-कल्पादिः}
{Mon} 
\cfoot{\rygdata{08:04--09:30}{10:56--12:22}{13:47--15:13}}
\caldata{JANUARY}{30}{\sunmonth{मकरः}{17}{}{माघः}{शिशिरऋतुः}{मङ्गलः}{हेमलम्बः}{उत्तरायणम्}{हेमन्तऋतुः}}
{\sunmoonrsdata{06:39}{18:05}{16:59}{06:10*}{12:22}
{\kalas{04:58 05:48 09:42 08:56 10:27 16:33 11:13 13:30 15:48 17:19 18:55 21:13 22:48 01:56*}}}
{\tnykdata{\anga{\tithi{14}{शुक्ल-चतुर्दशी}}{\time{39-20}{22:23}}\hspace{1ex}}%
{\anga{पुनर्वसुः}{\time{34-4}{20:16}}\hspace{1ex}}{चन्द्रराशिः—\mbox{मिथुनम्\RIGHTarrow{14:58}}}%
{\anga{विष्कम्भः}{\time{9-34}{10:28}}\hspace{1ex}\anga{प्रीतिः}{\time{59-20}{06:23*}}\hspace{1ex}\uanga{आयुष्मान्}}%
{\anga{गरः}{\time{13-43}{12:08}}\hspace{1ex}\anga{वणिजः}{\time{39-20}{22:23}}\hspace{1ex}\uanga{विष्टिः}}{}
}
{देवी-पर्व-११\eventsep \tamil{கபாலீ தெப்போத்ஸவம்}\eventsep पञ्च-पर्व-पूजा (पूर्णिमा)\eventsep मकर-पुष्योत्सवः}
{Tue} 
\cfoot{\rygdata{15:13--16:39}{09:30--10:56}{12:22--13:48}}
\caldata{JANUARY}{31}{\sunmonth{मकरः}{18}{}{माघः}{शिशिरऋतुः}{बुधः}{हेमलम्बः}{उत्तरायणम्}{हेमन्तऋतुः}}
{\sunmoonrsdata{06:39}{18:05}{18:03}{---}{12:22}
{\kalas{04:58 05:48 09:42 08:56 10:27 16:34 11:13 13:31 15:48 17:19 18:56 21:14 22:48 01:56*}}}
{\tnykdata{\anga{\tithi{15}{पौर्णमासी}}{\time{30-44}{18:56}}\hspace{1ex}}%
{\anga{पुष्यः}{\time{27-18}{17:34}}\hspace{1ex}}{चन्द्रराशिः—\mbox{कर्कटः}}%
{\anga{आयुष्मान्}{\time{49-21}{02:23*}}\hspace{1ex}\uanga{सौभाग्यः}}%
{\anga{विष्टिः}{\time{4-59}{08:38}}\hspace{1ex}\anga{बवः}{\time{30-44}{18:56}}\hspace{1ex}\anga{बालवः}{\time{56-37}{05:18*}}\hspace{1ex}\uanga{कौलवः}}{}
}
{आयुष्मद्-बव-सौम्य-संयॊगः\eventsep चन्द्र-ग्रहणम् (राहुग्रस्तोदय)~18:03\RIGHTarrow{}20:41\eventsep \tamil{கபாலீ தெப்போத்ஸவம்}\eventsep ललिता-जयन्ती\eventsep माघ-पूर्णिमा\eventsep माघ-पूर्णिमा-स्नानम्\eventsep पार्वणव्रतम् पूर्णिमायाम्\eventsep पूर्णिमा-व्रतम्\eventsep वेङ्कटाचले पूर्णिमा-गरुड-सेवा}
{Wed} 
\cfoot{\rygdata{12:22--13:48}{08:04--09:30}{10:56--12:22}}
\caldata{FEBRUARY}{1}{\sunmonth{मकरः}{19}{}{माघः}{शिशिरऋतुः}{गुरुः}{हेमलम्बः}{उत्तरायणम्}{हेमन्तऋतुः}}
{\sunmoonsrdata{06:38}{18:06}{19:05}{07:04}{12:22}
{\kalas{04:58 05:48 09:42 08:56 10:27 16:34 11:13 13:31 15:48 17:20 18:56 21:14 22:48 01:56*}}}
{\tnykdata{\anga{\tithi{16}{कृष्ण-प्रथमा}}{\time{22-42}{15:43}}\hspace{1ex}}%
{\anga{आश्रेषा}{\time{21-3}{15:04}}\hspace{1ex}}{चन्द्रराशिः—\mbox{कर्कटः\RIGHTarrow{15:04}}}%
{\anga{सौभाग्यः}{\time{39-58}{22:38}}\hspace{1ex}\uanga{शोभनः}}%
{\anga{कौलवः}{\time{22-42}{15:43}}\hspace{1ex}\anga{तैतिलः}{\time{49-1}{02:15*}}\hspace{1ex}\uanga{गरः}}{}
}
{काञ्ची ५१ जगद्गुरु श्री-विद्यातीर्थेन्द्र सरस्वती आराधना~\#{६३३}\eventsep पार्वण-प्रायश्चित्तावकाशः दर्शे\eventsep पूर्णमासेष्टिः\eventsep स्थालीपाकः}
{Thu} 
\cfoot{\rygdata{13:48--15:14}{06:38--08:04}{09:30--10:56}}
\caldata{FEBRUARY}{2}{\sunmonth{मकरः}{20}{}{माघः}{शिशिरऋतुः}{शुक्रः}{हेमलम्बः}{उत्तरायणम्}{हेमन्तऋतुः}}
{\sunmoonsrdata{06:38}{18:06}{20:03}{07:54}{12:22}
{\kalas{04:58 05:48 09:42 08:56 10:28 16:34 11:13 13:31 15:49 17:20 18:56 21:14 22:48 01:56*}}}
{\tnykdata{\anga{\tithi{17}{कृष्ण-द्वितीया}}{\time{15-38}{12:54}}\hspace{1ex}}%
{\anga{मघा}{\time{15-47}{12:57}}\hspace{1ex}}{चन्द्रराशिः—\mbox{सिंहः}}%
{\anga{शोभनः}{\time{31-28}{19:14}}\hspace{1ex}\uanga{अतिगण्डः}}%
{\anga{गरः}{\time{15-38}{12:54}}\hspace{1ex}\anga{वणिजः}{\time{42-34}{23:40}}\hspace{1ex}\uanga{विष्टिः}}{}
}
{\tamil{தை~வெள்ளிக்கிழமை}\eventsep \tamil{திருமழிசையாழ்வார் திருநக்ஷத்திரம்}}
{Fri} 
\cfoot{\rygdata{10:56--12:22}{15:14--16:40}{08:04--09:30}}
\caldata{FEBRUARY}{3}{\sunmonth{मकरः}{21}{}{माघः}{शिशिरऋतुः}{शनिः}{हेमलम्बः}{उत्तरायणम्}{हेमन्तऋतुः}}
{\sunmoonsrdata{06:38}{18:07}{20:59}{08:40}{12:22}
{\kalas{04:58 05:48 09:42 08:56 10:28 16:35 11:13 13:31 15:49 17:21 18:57 21:14 22:48 01:56*}}}
{\tnykdata{\anga{\tithi{18}{कृष्ण-तृतीया}}{\time{9-54}{10:36}}\hspace{1ex}}%
{\anga{पूर्वफल्गुनी}{\time{11-49}{11:22}}\hspace{1ex}}{चन्द्रराशिः—\mbox{सिंहः\RIGHTarrow{17:04}}}%
{\anga{अतिगण्डः}{\time{24-7}{16:17}}\hspace{1ex}\uanga{सुकर्म}}%
{\anga{विष्टिः}{\time{9-54}{10:36}}\hspace{1ex}\anga{बवः}{\time{37-38}{21:41}}\hspace{1ex}\uanga{बालवः}}{}
}
{द्विजप्रिय-महागणपति सङ्कटहर-चतुर्थी-व्रतम्}
{Sat} 
\cfoot{\rygdata{09:30--10:56}{13:48--15:14}{06:38--08:04}}
\caldata{FEBRUARY}{4}{\sunmonth{मकरः}{22}{}{माघः}{शिशिरऋतुः}{भानुः}{हेमलम्बः}{उत्तरायणम्}{हेमन्तऋतुः}}
{\sunmoonsrdata{06:38}{18:07}{21:52}{09:23}{12:22}
{\kalas{04:58 05:48 09:42 08:56 10:28 16:35 11:14 13:31 15:49 17:21 18:57 21:15 22:49 01:56*}}}
{\tnykdata{\anga{\tithi{19}{कृष्ण-चतुर्थी}}{\time{5-49}{08:58}}\hspace{1ex}}%
{\anga{उत्तरफल्गुनी}{\time{9-30}{10:26}}\hspace{1ex}}{चन्द्रराशिः—\mbox{कन्या}}%
{\anga{सुकर्म}{\time{18-9}{13:54}}\hspace{1ex}\uanga{धृतिः}}%
{\anga{बालवः}{\time{5-49}{08:58}}\hspace{1ex}\anga{कौलवः}{\time{34-28}{20:25}}\hspace{1ex}\uanga{तैतिलः}}{}
}
{आदित्यहस्त-पुण्यकालः~10:26\RIGHTarrow{}\eventsep \tamil{சண்டேஶ்வர நாயனார் (19) குருபூஜை}}
{Sun} 
\cfoot{\rygdata{16:41--18:07}{12:22--13:49}{15:15--16:41}}
\caldata{FEBRUARY}{5}{\sunmonth{मकरः}{23}{}{माघः}{शिशिरऋतुः}{सोमः}{हेमलम्बः}{उत्तरायणम्}{हेमन्तऋतुः}}
{\sunmoonsrdata{06:38}{18:07}{22:44}{10:05}{12:22}
{\kalas{04:58 05:48 09:42 08:56 10:28 16:35 11:14 13:32 15:49 17:21 18:57 21:15 22:49 01:56*}}}
{\tnykdata{\anga{\tithi{20}{कृष्ण-पञ्चमी}}{\time{3-38}{08:05}}\hspace{1ex}}%
{\anga{हस्तः}{\time{9-3}{10:15}}\hspace{1ex}}{चन्द्रराशिः—\mbox{कन्या\RIGHTarrow{22:27}}}%
{\anga{धृतिः}{\time{13-43}{12:07}}\hspace{1ex}\uanga{शूलः}}%
{\anga{तैतिलः}{\time{3-38}{08:05}}\hspace{1ex}\anga{गरः}{\time{33-17}{19:57}}\hspace{1ex}\uanga{वणिजः}}{}
}
{}
{Mon} 
\cfoot{\rygdata{08:04--09:30}{10:56--12:23}{13:49--15:15}}
\caldata{FEBRUARY}{6}{\sunmonth{मकरः}{24}{}{माघः}{शिशिरऋतुः}{मङ्गलः}{हेमलम्बः}{उत्तरायणम्}{हेमन्तऋतुः}}
{\sunmoonsrdata{06:38}{18:08}{23:35}{10:47}{12:23}
{\kalas{04:57 05:47 09:42 08:55 10:27 16:36 11:14 13:32 15:50 17:22 18:58 21:15 22:49 01:56*}}}
{\tnykdata{\anga{\tithi{21}{कृष्ण-षष्ठी}}{\time{3-28}{08:01}}\hspace{1ex}}%
{\anga{चित्रा}{\time{10-34}{10:51}}\hspace{1ex}}{चन्द्रराशिः—\mbox{तुला}}%
{\anga{शूलः}{\time{10-51}{10:58}}\hspace{1ex}\uanga{गण्डः}}%
{\anga{वणिजः}{\time{3-28}{08:01}}\hspace{1ex}\anga{विष्टिः}{\time{34-8}{20:17}}\hspace{1ex}\uanga{बवः}}{}
}
{माघ-अष्टका-पूर्वेद्युः\eventsep निक्षुभार्क-सप्तमी\eventsep यशोदा-जयन्ती}
{Tue} 
\cfoot{\rygdata{15:15--16:41}{09:30--10:56}{12:23--13:49}}
\caldata{FEBRUARY}{7}{\sunmonth{मकरः}{25}{}{माघः}{शिशिरऋतुः}{बुधः}{हेमलम्बः}{उत्तरायणम्}{हेमन्तऋतुः}}
{\sunmoonsrdata{06:37}{18:08}{00:26*}{11:29}{12:23}
{\kalas{04:57 05:47 09:41 08:55 10:27 16:36 11:14 13:32 15:50 17:22 18:58 21:15 22:49 01:56*}}}
{\tnykdata{\anga{\tithi{22}{कृष्ण-सप्तमी}}{\time{5-18}{08:45}}\hspace{1ex}}%
{\anga{स्वाती}{\time{14-1}{12:14}}\hspace{1ex}}{चन्द्रराशिः—\mbox{तुला}}%
{\anga{गण्डः}{\time{9-32}{10:26}}\hspace{1ex}\uanga{वृद्धिः}}%
{\anga{बवः}{\time{5-18}{08:45}}\hspace{1ex}\anga{बालवः}{\time{36-55}{21:23}}\hspace{1ex}\uanga{कौलवः}}{}
}
{काञ्ची ६६ जगद्गुरु श्री-चन्द्रशेखरेन्द्र सरस्वती ६ आराधना~\#{१११}\eventsep माघ-अष्टका-श्राद्धम्\eventsep पञ्च-पर्व-पूजा (अष्टमी)}
{Wed} 
\cfoot{\rygdata{12:23--13:49}{08:04--09:30}{10:56--12:23}}
\caldata{FEBRUARY}{8}{\sunmonth{मकरः}{26}{}{माघः}{शिशिरऋतुः}{गुरुः}{हेमलम्बः}{उत्तरायणम्}{हेमन्तऋतुः}}
{\sunmoonsrdata{06:37}{18:08}{01:17*}{12:13}{12:23}
{\kalas{04:57 05:47 09:41 08:55 10:27 16:36 11:14 13:32 15:50 17:22 18:58 21:15 22:49 01:56*}}}
{\tnykdata{\anga{\tithi{23}{कृष्ण-अष्टमी}}{\time{8-58}{10:12}}\hspace{1ex}}%
{\anga{विशाखा}{\time{19-10}{14:17}}\hspace{1ex}}{चन्द्रराशिः—\mbox{तुला\RIGHTarrow{07:43}}}%
{\anga{वृद्धिः}{\time{9-35}{10:27}}\hspace{1ex}\uanga{ध्रुवः}}%
{\anga{कौलवः}{\time{8-58}{10:12}}\hspace{1ex}\anga{तैतिलः}{\time{41-23}{23:10}}\hspace{1ex}\uanga{गरः}}{}
}
{माघ-अन्वष्टका-श्राद्धम्\eventsep \tamil{திருநீலகண்ட நாயனார் (1) குருபூஜை}}
{Thu} 
\cfoot{\rygdata{13:49--15:15}{06:37--08:03}{09:30--10:56}}
\caldata{FEBRUARY}{9}{\sunmonth{मकरः}{27}{}{माघः}{शिशिरऋतुः}{शुक्रः}{हेमलम्बः}{उत्तरायणम्}{हेमन्तऋतुः}}
{\sunmoonsrdata{06:37}{18:09}{02:08*}{12:58}{12:23}
{\kalas{04:57 05:47 09:41 08:55 10:27 16:37 11:14 13:32 15:50 17:23 18:59 21:16 22:49 01:56*}}}
{\tnykdata{\anga{\tithi{24}{कृष्ण-नवमी}}{\time{14-7}{12:16}}\hspace{1ex}}%
{\anga{अनूराधा}{\time{25-41}{16:53}}\hspace{1ex}}{चन्द्रराशिः—\mbox{वृश्चिकः}}%
{\anga{ध्रुवः}{\time{10-45}{10:55}}\hspace{1ex}\uanga{व्याघातः}}%
{\anga{गरः}{\time{14-7}{12:16}}\hspace{1ex}\anga{वणिजः}{\time{47-7}{01:27*}}\hspace{1ex}\uanga{विष्टिः}}{}
}
{प्रोक्लस्-जन्म~\#{१६०६}\eventsep \tamil{தை~வெள்ளிக்கிழமை}}
{Fri} 
\cfoot{\rygdata{10:56--12:23}{15:16--16:42}{08:03--09:30}}
\caldata{FEBRUARY}{10}{\sunmonth{मकरः}{28}{}{माघः}{शिशिरऋतुः}{शनिः}{हेमलम्बः}{उत्तरायणम्}{हेमन्तऋतुः}}
{\sunmoonsrdata{06:36}{18:09}{02:57*}{13:45}{12:23}
{\kalas{04:57 05:47 09:41 08:55 10:27 16:37 11:13 13:32 15:51 17:23 18:59 21:16 22:49 01:56*}}}
{\tnykdata{\anga{\tithi{25}{कृष्ण-दशमी}}{\time{20-18}{14:44}}\hspace{1ex}}%
{\anga{ज्येष्ठा}{\time{33-5}{19:50}}\hspace{1ex}}{चन्द्रराशिः—\mbox{वृश्चिकः\RIGHTarrow{19:50}}}%
{\anga{व्याघातः}{\time{12-42}{11:41}}\hspace{1ex}\uanga{हर्षणः}}%
{\anga{विष्टिः}{\time{20-18}{14:44}}\hspace{1ex}\anga{बवः}{\time{53-36}{04:03*}}\hspace{1ex}\uanga{बालवः}}{}
}
{}
{Sat} 
\cfoot{\rygdata{09:30--10:56}{13:49--15:16}{06:36--08:03}}
\caldata{FEBRUARY}{11}{\sunmonth{मकरः}{29}{}{माघः}{शिशिरऋतुः}{भानुः}{हेमलम्बः}{उत्तरायणम्}{हेमन्तऋतुः}}
{\sunmoonsrdata{06:36}{18:09}{03:46*}{14:34}{12:23}
{\kalas{04:56 05:46 09:41 08:55 10:27 16:37 11:13 13:32 15:51 17:23 18:59 21:16 22:49 01:56*}}}
{\tnykdata{\anga{\tithi{26}{कृष्ण-एकादशी}}{\time{27-0}{17:24}}\hspace{1ex}}%
{\anga{मूला}{\time{40-52}{22:57}}\hspace{1ex}}{चन्द्रराशिः—\mbox{धनुः}}%
{\anga{हर्षणः}{\time{15-4}{12:38}}\hspace{1ex}\uanga{वज्रम्}}%
{\anga{बालवः}{\time{27-0}{17:24}}\hspace{1ex}\uanga{कौलवः}}{}
}
{सर्व-विजया-एकादशी}
{Sun} 
\cfoot{\rygdata{16:43--18:09}{12:23--13:49}{15:16--16:43}}
\caldata{FEBRUARY}{12}{\sunmonth{मकरः}{30}{\mbox{मकरः{\tiny\RIGHTarrow}{02:29*}}}{माघः}{शिशिरऋतुः}{सोमः}{हेमलम्बः}{उत्तरायणम्}{हेमन्तऋतुः}}
{\sunmoonsrdata{06:36}{18:10}{04:33*}{15:24}{12:23}
{\kalas{04:56 05:46 09:41 08:55 10:27 16:37 11:13 13:32 15:51 17:24 18:59 21:16 22:49 01:56*}}}
{\tnykdata{\anga{\tithi{27}{कृष्ण-द्वादशी}}{\time{33-41}{20:04}}\hspace{1ex}}%
{\anga{पूर्वाषाढा}{\time{48-34}{02:02*}}\hspace{1ex}}{चन्द्रराशिः—\mbox{धनुः}}%
{\anga{वज्रम्}{\time{17-31}{13:36}}\hspace{1ex}\uanga{सिद्धिः}}%
{\anga{कौलवः}{\time{0-22}{06:45}}\hspace{1ex}\anga{तैतिलः}{\time{33-41}{20:04}}\hspace{1ex}\uanga{गरः}}{}
}
{सेङ्गालिपुरम्-मुत्तण्णावाळ्-आराधना~\#{१२५}}
{Mon} 
\cfoot{\rygdata{08:02--09:29}{10:56--12:23}{13:50--15:16}}
\caldata{FEBRUARY}{13}{\sunmonth{कुम्भः}{1}{}{माघः}{शिशिरऋतुः}{मङ्गलः}{हेमलम्बः}{उत्तरायणम्}{शिशिरऋतुः}}
{\sunmoonsrdata{06:35}{18:10}{05:18*}{16:14}{12:23}
{\kalas{04:56 05:46 09:41 08:54 10:27 16:37 11:13 13:32 15:51 17:24 19:00 21:16 22:49 01:56*}}}
{\tnykdata{\anga{\tithi{28}{कृष्ण-त्रयोदशी}}{\time{39-57}{22:35}}\hspace{1ex}}%
{\anga{उत्तराषाढा}{\time{55-48}{04:55*}}\hspace{1ex}}{चन्द्रराशिः—\mbox{धनुः\RIGHTarrow{08:46}}}%
{\anga{सिद्धिः}{\time{19-43}{14:29}}\hspace{1ex}\uanga{व्यतीपातः}}%
{\anga{गरः}{\time{6-54}{09:21}}\hspace{1ex}\anga{वणिजः}{\time{39-57}{22:35}}\hspace{1ex}\uanga{विष्टिः}}{}
}
{कुम्भ-रवि-सङ्क्रमण-विष्णुपदी-पुण्यकालः~06:35\RIGHTarrow{}08:53\eventsep \tamil{மாசி~செவ்வாய்}\eventsep प्रदोष-व्रतम्~18:10\RIGHTarrow{}19:43\eventsep सङ्क्रमण-दिन-पूर्वाह्ण-पुण्यकालः~06:35\RIGHTarrow{}12:23}
{Tue} 
\cfoot{\rygdata{15:16--16:43}{09:29--10:56}{12:23--13:50}}
\caldata{FEBRUARY}{14}{\sunmonth{कुम्भः}{2}{}{माघः}{शिशिरऋतुः}{बुधः}{हेमलम्बः}{उत्तरायणम्}{शिशिरऋतुः}}
{\sunmoonsrdata{06:35}{18:10}{06:00*}{17:04}{12:23}
{\kalas{04:56 05:45 09:40 08:54 10:27 16:38 11:13 13:32 15:51 17:24 19:00 21:16 22:50 01:56*}}}
{\tnykdata{\anga{\tithi{29}{कृष्ण-चतुर्दशी}}{\time{45-28}{00:46*}}\hspace{1ex}}%
{\fullanga{श्रवणः}}{चन्द्रराशिः—\mbox{मकरः}}%
{\anga{व्यतीपातः}{\time{21-26}{15:10}}\hspace{1ex}\uanga{वरीयान्}}%
{\anga{विष्टिः}{\time{12-50}{11:43}}\hspace{1ex}\anga{शकुनिः}{\time{45-28}{00:46*}}\hspace{1ex}\uanga{चतुष्पात्}}{}
}
{मासशिवरात्रिः\eventsep महाशिवरात्रिः\eventsep पञ्च-पर्व-पूजा (चतुर्दशी)\eventsep व्यतीपात-श्राद्धम्\eventsep माघ-यम-तर्पणम्\eventsep श्रवण-व्रतम्}
{Wed} 
\cfoot{\rygdata{12:23--13:50}{08:02--09:29}{10:56--12:23}}
\caldata{FEBRUARY}{15}{\sunmonth{कुम्भः}{3}{}{माघः}{शिशिरऋतुः}{गुरुः}{हेमलम्बः}{उत्तरायणम्}{शिशिरऋतुः}}
{\sunmoonsrdata{06:35}{18:11}{---}{17:54}{12:23}
{\kalas{04:55 05:45 09:40 08:54 10:27 16:38 11:13 13:32 15:51 17:24 19:00 21:17 22:50 01:55*}}}
{\tnykdata{\anga{\tithi{30}{अमावास्या}}{\time{50-0}{02:35*}}\hspace{1ex}}%
{\anga{श्रवणः}{\time{2-15}{07:29}}\hspace{1ex}}{चन्द्रराशिः—\mbox{मकरः\RIGHTarrow{20:38}}}%
{\anga{वरीयान्}{\time{22-28}{15:34}}\hspace{1ex}\uanga{परिघः}}%
{\anga{चतुष्पात्}{\time{17-52}{13:44}}\hspace{1ex}\anga{नाग}{\time{50-0}{02:35*}}\hspace{1ex}\uanga{किंस्तुघ्नः}}{}
}
{कलियुगादिः\eventsep माघ-स्नानपूर्तिः\eventsep पार्वणव्रतम् अमावास्यायाम्\eventsep पञ्च-पर्व-पूजा (अमावास्या)\eventsep पुरन्दरदास-आराधना~\#{४५४}\eventsep सर्व-माघ-अमावास्या (अलभ्यम्–श्रविष्ठा, पुष्कला)}
{Thu} 
\cfoot{\rygdata{13:50--15:17}{06:35--08:02}{09:29--10:56}}
\caldata{FEBRUARY}{16}{\sunmonth{कुम्भः}{4}{}{फाल्गुनः}{शिशिरऋतुः}{शुक्रः}{हेमलम्बः}{उत्तरायणम्}{शिशिरऋतुः}}
{\sunmoonrsdata{06:34}{18:11}{06:41}{18:43}{12:23}
{\kalas{04:55 05:45 09:40 08:54 10:26 16:38 11:13 13:32 15:52 17:25 19:00 21:17 22:50 01:55*}}}
{\tnykdata{\anga{\tithi{1}{शुक्ल-प्रथमा}}{\time{53-26}{03:57*}}\hspace{1ex}}%
{\anga{श्रविष्ठा}{\time{7-44}{09:40}}\hspace{1ex}}{चन्द्रराशिः—\mbox{कुम्भः}}%
{\anga{परिघः}{\time{22-43}{15:40}}\hspace{1ex}\uanga{शिवः}}%
{\anga{किंस्तुघ्नः}{\time{21-52}{15:19}}\hspace{1ex}\anga{बवः}{\time{53-26}{03:57*}}\hspace{1ex}\uanga{बालवः}}{}
}
{पयोव्रत-आरम्भः\eventsep दर्शेष्टिः\eventsep काञ्ची ६७ जगद्गुरु श्री-महादेवेन्द्र सरस्वती ५ आराधना~\#{१११}\eventsep पार्वण-प्रायश्चित्तावकाशः पौर्णमास्याम्\eventsep स्थालीपाकः}
{Fri} 
\cfoot{\rygdata{10:56--12:23}{15:17--16:44}{08:01--09:28}}
\caldata{FEBRUARY}{17}{\sunmonth{कुम्भः}{5}{}{फाल्गुनः}{शिशिरऋतुः}{शनिः}{हेमलम्बः}{उत्तरायणम्}{शिशिरऋतुः}}
{\sunmoonrsdata{06:34}{18:11}{07:21}{19:31}{12:23}
{\kalas{04:55 05:44 09:40 08:53 10:26 16:38 11:13 13:32 15:52 17:25 19:01 21:17 22:50 01:55*}}}
{\tnykdata{\anga{\tithi{2}{शुक्ल-द्वितीया}}{\time{55-42}{04:51*}}\hspace{1ex}}%
{\anga{शतभिषक्}{\time{12-9}{11:26}}\hspace{1ex}}{चन्द्रराशिः—\mbox{कुम्भः\RIGHTarrow{06:27*}}}%
{\anga{शिवः}{\time{22-6}{15:24}}\hspace{1ex}\uanga{सिद्धः}}%
{\anga{बालवः}{\time{24-43}{16:27}}\hspace{1ex}\anga{कौलवः}{\time{55-42}{04:51*}}\hspace{1ex}\uanga{तैतिलः}}{}
}
{चन्द्र-दर्शनम्~18:11\RIGHTarrow{}19:31\eventsep \tamil{கொச்செங்கட் சோழ நாயனார் (59) குருபூஜை}\eventsep फूलेरा-दूज्\eventsep रामकृष्ण-परमहंस-जयन्ती~\#{१८३}}
{Sat} 
\cfoot{\rygdata{09:28--10:55}{13:50--15:17}{06:34--08:01}}
\caldata{FEBRUARY}{18}{\sunmonth{कुम्भः}{6}{}{फाल्गुनः}{शिशिरऋतुः}{भानुः}{हेमलम्बः}{उत्तरायणम्}{शिशिरऋतुः}}
{\sunmoonrsdata{06:33}{18:11}{08:00}{20:20}{12:22}
{\kalas{04:55 05:44 09:40 08:53 10:26 16:38 11:13 13:32 15:52 17:25 19:01 21:17 22:50 01:55*}}}
{\tnykdata{\anga{\tithi{3}{शुक्ल-तृतीया}}{\time{56-48}{05:17*}}\hspace{1ex}}%
{\anga{पूर्वप्रोष्ठपदा}{\time{15-27}{12:44}}\hspace{1ex}}{चन्द्रराशिः—\mbox{मीनः}}%
{\anga{सिद्धः}{\time{20-36}{14:48}}\hspace{1ex}\uanga{साध्यः}}%
{\anga{तैतिलः}{\time{26-24}{17:07}}\hspace{1ex}\anga{गरः}{\time{56-48}{05:17*}}\hspace{1ex}\uanga{वणिजः}}{}
}
{रवि-सङ्क्रमण-पुण्यकालः~16:23\RIGHTarrow{}18:11\eventsep सायन-सङ्क्रमण-दिन-अपराह्ण-पुण्यकालः~12:22\RIGHTarrow{}18:11\eventsep तपस्य-मासः~22:47\RIGHTarrow{}\eventsep \tamil{திருச்செந்தூர் முருகன் மாசித் திருவிழா தொடக்கம்}}
{Sun} 
\cfoot{\rygdata{16:44--18:11}{12:23--13:50}{15:17--16:44}}
\caldata{FEBRUARY}{19}{\sunmonth{कुम्भः}{7}{}{फाल्गुनः}{शिशिरऋतुः}{सोमः}{हेमलम्बः}{उत्तरायणम्}{शिशिरऋतुः}}
{\sunmoonrsdata{06:33}{18:12}{08:39}{21:11}{12:22}
{\kalas{04:54 05:44 09:39 08:53 10:26 16:39 11:13 13:32 15:52 17:25 19:01 21:17 22:50 01:55*}}}
{\tnykdata{\anga{\tithi{4}{शुक्ल-चतुर्थी}}{\time{56-44}{05:15*}}\hspace{1ex}}%
{\anga{उत्तरप्रोष्ठपदा}{\time{17-37}{13:36}}\hspace{1ex}}{चन्द्रराशिः—\mbox{मीनः}}%
{\anga{साध्यः}{\time{18-12}{13:50}}\hspace{1ex}\uanga{शुभः}}%
{\anga{वणिजः}{\time{26-55}{17:19}}\hspace{1ex}\anga{विष्टिः}{\time{56-44}{05:15*}}\hspace{1ex}\uanga{बवः}}{}
}
{काञ्ची ६८ जगद्गुरु श्री-चन्द्रशेखरेन्द्र सरस्वती ७ आश्रम-स्वीकार-जयन्ती~\#{११०}\eventsep \tamil{திருச்செந்தூர் முருகன் மாசித் திருவிழா 2ம் நாள்}\eventsep शुक्ल-चतुर्थी-व्रतम्}
{Mon} 
\cfoot{\rygdata{08:00--09:28}{10:55--12:22}{13:50--15:17}}
\caldata{FEBRUARY}{20}{\sunmonth{कुम्भः}{8}{}{फाल्गुनः}{शिशिरऋतुः}{मङ्गलः}{हेमलम्बः}{उत्तरायणम्}{शिशिरऋतुः}}
{\sunmoonrsdata{06:33}{18:12}{09:20}{22:03}{12:22}
{\kalas{04:54 05:43 09:39 08:52 10:26 16:39 11:12 13:32 15:52 17:25 19:01 21:17 22:50 01:55*}}}
{\tnykdata{\anga{\tithi{5}{शुक्ल-पञ्चमी}}{\time{55-32}{04:46*}}\hspace{1ex}}%
{\anga{रेवती}{\time{18-40}{14:01}}\hspace{1ex}}{चन्द्रराशिः—\mbox{मीनः\RIGHTarrow{14:01}}}%
{\anga{शुभः}{\time{14-56}{12:31}}\hspace{1ex}\uanga{शुक्लः}}%
{\anga{बवः}{\time{26-17}{17:04}}\hspace{1ex}\anga{बालवः}{\time{55-32}{04:46*}}\hspace{1ex}\uanga{कौलवः}}{}
}
{भौमाश्विनी-पुण्यकालः~14:01\RIGHTarrow{}\eventsep \tamil{மாசி~செவ்வாய்}\eventsep \tamil{திருச்செந்தூர் முருகன் மாசித் திருவிழா 3ம் நாள்—முருகன் பவனி}}
{Tue} 
\cfoot{\rygdata{15:17--16:45}{09:27--10:55}{12:22--13:50}}
\caldata{FEBRUARY}{21}{\sunmonth{कुम्भः}{9}{}{फाल्गुनः}{शिशिरऋतुः}{बुधः}{हेमलम्बः}{उत्तरायणम्}{शिशिरऋतुः}}
{\sunmoonrsdata{06:32}{18:12}{10:03}{22:57}{12:22}
{\kalas{04:53 05:43 09:39 08:52 10:25 16:39 11:12 13:32 15:52 17:26 19:02 21:17 22:50 01:54*}}}
{\tnykdata{\anga{\tithi{6}{शुक्ल-षष्ठी}}{\time{53-14}{03:50*}}\hspace{1ex}}%
{\anga{अश्विनी}{\time{18-38}{13:59}}\hspace{1ex}}{चन्द्रराशिः—\mbox{मेषः}}%
{\anga{शुक्लः}{\time{10-46}{10:51}}\hspace{1ex}\uanga{ब्रह्म}}%
{\anga{कौलवः}{\time{24-32}{16:21}}\hspace{1ex}\anga{तैतिलः}{\time{53-14}{03:50*}}\hspace{1ex}\uanga{गरः}}{}
}
{षष्ठी-व्रतम्\eventsep \tamil{திருச்செந்தூர் முருகன் மாசித் திருவிழா 4ம் நாள்}}
{Wed} 
\cfoot{\rygdata{12:22--13:50}{08:00--09:27}{10:55--12:22}}
\caldata{FEBRUARY}{22}{\sunmonth{कुम्भः}{10}{}{फाल्गुनः}{शिशिरऋतुः}{गुरुः}{हेमलम्बः}{उत्तरायणम्}{शिशिरऋतुः}}
{\sunmoonrsdata{06:32}{18:12}{10:50}{23:54}{12:22}
{\kalas{04:53 05:42 09:39 08:52 10:25 16:39 11:12 13:32 15:52 17:26 19:02 21:17 22:50 01:54*}}}
{\tnykdata{\anga{\tithi{7}{शुक्ल-सप्तमी}}{\time{49-52}{02:29*}}\hspace{1ex}}%
{\anga{अपभरणी}{\time{17-31}{13:32}}\hspace{1ex}}{चन्द्रराशिः—\mbox{मेषः\RIGHTarrow{19:22}}}%
{\anga{ब्रह्म}{\time{5-45}{08:50}}\hspace{1ex}\anga{इन्द्रः}{\time{59-53}{06:29*}}\hspace{1ex}\uanga{वैधृतिः}}%
{\anga{गरः}{\time{21-41}{15:12}}\hspace{1ex}\anga{वणिजः}{\time{49-52}{02:29*}}\hspace{1ex}\uanga{विष्टिः}}{}
}
{कृत्तिका-व्रतम्\eventsep नन्दा-सप्तमी\eventsep \tamil{திருச்செந்தூர் முருகன் மாசித் திருவிழா 5ம் நாள்}\eventsep श्री-राघवेन्द्र-स्वामि-जयन्ती~\#{४२४}}
{Thu} 
\cfoot{\rygdata{13:50--15:17}{06:32--07:59}{09:27--10:54}}
\caldata{FEBRUARY}{23}{\sunmonth{कुम्भः}{11}{}{फाल्गुनः}{शिशिरऋतुः}{शुक्रः}{हेमलम्बः}{उत्तरायणम्}{शिशिरऋतुः}}
{\sunmoonrsdata{06:31}{18:13}{11:42}{00:54*}{12:22}
{\kalas{04:53 05:42 09:38 08:51 10:25 16:39 11:12 13:32 15:52 17:26 19:02 21:17 22:49 01:54*}}}
{\tnykdata{\anga{\tithi{8}{शुक्ल-अष्टमी}}{\time{45-29}{00:43*}}\hspace{1ex}}%
{\anga{कृत्तिका}{\time{15-24}{12:41}}\hspace{1ex}}{चन्द्रराशिः—\mbox{वृषभः}}%
{\anga{वैधृतिः}{\time{53-15}{03:49*}}\hspace{1ex}\uanga{विष्कम्भः}}%
{\anga{विष्टिः}{\time{17-48}{13:39}}\hspace{1ex}\anga{बवः}{\time{45-29}{00:43*}}\hspace{1ex}\uanga{बालवः}}{}
}
{\tamil{திருச்செந்தூர் முருகன் மாசித் திருவிழா 6ம் நாள்—வெள்ளித் தேர் பவனி}\eventsep वैधृति-श्राद्धम्}
{Fri} 
\cfoot{\rygdata{10:54--12:22}{15:17--16:45}{07:59--09:27}}
\caldata{FEBRUARY}{24}{\sunmonth{कुम्भः}{12}{}{फाल्गुनः}{शिशिरऋतुः}{शनिः}{हेमलम्बः}{उत्तरायणम्}{शिशिरऋतुः}}
{\sunmoonrsdata{06:31}{18:13}{12:38}{01:56*}{12:22}
{\kalas{04:52 05:42 09:38 08:51 10:25 16:39 11:12 13:32 15:52 17:26 19:02 21:17 22:49 01:54*}}}
{\tnykdata{\anga{\tithi{9}{शुक्ल-नवमी}}{\time{40-12}{22:36}}\hspace{1ex}}%
{\anga{रोहिणी}{\time{12-19}{11:27}}\hspace{1ex}}{चन्द्रराशिः—\mbox{वृषभः\RIGHTarrow{22:42}}}%
{\anga{विष्कम्भः}{\time{45-54}{00:52*}}\hspace{1ex}\uanga{प्रीतिः}}%
{\anga{बालवः}{\time{12-57}{11:42}}\hspace{1ex}\anga{कौलवः}{\time{40-12}{22:36}}\hspace{1ex}\uanga{तैतिलः}}{}
}
{\tamil{திருச்செந்தூர் முருகன் மாசித் திருவிழா 7ம் நாள்—உருகு சத்தச் சேவை/சிகப்பு சாத்தி அலங்காரம்}\eventsep शनिरोहिणी-पुण्यकालः\RIGHTarrow{}11:26}
{Sat} 
\cfoot{\rygdata{09:26--10:54}{13:50--15:17}{06:31--07:58}}
\caldata{FEBRUARY}{25}{\sunmonth{कुम्भः}{13}{}{फाल्गुनः}{शिशिरऋतुः}{भानुः}{हेमलम्बः}{उत्तरायणम्}{शिशिरऋतुः}}
{\sunmoonrsdata{06:30}{18:13}{13:38}{02:56*}{12:22}
{\kalas{04:52 05:41 09:38 08:51 10:24 16:39 11:11 13:32 15:52 17:26 19:02 21:17 22:49 01:53*}}}
{\tnykdata{\anga{\tithi{10}{शुक्ल-दशमी}}{\time{34-8}{20:10}}\hspace{1ex}}%
{\anga{मृगशीर्षम्}{\time{8-24}{09:52}}\hspace{1ex}}{चन्द्रराशिः—\mbox{मिथुनम्}}%
{\anga{प्रीतिः}{\time{37-55}{21:41}}\hspace{1ex}\uanga{आयुष्मान्}}%
{\anga{तैतिलः}{\time{7-16}{09:25}}\hspace{1ex}\anga{गरः}{\time{34-8}{20:10}}\hspace{1ex}\uanga{वणिजः}}{}
}
{\tamil{திருச்செந்தூர் முருகன் மாசித் திருவிழா 8ம் நாள்—பச்சை சாத்தி அலங்காரம்}\eventsep वेङ्कटाचले प्लवोत्सव-प्रारम्भः}
{Sun} 
\cfoot{\rygdata{16:45--18:13}{12:22--13:50}{15:17--16:45}}
\caldata{FEBRUARY}{26}{\sunmonth{कुम्भः}{14}{}{फाल्गुनः}{शिशिरऋतुः}{सोमः}{हेमलम्बः}{उत्तरायणम्}{शिशिरऋतुः}}
{\sunmoonrsdata{06:30}{18:13}{14:41}{03:55*}{12:22}
{\kalas{04:51 05:41 09:37 08:50 10:24 16:39 11:11 13:32 15:52 17:26 19:02 21:17 22:49 01:53*}}}
{\tnykdata{\anga{\tithi{11}{शुक्ल-एकादशी}}{\time{27-27}{17:29}}\hspace{1ex}}%
{\anga{आर्द्रा}{\time{3-47}{08:01}}\hspace{1ex}\anga{पुनर्वसुः}{\time{58-40}{05:58*}}\hspace{1ex}}{चन्द्रराशिः—\mbox{मिथुनम्\RIGHTarrow{00:30*}}}%
{\anga{आयुष्मान्}{\time{29-29}{18:17}}\hspace{1ex}\uanga{सौभाग्यः}}%
{\anga{वणिजः}{\time{0-52}{06:51}}\hspace{1ex}\anga{विष्टिः}{\time{27-27}{17:29}}\hspace{1ex}\anga{बवः}{\time{53-57}{04:05*}}\hspace{1ex}\uanga{बालवः}}{}
}
{\tamil{குலஶேகர ஆழ்வார் திருநக்ஷத்திரம்}\eventsep रंगभरी एकादशी\eventsep सर्व-आमलकी-एकादशी\eventsep \tamil{திருச்செந்தூர் முருகன் மாசித் திருவிழா 9ம் நாள்—தங்க கைலாச வாஹனம்}\eventsep वेङ्कटाचले प्लवोत्सवः}
{Mon} 
\cfoot{\rygdata{07:58--09:26}{10:54--12:22}{13:49--15:17}}
\caldata{FEBRUARY}{27}{\sunmonth{कुम्भः}{15}{}{फाल्गुनः}{शिशिरऋतुः}{मङ्गलः}{हेमलम्बः}{उत्तरायणम्}{शिशिरऋतुः}}
{\sunmoonrsdata{06:29}{18:13}{15:44}{04:50*}{12:21}
{\kalas{04:51 05:40 09:37 08:50 10:24 16:40 11:11 13:32 15:53 17:27 19:02 21:17 22:49 01:53*}}}
{\tnykdata{\anga{\tithi{12}{शुक्ल-द्वादशी}}{\time{20-24}{14:39}}\hspace{1ex}}%
{\anga{पुष्यः}{\time{53-21}{03:50*}}\hspace{1ex}}{चन्द्रराशिः—\mbox{कर्कटः}}%
{\anga{सौभाग्यः}{\time{20-44}{14:47}}\hspace{1ex}\uanga{शोभनः}}%
{\anga{बालवः}{\time{20-24}{14:39}}\hspace{1ex}\anga{कौलवः}{\time{46-48}{01:12*}}\hspace{1ex}\uanga{तैतिलः}}{}
}
{पयोव्रत-समापनम्\eventsep गोविन्द-महाद्वादशी\eventsep जया-महाद्वादशी\eventsep \tamil{மாசி~செவ்வாய்}\eventsep नरसिंह-द्वादशी\eventsep प्रदोष-व्रतम्~18:13\RIGHTarrow{}19:45\eventsep \tamil{திருச்செந்தூர் முருகன் மாசித் திருவிழா 10ம் நாள்—தேர்}\eventsep वेङ्कटाचले प्लवोत्सवः}
{Tue} 
\cfoot{\rygdata{15:17--16:45}{09:25--10:53}{12:21--13:49}}
\caldata{FEBRUARY}{28}{\sunmonth{कुम्भः}{16}{}{फाल्गुनः}{शिशिरऋतुः}{बुधः}{हेमलम्बः}{उत्तरायणम्}{शिशिरऋतुः}}
{\sunmoonrsdata{06:29}{18:14}{16:45}{05:40*}{12:21}
{\kalas{04:51 05:40 09:37 08:50 10:24 16:40 11:11 13:32 15:53 17:27 19:03 21:17 22:49 01:53*}}}
{\tnykdata{\anga{\tithi{13}{शुक्ल-त्रयोदशी}}{\time{13-13}{11:46}}\hspace{1ex}}%
{\anga{आश्रेषा}{\time{48-5}{01:43*}}\hspace{1ex}}{चन्द्रराशिः—\mbox{कर्कटः\RIGHTarrow{01:43*}}}%
{\anga{शोभनः}{\time{11-53}{11:14}}\hspace{1ex}\uanga{अतिगण्डः}}%
{\anga{तैतिलः}{\time{13-13}{11:46}}\hspace{1ex}\anga{गरः}{\time{39-39}{22:21}}\hspace{1ex}\uanga{वणिजः}}{}
}
{\tamil{நடராஜர் மஹாபிஷேகம்}\eventsep \tamil{திருச்செந்தூர் முருகன் தெப்பம்}\eventsep वेङ्कटाचले प्लवोत्सवः}
{Wed} 
\cfoot{\rygdata{12:21--13:49}{07:57--09:25}{10:53--12:21}}
\caldata{MARCH}{1}{\sunmonth{कुम्भः}{17}{}{फाल्गुनः}{शिशिरऋतुः}{गुरुः}{हेमलम्बः}{उत्तरायणम्}{शिशिरऋतुः}}
{\sunmoonrsdata{06:28}{18:14}{17:45}{---}{12:21}
{\kalas{04:50 05:39 09:36 08:49 10:23 16:40 11:10 13:32 15:53 17:27 19:03 21:17 22:49 01:52*}}}
{\tnykdata{\anga{\tithi{14}{शुक्ल-चतुर्दशी}}{\time{6-13}{08:57}}\hspace{1ex}\anga{\tithi{15}{पौर्णमासी}}{\time{59-42}{06:21*}}\hspace{1ex}\avamA{}}%
{\anga{मघा}{\time{43-13}{23:46}}\hspace{1ex}}{चन्द्रराशिः—\mbox{सिंहः}}%
{\anga{अतिगण्डः}{\time{3-11}{07:44}}\hspace{1ex}\anga{सुकर्म}{\time{54-51}{04:25*}}\hspace{1ex}\uanga{धृतिः}}%
{\anga{वणिजः}{\time{6-13}{08:57}}\hspace{1ex}\anga{विष्टिः}{\time{32-52}{19:37}}\hspace{1ex}\anga{बवः}{\time{59-42}{06:21*}}\hspace{1ex}\uanga{बालवः}}{}
}
{चैतन्य-महाप्रभु-जयन्ती~\#{५३३}\eventsep होलिका-पूर्णिमा\eventsep काम-दहनम्\eventsep कुम्भमाघोत्सवः\eventsep मन्वादिः-(रुद्रः-[१२])\eventsep पार्वणव्रतम् पूर्णिमायाम्\eventsep पूर्णिमा-व्रतम्\eventsep पञ्च-पर्व-पूजा (पूर्णिमा)\eventsep \tamil{திருச்செந்தூர் மாசித் திருவிழா நிறைவு}\eventsep वेङ्कटाचले पूर्णिमा-गरुड-सेवा\eventsep वेङ्कटाचले प्लवोत्सव-समापनम्}
{Thu} 
\cfoot{\rygdata{13:49--15:17}{06:28--07:56}{09:25--10:53}}
\caldata{MARCH}{2}{\sunmonth{कुम्भः}{18}{}{फाल्गुनः}{शिशिरऋतुः}{शुक्रः}{हेमलम्बः}{उत्तरायणम्}{शिशिरऋतुः}}
{\sunmoonsrdata{06:27}{18:14}{18:42}{06:28}{12:21}
{\kalas{04:50 05:39 09:36 08:49 10:23 16:40 11:10 13:31 15:53 17:27 19:03 21:17 22:49 01:52*}}}
{\tnykdata{\anga{\tithi{16}{कृष्ण-प्रथमा}}{\time{54-3}{04:05*}}\hspace{1ex}}%
{\anga{पूर्वफल्गुनी}{\time{39-6}{22:06}}\hspace{1ex}}{चन्द्रराशिः—\mbox{सिंहः\RIGHTarrow{03:45*}}}%
{\anga{धृतिः}{\time{47-13}{01:21*}}\hspace{1ex}\uanga{शूलः}}%
{\anga{बालवः}{\time{26-45}{17:10}}\hspace{1ex}\anga{कौलवः}{\time{54-3}{04:05*}}\hspace{1ex}\uanga{तैतिलः}}{}
}
{आम्र-कुसुम-प्राशनम्\eventsep होलि\eventsep पार्वण-प्रायश्चित्तावकाशः दर्शे\eventsep पूर्णमासेष्टिः\eventsep स्थालीपाकः}
{Fri} 
\cfoot{\rygdata{10:52--12:21}{15:17--16:46}{07:56--09:24}}
\caldata{MARCH}{3}{\sunmonth{कुम्भः}{19}{}{फाल्गुनः}{शिशिरऋतुः}{शनिः}{हेमलम्बः}{उत्तरायणम्}{शिशिरऋतुः}}
{\sunmoonsrdata{06:27}{18:14}{19:37}{07:13}{12:21}
{\kalas{04:49 05:38 09:36 08:48 10:23 16:40 11:10 13:31 15:53 17:27 19:03 21:17 22:49 01:52*}}}
{\tnykdata{\anga{\tithi{17}{कृष्ण-द्वितीया}}{\time{49-37}{02:18*}}\hspace{1ex}}%
{\anga{उत्तरफल्गुनी}{\time{36-5}{20:53}}\hspace{1ex}}{चन्द्रराशिः—\mbox{कन्या}}%
{\anga{शूलः}{\time{40-32}{22:40}}\hspace{1ex}\uanga{गण्डः}}%
{\anga{तैतिलः}{\time{21-41}{15:07}}\hspace{1ex}\anga{गरः}{\time{49-37}{02:18*}}\hspace{1ex}\uanga{वणिजः}}{}
}
{}
{Sat} 
\cfoot{\rygdata{09:24--10:52}{13:49--15:17}{06:27--07:55}}
\caldata{MARCH}{4}{\sunmonth{कुम्भः}{20}{}{फाल्गुनः}{शिशिरऋतुः}{भानुः}{हेमलम्बः}{उत्तरायणम्}{शिशिरऋतुः}}
{\sunmoonsrdata{06:26}{18:14}{20:30}{07:56}{12:20}
{\kalas{04:49 05:38 09:35 08:48 10:22 16:40 11:10 13:31 15:53 17:27 19:03 21:17 22:49 01:51*}}}
{\tnykdata{\anga{\tithi{18}{कृष्ण-तृतीया}}{\time{46-42}{01:07*}}\hspace{1ex}}%
{\anga{हस्तः}{\time{34-30}{20:14}}\hspace{1ex}}{चन्द्रराशिः—\mbox{कन्या}}%
{\anga{गण्डः}{\time{35-1}{20:27}}\hspace{1ex}\uanga{वृद्धिः}}%
{\anga{वणिजः}{\time{17-58}{13:38}}\hspace{1ex}\anga{विष्टिः}{\time{46-42}{01:07*}}\hspace{1ex}\uanga{बवः}}{}
}
{आदित्यहस्त-पुण्यकालः\eventsep ब्रह्म-कल्पादिः\eventsep छत्रपति-शिवाजी-जयन्ती~\#{३८९}\eventsep \tamil{எறிபத்த நாயனார் (7) குருபூஜை}}
{Sun} 
\cfoot{\rygdata{16:46--18:14}{12:20--13:49}{15:17--16:46}}
\caldata{MARCH}{5}{\sunmonth{कुम्भः}{21}{}{फाल्गुनः}{शिशिरऋतुः}{सोमः}{हेमलम्बः}{उत्तरायणम्}{शिशिरऋतुः}}
{\sunmoonsrdata{06:26}{18:14}{21:23}{08:39}{12:20}
{\kalas{04:48 05:37 09:35 08:48 10:22 16:40 11:09 13:31 15:53 17:27 19:03 21:17 22:48 01:51*}}}
{\tnykdata{\anga{\tithi{19}{कृष्ण-चतुर्थी}}{\time{45-33}{00:39*}}\hspace{1ex}}%
{\anga{चित्रा}{\time{34-35}{20:16}}\hspace{1ex}}{चन्द्रराशिः—\mbox{कन्या\RIGHTarrow{08:10}}}%
{\anga{वृद्धिः}{\time{30-52}{18:47}}\hspace{1ex}\uanga{ध्रुवः}}%
{\anga{बवः}{\time{15-54}{12:48}}\hspace{1ex}\anga{बालवः}{\time{45-33}{00:39*}}\hspace{1ex}\uanga{कौलवः}}{}
}
{भालचन्द्र-महागणपति सङ्कटहर-चतुर्थी-व्रतम्}
{Mon} 
\cfoot{\rygdata{07:54--09:23}{10:52--12:20}{13:49--15:17}}
\caldata{MARCH}{6}{\sunmonth{कुम्भः}{22}{}{फाल्गुनः}{शिशिरऋतुः}{मङ्गलः}{हेमलम्बः}{उत्तरायणम्}{शिशिरऋतुः}}
{\sunmoonsrdata{06:25}{18:14}{22:16}{09:22}{12:20}
{\kalas{04:48 05:36 09:34 08:47 10:22 16:40 11:09 13:31 15:53 17:27 19:03 21:17 22:48 01:51*}}}
{\tnykdata{\anga{\tithi{20}{कृष्ण-पञ्चमी}}{\time{46-19}{00:57*}}\hspace{1ex}}%
{\anga{स्वाती}{\time{36-32}{21:02}}\hspace{1ex}}{चन्द्रराशिः—\mbox{तुला}}%
{\anga{ध्रुवः}{\time{28-13}{17:42}}\hspace{1ex}\uanga{व्याघातः}}%
{\anga{कौलवः}{\time{15-42}{12:42}}\hspace{1ex}\anga{तैतिलः}{\time{46-19}{00:57*}}\hspace{1ex}\uanga{गरः}}{}
}
{\tamil{மாசி~செவ்வாய்}\eventsep रङ्ग-पञ्चमी}
{Tue} 
\cfoot{\rygdata{15:17--16:46}{09:23--10:51}{12:20--13:49}}
\caldata{MARCH}{7}{\sunmonth{कुम्भः}{23}{}{फाल्गुनः}{शिशिरऋतुः}{बुधः}{हेमलम्बः}{उत्तरायणम्}{शिशिरऋतुः}}
{\sunmoonsrdata{06:25}{18:15}{23:08}{10:06}{12:20}
{\kalas{04:47 05:36 09:34 08:47 10:21 16:40 11:09 13:31 15:53 17:27 19:03 21:17 22:48 01:51*}}}
{\tnykdata{\anga{\tithi{21}{कृष्ण-षष्ठी}}{\time{48-59}{02:00*}}\hspace{1ex}}%
{\anga{विशाखा}{\time{40-19}{22:33}}\hspace{1ex}}{चन्द्रराशिः—\mbox{तुला\RIGHTarrow{16:06}}}%
{\anga{व्याघातः}{\time{27-2}{17:13}}\hspace{1ex}\uanga{हर्षणः}}%
{\anga{गरः}{\time{17-26}{13:23}}\hspace{1ex}\anga{वणिजः}{\time{48-59}{02:00*}}\hspace{1ex}\uanga{विष्टिः}}{}
}
{}
{Wed} 
\cfoot{\rygdata{12:20--13:48}{07:53--09:22}{10:51--12:20}}
\caldata{MARCH}{8}{\sunmonth{कुम्भः}{24}{}{फाल्गुनः}{शिशिरऋतुः}{गुरुः}{हेमलम्बः}{उत्तरायणम्}{शिशिरऋतुः}}
{\sunmoonsrdata{06:24}{18:15}{00:00*}{10:52}{12:19}
{\kalas{04:47 05:35 09:33 08:46 10:21 16:40 11:08 13:31 15:53 17:27 19:03 21:17 22:48 01:50*}}}
{\tnykdata{\anga{\tithi{22}{कृष्ण-सप्तमी}}{\time{53-21}{03:45*}}\hspace{1ex}}%
{\anga{अनूराधा}{\time{45-47}{00:43*}}\hspace{1ex}}{चन्द्रराशिः—\mbox{वृश्चिकः}}%
{\anga{हर्षणः}{\time{27-13}{17:17}}\hspace{1ex}\uanga{वज्रम्}}%
{\anga{विष्टिः}{\time{20-59}{14:48}}\hspace{1ex}\anga{बवः}{\time{53-21}{03:45*}}\hspace{1ex}\uanga{बालवः}}{}
}
{फाल्गुन-अष्टका-पूर्वेद्युः}
{Thu} 
\cfoot{\rygdata{13:48--15:17}{06:24--07:53}{09:22--10:51}}
\caldata{MARCH}{9}{\sunmonth{कुम्भः}{25}{}{फाल्गुनः}{शिशिरऋतुः}{शुक्रः}{हेमलम्बः}{उत्तरायणम्}{शिशिरऋतुः}}
{\sunmoonsrdata{06:23}{18:15}{00:50*}{11:39}{12:19}
{\kalas{04:46 05:35 09:33 08:46 10:21 16:40 11:08 13:30 15:53 17:27 19:03 21:17 22:48 01:50*}}}
{\tnykdata{\anga{\tithi{23}{कृष्ण-अष्टमी}}{\time{59-2}{06:01*}}\hspace{1ex}}%
{\anga{ज्येष्ठा}{\time{52-34}{03:25*}}\hspace{1ex}}{चन्द्रराशिः—\mbox{वृश्चिकः\RIGHTarrow{03:25*}}}%
{\anga{वज्रम्}{\time{28-33}{17:49}}\hspace{1ex}\uanga{सिद्धिः}}%
{\anga{बालवः}{\time{26-4}{16:49}}\hspace{1ex}\anga{कौलवः}{\time{59-2}{06:01*}}\hspace{1ex}\uanga{तैतिलः}}{}
}
{पञ्च-पर्व-पूजा (अष्टमी)\eventsep फाल्गुन-अष्टका-श्राद्धम्}
{Fri} 
\cfoot{\rygdata{10:50--12:19}{15:17--16:46}{07:52--09:21}}
\caldata{MARCH}{10}{\sunmonth{कुम्भः}{26}{}{फाल्गुनः}{शिशिरऋतुः}{शनिः}{हेमलम्बः}{उत्तरायणम्}{शिशिरऋतुः}}
{\sunmoonsrdata{06:23}{18:15}{01:40*}{12:27}{12:19}
{\kalas{04:46 05:34 09:33 08:45 10:20 16:40 11:08 13:30 15:52 17:28 19:03 21:17 22:48 01:50*}}}
{\tnykdata{\fulltithi{\tithi{24}{कृष्ण-नवमी}}}%
{\fullanga{मूला}}{चन्द्रराशिः—\mbox{धनुः}}%
{\anga{सिद्धिः}{\time{30-39}{18:39}}\hspace{1ex}\uanga{व्यतीपातः}}%
{\anga{तैतिलः}{\time{32-13}{19:16}}\hspace{1ex}\uanga{गरः}}{}
}
{फाल्गुन-अन्वष्टका-श्राद्धम्}
{Sat} 
\cfoot{\rygdata{09:21--10:50}{13:48--15:17}{06:23--07:52}}
\caldata{MARCH}{11}{\sunmonth{कुम्भः}{27}{}{फाल्गुनः}{शिशिरऋतुः}{भानुः}{हेमलम्बः}{उत्तरायणम्}{शिशिरऋतुः}}
{\sunmoonsrdata{06:22}{18:15}{02:27*}{13:17}{12:19}
{\kalas{04:45 05:34 09:32 08:45 10:20 16:40 11:07 13:30 15:52 17:28 19:04 21:17 22:48 01:49*}}}
{\tnykdata{\anga{\tithi{24}{कृष्ण-नवमी}}{\time{5-32}{08:35}}\hspace{1ex}}%
{\anga{मूला}{\time{0-10}{06:26}}\hspace{1ex}}{चन्द्रराशिः—\mbox{धनुः}}%
{\anga{व्यतीपातः}{\time{33-8}{19:38}}\hspace{1ex}\uanga{वरीयान्}}%
{\anga{गरः}{\time{5-32}{08:35}}\hspace{1ex}\anga{वणिजः}{\time{38-51}{21:55}}\hspace{1ex}\uanga{विष्टिः}}{}
}
{व्यतीपात-श्राद्धम्}
{Sun} 
\cfoot{\rygdata{16:46--18:15}{12:19--13:48}{15:17--16:46}}
\caldata{MARCH}{12}{\sunmonth{कुम्भः}{28}{}{फाल्गुनः}{शिशिरऋतुः}{सोमः}{हेमलम्बः}{उत्तरायणम्}{शिशिरऋतुः}}
{\sunmoonsrdata{06:22}{18:15}{03:13*}{14:07}{12:18}
{\kalas{04:45 05:33 09:32 08:44 10:19 16:40 11:07 13:30 15:52 17:28 19:04 21:17 22:47 01:49*}}}
{\tnykdata{\anga{\tithi{25}{कृष्ण-दशमी}}{\time{12-9}{11:13}}\hspace{1ex}}%
{\anga{पूर्वाषाढा}{\time{7-57}{09:32}}\hspace{1ex}}{चन्द्रराशिः—\mbox{धनुः\RIGHTarrow{16:18}}}%
{\anga{वरीयान्}{\time{35-33}{20:35}}\hspace{1ex}\uanga{परिघः}}%
{\anga{विष्टिः}{\time{12-9}{11:13}}\hspace{1ex}\anga{बवः}{\time{45-19}{00:29*}}\hspace{1ex}\uanga{बालवः}}{}
}
{देवी-पर्व-१२\eventsep \tamil{காரி நாயனார் (47) குருபூஜை}}
{Mon} 
\cfoot{\rygdata{07:51--09:20}{10:49--12:18}{13:48--15:17}}
\caldata{MARCH}{13}{\sunmonth{कुम्भः}{29}{}{फाल्गुनः}{शिशिरऋतुः}{मङ्गलः}{हेमलम्बः}{उत्तरायणम्}{शिशिरऋतुः}}
{\sunmoonsrdata{06:21}{18:15}{03:56*}{14:57}{12:18}
{\kalas{04:44 05:32 09:31 08:44 10:19 16:40 11:07 13:30 15:52 17:28 19:04 21:16 22:47 01:48*}}}
{\tnykdata{\anga{\tithi{26}{कृष्ण-एकादशी}}{\time{18-20}{13:41}}\hspace{1ex}}%
{\anga{उत्तराषाढा}{\time{15-20}{12:29}}\hspace{1ex}}{चन्द्रराशिः—\mbox{मकरः}}%
{\anga{परिघः}{\time{37-28}{21:20}}\hspace{1ex}\uanga{शिवः}}%
{\anga{बालवः}{\time{18-20}{13:41}}\hspace{1ex}\anga{कौलवः}{\time{51-4}{02:47*}}\hspace{1ex}\uanga{तैतिलः}}{}
}
{काञ्ची ७० जगद्गुरु श्री-शङ्कर विजयेन्द्र सरस्वती जयन्ती~\#{५०}\eventsep \tamil{மாசி~செவ்வாய்}\eventsep सर्व-पापमोचनी-एकादशी}
{Tue} 
\cfoot{\rygdata{15:17--16:46}{09:20--10:49}{12:18--13:47}}
\caldata{MARCH}{14}{\sunmonth{कुम्भः}{30}{\mbox{कुम्भः{\tiny\RIGHTarrow}{23:27}}}{फाल्गुनः}{शिशिरऋतुः}{बुधः}{हेमलम्बः}{उत्तरायणम्}{शिशिरऋतुः}}
{\sunmoonsrdata{06:20}{18:15}{04:38*}{15:47}{12:18}
{\kalas{04:44 05:32 09:31 08:43 10:19 16:40 11:06 13:29 15:52 17:28 19:04 21:16 22:47 01:48*}}}
{\tnykdata{\anga{\tithi{27}{कृष्ण-द्वादशी}}{\time{23-33}{15:45}}\hspace{1ex}}%
{\anga{श्रवणः}{\time{21-50}{15:04}}\hspace{1ex}}{चन्द्रराशिः—\mbox{मकरः\RIGHTarrow{04:11*}}}%
{\anga{शिवः}{\time{38-36}{21:47}}\hspace{1ex}\uanga{सिद्धः}}%
{\anga{तैतिलः}{\time{23-33}{15:45}}\hspace{1ex}\anga{गरः}{\time{55-40}{04:37*}}\hspace{1ex}\uanga{वणिजः}}{}
}
{सावित्री-व्रतम्\eventsep प्रदोष-व्रतम्~18:15\RIGHTarrow{}19:46\eventsep रवि-सङ्क्रमण-पुण्यकालः~17:02\RIGHTarrow{}18:15\eventsep सङ्क्रमण-दिन-अपराह्ण-पुण्यकालः~12:18\RIGHTarrow{}18:15\eventsep विजया/श्रवण-महाद्वादशी\eventsep श्रवण-व्रतम्}
{Wed} 
\cfoot{\rygdata{12:18--13:47}{07:50--09:19}{10:48--12:18}}
\caldata{MARCH}{15}{\sunmonth{मीनः}{1}{}{फाल्गुनः}{शिशिरऋतुः}{गुरुः}{हेमलम्बः}{उत्तरायणम्}{शिशिरऋतुः}}
{\sunmoonsrdata{06:20}{18:15}{05:18*}{16:36}{12:18}
{\kalas{04:43 05:31 09:30 08:43 10:18 16:40 11:06 13:29 15:52 17:28 19:04 21:16 22:47 01:48*}}}
{\tnykdata{\anga{\tithi{28}{कृष्ण-त्रयोदशी}}{\time{27-28}{17:19}}\hspace{1ex}}%
{\anga{श्रविष्ठा}{\time{27-6}{17:10}}\hspace{1ex}}{चन्द्रराशिः—\mbox{कुम्भः}}%
{\anga{सिद्धः}{\time{38-43}{21:49}}\hspace{1ex}\uanga{साध्यः}}%
{\anga{वणिजः}{\time{27-28}{17:19}}\hspace{1ex}\anga{विष्टिः}{\time{58-53}{05:53*}}\hspace{1ex}\uanga{शकुनिः}}{}
}
{मासशिवरात्रिः\eventsep पञ्च-पर्व-पूजा (चतुर्दशी)}
{Thu} 
\cfoot{\rygdata{13:47--15:16}{06:20--07:49}{09:19--10:48}}
\caldata{MARCH}{16}{\sunmonth{मीनः}{2}{}{फाल्गुनः}{शिशिरऋतुः}{शुक्रः}{हेमलम्बः}{उत्तरायणम्}{शिशिरऋतुः}}
{\sunmoonsrdata{06:19}{18:16}{05:57*}{17:25}{12:17}
{\kalas{04:42 05:31 09:30 08:42 10:18 16:40 11:06 13:29 15:52 17:28 19:04 21:16 22:47 01:47*}}}
{\tnykdata{\anga{\tithi{29}{कृष्ण-चतुर्दशी}}{\time{29-57}{18:18}}\hspace{1ex}}%
{\anga{शतभिषक्}{\time{30-59}{18:43}}\hspace{1ex}}{चन्द्रराशिः—\mbox{कुम्भः}}%
{\anga{साध्यः}{\time{37-44}{21:25}}\hspace{1ex}\uanga{शुभः}}%
{\anga{शकुनिः}{\time{29-57}{18:18}}\hspace{1ex}\uanga{चतुष्पात्}}{}
}
{पञ्च-पर्व-पूजा (अमावास्या)}
{Fri} 
\cfoot{\rygdata{10:48--12:17}{15:16--16:46}{07:49--09:18}}
\caldata{MARCH}{17}{\sunmonth{मीनः}{3}{}{फाल्गुनः}{शिशिरऋतुः}{शनिः}{हेमलम्बः}{उत्तरायणम्}{शिशिरऋतुः}}
{\sunmoonsrdata{06:18}{18:16}{---}{18:15}{12:17}
{\kalas{04:42 05:30 09:30 08:42 10:17 16:40 11:05 13:29 15:52 17:28 19:04 21:16 22:46 01:47*}}}
{\tnykdata{\anga{\tithi{30}{अमावास्या}}{\time{30-57}{18:41}}\hspace{1ex}}%
{\anga{पूर्वप्रोष्ठपदा}{\time{33-27}{19:41}}\hspace{1ex}}{चन्द्रराशिः—\mbox{कुम्भः\RIGHTarrow{13:30}}}%
{\anga{शुभः}{\time{35-36}{20:33}}\hspace{1ex}\uanga{शुक्लः}}%
{\anga{चतुष्पात्}{\time{0-38}{06:34}}\hspace{1ex}\anga{नाग}{\time{30-57}{18:41}}\hspace{1ex}\uanga{किंस्तुघ्नः}}{}
}
{काञ्ची ६५ जगद्गुरु श्री-सुदर्शन महादेवेन्द्र सरस्वती आराधना~\#{१२८}\eventsep मन्वादिः-(रैवतः-[५])\eventsep पार्वणव्रतम् अमावास्यायाम्\eventsep सर्व-फाल्गुन-अमावास्या (अलभ्यम्–पूर्वप्रोष्ठपदा)}
{Sat} 
\cfoot{\rygdata{09:18--10:47}{13:47--15:16}{06:18--07:48}}
\caldata{MARCH}{18}{\sunmonth{मीनः}{4}{}{चैत्रः}{वसन्तऋतुः}{भानुः}{हेमलम्बः}{उत्तरायणम्}{शिशिरऋतुः}}
{\sunmoonrsdata{06:18}{18:16}{06:37}{19:05}{12:17}
{\kalas{04:41 05:30 09:29 08:41 10:17 16:40 11:05 13:28 15:52 17:28 19:04 21:16 22:46 01:46*}}}
{\tnykdata{\anga{\tithi{1}{शुक्ल-प्रथमा}}{\time{30-34}{18:31}}\hspace{1ex}}%
{\anga{उत्तरप्रोष्ठपदा}{\time{34-36}{20:08}}\hspace{1ex}}{चन्द्रराशिः—\mbox{मीनः}}%
{\anga{शुक्लः}{\time{32-25}{19:16}}\hspace{1ex}\uanga{ब्रह्म}}%
{\anga{किंस्तुघ्नः}{\time{0-56}{06:40}}\hspace{1ex}\anga{बवः}{\time{30-34}{18:31}}\hspace{1ex}\anga{बालवः}{\time{59-54}{06:15*}}\hspace{1ex}\uanga{कौलवः}}{}
}
{दर्शेष्टिः\eventsep काञ्ची १५ जगद्गुरु श्री-गङ्गाधरेन्द्र सरस्वती आराधना~\#{१६९०}\eventsep काञ्ची २७ जगद्गुरु श्री-चिद्विलासेन्द्र सरस्वती आराधना~\#{१४४२}\eventsep काञ्ची ५२ जगद्गुरु श्री-शङ्करानन्देन्द्र सरस्वती आराधना~\#{६०२}\eventsep पार्वण-प्रायश्चित्तावकाशः पौर्णमास्याम्\eventsep स्थालीपाकः\eventsep वसन्तनवरात्र-आरम्भः\eventsep युगादिः/चान्द्रमान-संवत्सरारम्भः\eventsep श्वेत-कल्पादिः}
{Sun} 
\cfoot{\rygdata{16:46--18:16}{12:17--13:46}{15:16--16:46}}
\caldata{MARCH}{19}{\sunmonth{मीनः}{5}{}{चैत्रः}{वसन्तऋतुः}{सोमः}{हेमलम्बः}{उत्तरायणम्}{शिशिरऋतुः}}
{\sunmoonrsdata{06:17}{18:16}{07:18}{19:58}{12:16}
{\kalas{04:41 05:29 09:29 08:41 10:17 16:40 11:04 13:28 15:52 17:28 19:04 21:16 22:46 01:46*}}}
{\tnykdata{\anga{\tithi{2}{शुक्ल-द्वितीया}}{\time{28-59}{17:53}}\hspace{1ex}}%
{\anga{रेवती}{\time{34-35}{20:07}}\hspace{1ex}}{चन्द्रराशिः—\mbox{मीनः\RIGHTarrow{20:07}}}%
{\anga{ब्रह्म}{\time{28-19}{17:37}}\hspace{1ex}\uanga{इन्द्रः}}%
{\anga{कौलवः}{\time{28-59}{17:53}}\hspace{1ex}\anga{तैतिलः}{\time{57-48}{05:24*}}\hspace{1ex}\uanga{गरः}}{}
}
{आन्दोलन-तृतीया\eventsep बालेन्दुव्रतम्\eventsep चन्द्र-दर्शनम्~18:16\RIGHTarrow{}19:58}
{Mon} 
\cfoot{\rygdata{07:47--09:17}{10:46--12:16}{13:46--15:16}}
\caldata{MARCH}{20}{\sunmonth{मीनः}{6}{}{चैत्रः}{वसन्तऋतुः}{मङ्गलः}{हेमलम्बः}{उत्तरायणम्}{शिशिरऋतुः}}
{\sunmoonrsdata{06:16}{18:16}{08:02}{20:52}{12:16}
{\kalas{04:40 05:28 09:28 08:40 10:16 16:40 11:04 13:28 15:52 17:28 19:04 21:16 22:46 01:46*}}}
{\tnykdata{\anga{\tithi{3}{शुक्ल-तृतीया}}{\time{26-24}{16:50}}\hspace{1ex}}%
{\anga{अश्विनी}{\time{33-35}{19:43}}\hspace{1ex}}{चन्द्रराशिः—\mbox{मेषः}}%
{\anga{इन्द्रः}{\time{23-24}{15:38}}\hspace{1ex}\uanga{वैधृतिः}}%
{\anga{गरः}{\time{26-24}{16:50}}\hspace{1ex}\anga{वणिजः}{\time{54-47}{04:11*}}\hspace{1ex}\uanga{विष्टिः}}{}
}
{(सायन) विषु-पुण्यकालः~17:45\RIGHTarrow{}18:16\eventsep अरुन्धती-व्रतम्\eventsep भौमाश्विनी-पुण्यकालः\eventsep गौरी-तृतीया/सौभाग्य-गौरी-व्रतम्\eventsep मधु-मासः/वसन्तऋतुः~21:45\RIGHTarrow{}\eventsep महातारा-जयन्ती\eventsep मन्वादिः-(उत्तमः-[३])\eventsep पार्वतीश्वरयोरान्दोलनव्रतम्\eventsep रवि-सङ्क्रमण-पुण्यकालः~15:21\RIGHTarrow{}18:16\eventsep सायन-सङ्क्रमण-दिन-अपराह्ण-पुण्यकालः~12:16\RIGHTarrow{}18:16\eventsep वैधृति-श्राद्धम्}
{Tue} 
\cfoot{\rygdata{15:16--16:46}{09:16--10:46}{12:16--13:46}}
\caldata{MARCH}{21}{\sunmonth{मीनः}{7}{}{चैत्रः}{वसन्तऋतुः}{बुधः}{हेमलम्बः}{उत्तरायणम्}{शिशिरऋतुः}}
{\sunmoonrsdata{06:16}{18:16}{08:48}{21:50}{12:16}
{\kalas{04:40 05:28 09:28 08:40 10:16 16:40 11:04 13:28 15:52 17:28 19:04 21:16 22:46 01:45*}}}
{\tnykdata{\anga{\tithi{4}{शुक्ल-चतुर्थी}}{\time{23-1}{15:28}}\hspace{1ex}}%
{\anga{अपभरणी}{\time{31-50}{19:00}}\hspace{1ex}}{चन्द्रराशिः—\mbox{मेषः\RIGHTarrow{00:47*}}}%
{\anga{वैधृतिः}{\time{17-51}{13:24}}\hspace{1ex}\uanga{विष्कम्भः}}%
{\anga{विष्टिः}{\time{23-1}{15:28}}\hspace{1ex}\anga{बवः}{\time{51-3}{02:41*}}\hspace{1ex}\uanga{बालवः}}{}
}
{कूर्म-कल्पादिः\eventsep कपालीश्वर-ध्वजारोहणम्\eventsep विषुवदिनम्\eventsep शुक्ल-चतुर्थी-व्रतम्}
{Wed} 
\cfoot{\rygdata{12:16--13:46}{07:46--09:16}{10:46--12:16}}
\caldata{MARCH}{22}{\sunmonth{मीनः}{8}{}{चैत्रः}{वसन्तऋतुः}{गुरुः}{हेमलम्बः}{उत्तरायणम्}{शिशिरऋतुः}}
{\sunmoonrsdata{06:15}{18:16}{09:39}{22:49}{12:16}
{\kalas{04:39 05:27 09:27 08:39 10:15 16:40 11:03 13:28 15:52 17:28 19:04 21:16 22:45 01:45*}}}
{\tnykdata{\anga{\tithi{5}{शुक्ल-पञ्चमी}}{\time{19-0}{13:51}}\hspace{1ex}}%
{\anga{कृत्तिका}{\time{29-30}{18:03}}\hspace{1ex}}{चन्द्रराशिः—\mbox{वृषभः}}%
{\anga{विष्कम्भः}{\time{11-47}{10:58}}\hspace{1ex}\uanga{प्रीतिः}}%
{\anga{बालवः}{\time{19-0}{13:51}}\hspace{1ex}\anga{कौलवः}{\time{46-48}{00:58*}}\hspace{1ex}\uanga{तैतिलः}}{}
}
{हय-पूजा\eventsep कृत्तिका-व्रतम्\eventsep \tamil{கபாலீ ஸூர்ய~சந்த்ர~வட்டம்}\eventsep लक्ष्मी-पञ्चमी\eventsep मुत्तुस्वामि-दीक्षित-जयन्ती~\#{२४४}\eventsep शालिहोत्र-व्रत-आरम्भः}
{Thu} 
\cfoot{\rygdata{13:46--15:16}{06:15--07:45}{09:15--10:45}}
\caldata{MARCH}{23}{\sunmonth{मीनः}{9}{}{चैत्रः}{वसन्तऋतुः}{शुक्रः}{हेमलम्बः}{उत्तरायणम्}{शिशिरऋतुः}}
{\sunmoonrsdata{06:14}{18:16}{10:33}{23:49}{12:15}
{\kalas{04:39 05:26 09:27 08:39 10:15 16:40 11:03 13:27 15:52 17:28 19:04 21:15 22:45 01:45*}}}
{\tnykdata{\anga{\tithi{6}{शुक्ल-षष्ठी}}{\time{14-31}{12:03}}\hspace{1ex}}%
{\anga{रोहिणी}{\time{26-43}{16:56}}\hspace{1ex}}{चन्द्रराशिः—\mbox{वृषभः\RIGHTarrow{04:19*}}}%
{\anga{प्रीतिः}{\time{5-20}{08:22}}\hspace{1ex}\anga{आयुष्मान्}{\time{58-33}{05:40*}}\hspace{1ex}\uanga{सौभाग्यः}}%
{\anga{तैतिलः}{\time{14-31}{12:03}}\hspace{1ex}\anga{गरः}{\time{42-7}{23:05}}\hspace{1ex}\uanga{वणिजः}}{}
}
{षष्ठी-व्रतम्\eventsep कपाल्यधिकार-नन्दी\eventsep \tamil{கபாலீ பூதண் பூதகீ}\eventsep \tamil{நேச நாயனார் (58) குருபூஜை}\eventsep यमुना-जयन्ती}
{Fri} 
\cfoot{\rygdata{10:45--12:15}{15:16--16:46}{07:44--09:15}}
\caldata{MARCH}{24}{\sunmonth{मीनः}{10}{}{चैत्रः}{वसन्तऋतुः}{शनिः}{हेमलम्बः}{उत्तरायणम्}{शिशिरऋतुः}}
{\sunmoonrsdata{06:14}{18:16}{11:31}{00:49*}{12:15}
{\kalas{04:38 05:26 09:26 08:38 10:15 16:40 11:03 13:27 15:52 17:28 19:04 21:15 22:45 01:44*}}}
{\tnykdata{\anga{\tithi{7}{शुक्ल-सप्तमी}}{\time{9-40}{10:06}}\hspace{1ex}}%
{\anga{मृगशीर्षम्}{\time{23-36}{15:40}}\hspace{1ex}}{चन्द्रराशिः—\mbox{मिथुनम्}}%
{\anga{सौभाग्यः}{\time{51-35}{02:52*}}\hspace{1ex}\uanga{शोभनः}}%
{\anga{वणिजः}{\time{9-40}{10:06}}\hspace{1ex}\anga{विष्टिः}{\time{37-7}{21:05}}\hspace{1ex}\uanga{बवः}}{}
}
{सूर्यस्य दमनकपूजा}
{Sat} 
\cfoot{\rygdata{09:14--10:45}{13:45--15:15}{06:14--07:44}}
\caldata{MARCH}{25}{\sunmonth{मीनः}{11}{}{चैत्रः}{वसन्तऋतुः}{भानुः}{हेमलम्बः}{उत्तरायणम्}{शिशिरऋतुः}}
{\sunmoonrsdata{06:13}{18:16}{12:32}{01:47*}{12:15}
{\kalas{04:37 05:25 09:26 08:38 10:14 16:40 11:02 13:27 15:52 17:28 19:04 21:15 22:45 01:44*}}}
{\tnykdata{\anga{\tithi{8}{शुक्ल-अष्टमी}}{\time{4-32}{08:02}}\hspace{1ex}\anga{\tithi{9}{शुक्ल-नवमी}}{\time{59-12}{05:54*}}\hspace{1ex}\avamA{}}%
{\anga{आर्द्रा}{\time{20-14}{14:19}}\hspace{1ex}}{चन्द्रराशिः—\mbox{मिथुनम्}}%
{\anga{शोभनः}{\time{44-26}{00:00*}}\hspace{1ex}\uanga{अतिगण्डः}}%
{\anga{बवः}{\time{4-32}{08:02}}\hspace{1ex}\anga{बालवः}{\time{31-53}{18:59}}\hspace{1ex}\anga{कौलवः}{\time{59-12}{05:54*}}\hspace{1ex}\uanga{तैतिलः}}{}
}
{अशोकाष्टमी\eventsep भवान्युत्पत्तिः\eventsep \tamil{கணநாத நாயனார் (37) குருபூஜை}\eventsep काञ्ची ४३ जगद्गुरु श्री-आनन्दघनेन्द्र सरस्वती आराधना~\#{१००५}\eventsep \tamil{கபாலீ சவுடல் விமானம்}\eventsep कपालि-वृषभ-वाहनम्\eventsep वसन्तनवरात्र-समापनम्\eventsep श्रीरामनवमी}
{Sun} 
\cfoot{\rygdata{16:46--18:16}{12:15--13:45}{15:15--16:46}}
\caldata{MARCH}{26}{\sunmonth{मीनः}{12}{}{चैत्रः}{वसन्तऋतुः}{सोमः}{हेमलम्बः}{उत्तरायणम्}{शिशिरऋतुः}}
{\sunmoonrsdata{06:12}{18:16}{13:33}{02:42*}{12:14}
{\kalas{04:37 05:25 09:25 08:37 10:14 16:40 11:02 13:27 15:51 17:28 19:04 21:15 22:45 01:43*}}}
{\tnykdata{\anga{\tithi{10}{शुक्ल-दशमी}}{\time{53-46}{03:43*}}\hspace{1ex}}%
{\anga{पुनर्वसुः}{\time{16-42}{12:54}}\hspace{1ex}}{चन्द्रराशिः—\mbox{मिथुनम्\RIGHTarrow{07:15}}}%
{\anga{अतिगण्डः}{\time{37-10}{21:05}}\hspace{1ex}\uanga{सुकर्म}}%
{\anga{तैतिलः}{\time{26-30}{16:49}}\hspace{1ex}\anga{गरः}{\time{53-46}{03:43*}}\hspace{1ex}\uanga{वणिजः}}{}
}
{धर्मराज-दशमी\eventsep \tamil{கபாலீ பல்லக்கு விழா}}
{Mon} 
\cfoot{\rygdata{07:43--09:13}{10:44--12:14}{13:45--15:15}}
\caldata{MARCH}{27}{\sunmonth{मीनः}{13}{}{चैत्रः}{वसन्तऋतुः}{मङ्गलः}{हेमलम्बः}{उत्तरायणम्}{शिशिरऋतुः}}
{\sunmoonrsdata{06:12}{18:16}{14:33}{03:32*}{12:14}
{\kalas{04:36 05:24 09:25 08:37 10:13 16:40 11:01 13:26 15:51 17:28 19:04 21:15 22:44 01:43*}}}
{\tnykdata{\anga{\tithi{11}{शुक्ल-एकादशी}}{\time{48-19}{01:32*}}\hspace{1ex}}%
{\anga{पुष्यः}{\time{13-6}{11:26}}\hspace{1ex}}{चन्द्रराशिः—\mbox{कर्कटः}}%
{\anga{सुकर्म}{\time{29-53}{18:09}}\hspace{1ex}\uanga{धृतिः}}%
{\anga{वणिजः}{\time{21-3}{14:37}}\hspace{1ex}\anga{विष्टिः}{\time{48-19}{01:32*}}\hspace{1ex}\uanga{बवः}}{}
}
{ऋषीणां दमनकपूजा\eventsep कपालीश्वरयात्रा\eventsep \tamil{முனையடுவார் நாயனார் (50) குருபூஜை}\eventsep समुद्र-मन्थनम्\eventsep सर्व-कामदा-एकादशी\eventsep श्रीकृष्णदोलोत्सवः}
{Tue} 
\cfoot{\rygdata{15:15--16:46}{09:13--10:43}{12:14--13:45}}
\caldata{MARCH}{28}{\sunmonth{मीनः}{14}{}{चैत्रः}{वसन्तऋतुः}{बुधः}{हेमलम्बः}{उत्तरायणम्}{शिशिरऋतुः}}
{\sunmoonrsdata{06:11}{18:16}{15:32}{04:19*}{12:14}
{\kalas{04:36 05:23 09:24 08:36 10:13 16:40 11:01 13:26 15:51 17:28 19:04 21:15 22:44 01:43*}}}
{\tnykdata{\anga{\tithi{12}{शुक्ल-द्वादशी}}{\time{43-0}{23:23}}\hspace{1ex}}%
{\anga{आश्रेषा}{\time{9-30}{09:59}}\hspace{1ex}}{चन्द्रराशिः—\mbox{कर्कटः\RIGHTarrow{09:59}}}%
{\anga{धृतिः}{\time{22-39}{15:15}}\hspace{1ex}\uanga{शूलः}}%
{\anga{बवः}{\time{15-39}{12:27}}\hspace{1ex}\anga{बालवः}{\time{43-0}{23:23}}\hspace{1ex}\uanga{कौलवः}}{}
}
{भ्रातृप्राप्ति-व्रत-आरम्भः\eventsep दमनकारोपण-द्वादशी\eventsep हरिवासरः\RIGHTarrow{}06:59\eventsep \tamil{கபாலீ அறுபத்து மூவர்}\eventsep तुलसी-जननं-क्षीरसागरतः\eventsep वेङ्कटाचले वसन्तोत्सव-प्रारम्भः\eventsep विष्णु-दमनकोत्सवः}
{Wed} 
\cfoot{\rygdata{12:14--13:44}{07:42--09:12}{10:43--12:14}}
\caldata{MARCH}{29}{\sunmonth{मीनः}{15}{}{चैत्रः}{वसन्तऋतुः}{गुरुः}{हेमलम्बः}{उत्तरायणम्}{शिशिरऋतुः}}
{\sunmoonrsdata{06:10}{18:16}{16:28}{05:04*}{12:13}
{\kalas{04:35 05:23 09:24 08:36 10:12 16:40 11:01 13:26 15:51 17:28 19:04 21:15 22:44 01:42*}}}
{\tnykdata{\anga{\tithi{13}{शुक्ल-त्रयोदशी}}{\time{38-0}{21:22}}\hspace{1ex}}%
{\anga{मघा}{\time{6-7}{08:37}}\hspace{1ex}}{चन्द्रराशिः—\mbox{सिंहः}}%
{\anga{शूलः}{\time{15-37}{12:25}}\hspace{1ex}\uanga{गण्डः}}%
{\anga{कौलवः}{\time{10-28}{10:22}}\hspace{1ex}\anga{तैतिलः}{\time{38-0}{21:22}}\hspace{1ex}\uanga{गरः}}{}
}
{दमनक-चोरी-उत्सवः\eventsep मदन-त्रयोदशी\eventsep प्रदोष-व्रतम्~18:16\RIGHTarrow{}19:46\eventsep वेङ्कटाचले वसन्तोत्सवः}
{Thu} 
\cfoot{\rygdata{13:44--15:15}{06:10--07:41}{09:12--10:43}}
\caldata{MARCH}{30}{\sunmonth{मीनः}{16}{}{चैत्रः}{वसन्तऋतुः}{शुक्रः}{हेमलम्बः}{उत्तरायणम्}{शिशिरऋतुः}}
{\sunmoonrsdata{06:10}{18:16}{17:23}{05:47*}{12:13}
{\kalas{04:35 05:22 09:24 08:35 10:12 16:40 11:00 13:26 15:51 17:28 19:04 21:15 22:44 01:42*}}}
{\tnykdata{\anga{\tithi{14}{शुक्ल-चतुर्दशी}}{\time{33-32}{19:35}}\hspace{1ex}}%
{\anga{पूर्वफल्गुनी}{\time{3-8}{07:25}}\hspace{1ex}}{चन्द्रराशिः—\mbox{सिंहः\RIGHTarrow{13:09}}}%
{\anga{गण्डः}{\time{8-57}{09:45}}\hspace{1ex}\uanga{वृद्धिः}}%
{\anga{गरः}{\time{5-42}{08:27}}\hspace{1ex}\anga{वणिजः}{\time{33-32}{19:35}}\hspace{1ex}\uanga{विष्टिः}}{}
}
{दमनक-चतुर्दशी\eventsep \tamil{கற்பகாம்பாள்–கபாலீஶ்வரர் திருக்கல்யாணம்}\eventsep मदन-चतुर्दशी\eventsep नृसिंह-दोलोत्सवः\eventsep मीनोत्तरफाल्गुनोत्सवः\eventsep पञ्च-पर्व-पूजा (पूर्णिमा)\eventsep वेङ्कटाचले पूर्णिमा-गरुड-सेवा\eventsep वेङ्कटाचले वसन्तोत्सव-समापनम्}
{Fri} 
\cfoot{\rygdata{10:42--12:13}{15:15--16:46}{07:41--09:11}}
\caldata{MARCH}{31}{\sunmonth{मीनः}{17}{}{चैत्रः}{वसन्तऋतुः}{शनिः}{हेमलम्बः}{उत्तरायणम्}{शिशिरऋतुः}}
{\sunmoonrsdata{06:09}{18:17}{18:17}{---}{12:13}
{\kalas{04:34 05:21 09:23 08:34 10:12 16:40 11:00 13:26 15:51 17:28 19:04 21:14 22:43 01:41*}}}
{\tnykdata{\anga{\tithi{15}{पौर्णमासी}}{\time{29-53}{18:06}}\hspace{1ex}}%
{\anga{उत्तरफल्गुनी}{\time{0-49}{06:29}}\hspace{1ex}\anga{हस्तः}{\time{59-24}{05:55*}}\hspace{1ex}}{चन्द्रराशिः—\mbox{कन्या}}%
{\anga{वृद्धिः}{\time{2-51}{07:17}}\hspace{1ex}\anga{ध्रुवः}{\time{57-29}{05:09*}}\hspace{1ex}\uanga{व्याघातः}}%
{\anga{विष्टिः}{\time{1-36}{06:48}}\hspace{1ex}\anga{बवः}{\time{29-53}{18:06}}\hspace{1ex}\anga{बालवः}{\time{58-26}{05:32*}}\hspace{1ex}\uanga{कौलवः}}{}
}
{चैत्र-पूर्णिमा\eventsep चित्रगुप्त-व्रतम्\eventsep गजेन्द्र-मोक्षः\eventsep उमा-कपालीश्वर-दर्शनम्\eventsep \tamil{கபாலீ விடையாற்றி தொடக்கம்}\eventsep मन्वादिः-(रौच्यः-[१३])\eventsep पार्वणव्रतम् पूर्णिमायाम्\eventsep पूर्णिमा-व्रतम्\eventsep श्री-हनूमत्-जयन्ती}
{Sat} 
\cfoot{\rygdata{09:11--10:42}{13:44--15:15}{06:09--07:40}}
\caldata{APRIL}{1}{\sunmonth{मीनः}{18}{}{चैत्रः}{वसन्तऋतुः}{भानुः}{हेमलम्बः}{उत्तरायणम्}{शिशिरऋतुः}}
{\sunmoonsrdata{06:08}{18:17}{19:10}{06:30}{12:12}
{\kalas{04:33 05:21 09:23 08:34 10:11 16:40 11:00 13:25 15:51 17:28 19:04 21:14 22:43 01:41*}}}
{\tnykdata{\anga{\tithi{16}{कृष्ण-प्रथमा}}{\time{27-20}{17:04}}\hspace{1ex}}%
{\anga{चित्रा}{\time{59-14}{05:50*}}\hspace{1ex}}{चन्द्रराशिः—\mbox{कन्या\RIGHTarrow{17:48}}}%
{\anga{व्याघातः}{\time{53-7}{03:23*}}\hspace{1ex}\uanga{हर्षणः}}%
{\anga{कौलवः}{\time{27-20}{17:04}}\hspace{1ex}\anga{तैतिलः}{\time{56-32}{04:45*}}\hspace{1ex}\uanga{गरः}}{}
}
{पार्वण-प्रायश्चित्तावकाशः दर्शे\eventsep पूर्णमासेष्टिः\eventsep स्थालीपाकः}
{Sun} 
\cfoot{\rygdata{16:46--18:17}{12:12--13:43}{15:15--16:46}}
\caldata{APRIL}{2}{\sunmonth{मीनः}{19}{}{चैत्रः}{वसन्तऋतुः}{सोमः}{हेमलम्बः}{उत्तरायणम्}{शिशिरऋतुः}}
{\sunmoonsrdata{06:08}{18:17}{20:03}{07:13}{12:12}
{\kalas{04:33 05:20 09:22 08:33 10:11 16:40 10:59 13:25 15:51 17:28 19:04 21:14 22:43 01:41*}}}
{\tnykdata{\anga{\tithi{17}{कृष्ण-द्वितीया}}{\time{26-8}{16:35}}\hspace{1ex}}%
{\fullanga{स्वाती}}{चन्द्रराशिः—\mbox{तुला}}%
{\anga{हर्षणः}{\time{49-55}{02:06*}}\hspace{1ex}\uanga{वज्रम्}}%
{\anga{गरः}{\time{26-8}{16:35}}\hspace{1ex}\anga{वणिजः}{\time{56-6}{04:34*}}\hspace{1ex}\uanga{विष्टिः}}{}
}
{काञ्ची ६० जगद्गुरु श्री-अद्वैतात्मप्रकाशेन्द्र सरस्वती आराधना~\#{३१५}\eventsep काञ्ची जगद्गुरु श्री-जयेन्द्र सरस्वती आश्रम-स्वीकार-जयन्ती~\#{६५}\eventsep \tamil{காரைக்கால் அம்மையார் (23) குருபூஜை}}
{Mon} 
\cfoot{\rygdata{07:39--09:10}{10:41--12:12}{13:43--15:14}}
\caldata{APRIL}{3}{\sunmonth{मीनः}{20}{}{चैत्रः}{वसन्तऋतुः}{मङ्गलः}{हेमलम्बः}{उत्तरायणम्}{शिशिरऋतुः}}
{\sunmoonsrdata{06:07}{18:17}{20:56}{07:57}{12:12}
{\kalas{04:32 05:20 09:22 08:33 10:10 16:39 10:59 13:25 15:51 17:28 19:04 21:14 22:43 01:40*}}}
{\tnykdata{\anga{\tithi{18}{कृष्ण-तृतीया}}{\time{26-30}{16:43}}\hspace{1ex}}%
{\anga{स्वाती}{\time{0-33}{06:20}}\hspace{1ex}}{चन्द्रराशिः—\mbox{तुला\RIGHTarrow{01:08*}}}%
{\anga{वज्रम्}{\time{48-1}{01:20*}}\hspace{1ex}\uanga{सिद्धिः}}%
{\anga{विष्टिः}{\time{26-30}{16:43}}\hspace{1ex}\anga{बवः}{\time{57-19}{05:03*}}\hspace{1ex}\uanga{बालवः}}{}
}
{अङ्गारकी विकट-महागणपति सङ्कटहर-चतुर्थी-व्रतम्}
{Tue} 
\cfoot{\rygdata{15:14--16:46}{09:09--10:41}{12:12--13:43}}
\caldata{APRIL}{4}{\sunmonth{मीनः}{21}{}{चैत्रः}{वसन्तऋतुः}{बुधः}{हेमलम्बः}{उत्तरायणम्}{शिशिरऋतुः}}
{\sunmoonsrdata{06:06}{18:17}{21:49}{08:43}{12:12}
{\kalas{04:32 05:19 09:21 08:32 10:10 16:39 10:59 13:25 15:51 17:28 19:04 21:14 22:43 01:40*}}}
{\tnykdata{\anga{\tithi{19}{कृष्ण-चतुर्थी}}{\time{28-34}{17:32}}\hspace{1ex}}%
{\anga{विशाखा}{\time{3-27}{07:29}}\hspace{1ex}}{चन्द्रराशिः—\mbox{वृश्चिकः}}%
{\anga{सिद्धिः}{\time{47-25}{01:04*}}\hspace{1ex}\uanga{व्यतीपातः}}%
{\anga{बालवः}{\time{28-34}{17:32}}\hspace{1ex}\uanga{कौलवः}}{}
}
{बुधानुराधा-पुण्यकालः~07:29\RIGHTarrow{}}
{Wed} 
\cfoot{\rygdata{12:12--13:43}{07:38--09:09}{10:40--12:12}}
\caldata{APRIL}{5}{\sunmonth{मीनः}{22}{}{चैत्रः}{वसन्तऋतुः}{गुरुः}{हेमलम्बः}{उत्तरायणम्}{शिशिरऋतुः}}
{\sunmoonsrdata{06:06}{18:17}{22:41}{09:30}{12:11}
{\kalas{04:31 05:18 09:21 08:32 10:09 16:39 10:58 13:24 15:51 17:28 19:04 21:14 22:42 01:40*}}}
{\tnykdata{\anga{\tithi{20}{कृष्ण-पञ्चमी}}{\time{32-17}{19:01}}\hspace{1ex}}%
{\anga{अनूराधा}{\time{7-58}{09:17}}\hspace{1ex}}{चन्द्रराशिः—\mbox{वृश्चिकः}}%
{\anga{व्यतीपातः}{\time{48-3}{01:19*}}\hspace{1ex}\uanga{वरीयान्}}%
{\anga{कौलवः}{\time{0-15}{06:12}}\hspace{1ex}\anga{तैतिलः}{\time{32-17}{19:01}}\hspace{1ex}\uanga{गरः}}{}
}
{वराह-जयन्ती\eventsep व्यतीपात-श्राद्धम्}
{Thu} 
\cfoot{\rygdata{13:43--15:14}{06:06--07:37}{09:09--10:40}}
\caldata{APRIL}{6}{\sunmonth{मीनः}{23}{}{चैत्रः}{वसन्तऋतुः}{शुक्रः}{हेमलम्बः}{उत्तरायणम्}{शिशिरऋतुः}}
{\sunmoonsrdata{06:05}{18:17}{23:32}{10:19}{12:11}
{\kalas{04:31 05:18 09:20 08:31 10:09 16:39 10:58 13:24 15:51 17:28 19:04 21:14 22:42 01:39*}}}
{\tnykdata{\anga{\tithi{21}{कृष्ण-षष्ठी}}{\time{37-25}{21:03}}\hspace{1ex}}%
{\anga{ज्येष्ठा}{\time{13-58}{11:41}}\hspace{1ex}}{चन्द्रराशिः—\mbox{वृश्चिकः\RIGHTarrow{11:41}}}%
{\anga{वरीयान्}{\time{49-41}{01:58*}}\hspace{1ex}\uanga{परिघः}}%
{\anga{गरः}{\time{4-42}{07:58}}\hspace{1ex}\anga{वणिजः}{\time{37-25}{21:03}}\hspace{1ex}\uanga{विष्टिः}}{}
}
{}
{Fri} 
\cfoot{\rygdata{10:40--12:11}{15:14--16:45}{07:37--09:08}}
\caldata{APRIL}{7}{\sunmonth{मीनः}{24}{}{चैत्रः}{वसन्तऋतुः}{शनिः}{हेमलम्बः}{उत्तरायणम्}{शिशिरऋतुः}}
{\sunmoonsrdata{06:04}{18:17}{00:20*}{11:09}{12:11}
{\kalas{04:30 05:17 09:20 08:31 10:09 16:39 10:57 13:24 15:50 17:28 19:04 21:14 22:42 01:39*}}}
{\tnykdata{\anga{\tithi{22}{कृष्ण-सप्तमी}}{\time{43-32}{23:30}}\hspace{1ex}}%
{\anga{मूला}{\time{21-5}{14:31}}\hspace{1ex}}{चन्द्रराशिः—\mbox{धनुः}}%
{\anga{परिघः}{\time{52-0}{02:53*}}\hspace{1ex}\uanga{शिवः}}%
{\anga{विष्टिः}{\time{10-24}{10:14}}\hspace{1ex}\anga{बवः}{\time{43-32}{23:30}}\hspace{1ex}\uanga{बालवः}}{}
}
{}
{Sat} 
\cfoot{\rygdata{09:08--10:39}{13:42--15:14}{06:04--07:36}}
\caldata{APRIL}{8}{\sunmonth{मीनः}{25}{}{चैत्रः}{वसन्तऋतुः}{भानुः}{हेमलम्बः}{उत्तरायणम्}{शिशिरऋतुः}}
{\sunmoonsrdata{06:04}{18:17}{01:07*}{11:59}{12:10}
{\kalas{04:30 05:17 09:19 08:30 10:08 16:39 10:57 13:24 15:50 17:28 19:04 21:14 22:42 01:38*}}}
{\tnykdata{\anga{\tithi{23}{कृष्ण-अष्टमी}}{\time{50-2}{02:05*}}\hspace{1ex}}%
{\anga{पूर्वाषाढा}{\time{28-47}{17:35}}\hspace{1ex}}{चन्द्रराशिः—\mbox{धनुः\RIGHTarrow{00:21*}}}%
{\anga{शिवः}{\time{54-33}{03:53*}}\hspace{1ex}\uanga{सिद्धः}}%
{\anga{बालवः}{\time{16-48}{12:47}}\hspace{1ex}\anga{कौलवः}{\time{50-2}{02:05*}}\hspace{1ex}\uanga{तैतिलः}}{}
}
{काञ्ची ५६ जगद्गुरु श्री-सर्वज्ञ सदाशिव बोधेन्द्र सरस्वती आराधना~\#{४८०}\eventsep पञ्च-पर्व-पूजा (अष्टमी)}
{Sun} 
\cfoot{\rygdata{16:45--18:17}{12:10--13:42}{15:14--16:45}}
\caldata{APRIL}{9}{\sunmonth{मीनः}{26}{}{चैत्रः}{वसन्तऋतुः}{सोमः}{हेमलम्बः}{उत्तरायणम्}{शिशिरऋतुः}}
{\sunmoonsrdata{06:03}{18:17}{01:51*}{12:49}{12:10}
{\kalas{04:29 05:16 09:19 08:30 10:08 16:39 10:57 13:24 15:50 17:28 19:04 21:14 22:42 01:38*}}}
{\tnykdata{\anga{\tithi{24}{कृष्ण-नवमी}}{\time{56-16}{04:34*}}\hspace{1ex}}%
{\anga{उत्तराषाढा}{\time{36-25}{20:37}}\hspace{1ex}}{चन्द्रराशिः—\mbox{मकरः}}%
{\anga{सिद्धः}{\time{56-52}{04:48*}}\hspace{1ex}\uanga{साध्यः}}%
{\anga{तैतिलः}{\time{23-15}{15:21}}\hspace{1ex}\anga{गरः}{\time{56-16}{04:34*}}\hspace{1ex}\uanga{वणिजः}}{}
}
{}
{Mon} 
\cfoot{\rygdata{07:35--09:07}{10:38--12:10}{13:42--15:14}}
\caldata{APRIL}{10}{\sunmonth{मीनः}{27}{}{चैत्रः}{वसन्तऋतुः}{मङ्गलः}{हेमलम्बः}{उत्तरायणम्}{शिशिरऋतुः}}
{\sunmoonsrdata{06:03}{18:17}{02:32*}{13:38}{12:10}
{\kalas{04:28 05:16 09:19 08:30 10:07 16:39 10:56 13:23 15:50 17:28 19:04 21:13 22:42 01:38*}}}
{\tnykdata{\fulltithi{\tithi{25}{कृष्ण-दशमी}}}%
{\anga{श्रवणः}{\time{43-21}{23:23}}\hspace{1ex}}{चन्द्रराशिः—\mbox{मकरः}}%
{\anga{साध्यः}{\time{58-28}{05:26*}}\hspace{1ex}\uanga{शुभः}}%
{\anga{वणिजः}{\time{29-5}{17:41}}\hspace{1ex}\uanga{विष्टिः}}{}
}
{\tamil{கபாலீ விடையாற்றி நிறைவு}\eventsep श्रवण-व्रतम्}
{Tue} 
\cfoot{\rygdata{15:14--16:45}{09:06--10:38}{12:10--13:42}}
\caldata{APRIL}{11}{\sunmonth{मीनः}{28}{}{चैत्रः}{वसन्तऋतुः}{बुधः}{हेमलम्बः}{उत्तरायणम्}{शिशिरऋतुः}}
{\sunmoonsrdata{06:02}{18:17}{03:13*}{14:27}{12:10}
{\kalas{04:28 05:15 09:18 08:29 10:07 16:39 10:56 13:23 15:50 17:28 19:04 21:13 22:41 01:37*}}}
{\tnykdata{\anga{\tithi{25}{कृष्ण-दशमी}}{\time{1-35}{06:40}}\hspace{1ex}}%
{\anga{श्रविष्ठा}{\time{49-2}{01:39*}}\hspace{1ex}}{चन्द्रराशिः—\mbox{मकरः\RIGHTarrow{12:35}}}%
{\anga{शुभः}{\time{59-1}{05:39*}}\hspace{1ex}\uanga{शुक्लः}}%
{\anga{विष्टिः}{\time{1-35}{06:40}}\hspace{1ex}\anga{बवः}{\time{33-43}{19:31}}\hspace{1ex}\uanga{बालवः}}{}
}
{}
{Wed} 
\cfoot{\rygdata{12:10--13:42}{07:34--09:06}{10:38--12:10}}
\caldata{APRIL}{12}{\sunmonth{मीनः}{29}{}{चैत्रः}{वसन्तऋतुः}{गुरुः}{हेमलम्बः}{उत्तरायणम्}{शिशिरऋतुः}}
{\sunmoonsrdata{06:01}{18:18}{03:52*}{15:16}{12:09}
{\kalas{04:27 05:14 09:18 08:29 10:07 16:39 10:56 13:23 15:50 17:28 19:04 21:13 22:41 01:37*}}}
{\tnykdata{\anga{\tithi{26}{कृष्ण-एकादशी}}{\time{5-28}{08:13}}\hspace{1ex}}%
{\anga{शतभिषक्}{\time{53-7}{03:16*}}\hspace{1ex}}{चन्द्रराशिः—\mbox{कुम्भः}}%
{\anga{शुक्लः}{\time{58-16}{05:20*}}\hspace{1ex}\uanga{ब्रह्म}}%
{\anga{बालवः}{\time{5-28}{08:13}}\hspace{1ex}\anga{कौलवः}{\time{36-46}{20:44}}\hspace{1ex}\uanga{तैतिलः}}{}
}
{\tamil{தண்டியடிகள் நாயனார் (30) குருபூஜை}\eventsep सर्व-वरूथिनी-एकादशी\eventsep वल्लभाचार्य-जयन्ती~\#{५४०}}
{Thu} 
\cfoot{\rygdata{13:41--15:13}{06:01--07:33}{09:05--10:37}}
\caldata{APRIL}{13}{\sunmonth{मीनः}{30}{}{चैत्रः}{वसन्तऋतुः}{शुक्रः}{हेमलम्बः}{उत्तरायणम्}{शिशिरऋतुः}}
{\sunmoonsrdata{06:01}{18:18}{04:32*}{16:05}{12:09}
{\kalas{04:27 05:14 09:17 08:28 10:06 16:39 10:55 13:23 15:50 17:28 19:04 21:13 22:41 01:37*}}}
{\tnykdata{\anga{\tithi{27}{कृष्ण-द्वादशी}}{\time{7-38}{09:04}}\hspace{1ex}}%
{\anga{पूर्वप्रोष्ठपदा}{\time{55-27}{04:12*}}\hspace{1ex}}{चन्द्रराशिः—\mbox{कुम्भः\RIGHTarrow{22:02}}}%
{\anga{ब्रह्म}{\time{56-7}{04:28*}}\hspace{1ex}\uanga{इन्द्रः}}%
{\anga{तैतिलः}{\time{7-38}{09:04}}\hspace{1ex}\anga{गरः}{\time{38-0}{21:13}}\hspace{1ex}\uanga{वणिजः}}{}
}
{देवी-पर्व-१\eventsep मत्स्य-जयन्ती\eventsep प्रदोष-व्रतम्~18:17\RIGHTarrow{}19:45}
{Fri} 
\cfoot{\rygdata{10:37--12:09}{15:13--16:45}{07:33--09:05}}
\caldata{APRIL}{14}{\sunmonth{मेषः}{1}{\mbox{मीनः{\tiny\RIGHTarrow}{07:57}}}{चैत्रः}{वसन्तऋतुः}{शनिः}{विलम्बः}{उत्तरायणम्}{वसन्तऋतुः}}
{\sunmoonsrdata{06:00}{18:18}{05:13*}{16:56}{12:09}
{\kalas{04:26 05:13 09:17 08:28 10:06 16:39 10:55 13:23 15:50 17:29 19:04 21:13 22:41 01:36*}}}
{\tnykdata{\anga{\tithi{28}{कृष्ण-त्रयोदशी}}{\time{7-58}{09:11}}\hspace{1ex}}%
{\anga{उत्तरप्रोष्ठपदा}{\time{56-4}{04:26*}}\hspace{1ex}}{चन्द्रराशिः—\mbox{मीनः}}%
{\anga{इन्द्रः}{\time{52-35}{03:02*}}\hspace{1ex}\uanga{वैधृतिः}}%
{\anga{वणिजः}{\time{7-58}{09:11}}\hspace{1ex}\anga{विष्टिः}{\time{37-28}{20:59}}\hspace{1ex}\uanga{शकुनिः}}{}
}
{मासशिवरात्रिः\eventsep मेष-सङ्क्रान्तिः (विलम्बः-संवत्सरः)\eventsep मेष-सङ्क्रमण-पुण्यकालः~06:00\RIGHTarrow{}11:57\eventsep पञ्चाङ्ग-पठनम्\eventsep पञ्च-पर्व-पूजा (चतुर्दशी)\eventsep रवि-सङ्क्रमण-पुण्यकालः~06:00\RIGHTarrow{}14:21\eventsep सङ्क्रमण-दिन-पूर्वाह्ण-पुण्यकालः~06:00\RIGHTarrow{}12:09\eventsep \tamil{விஷுக்கனி}}
{Sat} 
\cfoot{\rygdata{09:05--10:37}{13:41--15:13}{06:00--07:32}}
\caldata{APRIL}{15}{\sunmonth{मेषः}{2}{}{चैत्रः}{वसन्तऋतुः}{भानुः}{विलम्बः}{उत्तरायणम्}{वसन्तऋतुः}}
{\sunmoonsrdata{06:00}{18:18}{05:56*}{17:48}{12:09}
{\kalas{04:26 05:13 09:16 08:27 10:06 16:39 10:55 13:22 15:50 17:29 19:04 21:13 22:41 01:36*}}}
{\tnykdata{\anga{\tithi{29}{कृष्ण-चतुर्दशी}}{\time{6-34}{08:37}}\hspace{1ex}}%
{\anga{रेवती}{\time{55-9}{04:03*}}\hspace{1ex}}{चन्द्रराशिः—\mbox{मीनः\RIGHTarrow{04:03*}}}%
{\anga{वैधृतिः}{\time{47-47}{01:07*}}\hspace{1ex}\uanga{विष्कम्भः}}%
{\anga{शकुनिः}{\time{6-34}{08:37}}\hspace{1ex}\anga{चतुष्पात्}{\time{35-16}{20:06}}\hspace{1ex}\uanga{नाग}}{}
}
{गङ्गा-स्नानम्\eventsep काञ्ची ४७ जगद्गुरु श्री-चन्द्रशेखरेन्द्र सरस्वती ३ आराधना~\#{८५३}\eventsep पार्वणव्रतम् अमावास्यायाम्\eventsep पञ्च-पर्व-पूजा (अमावास्या)\eventsep सर्व-चैत्र-अमावास्या\eventsep वैधृति-श्राद्धम्}
{Sun} 
\cfoot{\rygdata{16:45--18:18}{12:09--13:41}{15:13--16:45}}
\caldata{APRIL}{16}{\sunmonth{मेषः}{3}{}{चैत्रः}{वसन्तऋतुः}{सोमः}{विलम्बः}{उत्तरायणम्}{वसन्तऋतुः}}
{\sunmoonsrdata{05:59}{18:18}{---}{18:43}{12:08}
{\kalas{04:25 05:12 09:16 08:27 10:05 16:39 10:55 13:22 15:50 17:29 19:05 21:13 22:41 01:36*}}}
{\tnykdata{\anga{\tithi{30}{अमावास्या}}{\time{3-39}{07:27}}\hspace{1ex}\anga{\tithi{1}{शुक्ल-प्रथमा}}{\time{59-29}{05:47*}}\hspace{1ex}\avamA{}}%
{\anga{अश्विनी}{\time{52-58}{03:10*}}\hspace{1ex}}{चन्द्रराशिः—\mbox{मेषः}}%
{\anga{विष्कम्भः}{\time{41-57}{22:46}}\hspace{1ex}\uanga{प्रीतिः}}%
{\anga{नाग}{\time{3-39}{07:27}}\hspace{1ex}\anga{किंस्तुघ्नः}{\time{31-42}{18:40}}\hspace{1ex}\anga{बवः}{\time{59-29}{05:47*}}\hspace{1ex}\uanga{बालवः}}{}
}
{भार्गव-राम-पूजा\eventsep दर्शेष्टिः\eventsep पार्वण-प्रायश्चित्तावकाशः पौर्णमास्याम्\eventsep पराशर-महर्षि-जयन्ती\eventsep सोमवती अमावास्या\eventsep स्थालीपाकः\eventsep वह्नि-व्रतम्\eventsep वैशाख-मास-आरम्भः}
{Mon} 
\cfoot{\rygdata{07:31--09:04}{10:36--12:08}{13:41--15:13}}
\caldata{APRIL}{17}{\sunmonth{मेषः}{4}{}{वैशाखः}{वसन्तऋतुः}{मङ्गलः}{विलम्बः}{उत्तरायणम्}{वसन्तऋतुः}}
{\sunmoonrsdata{05:58}{18:18}{06:42}{19:41}{12:08}
{\kalas{04:25 05:12 09:16 08:26 10:05 16:39 10:54 13:22 15:50 17:29 19:05 21:13 22:40 01:35*}}}
{\tnykdata{\anga{\tithi{2}{शुक्ल-द्वितीया}}{\time{54-26}{03:45*}}\hspace{1ex}}%
{\anga{अपभरणी}{\time{49-52}{01:55*}}\hspace{1ex}}{चन्द्रराशिः—\mbox{मेषः}}%
{\anga{प्रीतिः}{\time{35-17}{20:06}}\hspace{1ex}\uanga{आयुष्मान्}}%
{\anga{बालवः}{\time{27-3}{16:48}}\hspace{1ex}\anga{कौलवः}{\time{54-26}{03:45*}}\hspace{1ex}\uanga{तैतिलः}}{}
}
{चन्द्र-दर्शनम्~18:18\RIGHTarrow{}19:41\eventsep \tamil{சிறுத்தொண்ட நாயனார் (35) குருபூஜை}}
{Tue} 
\cfoot{\rygdata{15:13--16:45}{09:03--10:36}{12:08--13:41}}
\caldata{APRIL}{18}{\sunmonth{मेषः}{5}{}{वैशाखः}{वसन्तऋतुः}{बुधः}{विलम्बः}{उत्तरायणम्}{वसन्तऋतुः}}
{\sunmoonrsdata{05:58}{18:18}{07:33}{20:41}{12:08}
{\kalas{04:24 05:11 09:15 08:26 10:05 16:39 10:54 13:22 15:50 17:29 19:05 21:13 22:40 01:35*}}}
{\tnykdata{\anga{\tithi{3}{शुक्ल-तृतीया}}{\time{48-49}{01:30*}}\hspace{1ex}}%
{\anga{कृत्तिका}{\time{46-10}{00:26*}}\hspace{1ex}}{चन्द्रराशिः—\mbox{मेषः\RIGHTarrow{07:34}}}%
{\anga{आयुष्मान्}{\time{28-5}{17:12}}\hspace{1ex}\uanga{सौभाग्यः}}%
{\anga{तैतिलः}{\time{21-41}{14:38}}\hspace{1ex}\anga{गरः}{\time{48-49}{01:30*}}\hspace{1ex}\uanga{वणिजः}}{}
}
{अक्षय्य-तृतीया\eventsep बलराम-जयन्ती\eventsep चन्दन-पूजा\eventsep देवी-पर्व-२\eventsep कृतयुगादिः\eventsep कृत्तिका-व्रतम्\eventsep प्रोक्लस्-मृत्युः~\#{१५३३}\eventsep राज-मातङ्गी-जयन्ती\eventsep श्यामा-शास्त्री-जयन्ती~\#{२५७}}
{Wed} 
\cfoot{\rygdata{12:08--13:40}{07:30--09:03}{10:35--12:08}}
\caldata{APRIL}{19}{\sunmonth{मेषः}{6}{}{वैशाखः}{वसन्तऋतुः}{गुरुः}{विलम्बः}{उत्तरायणम्}{वसन्तऋतुः}}
{\sunmoonrsdata{05:57}{18:18}{08:27}{21:43}{12:08}
{\kalas{04:24 05:11 09:15 08:26 10:04 16:39 10:54 13:22 15:50 17:29 19:05 21:13 22:40 01:35*}}}
{\tnykdata{\anga{\tithi{4}{शुक्ल-चतुर्थी}}{\time{42-56}{23:08}}\hspace{1ex}}%
{\anga{रोहिणी}{\time{42-11}{22:50}}\hspace{1ex}}{चन्द्रराशिः—\mbox{वृषभः}}%
{\anga{सौभाग्यः}{\time{20-35}{14:11}}\hspace{1ex}\uanga{शोभनः}}%
{\anga{वणिजः}{\time{15-54}{12:19}}\hspace{1ex}\anga{विष्टिः}{\time{42-56}{23:08}}\hspace{1ex}\uanga{बवः}}{}
}
{बगलामुखी-जयन्ती\eventsep \tamil{மங்கையர்க்கரசியார் நாயனார் (57) குருபூஜை}\eventsep वार्ता-गौरी-व्रतम्\eventsep शुक्ल-चतुर्थी-व्रतम्}
{Thu} 
\cfoot{\rygdata{13:40--15:13}{05:57--07:30}{09:03--10:35}}
\caldata{APRIL}{20}{\sunmonth{मेषः}{7}{}{वैशाखः}{वसन्तऋतुः}{शुक्रः}{विलम्बः}{उत्तरायणम्}{वसन्तऋतुः}}
{\sunmoonrsdata{05:57}{18:18}{09:25}{22:44}{12:08}
{\kalas{04:24 05:10 09:14 08:25 10:04 16:39 10:53 13:22 15:50 17:29 19:05 21:13 22:40 01:35*}}}
{\tnykdata{\anga{\tithi{5}{शुक्ल-पञ्चमी}}{\time{37-1}{20:45}}\hspace{1ex}}%
{\anga{मृगशीर्षम्}{\time{38-10}{21:13}}\hspace{1ex}}{चन्द्रराशिः—\mbox{वृषभः\RIGHTarrow{10:01}}}%
{\anga{शोभनः}{\time{12-58}{11:08}}\hspace{1ex}\uanga{अतिगण्डः}}%
{\anga{बवः}{\time{9-58}{09:56}}\hspace{1ex}\anga{बालवः}{\time{37-1}{20:45}}\hspace{1ex}\uanga{कौलवः}}{}
}
{(सायन) विष्णुपदी-पुण्यकालः~05:57\RIGHTarrow{}15:06\eventsep लावण्य-गौरी-व्रतम्\eventsep माधव-मासः~08:42\RIGHTarrow{}\eventsep सायन-सङ्क्रमण-दिन-पूर्वाह्ण-पुण्यकालः~05:57\RIGHTarrow{}12:08\eventsep सूरदास-जयन्ती~\#{५४१}\eventsep सर्प-पूजा\eventsep शङ्कर-जयन्ती~\#{२५२७}}
{Fri} 
\cfoot{\rygdata{10:35--12:08}{15:13--16:46}{07:29--09:02}}
\caldata{APRIL}{21}{\sunmonth{मेषः}{8}{}{वैशाखः}{वसन्तऋतुः}{शनिः}{विलम्बः}{उत्तरायणम्}{वसन्तऋतुः}}
{\sunmoonrsdata{05:56}{18:19}{10:26}{23:43}{12:07}
{\kalas{04:23 05:10 09:14 08:25 10:04 16:39 10:53 13:22 15:50 17:29 19:05 21:13 22:40 01:34*}}}
{\tnykdata{\anga{\tithi{6}{शुक्ल-षष्ठी}}{\time{31-18}{18:27}}\hspace{1ex}}%
{\anga{आर्द्रा}{\time{34-21}{19:41}}\hspace{1ex}}{चन्द्रराशिः—\mbox{मिथुनम्}}%
{\anga{अतिगण्डः}{\time{5-27}{08:07}}\hspace{1ex}\anga{सुकर्म}{\time{58-8}{05:11*}}\hspace{1ex}\uanga{धृतिः}}%
{\anga{कौलवः}{\time{4-8}{07:36}}\hspace{1ex}\anga{तैतिलः}{\time{31-18}{18:27}}\hspace{1ex}\anga{गरः}{\time{58-32}{05:21*}}\hspace{1ex}\uanga{वणिजः}}{}
}
{षष्ठी-व्रतम्\eventsep काञ्ची ४० जगद्गुरु श्री-महादेवेन्द्र सरस्वती २ आराधना~\#{११०४}\eventsep रामानुज-जन्म-नक्षत्रम्~\#{१००२}\eventsep रामानुज-जयन्ती~\#{१००२}\eventsep \tamil{விறன்மிண்ட நாயனார் (5) குருபூஜை}}
{Sat} 
\cfoot{\rygdata{09:02--10:35}{13:40--15:13}{05:56--07:29}}
\caldata{APRIL}{22}{\sunmonth{मेषः}{9}{}{वैशाखः}{वसन्तऋतुः}{भानुः}{विलम्बः}{उत्तरायणम्}{वसन्तऋतुः}}
{\sunmoonrsdata{05:56}{18:19}{11:28}{00:38*}{12:07}
{\kalas{04:23 05:09 09:14 08:24 10:03 16:40 10:53 13:21 15:50 17:29 19:05 21:13 22:40 01:34*}}}
{\tnykdata{\anga{\tithi{7}{शुक्ल-सप्तमी}}{\time{25-53}{16:17}}\hspace{1ex}}%
{\anga{पुनर्वसुः}{\time{30-51}{18:16}}\hspace{1ex}}{चन्द्रराशिः—\mbox{मिथुनम्\RIGHTarrow{12:37}}}%
{\anga{धृतिः}{\time{51-7}{02:23*}}\hspace{1ex}\uanga{शूलः}}%
{\anga{वणिजः}{\time{25-53}{16:17}}\hspace{1ex}\anga{विष्टिः}{\time{53-18}{03:15*}}\hspace{1ex}\uanga{बवः}}{}
}
{गङ्गा-सप्तमी\eventsep रविपुष्ययोग-पुण्यकालः~18:16\RIGHTarrow{}\eventsep त्यागराज-जयन्ती~\#{२५२}\eventsep विद्यारण्य-स्वामि-जयन्ती\eventsep विजया-भानुसप्तमी\eventsep शर्करा-सप्तमी}
{Sun} 
\cfoot{\rygdata{16:46--18:19}{12:07--13:40}{15:13--16:46}}
\caldata{APRIL}{23}{\sunmonth{मेषः}{10}{}{वैशाखः}{वसन्तऋतुः}{सोमः}{विलम्बः}{उत्तरायणम्}{वसन्तऋतुः}}
{\sunmoonrsdata{05:55}{18:19}{12:28}{01:29*}{12:07}
{\kalas{04:22 05:09 09:13 08:24 10:03 16:40 10:53 13:21 15:50 17:29 19:05 21:13 22:40 01:34*}}}
{\tnykdata{\anga{\tithi{8}{शुक्ल-अष्टमी}}{\time{20-51}{14:16}}\hspace{1ex}}%
{\anga{पुष्यः}{\time{27-45}{17:01}}\hspace{1ex}}{चन्द्रराशिः—\mbox{कर्कटः}}%
{\anga{शूलः}{\time{44-27}{23:42}}\hspace{1ex}\uanga{गण्डः}}%
{\anga{बवः}{\time{20-51}{14:16}}\hspace{1ex}\anga{बालवः}{\time{48-30}{01:19*}}\hspace{1ex}\uanga{कौलवः}}{}
}
{काञ्ची २६ जगद्गुरु श्री-प्रज्ञाघनेन्द्र सरस्वती आराधना~\#{१४५५}}
{Mon} 
\cfoot{\rygdata{07:28--09:01}{10:34--12:07}{13:40--15:13}}
\caldata{APRIL}{24}{\sunmonth{मेषः}{11}{}{वैशाखः}{वसन्तऋतुः}{मङ्गलः}{विलम्बः}{उत्तरायणम्}{वसन्तऋतुः}}
{\sunmoonrsdata{05:55}{18:19}{13:26}{02:17*}{12:07}
{\kalas{04:22 05:08 09:13 08:24 10:03 16:40 10:52 13:21 15:50 17:29 19:05 21:13 22:40 01:33*}}}
{\tnykdata{\anga{\tithi{9}{शुक्ल-नवमी}}{\time{16-17}{12:25}}\hspace{1ex}}%
{\anga{आश्रेषा}{\time{25-6}{15:57}}\hspace{1ex}}{चन्द्रराशिः—\mbox{कर्कटः\RIGHTarrow{15:57}}}%
{\anga{गण्डः}{\time{38-9}{21:11}}\hspace{1ex}\uanga{वृद्धिः}}%
{\anga{कौलवः}{\time{16-17}{12:25}}\hspace{1ex}\anga{तैतिलः}{\time{44-9}{23:34}}\hspace{1ex}\uanga{गरः}}{}
}
{नॆरूर्-श्री-सदाशिव-ब्रह्मेन्द्र-आराधना~\#{१०४}\eventsep पुरी गोवर्धन-मठ-प्रतिष्ठापन-जयन्ती~\#{२५०३}\eventsep सीतानवमी\eventsep सिंहाचलं-चन्दन-महोत्सवः\eventsep वेङ्कटाचले पद्मावती-परिणयोत्सव-प्रारम्भः (गज-वाहनम्)\eventsep वसिष्ठ-महर्षि-जयन्ती}
{Tue} 
\cfoot{\rygdata{15:13--16:46}{09:01--10:34}{12:07--13:40}}
\caldata{APRIL}{25}{\sunmonth{मेषः}{12}{}{वैशाखः}{वसन्तऋतुः}{बुधः}{विलम्बः}{उत्तरायणम्}{वसन्तऋतुः}}
{\sunmoonrsdata{05:54}{18:19}{14:22}{03:01*}{12:07}
{\kalas{04:21 05:08 09:13 08:23 10:02 16:40 10:52 13:21 15:50 17:29 19:05 21:13 22:39 01:33*}}}
{\tnykdata{\anga{\tithi{10}{शुक्ल-दशमी}}{\time{12-10}{10:46}}\hspace{1ex}}%
{\anga{मघा}{\time{22-55}{15:04}}\hspace{1ex}}{चन्द्रराशिः—\mbox{सिंहः}}%
{\anga{वृद्धिः}{\time{32-16}{18:49}}\hspace{1ex}\uanga{ध्रुवः}}%
{\anga{गरः}{\time{12-10}{10:46}}\hspace{1ex}\anga{वणिजः}{\time{40-17}{22:01}}\hspace{1ex}\uanga{विष्टिः}}{}
}
{काञ्ची १ जगद्गुरु श्री-आदि-शङ्कर भगवत्पाद आराधना~\#{२४९४}\eventsep निमिषाम्बा-जयन्ती\eventsep वेङ्कटाचले पद्मावती-परिणयम् (अश्व-वाहनम्)\eventsep श्री-वासवी-जयन्ती}
{Wed} 
\cfoot{\rygdata{12:07--13:40}{07:27--09:00}{10:33--12:07}}
\caldata{APRIL}{26}{\sunmonth{मेषः}{13}{}{वैशाखः}{वसन्तऋतुः}{गुरुः}{विलम्बः}{उत्तरायणम्}{वसन्तऋतुः}}
{\sunmoonrsdata{05:54}{18:19}{15:16}{03:43*}{12:06}
{\kalas{04:21 05:07 09:12 08:23 10:02 16:40 10:52 13:21 15:50 17:29 19:05 21:13 22:39 01:33*}}}
{\tnykdata{\anga{\tithi{11}{शुक्ल-एकादशी}}{\time{8-35}{09:20}}\hspace{1ex}}%
{\anga{पूर्वफल्गुनी}{\time{21-16}{14:24}}\hspace{1ex}}{चन्द्रराशिः—\mbox{सिंहः\RIGHTarrow{20:16}}}%
{\anga{ध्रुवः}{\time{26-49}{16:37}}\hspace{1ex}\uanga{व्याघातः}}%
{\anga{विष्टिः}{\time{8-35}{09:20}}\hspace{1ex}\anga{बवः}{\time{36-59}{20:42}}\hspace{1ex}\uanga{बालवः}}{}
}
{बुध-जयन्ती\eventsep सर्व-मोहिनी-एकादशी\eventsep वेङ्कटाचले पद्मावती-परिणयोत्सव-समापनम् (गरुड-वाहनम्)}
{Thu} 
\cfoot{\rygdata{13:40--15:13}{05:54--07:27}{09:00--10:33}}
\caldata{APRIL}{27}{\sunmonth{मेषः}{14}{}{वैशाखः}{वसन्तऋतुः}{शुक्रः}{विलम्बः}{उत्तरायणम्}{वसन्तऋतुः}}
{\sunmoonrsdata{05:53}{18:19}{16:08}{04:25*}{12:06}
{\kalas{04:21 05:07 09:12 08:22 10:02 16:40 10:52 13:21 15:50 17:30 19:06 21:13 22:39 01:33*}}}
{\tnykdata{\anga{\tithi{12}{शुक्ल-द्वादशी}}{\time{5-35}{08:07}}\hspace{1ex}}%
{\anga{उत्तरफल्गुनी}{\time{20-13}{13:59}}\hspace{1ex}}{चन्द्रराशिः—\mbox{कन्या}}%
{\anga{व्याघातः}{\time{21-52}{14:38}}\hspace{1ex}\uanga{हर्षणः}}%
{\anga{बालवः}{\time{5-35}{08:07}}\hspace{1ex}\anga{कौलवः}{\time{34-19}{19:37}}\hspace{1ex}\uanga{तैतिलः}}{}
}
{गिरिजा-कल्याणम्\eventsep \tamil{மீனாக்ஷீ திருக்கல்யாணம்}\eventsep मधुसूदन-पूजा\eventsep परशुराम-द्वादशी\eventsep प्रदोष-व्रतम्~18:19\RIGHTarrow{}19:46\eventsep रुक्मिणी-द्वादशी}
{Fri} 
\cfoot{\rygdata{10:33--12:06}{15:13--16:46}{07:26--09:00}}
\caldata{APRIL}{28}{\sunmonth{मेषः}{15}{}{वैशाखः}{वसन्तऋतुः}{शनिः}{विलम्बः}{उत्तरायणम्}{वसन्तऋतुः}}
{\sunmoonrsdata{05:53}{18:20}{17:01}{05:07*}{12:06}
{\kalas{04:20 05:07 09:12 08:22 10:02 16:40 10:51 13:21 15:50 17:30 19:06 21:13 22:39 01:32*}}}
{\tnykdata{\anga{\tithi{13}{शुक्ल-त्रयोदशी}}{\time{3-17}{07:12}}\hspace{1ex}}%
{\anga{हस्तः}{\time{19-55}{13:51}}\hspace{1ex}}{चन्द्रराशिः—\mbox{कन्या\RIGHTarrow{01:55*}}}%
{\anga{हर्षणः}{\time{17-31}{12:53}}\hspace{1ex}\uanga{वज्रम्}}%
{\anga{तैतिलः}{\time{3-17}{07:12}}\hspace{1ex}\anga{गरः}{\time{32-27}{18:52}}\hspace{1ex}\uanga{वणिजः}}{}
}
{छिन्नमस्ता-जयन्ती\eventsep काञ्ची ३९ जगद्गुरु श्री-सच्चिद्विलासेन्द्र सरस्वती आराधना~\#{११४६}\eventsep नृसिंह-जयन्ती}
{Sat} 
\cfoot{\rygdata{08:59--10:33}{13:39--15:13}{05:53--07:26}}
\caldata{APRIL}{29}{\sunmonth{मेषः}{16}{}{वैशाखः}{वसन्तऋतुः}{भानुः}{विलम्बः}{उत्तरायणम्}{वसन्तऋतुः}}
{\sunmoonrsdata{05:52}{18:20}{17:53}{05:51*}{12:06}
{\kalas{04:20 05:06 09:12 08:22 10:01 16:40 10:51 13:21 15:50 17:30 19:06 21:13 22:39 01:32*}}}
{\tnykdata{\anga{\tithi{14}{शुक्ल-चतुर्दशी}}{\time{1-52}{06:37}}\hspace{1ex}}%
{\anga{चित्रा}{\time{20-32}{14:05}}\hspace{1ex}}{चन्द्रराशिः—\mbox{तुला}}%
{\anga{वज्रम्}{\time{13-55}{11:26}}\hspace{1ex}\uanga{सिद्धिः}}%
{\anga{वणिजः}{\time{1-52}{06:37}}\hspace{1ex}\anga{विष्टिः}{\time{31-32}{18:29}}\hspace{1ex}\uanga{बवः}}{}
}
{चित्रा-पूर्णिमा\eventsep \tamil{இசைஞானியார் நாயனார் (62) குருபூஜை}\eventsep \tamil{மதுரகவி ஆழ்வார் திருநக்ஷத்திரம்}\eventsep पार्वणव्रतम् पूर्णिमायाम्\eventsep पञ्च-पर्व-पूजा (पूर्णिमा)\eventsep वेङ्कटाचले पूर्णिमा-गरुड-सेवा}
{Sun} 
\cfoot{\rygdata{16:46--18:20}{12:06--13:39}{15:13--16:46}}
\caldata{APRIL}{30}{\sunmonth{मेषः}{17}{}{वैशाखः}{वसन्तऋतुः}{सोमः}{विलम्बः}{उत्तरायणम्}{वसन्तऋतुः}}
{\sunmoonrsdata{05:52}{18:20}{18:46}{---}{12:06}
{\kalas{04:19 05:06 09:11 08:21 10:01 16:40 10:51 13:21 15:50 17:30 19:06 21:13 22:39 01:32*}}}
{\tnykdata{\anga{\tithi{15}{पौर्णमासी}}{\time{1-30}{06:28}}\hspace{1ex}}%
{\anga{स्वाती}{\time{22-13}{14:45}}\hspace{1ex}}{चन्द्रराशिः—\mbox{तुला}}%
{\anga{सिद्धिः}{\time{11-9}{10:20}}\hspace{1ex}\uanga{व्यतीपातः}}%
{\anga{बवः}{\time{1-30}{06:28}}\hspace{1ex}\anga{बालवः}{\time{31-45}{18:34}}\hspace{1ex}\uanga{कौलवः}}{}
}
{अन्नमाचार्य-जयन्ती\eventsep अर्धनारीश्वर-व्रतम्\eventsep काञ्ची कामकोटि-मठ-प्रतिष्ठापन-जयन्ती~\#{२५००}\eventsep पार्वण-प्रायश्चित्तावकाशः दर्शे\eventsep पूर्णमासेष्टिः\eventsep पूर्णिमा-व्रतम्\eventsep सम्पत्-गौरी-व्रतम्\eventsep स्थालीपाकः\eventsep \tamil{திருக்குறிப்புத் தொண்ட நாயனார் (18) குருபூஜை}\eventsep वैशाख-पूर्णिमा-स्नानम्\eventsep व्यतीपात-श्राद्धम्\eventsep शरभ-जयन्ती}
{Mon} 
\cfoot{\rygdata{07:25--08:59}{10:32--12:06}{13:39--15:13}}
\caldata{MAY}{1}{\sunmonth{मेषः}{18}{}{वैशाखः}{वसन्तऋतुः}{मङ्गलः}{विलम्बः}{उत्तरायणम्}{वसन्तऋतुः}}
{\sunmoonsrdata{05:51}{18:20}{19:39}{06:36}{12:06}
{\kalas{04:19 05:05 09:11 08:21 10:01 16:40 10:51 13:21 15:50 17:30 19:06 21:13 22:39 01:32*}}}
{\tnykdata{\anga{\tithi{16}{कृष्ण-प्रथमा}}{\time{2-20}{06:47}}\hspace{1ex}}%
{\anga{विशाखा}{\time{25-8}{15:55}}\hspace{1ex}}{चन्द्रराशिः—\mbox{तुला\RIGHTarrow{09:34}}}%
{\anga{व्यतीपातः}{\time{9-23}{09:37}}\hspace{1ex}\uanga{वरीयान्}}%
{\anga{कौलवः}{\time{2-20}{06:47}}\hspace{1ex}\anga{तैतिलः}{\time{33-15}{19:09}}\hspace{1ex}\uanga{गरः}}{}
}
{}
{Tue} 
\cfoot{\rygdata{15:13--16:46}{08:58--10:32}{12:06--13:39}}
\caldata{MAY}{2}{\sunmonth{मेषः}{19}{}{वैशाखः}{वसन्तऋतुः}{बुधः}{विलम्बः}{उत्तरायणम्}{वसन्तऋतुः}}
{\sunmoonsrdata{05:51}{18:20}{20:32}{07:22}{12:06}
{\kalas{04:19 05:05 09:11 08:21 10:01 16:40 10:51 13:20 15:50 17:30 19:06 21:13 22:39 01:32*}}}
{\tnykdata{\anga{\tithi{17}{कृष्ण-द्वितीया}}{\time{4-31}{07:40}}\hspace{1ex}}%
{\anga{अनूराधा}{\time{29-24}{17:36}}\hspace{1ex}}{चन्द्रराशिः—\mbox{वृश्चिकः}}%
{\anga{वरीयान्}{\time{8-39}{09:19}}\hspace{1ex}\uanga{परिघः}}%
{\anga{गरः}{\time{4-31}{07:40}}\hspace{1ex}\anga{वणिजः}{\time{36-8}{20:18}}\hspace{1ex}\uanga{विष्टिः}}{}
}
{बुधानुराधा-पुण्यकालः\RIGHTarrow{}17:36\eventsep नारद-जयन्ती\eventsep पार्थिव-कल्पादिः}
{Wed} 
\cfoot{\rygdata{12:06--13:39}{07:25--08:58}{10:32--12:06}}
\caldata{MAY}{3}{\sunmonth{मेषः}{20}{}{वैशाखः}{वसन्तऋतुः}{गुरुः}{विलम्बः}{उत्तरायणम्}{वसन्तऋतुः}}
{\sunmoonsrdata{05:51}{18:20}{21:23}{08:11}{12:05}
{\kalas{04:18 05:04 09:10 08:20 10:00 16:40 10:50 13:20 15:50 17:30 19:06 21:13 22:39 01:32*}}}
{\tnykdata{\anga{\tithi{18}{कृष्ण-तृतीया}}{\time{8-6}{09:05}}\hspace{1ex}}%
{\anga{ज्येष्ठा}{\time{34-58}{19:50}}\hspace{1ex}}{चन्द्रराशिः—\mbox{वृश्चिकः\RIGHTarrow{19:50}}}%
{\anga{परिघः}{\time{9-1}{09:27}}\hspace{1ex}\uanga{शिवः}}%
{\anga{विष्टिः}{\time{8-6}{09:05}}\hspace{1ex}\anga{बवः}{\time{40-22}{22:00}}\hspace{1ex}\uanga{बालवः}}{}
}
{एकदन्त-महागणपति सङ्कटहर-चतुर्थी-व्रतम्\eventsep काञ्ची ३० जगद्गुरु श्री-बोधेन्द्र सरस्वती २ आराधना~\#{१३६४}}
{Thu} 
\cfoot{\rygdata{13:39--15:13}{05:51--07:24}{08:58--10:32}}
\caldata{MAY}{4}{\sunmonth{मेषः}{21}{}{वैशाखः}{वसन्तऋतुः}{शुक्रः}{विलम्बः}{उत्तरायणम्}{वसन्तऋतुः}}
{\sunmoonsrdata{05:50}{18:21}{22:13}{09:01}{12:05}
{\kalas{04:18 05:04 09:10 08:20 10:00 16:40 10:50 13:20 15:50 17:31 19:06 21:13 22:39 01:31*}}}
{\tnykdata{\anga{\tithi{19}{कृष्ण-चतुर्थी}}{\time{12-57}{11:01}}\hspace{1ex}}%
{\anga{मूला}{\time{41-42}{22:31}}\hspace{1ex}}{चन्द्रराशिः—\mbox{धनुः}}%
{\anga{शिवः}{\time{10-22}{09:59}}\hspace{1ex}\uanga{सिद्धः}}%
{\anga{बालवः}{\time{12-57}{11:01}}\hspace{1ex}\anga{कौलवः}{\time{45-47}{00:09*}}\hspace{1ex}\uanga{तैतिलः}}{}
}
{सावित्री-व्रतम्}
{Fri} 
\cfoot{\rygdata{10:32--12:05}{15:13--16:47}{07:24--08:58}}
\caldata{MAY}{5}{\sunmonth{मेषः}{22}{}{वैशाखः}{वसन्तऋतुः}{शनिः}{विलम्बः}{उत्तरायणम्}{वसन्तऋतुः}}
{\sunmoonsrdata{05:50}{18:21}{23:00}{09:51}{12:05}
{\kalas{04:18 05:04 09:10 08:20 10:00 16:41 10:50 13:20 15:51 17:31 19:07 21:13 22:39 01:31*}}}
{\tnykdata{\anga{\tithi{20}{कृष्ण-पञ्चमी}}{\time{18-50}{13:22}}\hspace{1ex}}%
{\anga{पूर्वाषाढा}{\time{49-13}{01:31*}}\hspace{1ex}}{चन्द्रराशिः—\mbox{धनुः}}%
{\anga{सिद्धः}{\time{12-29}{10:50}}\hspace{1ex}\uanga{साध्यः}}%
{\anga{तैतिलः}{\time{18-50}{13:22}}\hspace{1ex}\anga{गरः}{\time{51-59}{02:38*}}\hspace{1ex}\uanga{वणिजः}}{}
}
{अग्निनक्षत्र-आरम्भः\eventsep कश्यप-महर्षि-जयन्ती}
{Sat} 
\cfoot{\rygdata{08:57--10:31}{13:39--15:13}{05:50--07:24}}
\caldata{MAY}{6}{\sunmonth{मेषः}{23}{}{वैशाखः}{वसन्तऋतुः}{भानुः}{विलम्बः}{उत्तरायणम्}{वसन्तऋतुः}}
{\sunmoonsrdata{05:49}{18:21}{23:45}{10:41}{12:05}
{\kalas{04:18 05:03 09:10 08:20 10:00 16:41 10:50 13:20 15:51 17:31 19:07 21:13 22:39 01:31*}}}
{\tnykdata{\anga{\tithi{21}{कृष्ण-षष्ठी}}{\time{25-14}{15:55}}\hspace{1ex}}%
{\anga{उत्तराषाढा}{\time{57-0}{04:38*}}\hspace{1ex}}{चन्द्रराशिः—\mbox{धनुः\RIGHTarrow{08:18}}}%
{\anga{साध्यः}{\time{15-4}{11:51}}\hspace{1ex}\uanga{शुभः}}%
{\anga{वणिजः}{\time{25-14}{15:55}}\hspace{1ex}\anga{विष्टिः}{\time{58-27}{05:13*}}\hspace{1ex}\uanga{बवः}}{}
}
{}
{Sun} 
\cfoot{\rygdata{16:47--18:21}{12:05--13:39}{15:13--16:47}}
\caldata{MAY}{7}{\sunmonth{मेषः}{24}{}{वैशाखः}{वसन्तऋतुः}{सोमः}{विलम्बः}{उत्तरायणम्}{वसन्तऋतुः}}
{\sunmoonsrdata{05:49}{18:21}{00:28*}{11:30}{12:05}
{\kalas{04:17 05:03 09:10 08:19 10:00 16:41 10:50 13:20 15:51 17:31 19:07 21:13 22:39 01:31*}}}
{\tnykdata{\anga{\tithi{22}{कृष्ण-सप्तमी}}{\time{31-36}{18:27}}\hspace{1ex}}%
{\fullanga{श्रवणः}}{चन्द्रराशिः—\mbox{मकरः}}%
{\anga{शुभः}{\time{17-41}{12:54}}\hspace{1ex}\uanga{शुक्लः}}%
{\anga{बवः}{\time{31-36}{18:27}}\hspace{1ex}\uanga{बालवः}}{}
}
{\tamil{நடராஜர் சித்திரை ஓணம் மஹாபிஷேகம்}\eventsep पञ्च-पर्व-पूजा (अष्टमी)\eventsep सोमश्रावणी-पुण्यकालः\eventsep श्रवण-व्रतम्}
{Mon} 
\cfoot{\rygdata{07:23--08:57}{10:31--12:05}{13:39--15:13}}
\caldata{MAY}{8}{\sunmonth{मेषः}{25}{}{वैशाखः}{वसन्तऋतुः}{मङ्गलः}{विलम्बः}{उत्तरायणम्}{वसन्तऋतुः}}
{\sunmoonsrdata{05:49}{18:21}{01:08*}{12:19}{12:05}
{\kalas{04:17 05:03 09:09 08:19 10:00 16:41 10:50 13:20 15:51 17:31 19:07 21:13 22:39 01:31*}}}
{\tnykdata{\anga{\tithi{23}{कृष्ण-अष्टमी}}{\time{37-14}{20:43}}\hspace{1ex}}%
{\anga{श्रवणः}{\time{4-28}{07:36}}\hspace{1ex}}{चन्द्रराशिः—\mbox{मकरः\RIGHTarrow{20:57}}}%
{\anga{शुक्लः}{\time{19-52}{13:46}}\hspace{1ex}\uanga{ब्रह्म}}%
{\anga{बालवः}{\time{4-33}{07:38}}\hspace{1ex}\anga{कौलवः}{\time{37-14}{20:43}}\hspace{1ex}\uanga{तैतिलः}}{}
}
{}
{Tue} 
\cfoot{\rygdata{15:13--16:47}{08:57--10:31}{12:05--13:39}}
\caldata{MAY}{9}{\sunmonth{मेषः}{26}{}{वैशाखः}{वसन्तऋतुः}{बुधः}{विलम्बः}{उत्तरायणम्}{वसन्तऋतुः}}
{\sunmoonsrdata{05:48}{18:22}{01:47*}{13:07}{12:05}
{\kalas{04:17 05:02 09:09 08:19 09:59 16:41 10:50 13:20 15:51 17:31 19:07 21:13 22:39 01:31*}}}
{\tnykdata{\anga{\tithi{24}{कृष्ण-नवमी}}{\time{41-35}{22:26}}\hspace{1ex}}%
{\anga{श्रविष्ठा}{\time{10-56}{10:11}}\hspace{1ex}}{चन्द्रराशिः—\mbox{कुम्भः}}%
{\anga{ब्रह्म}{\time{21-11}{14:17}}\hspace{1ex}\uanga{इन्द्रः}}%
{\anga{तैतिलः}{\time{9-37}{09:39}}\hspace{1ex}\anga{गरः}{\time{41-35}{22:26}}\hspace{1ex}\uanga{वणिजः}}{}
}
{काञ्ची ७ जगद्गुरु श्री-आनन्दज्ञानेन्द्र सरस्वती आराधना~\#{२०७३}}
{Wed} 
\cfoot{\rygdata{12:05--13:39}{07:22--08:57}{10:31--12:05}}
\caldata{MAY}{10}{\sunmonth{मेषः}{27}{}{वैशाखः}{वसन्तऋतुः}{गुरुः}{विलम्बः}{उत्तरायणम्}{वसन्तऋतुः}}
{\sunmoonsrdata{05:48}{18:22}{02:26*}{13:55}{12:05}
{\kalas{04:17 05:02 09:09 08:19 09:59 16:41 10:50 13:20 15:51 17:32 19:08 21:13 22:39 01:31*}}}
{\tnykdata{\anga{\tithi{25}{कृष्ण-दशमी}}{\time{44-9}{23:28}}\hspace{1ex}}%
{\anga{शतभिषक्}{\time{15-54}{12:10}}\hspace{1ex}}{चन्द्रराशिः—\mbox{कुम्भः}}%
{\anga{इन्द्रः}{\time{21-14}{14:18}}\hspace{1ex}\uanga{वैधृतिः}}%
{\anga{वणिजः}{\time{13-7}{11:03}}\hspace{1ex}\anga{विष्टिः}{\time{44-9}{23:28}}\hspace{1ex}\uanga{बवः}}{}
}
{\tamil{திருநாவுக்கரச நாயனார் (20) குருபூஜை}\eventsep वैधृति-श्राद्धम्}
{Thu} 
\cfoot{\rygdata{13:39--15:13}{05:48--07:22}{08:56--10:31}}
\caldata{MAY}{11}{\sunmonth{मेषः}{28}{}{वैशाखः}{वसन्तऋतुः}{शुक्रः}{विलम्बः}{उत्तरायणम्}{वसन्तऋतुः}}
{\sunmoonsrdata{05:48}{18:22}{03:06*}{14:45}{12:05}
{\kalas{04:16 05:02 09:09 08:19 09:59 16:41 10:50 13:20 15:51 17:32 19:08 21:13 22:39 01:30*}}}
{\tnykdata{\anga{\tithi{26}{कृष्ण-एकादशी}}{\time{44-44}{23:42}}\hspace{1ex}}%
{\anga{पूर्वप्रोष्ठपदा}{\time{18-59}{13:24}}\hspace{1ex}}{चन्द्रराशिः—\mbox{कुम्भः\RIGHTarrow{07:10}}}%
{\anga{वैधृतिः}{\time{19-47}{13:43}}\hspace{1ex}\uanga{विष्कम्भः}}%
{\anga{बवः}{\time{14-43}{11:41}}\hspace{1ex}\anga{बालवः}{\time{44-44}{23:42}}\hspace{1ex}\uanga{कौलवः}}{}
}
{भद्रकाळी-जयन्ती\eventsep सर्व-अपरा-एकादशी}
{Fri} 
\cfoot{\rygdata{10:31--12:05}{15:13--16:48}{07:22--08:56}}
\caldata{MAY}{12}{\sunmonth{मेषः}{29}{}{वैशाखः}{वसन्तऋतुः}{शनिः}{विलम्बः}{उत्तरायणम्}{वसन्तऋतुः}}
{\sunmoonsrdata{05:47}{18:22}{03:48*}{15:36}{12:05}
{\kalas{04:16 05:02 09:09 08:18 09:59 16:42 10:49 13:20 15:51 17:32 19:08 21:14 22:39 01:30*}}}
{\tnykdata{\anga{\tithi{27}{कृष्ण-द्वादशी}}{\time{43-17}{23:06}}\hspace{1ex}}%
{\anga{उत्तरप्रोष्ठपदा}{\time{20-6}{13:50}}\hspace{1ex}}{चन्द्रराशिः—\mbox{मीनः}}%
{\anga{विष्कम्भः}{\time{16-44}{12:29}}\hspace{1ex}\uanga{प्रीतिः}}%
{\anga{कौलवः}{\time{14-16}{11:30}}\hspace{1ex}\anga{तैतिलः}{\time{43-17}{23:06}}\hspace{1ex}\uanga{गरः}}{}
}
{}
{Sat} 
\cfoot{\rygdata{08:56--10:31}{13:39--15:14}{05:47--07:22}}
\caldata{MAY}{13}{\sunmonth{मेषः}{30}{}{वैशाखः}{वसन्तऋतुः}{भानुः}{विलम्बः}{उत्तरायणम्}{वसन्तऋतुः}}
{\sunmoonsrdata{05:47}{18:23}{04:32*}{16:29}{12:05}
{\kalas{04:16 05:01 09:09 08:18 09:59 16:42 10:49 13:20 15:51 17:32 19:08 21:14 22:39 01:30*}}}
{\tnykdata{\anga{\tithi{28}{कृष्ण-त्रयोदशी}}{\time{39-56}{21:46}}\hspace{1ex}}%
{\anga{रेवती}{\time{19-16}{13:30}}\hspace{1ex}}{चन्द्रराशिः—\mbox{मीनः\RIGHTarrow{13:30}}}%
{\anga{प्रीतिः}{\time{12-6}{10:38}}\hspace{1ex}\uanga{आयुष्मान्}}%
{\anga{गरः}{\time{11-50}{10:31}}\hspace{1ex}\anga{वणिजः}{\time{39-56}{21:46}}\hspace{1ex}\uanga{विष्टिः}}{}
}
{मासशिवरात्रिः\eventsep पञ्च-पर्व-पूजा (चतुर्दशी)\eventsep प्रदोष-व्रतम्~18:22\RIGHTarrow{}19:48\eventsep रमण-महर्षि-आराधना~\#{६८}}
{Sun} 
\cfoot{\rygdata{16:48--18:23}{12:05--13:39}{15:14--16:48}}
\caldata{MAY}{14}{\sunmonth{मेषः}{31}{\mbox{मेषः{\tiny\RIGHTarrow}{04:46*}}}{वैशाखः}{वसन्तऋतुः}{सोमः}{विलम्बः}{उत्तरायणम्}{वसन्तऋतुः}}
{\sunmoonsrdata{05:47}{18:23}{05:21*}{17:26}{12:05}
{\kalas{04:16 05:01 09:08 08:18 09:59 16:42 10:49 13:20 15:52 17:32 19:08 21:14 22:39 01:30*}}}
{\tnykdata{\anga{\tithi{29}{कृष्ण-चतुर्दशी}}{\time{34-59}{19:46}}\hspace{1ex}}%
{\anga{अश्विनी}{\time{16-44}{12:29}}\hspace{1ex}}{चन्द्रराशिः—\mbox{मेषः}}%
{\anga{आयुष्मान्}{\time{6-3}{08:12}}\hspace{1ex}\anga{सौभाग्यः}{\time{58-49}{05:19*}}\hspace{1ex}\uanga{शोभनः}}%
{\anga{विष्टिः}{\time{7-39}{08:50}}\hspace{1ex}\anga{शकुनिः}{\time{34-59}{19:46}}\hspace{1ex}\uanga{चतुष्पात्}}{}
}
{बोधायन वैशाख-अमावास्या\eventsep काञ्ची ३ जगद्गुरु श्री-सर्वज्ञात्मेन्द्र सरस्वती आराधना~\#{२३८२}\eventsep पञ्च-पर्व-पूजा (अमावास्या)}
{Mon} 
\cfoot{\rygdata{07:21--08:56}{10:30--12:05}{13:39--15:14}}
\caldata{MAY}{15}{\sunmonth{वृषभः}{1}{}{वैशाखः}{वसन्तऋतुः}{मङ्गलः}{विलम्बः}{उत्तरायणम्}{वसन्तऋतुः}}
{\sunmoonsrdata{05:47}{18:23}{---}{18:26}{12:05}
{\kalas{04:16 05:01 09:08 08:18 09:59 16:42 10:49 13:21 15:52 17:33 19:09 21:14 22:39 01:30*}}}
{\tnykdata{\anga{\tithi{30}{अमावास्या}}{\time{28-46}{17:17}}\hspace{1ex}}%
{\anga{अपभरणी}{\time{12-50}{10:55}}\hspace{1ex}}{चन्द्रराशिः—\mbox{मेषः\RIGHTarrow{16:27}}}%
{\anga{शोभनः}{\time{50-42}{02:04*}}\hspace{1ex}\uanga{अतिगण्डः}}%
{\anga{चतुष्पात्}{\time{2-0}{06:35}}\hspace{1ex}\anga{नाग}{\time{28-46}{17:17}}\hspace{1ex}\anga{किंस्तुघ्नः}{\time{55-19}{03:55*}}\hspace{1ex}\uanga{बवः}}{}
}
{बोधायन-इष्टिः\eventsep कृत्तिका-व्रतम्\eventsep पार्वणव्रतम् अमावास्यायाम्\eventsep सङ्क्रमण-दिन-पूर्वाह्ण-पुण्यकालः~05:47\RIGHTarrow{}12:05\eventsep वृषभ-रवि-सङ्क्रमण-विष्णुपदी-पुण्यकालः~05:47\RIGHTarrow{}11:10\eventsep वैशाख-अमावास्या (अलभ्यम्–पुष्कला)\eventsep वैशाख-मास-समापनम्\eventsep वैशाख-स्नानपूर्तिः\eventsep शनि-जयन्ती\eventsep शुक-महर्षि-जयन्ती}
{Tue} 
\cfoot{\rygdata{15:14--16:49}{08:56--10:30}{12:05--13:39}}
\caldata{MAY}{16}{\sunmonth{वृषभः}{2}{}{ज्यैष्ठः-(अधिकः)}{ग्रीष्मऋतुः}{बुधः}{विलम्बः}{उत्तरायणम्}{वसन्तऋतुः}}
{\sunmoonrsdata{05:46}{18:23}{06:15}{19:29}{12:05}
{\kalas{04:15 05:01 09:08 08:18 09:59 16:42 10:49 13:21 15:52 17:33 19:09 21:14 22:39 01:30*}}}
{\tnykdata{\anga{\tithi{1}{शुक्ल-प्रथमा}}{\time{21-43}{14:28}}\hspace{1ex}}%
{\anga{कृत्तिका}{\time{7-57}{08:57}}\hspace{1ex}}{चन्द्रराशिः—\mbox{वृषभः}}%
{\anga{अतिगण्डः}{\time{42-2}{22:35}}\hspace{1ex}\uanga{सुकर्म}}%
{\anga{बवः}{\time{21-43}{14:28}}\hspace{1ex}\anga{बालवः}{\time{47-59}{00:58*}}\hspace{1ex}\uanga{कौलवः}}{}
}
{चन्द्र-दर्शनम्~18:23\RIGHTarrow{}19:29\eventsep दर्शेष्टिः\eventsep पार्वण-प्रायश्चित्तावकाशः पौर्णमास्याम्\eventsep स्थालीपाकः}
{Wed} 
\cfoot{\rygdata{12:05--13:39}{07:21--08:56}{10:30--12:05}}
\caldata{MAY}{17}{\sunmonth{वृषभः}{3}{}{ज्यैष्ठः-(अधिकः)}{ग्रीष्मऋतुः}{गुरुः}{विलम्बः}{उत्तरायणम्}{वसन्तऋतुः}}
{\sunmoonrsdata{05:46}{18:24}{07:14}{20:32}{12:05}
{\kalas{04:15 05:01 09:08 08:18 09:59 16:43 10:49 13:21 15:52 17:33 19:09 21:14 22:40 01:30*}}}
{\tnykdata{\anga{\tithi{2}{शुक्ल-द्वितीया}}{\time{14-11}{11:27}}\hspace{1ex}}%
{\anga{रोहिणी}{\time{2-30}{06:47}}\hspace{1ex}\anga{मृगशीर्षम्}{\time{56-53}{04:32*}}\hspace{1ex}}{चन्द्रराशिः—\mbox{वृषभः\RIGHTarrow{17:39}}}%
{\anga{सुकर्म}{\time{33-5}{19:01}}\hspace{1ex}\uanga{धृतिः}}%
{\anga{कौलवः}{\time{14-11}{11:27}}\hspace{1ex}\anga{तैतिलः}{\time{40-22}{21:55}}\hspace{1ex}\uanga{गरः}}{}
}
{}
{Thu} 
\cfoot{\rygdata{13:40--15:14}{05:46--07:21}{08:56--10:30}}
\caldata{MAY}{18}{\sunmonth{वृषभः}{4}{}{ज्यैष्ठः-(अधिकः)}{ग्रीष्मऋतुः}{शुक्रः}{विलम्बः}{उत्तरायणम्}{वसन्तऋतुः}}
{\sunmoonrsdata{05:46}{18:24}{08:16}{21:35}{12:05}
{\kalas{04:15 05:00 09:08 08:18 09:59 16:43 10:49 13:21 15:52 17:33 19:09 21:14 22:40 01:30*}}}
{\tnykdata{\anga{\tithi{3}{शुक्ल-तृतीया}}{\time{6-36}{08:24}}\hspace{1ex}\anga{\tithi{4}{शुक्ल-चतुर्थी}}{\time{59-14}{05:28*}}\hspace{1ex}\avamA{}}%
{\anga{आर्द्रा}{\time{51-29}{02:22*}}\hspace{1ex}}{चन्द्रराशिः—\mbox{मिथुनम्}}%
{\anga{धृतिः}{\time{24-12}{15:27}}\hspace{1ex}\uanga{शूलः}}%
{\anga{गरः}{\time{6-36}{08:24}}\hspace{1ex}\anga{वणिजः}{\time{32-52}{18:55}}\hspace{1ex}\anga{विष्टिः}{\time{59-14}{05:28*}}\hspace{1ex}\uanga{बवः}}{}
}
{शुक्ल-चतुर्थी-व्रतम्}
{Fri} 
\cfoot{\rygdata{10:30--12:05}{15:14--16:49}{07:21--08:55}}
\caldata{MAY}{19}{\sunmonth{वृषभः}{5}{}{ज्यैष्ठः-(अधिकः)}{ग्रीष्मऋतुः}{शनिः}{विलम्बः}{उत्तरायणम्}{वसन्तऋतुः}}
{\sunmoonrsdata{05:46}{18:24}{09:19}{22:33}{12:05}
{\kalas{04:15 05:00 09:08 08:17 09:59 16:43 10:49 13:21 15:52 17:34 19:10 21:14 22:40 01:30*}}}
{\tnykdata{\anga{\tithi{5}{शुक्ल-पञ्चमी}}{\time{52-24}{02:44*}}\hspace{1ex}}%
{\anga{पुनर्वसुः}{\time{46-34}{00:24*}}\hspace{1ex}}{चन्द्रराशिः—\mbox{मिथुनम्\RIGHTarrow{18:52}}}%
{\anga{शूलः}{\time{15-36}{12:00}}\hspace{1ex}\uanga{गण्डः}}%
{\anga{बवः}{\time{25-45}{16:04}}\hspace{1ex}\anga{बालवः}{\time{52-24}{02:44*}}\hspace{1ex}\uanga{कौलवः}}{}
}
{}
{Sat} 
\cfoot{\rygdata{08:55--10:30}{13:40--15:15}{05:46--07:21}}
\caldata{MAY}{20}{\sunmonth{वृषभः}{6}{}{ज्यैष्ठः-(अधिकः)}{ग्रीष्मऋतुः}{भानुः}{विलम्बः}{उत्तरायणम्}{वसन्तऋतुः}}
{\sunmoonrsdata{05:46}{18:24}{10:21}{23:27}{12:05}
{\kalas{04:15 05:00 09:08 08:17 09:59 16:43 10:49 13:21 15:53 17:34 19:10 21:15 22:40 01:30*}}}
{\tnykdata{\anga{\tithi{6}{शुक्ल-षष्ठी}}{\time{46-19}{00:18*}}\hspace{1ex}}%
{\anga{पुष्यः}{\time{42-23}{22:43}}\hspace{1ex}}{चन्द्रराशिः—\mbox{कर्कटः}}%
{\anga{गण्डः}{\time{7-29}{08:46}}\hspace{1ex}\uanga{वृद्धिः}}%
{\anga{कौलवः}{\time{19-16}{13:28}}\hspace{1ex}\anga{तैतिलः}{\time{46-19}{00:18*}}\hspace{1ex}\uanga{गरः}}{}
}
{षष्ठी-व्रतम्\eventsep \tamil{நமிநந்தியடிகள் நாயனார் (26) குருபூஜை}\eventsep रविपुष्ययोग-पुण्यकालः}
{Sun} 
\cfoot{\rygdata{16:50--18:24}{12:05--13:40}{15:15--16:50}}
\caldata{MAY}{21}{\sunmonth{वृषभः}{7}{}{ज्यैष्ठः-(अधिकः)}{ग्रीष्मऋतुः}{सोमः}{विलम्बः}{उत्तरायणम्}{वसन्तऋतुः}}
{\sunmoonrsdata{05:46}{18:25}{11:21}{00:15*}{12:05}
{\kalas{04:15 05:00 09:08 08:17 09:59 16:43 10:49 13:21 15:53 17:34 19:10 21:15 22:40 01:30*}}}
{\tnykdata{\anga{\tithi{7}{शुक्ल-सप्तमी}}{\time{41-7}{22:13}}\hspace{1ex}}%
{\anga{आश्रेषा}{\time{39-3}{21:23}}\hspace{1ex}}{चन्द्रराशिः—\mbox{कर्कटः\RIGHTarrow{21:23}}}%
{\anga{वृद्धिः}{\time{0-2}{05:47}}\hspace{1ex}\anga{ध्रुवः}{\time{53-20}{03:06*}}\hspace{1ex}\uanga{व्याघातः}}%
{\anga{गरः}{\time{13-36}{11:12}}\hspace{1ex}\anga{वणिजः}{\time{41-7}{22:13}}\hspace{1ex}\uanga{विष्टिः}}{}
}
{(सायन) षडशीति-पुण्यकालः~07:44\RIGHTarrow{}18:25\eventsep रवि-सङ्क्रमण-पुण्यकालः~05:45\RIGHTarrow{}14:08\eventsep सायन-सङ्क्रमण-दिन-पूर्वाह्ण-पुण्यकालः~05:45\RIGHTarrow{}12:05\eventsep \tamil{ஸோமாஸிமார நாயனார் (32) குருபூஜை}\eventsep शुक्र-मासः/ग्रीष्मऋतुः~07:44\RIGHTarrow{}}
{Mon} 
\cfoot{\rygdata{07:20--08:55}{10:30--12:05}{13:40--15:15}}
\caldata{MAY}{22}{\sunmonth{वृषभः}{8}{}{ज्यैष्ठः-(अधिकः)}{ग्रीष्मऋतुः}{मङ्गलः}{विलम्बः}{उत्तरायणम्}{वसन्तऋतुः}}
{\sunmoonrsdata{05:45}{18:25}{12:18}{01:01*}{12:05}
{\kalas{04:15 05:00 09:08 08:17 09:59 16:44 10:49 13:21 15:53 17:34 19:10 21:15 22:40 01:30*}}}
{\tnykdata{\anga{\tithi{8}{शुक्ल-अष्टमी}}{\time{36-53}{20:31}}\hspace{1ex}}%
{\anga{मघा}{\time{36-42}{20:26}}\hspace{1ex}}{चन्द्रराशिः—\mbox{सिंहः}}%
{\anga{व्याघातः}{\time{47-25}{00:44*}}\hspace{1ex}\uanga{हर्षणः}}%
{\anga{विष्टिः}{\time{8-53}{09:19}}\hspace{1ex}\anga{बवः}{\time{36-53}{20:31}}\hspace{1ex}\uanga{बालवः}}{}
}
{\tamil{மயிலை~வெள்ளீஶ்வரர்~ப்ரஹ்மோத்ஸவம்}}
{Tue} 
\cfoot{\rygdata{15:15--16:50}{08:55--10:30}{12:05--13:40}}
\caldata{MAY}{23}{\sunmonth{वृषभः}{9}{}{ज्यैष्ठः-(अधिकः)}{ग्रीष्मऋतुः}{बुधः}{विलम्बः}{उत्तरायणम्}{वसन्तऋतुः}}
{\sunmoonrsdata{05:45}{18:25}{13:12}{01:43*}{12:05}
{\kalas{04:15 05:00 09:08 08:17 09:59 16:44 10:49 13:21 15:53 17:35 19:11 21:15 22:40 01:30*}}}
{\tnykdata{\anga{\tithi{9}{शुक्ल-नवमी}}{\time{33-38}{19:13}}\hspace{1ex}}%
{\anga{पूर्वफल्गुनी}{\time{35-19}{19:53}}\hspace{1ex}}{चन्द्रराशिः—\mbox{सिंहः\RIGHTarrow{01:49*}}}%
{\anga{हर्षणः}{\time{42-19}{22:41}}\hspace{1ex}\uanga{वज्रम्}}%
{\anga{बालवः}{\time{5-8}{07:49}}\hspace{1ex}\anga{कौलवः}{\time{33-38}{19:13}}\hspace{1ex}\uanga{तैतिलः}}{}
}
{}
{Wed} 
\cfoot{\rygdata{12:05--13:40}{07:20--08:55}{10:30--12:05}}
\caldata{MAY}{24}{\sunmonth{वृषभः}{10}{}{ज्यैष्ठः-(अधिकः)}{ग्रीष्मऋतुः}{गुरुः}{विलम्बः}{उत्तरायणम्}{वसन्तऋतुः}}
{\sunmoonrsdata{05:45}{18:26}{14:05}{02:25*}{12:05}
{\kalas{04:14 05:00 09:08 08:17 09:59 16:44 10:49 13:21 15:54 17:35 19:11 21:15 22:40 01:30*}}}
{\tnykdata{\anga{\tithi{10}{शुक्ल-दशमी}}{\time{31-22}{18:18}}\hspace{1ex}}%
{\anga{उत्तरफल्गुनी}{\time{34-55}{19:43}}\hspace{1ex}}{चन्द्रराशिः—\mbox{कन्या}}%
{\anga{वज्रम्}{\time{38-0}{20:57}}\hspace{1ex}\uanga{सिद्धिः}}%
{\anga{तैतिलः}{\time{2-22}{06:42}}\hspace{1ex}\anga{गरः}{\time{31-22}{18:18}}\hspace{1ex}\uanga{वणिजः}}{}
}
{}
{Thu} 
\cfoot{\rygdata{13:40--15:15}{05:45--07:20}{08:55--10:30}}
\caldata{MAY}{25}{\sunmonth{वृषभः}{11}{}{ज्यैष्ठः-(अधिकः)}{ग्रीष्मऋतुः}{शुक्रः}{विलम्बः}{उत्तरायणम्}{वसन्तऋतुः}}
{\sunmoonrsdata{05:45}{18:26}{14:56}{03:06*}{12:05}
{\kalas{04:14 05:00 09:08 08:17 09:59 16:44 10:49 13:22 15:54 17:35 19:11 21:16 22:40 01:30*}}}
{\tnykdata{\anga{\tithi{11}{शुक्ल-एकादशी}}{\time{30-5}{17:47}}\hspace{1ex}}%
{\anga{हस्तः}{\time{35-30}{19:57}}\hspace{1ex}}{चन्द्रराशिः—\mbox{कन्या}}%
{\anga{सिद्धिः}{\time{34-30}{19:33}}\hspace{1ex}\uanga{व्यतीपातः}}%
{\anga{वणिजः}{\time{0-36}{06:00}}\hspace{1ex}\anga{विष्टिः}{\time{30-5}{17:47}}\hspace{1ex}\anga{बवः}{\time{59-49}{05:41*}}\hspace{1ex}\uanga{बालवः}}{}
}
{सर्व-पद्मिनी-एकादशी}
{Fri} 
\cfoot{\rygdata{10:30--12:05}{15:16--16:51}{07:20--08:55}}
\caldata{MAY}{26}{\sunmonth{वृषभः}{12}{}{ज्यैष्ठः-(अधिकः)}{ग्रीष्मऋतुः}{शनिः}{विलम्बः}{उत्तरायणम्}{वसन्तऋतुः}}
{\sunmoonrsdata{05:45}{18:26}{15:48}{03:48*}{12:05}
{\kalas{04:14 05:00 09:08 08:17 09:59 16:45 10:49 13:22 15:54 17:35 19:11 21:16 22:41 01:30*}}}
{\tnykdata{\anga{\tithi{12}{शुक्ल-द्वादशी}}{\time{29-48}{17:40}}\hspace{1ex}}%
{\anga{चित्रा}{\time{37-3}{20:34}}\hspace{1ex}}{चन्द्रराशिः—\mbox{कन्या\RIGHTarrow{08:13}}}%
{\anga{व्यतीपातः}{\time{31-46}{18:28}}\hspace{1ex}\uanga{वरीयान्}}%
{\anga{बालवः}{\time{29-48}{17:40}}\hspace{1ex}\uanga{कौलवः}}{}
}
{व्यतीपात-श्राद्धम्\eventsep शनि-प्रदोष-व्रतम्~18:26\RIGHTarrow{}19:51}
{Sat} 
\cfoot{\rygdata{08:55--10:30}{13:41--15:16}{05:45--07:20}}
\caldata{MAY}{27}{\sunmonth{वृषभः}{13}{}{ज्यैष्ठः-(अधिकः)}{ग्रीष्मऋतुः}{भानुः}{विलम्बः}{उत्तरायणम्}{वसन्तऋतुः}}
{\sunmoonrsdata{05:45}{18:26}{16:39}{04:31*}{12:06}
{\kalas{04:14 05:00 09:08 08:17 09:59 16:45 10:50 13:22 15:54 17:36 19:12 21:16 22:41 01:30*}}}
{\tnykdata{\anga{\tithi{13}{शुक्ल-त्रयोदशी}}{\time{30-31}{17:57}}\hspace{1ex}}%
{\anga{स्वाती}{\time{39-36}{21:36}}\hspace{1ex}}{चन्द्रराशिः—\mbox{तुला}}%
{\anga{वरीयान्}{\time{29-51}{17:41}}\hspace{1ex}\uanga{परिघः}}%
{\anga{कौलवः}{\time{0-2}{05:46}}\hspace{1ex}\anga{तैतिलः}{\time{30-31}{17:57}}\hspace{1ex}\uanga{गरः}}{}
}
{}
{Sun} 
\cfoot{\rygdata{16:51--18:26}{12:06--13:41}{15:16--16:51}}
\caldata{MAY}{28}{\sunmonth{वृषभः}{14}{}{ज्यैष्ठः-(अधिकः)}{ग्रीष्मऋतुः}{सोमः}{विलम्बः}{उत्तरायणम्}{वसन्तऋतुः}}
{\sunmoonrsdata{05:45}{18:27}{17:32}{05:17*}{12:06}
{\kalas{04:14 05:00 09:08 08:17 09:59 16:45 10:50 13:22 15:54 17:36 19:12 21:16 22:41 01:31*}}}
{\tnykdata{\anga{\tithi{14}{शुक्ल-चतुर्दशी}}{\time{32-18}{18:40}}\hspace{1ex}}%
{\anga{विशाखा}{\time{43-12}{23:02}}\hspace{1ex}}{चन्द्रराशिः—\mbox{तुला\RIGHTarrow{16:38}}}%
{\anga{परिघः}{\time{28-45}{17:15}}\hspace{1ex}\uanga{शिवः}}%
{\anga{गरः}{\time{1-17}{06:16}}\hspace{1ex}\anga{वणिजः}{\time{32-18}{18:40}}\hspace{1ex}\uanga{विष्टिः}}{}
}
{अग्निनक्षत्र-समापनम्\eventsep कृत्तिकावैषाखोत्सवः\eventsep \tamil{நம்மாழ்வார் திருநக்ஷத்திரம்}\eventsep पञ्च-पर्व-पूजा (पूर्णिमा)\eventsep \tamil{வைகாசி~விஶாகம்}}
{Mon} 
\cfoot{\rygdata{07:20--08:55}{10:31--12:06}{13:41--15:16}}
\caldata{MAY}{29}{\sunmonth{वृषभः}{15}{}{ज्यैष्ठः-(अधिकः)}{ग्रीष्मऋतुः}{मङ्गलः}{विलम्बः}{उत्तरायणम्}{वसन्तऋतुः}}
{\sunmoonrsdata{05:45}{18:27}{18:24}{---}{12:06}
{\kalas{04:14 04:59 09:08 08:17 09:59 16:45 10:50 13:22 15:55 17:36 19:12 21:16 22:41 01:31*}}}
{\tnykdata{\anga{\tithi{15}{पौर्णमासी}}{\time{35-11}{19:49}}\hspace{1ex}}%
{\anga{अनूराधा}{\time{47-51}{00:53*}}\hspace{1ex}}{चन्द्रराशिः—\mbox{वृश्चिकः}}%
{\anga{शिवः}{\time{28-31}{17:09}}\hspace{1ex}\uanga{सिद्धः}}%
{\anga{विष्टिः}{\time{3-36}{07:11}}\hspace{1ex}\anga{बवः}{\time{35-11}{19:49}}\hspace{1ex}\uanga{बालवः}}{}
}
{काञ्ची ६८ जगद्गुरु श्री-चन्द्रशेखरेन्द्र सरस्वती ७ जयन्ती~\#{१२५}\eventsep पार्वणव्रतम् पूर्णिमायाम्\eventsep पूर्णिमा-व्रतम्\eventsep वेङ्कटाचले पूर्णिमा-गरुड-सेवा}
{Tue} 
\cfoot{\rygdata{15:16--16:52}{08:55--10:31}{12:06--13:41}}
\caldata{MAY}{30}{\sunmonth{वृषभः}{16}{}{ज्यैष्ठः-(अधिकः)}{ग्रीष्मऋतुः}{बुधः}{विलम्बः}{उत्तरायणम्}{वसन्तऋतुः}}
{\sunmoonsrdata{05:45}{18:27}{19:16}{06:05}{12:06}
{\kalas{04:14 04:59 09:08 08:17 09:59 16:46 10:50 13:22 15:55 17:36 19:13 21:17 22:41 01:31*}}}
{\tnykdata{\anga{\tithi{16}{कृष्ण-प्रथमा}}{\time{39-9}{21:24}}\hspace{1ex}}%
{\anga{ज्येष्ठा}{\time{53-34}{03:10*}}\hspace{1ex}}{चन्द्रराशिः—\mbox{वृश्चिकः\RIGHTarrow{03:10*}}}%
{\anga{सिद्धः}{\time{29-8}{17:24}}\hspace{1ex}\uanga{साध्यः}}%
{\anga{बालवः}{\time{7-1}{08:33}}\hspace{1ex}\anga{कौलवः}{\time{39-9}{21:24}}\hspace{1ex}\uanga{तैतिलः}}{}
}
{पार्वण-प्रायश्चित्तावकाशः दर्शे\eventsep पूर्णमासेष्टिः\eventsep स्थालीपाकः}
{Wed} 
\cfoot{\rygdata{12:06--13:41}{07:20--08:55}{10:31--12:06}}
\caldata{MAY}{31}{\sunmonth{वृषभः}{17}{}{ज्यैष्ठः-(अधिकः)}{ग्रीष्मऋतुः}{गुरुः}{विलम्बः}{उत्तरायणम्}{वसन्तऋतुः}}
{\sunmoonsrdata{05:45}{18:28}{20:07}{06:54}{12:06}
{\kalas{04:14 04:59 09:08 08:17 09:59 16:46 10:50 13:22 15:55 17:37 19:13 21:17 22:41 01:31*}}}
{\tnykdata{\anga{\tithi{17}{कृष्ण-द्वितीया}}{\time{44-8}{23:24}}\hspace{1ex}}%
{\fullanga{मूला}}{चन्द्रराशिः—\mbox{धनुः}}%
{\anga{साध्यः}{\time{30-33}{17:58}}\hspace{1ex}\uanga{शुभः}}%
{\anga{तैतिलः}{\time{11-31}{10:21}}\hspace{1ex}\anga{गरः}{\time{44-8}{23:24}}\hspace{1ex}\uanga{वणिजः}}{}
}
{काञ्ची जगद्गुरु श्री-शङ्कर विजयेन्द्र सरस्वती आश्रम-स्वीकार-जयन्ती~\#{३६}\eventsep \tamil{முருக நாயனார் (15) குருபூஜை}\eventsep \tamil{திருஞானஸம்பந்தமூர்த்தி நாயனார் (27) குருபூஜை}\eventsep \tamil{திருநீலகண்ட யாழ்ப்பாண நாயனார் (60) குருபூஜை}\eventsep \tamil{திருநீலநக்க நாயனார் (25) குருபூஜை}}
{Thu} 
\cfoot{\rygdata{13:42--15:17}{05:45--07:20}{08:55--10:31}}
\caldata{JUNE}{1}{\sunmonth{वृषभः}{18}{}{ज्यैष्ठः-(अधिकः)}{ग्रीष्मऋतुः}{शुक्रः}{विलम्बः}{उत्तरायणम्}{वसन्तऋतुः}}
{\sunmoonsrdata{05:45}{18:28}{20:55}{07:44}{12:06}
{\kalas{04:14 04:59 09:08 08:17 09:59 16:46 10:50 13:23 15:55 17:37 19:13 21:17 22:42 01:31*}}}
{\tnykdata{\anga{\tithi{18}{कृष्ण-तृतीया}}{\time{49-59}{01:44*}}\hspace{1ex}}%
{\anga{मूला}{\time{0-14}{05:50}}\hspace{1ex}}{चन्द्रराशिः—\mbox{धनुः}}%
{\anga{शुभः}{\time{32-40}{18:49}}\hspace{1ex}\uanga{शुक्लः}}%
{\anga{वणिजः}{\time{16-58}{12:32}}\hspace{1ex}\anga{विष्टिः}{\time{49-59}{01:44*}}\hspace{1ex}\uanga{बवः}}{}
}
{}
{Fri} 
\cfoot{\rygdata{10:31--12:06}{15:17--16:53}{07:20--08:55}}
\caldata{JUNE}{2}{\sunmonth{वृषभः}{19}{}{ज्यैष्ठः-(अधिकः)}{ग्रीष्मऋतुः}{शनिः}{विलम्बः}{उत्तरायणम्}{वसन्तऋतुः}}
{\sunmoonsrdata{05:45}{18:28}{21:41}{08:34}{12:06}
{\kalas{04:14 05:00 09:08 08:17 09:59 16:46 10:50 13:23 15:56 17:37 19:13 21:17 22:42 01:31*}}}
{\tnykdata{\anga{\tithi{19}{कृष्ण-चतुर्थी}}{\time{56-21}{04:17*}}\hspace{1ex}}%
{\anga{पूर्वाषाढा}{\time{7-40}{08:49}}\hspace{1ex}}{चन्द्रराशिः—\mbox{धनुः\RIGHTarrow{15:35}}}%
{\anga{शुक्लः}{\time{35-15}{19:51}}\hspace{1ex}\uanga{ब्रह्म}}%
{\anga{बवः}{\time{23-7}{15:00}}\hspace{1ex}\anga{बालवः}{\time{56-21}{04:17*}}\hspace{1ex}\uanga{कौलवः}}{}
}
{विभुवन-महागणपति सङ्कटहर-चतुर्थी-व्रतम्}
{Sat} 
\cfoot{\rygdata{08:56--10:31}{13:42--15:17}{05:45--07:20}}
\caldata{JUNE}{3}{\sunmonth{वृषभः}{20}{}{ज्यैष्ठः-(अधिकः)}{ग्रीष्मऋतुः}{भानुः}{विलम्बः}{उत्तरायणम्}{वसन्तऋतुः}}
{\sunmoonsrdata{05:45}{18:29}{22:24}{09:24}{12:07}
{\kalas{04:15 05:00 09:08 08:17 09:59 16:47 10:50 13:23 15:56 17:38 19:14 21:18 22:42 01:31*}}}
{\tnykdata{\fulltithi{\tithi{20}{कृष्ण-पञ्चमी}}}%
{\anga{उत्तराषाढा}{\time{15-29}{11:57}}\hspace{1ex}}{चन्द्रराशिः—\mbox{मकरः}}%
{\anga{ब्रह्म}{\time{38-2}{20:57}}\hspace{1ex}\uanga{इन्द्रः}}%
{\anga{कौलवः}{\time{29-36}{17:35}}\hspace{1ex}\uanga{तैतिलः}}{}
}
{}
{Sun} 
\cfoot{\rygdata{16:53--18:29}{12:07--13:42}{15:18--16:53}}
\caldata{JUNE}{4}{\sunmonth{वृषभः}{21}{}{ज्यैष्ठः-(अधिकः)}{ग्रीष्मऋतुः}{सोमः}{विलम्बः}{उत्तरायणम्}{वसन्तऋतुः}}
{\sunmoonsrdata{05:45}{18:29}{23:05}{10:13}{12:07}
{\kalas{04:15 05:00 09:08 08:18 09:59 16:47 10:50 13:23 15:56 17:38 19:14 21:18 22:42 01:31*}}}
{\tnykdata{\anga{\tithi{20}{कृष्ण-पञ्चमी}}{\time{2-48}{06:52}}\hspace{1ex}}%
{\anga{श्रवणः}{\time{23-16}{15:03}}\hspace{1ex}}{चन्द्रराशिः—\mbox{मकरः\RIGHTarrow{04:32*}}}%
{\anga{इन्द्रः}{\time{40-36}{21:59}}\hspace{1ex}\uanga{वैधृतिः}}%
{\anga{तैतिलः}{\time{2-48}{06:52}}\hspace{1ex}\anga{गरः}{\time{35-54}{20:06}}\hspace{1ex}\uanga{वणिजः}}{}
}
{सोमश्रावणी-पुण्यकालः\RIGHTarrow{}15:03\eventsep श्रवण-व्रतम्}
{Mon} 
\cfoot{\rygdata{07:20--08:56}{10:31--12:07}{13:42--15:18}}
\caldata{JUNE}{5}{\sunmonth{वृषभः}{22}{}{ज्यैष्ठः-(अधिकः)}{ग्रीष्मऋतुः}{मङ्गलः}{विलम्बः}{उत्तरायणम्}{वसन्तऋतुः}}
{\sunmoonsrdata{05:45}{18:29}{23:44}{11:00}{12:07}
{\kalas{04:15 05:00 09:09 08:18 10:00 16:47 10:51 13:23 15:56 17:38 19:14 21:18 22:42 01:31*}}}
{\tnykdata{\anga{\tithi{21}{कृष्ण-षष्ठी}}{\time{8-48}{09:16}}\hspace{1ex}}%
{\anga{श्रविष्ठा}{\time{30-24}{17:55}}\hspace{1ex}}{चन्द्रराशिः—\mbox{कुम्भः}}%
{\anga{वैधृतिः}{\time{42-35}{22:47}}\hspace{1ex}\uanga{विष्कम्भः}}%
{\anga{वणिजः}{\time{8-48}{09:16}}\hspace{1ex}\anga{विष्टिः}{\time{41-26}{22:19}}\hspace{1ex}\uanga{बवः}}{}
}
{वैधृति-श्राद्धम्}
{Tue} 
\cfoot{\rygdata{15:18--16:54}{08:56--10:31}{12:07--13:42}}
\caldata{JUNE}{6}{\sunmonth{वृषभः}{23}{}{ज्यैष्ठः-(अधिकः)}{ग्रीष्मऋतुः}{बुधः}{विलम्बः}{उत्तरायणम्}{वसन्तऋतुः}}
{\sunmoonsrdata{05:45}{18:29}{00:22*}{11:48}{12:07}
{\kalas{04:15 05:00 09:09 08:18 10:00 16:48 10:51 13:24 15:57 17:38 19:15 21:18 22:43 01:32*}}}
{\tnykdata{\anga{\tithi{22}{कृष्ण-सप्तमी}}{\time{13-45}{11:15}}\hspace{1ex}}%
{\anga{शतभिषक्}{\time{36-24}{20:18}}\hspace{1ex}}{चन्द्रराशिः—\mbox{कुम्भः}}%
{\anga{विष्कम्भः}{\time{43-33}{23:10}}\hspace{1ex}\uanga{प्रीतिः}}%
{\anga{बवः}{\time{13-45}{11:15}}\hspace{1ex}\anga{बालवः}{\time{45-41}{00:01*}}\hspace{1ex}\uanga{कौलवः}}{}
}
{पञ्च-पर्व-पूजा (अष्टमी)}
{Wed} 
\cfoot{\rygdata{12:07--13:43}{07:20--08:56}{10:32--12:07}}
\caldata{JUNE}{7}{\sunmonth{वृषभः}{24}{}{ज्यैष्ठः-(अधिकः)}{ग्रीष्मऋतुः}{गुरुः}{विलम्बः}{उत्तरायणम्}{वसन्तऋतुः}}
{\sunmoonsrdata{05:45}{18:30}{01:01*}{12:36}{12:07}
{\kalas{04:15 05:00 09:09 08:18 10:00 16:48 10:51 13:24 15:57 17:39 19:15 21:19 22:43 01:32*}}}
{\tnykdata{\anga{\tithi{23}{कृष्ण-अष्टमी}}{\time{17-10}{12:37}}\hspace{1ex}}%
{\anga{पूर्वप्रोष्ठपदा}{\time{40-45}{22:03}}\hspace{1ex}}{चन्द्रराशिः—\mbox{कुम्भः\RIGHTarrow{15:41}}}%
{\anga{प्रीतिः}{\time{43-12}{23:02}}\hspace{1ex}\uanga{आयुष्मान्}}%
{\anga{कौलवः}{\time{17-10}{12:37}}\hspace{1ex}\anga{तैतिलः}{\time{48-10}{01:01*}}\hspace{1ex}\uanga{गरः}}{}
}
{}
{Thu} 
\cfoot{\rygdata{13:43--15:19}{05:45--07:20}{08:56--10:32}}
\caldata{JUNE}{8}{\sunmonth{वृषभः}{25}{}{ज्यैष्ठः-(अधिकः)}{ग्रीष्मऋतुः}{शुक्रः}{विलम्बः}{उत्तरायणम्}{वसन्तऋतुः}}
{\sunmoonsrdata{05:45}{18:30}{01:40*}{13:24}{12:07}
{\kalas{04:15 05:00 09:09 08:18 10:00 16:48 10:51 13:24 15:57 17:39 19:15 21:19 22:43 01:32*}}}
{\tnykdata{\anga{\tithi{24}{कृष्ण-नवमी}}{\time{18-39}{13:13}}\hspace{1ex}}%
{\anga{उत्तरप्रोष्ठपदा}{\time{43-8}{23:00}}\hspace{1ex}}{चन्द्रराशिः—\mbox{मीनः}}%
{\anga{आयुष्मान्}{\time{41-15}{22:15}}\hspace{1ex}\uanga{सौभाग्यः}}%
{\anga{गरः}{\time{18-39}{13:13}}\hspace{1ex}\anga{वणिजः}{\time{48-38}{01:12*}}\hspace{1ex}\uanga{विष्टिः}}{}
}
{}
{Fri} 
\cfoot{\rygdata{10:32--12:07}{15:19--16:54}{07:21--08:56}}
\caldata{JUNE}{9}{\sunmonth{वृषभः}{26}{}{ज्यैष्ठः-(अधिकः)}{ग्रीष्मऋतुः}{शनिः}{विलम्बः}{उत्तरायणम्}{वसन्तऋतुः}}
{\sunmoonsrdata{05:45}{18:30}{02:22*}{14:15}{12:08}
{\kalas{04:15 05:00 09:09 08:18 10:00 16:48 10:51 13:24 15:57 17:39 19:15 21:19 22:43 01:32*}}}
{\tnykdata{\anga{\tithi{25}{कृष्ण-दशमी}}{\time{18-3}{12:59}}\hspace{1ex}}%
{\anga{रेवती}{\time{43-27}{23:08}}\hspace{1ex}}{चन्द्रराशिः—\mbox{मीनः\RIGHTarrow{23:08}}}%
{\anga{सौभाग्यः}{\time{37-37}{20:48}}\hspace{1ex}\uanga{शोभनः}}%
{\anga{विष्टिः}{\time{18-3}{12:59}}\hspace{1ex}\anga{बवः}{\time{46-58}{00:32*}}\hspace{1ex}\uanga{बालवः}}{}
}
{}
{Sat} 
\cfoot{\rygdata{08:56--10:32}{13:43--15:19}{05:45--07:21}}
\caldata{JUNE}{10}{\sunmonth{वृषभः}{27}{}{ज्यैष्ठः-(अधिकः)}{ग्रीष्मऋतुः}{भानुः}{विलम्बः}{उत्तरायणम्}{वसन्तऋतुः}}
{\sunmoonsrdata{05:45}{18:31}{03:09*}{15:10}{12:08}
{\kalas{04:15 05:00 09:09 08:18 10:00 16:49 10:51 13:24 15:58 17:40 19:16 21:19 22:44 01:32*}}}
{\tnykdata{\anga{\tithi{26}{कृष्ण-एकादशी}}{\time{15-22}{11:54}}\hspace{1ex}}%
{\anga{अश्विनी}{\time{41-45}{22:27}}\hspace{1ex}}{चन्द्रराशिः—\mbox{मेषः}}%
{\anga{शोभनः}{\time{32-21}{18:42}}\hspace{1ex}\uanga{अतिगण्डः}}%
{\anga{बालवः}{\time{15-22}{11:54}}\hspace{1ex}\anga{कौलवः}{\time{43-18}{23:04}}\hspace{1ex}\uanga{तैतिलः}}{}
}
{सर्व-परमा-एकादशी}
{Sun} 
\cfoot{\rygdata{16:55--18:31}{12:08--13:44}{15:19--16:55}}
\caldata{JUNE}{11}{\sunmonth{वृषभः}{28}{}{ज्यैष्ठः-(अधिकः)}{ग्रीष्मऋतुः}{सोमः}{विलम्बः}{उत्तरायणम्}{वसन्तऋतुः}}
{\sunmoonsrdata{05:45}{18:31}{04:00*}{16:07}{12:08}
{\kalas{04:15 05:00 09:09 08:18 10:00 16:49 10:52 13:25 15:58 17:40 19:16 21:20 22:44 01:33*}}}
{\tnykdata{\anga{\tithi{27}{कृष्ण-द्वादशी}}{\time{10-45}{10:04}}\hspace{1ex}}%
{\anga{अपभरणी}{\time{38-16}{21:04}}\hspace{1ex}}{चन्द्रराशिः—\mbox{मेषः\RIGHTarrow{02:37*}}}%
{\anga{अतिगण्डः}{\time{25-34}{15:59}}\hspace{1ex}\uanga{सुकर्म}}%
{\anga{तैतिलः}{\time{10-45}{10:04}}\hspace{1ex}\anga{गरः}{\time{37-49}{20:53}}\hspace{1ex}\uanga{वणिजः}}{}
}
{चिदम्बरे ध्वजारोहणम्/पञ्चमूर्ति रथोत्सवः\eventsep \tamil{கழற்சிங்க நாயனார் (51) குருபூஜை}\eventsep सोम-प्रदोष-व्रतम्~18:31\RIGHTarrow{}19:55}
{Mon} 
\cfoot{\rygdata{07:21--08:57}{10:32--12:08}{13:44--15:20}}
\caldata{JUNE}{12}{\sunmonth{वृषभः}{29}{}{ज्यैष्ठः-(अधिकः)}{ग्रीष्मऋतुः}{मङ्गलः}{विलम्बः}{उत्तरायणम्}{वसन्तऋतुः}}
{\sunmoonsrdata{05:45}{18:31}{04:56*}{17:09}{12:08}
{\kalas{04:16 05:00 09:10 08:19 10:01 16:49 10:52 13:25 15:58 17:40 19:16 21:20 22:44 01:33*}}}
{\tnykdata{\anga{\tithi{28}{कृष्ण-त्रयोदशी}}{\time{4-31}{07:34}}\hspace{1ex}\anga{\tithi{29}{कृष्ण-चतुर्दशी}}{\time{57-0}{04:34*}}\hspace{1ex}\avamA{}}%
{\anga{कृत्तिका}{\time{33-19}{19:05}}\hspace{1ex}}{चन्द्रराशिः—\mbox{वृषभः}}%
{\anga{सुकर्म}{\time{17-31}{12:46}}\hspace{1ex}\uanga{धृतिः}}%
{\anga{वणिजः}{\time{4-31}{07:34}}\hspace{1ex}\anga{विष्टिः}{\time{30-54}{18:07}}\hspace{1ex}\anga{शकुनिः}{\time{57-0}{04:34*}}\hspace{1ex}\uanga{चतुष्पात्}}{}
}
{चिदम्बरे रजत चन्द्रप्रभ वाहनम्\eventsep कृत्तिका-व्रतम्\eventsep मासशिवरात्रिः\eventsep पञ्च-पर्व-पूजा (चतुर्दशी)}
{Tue} 
\cfoot{\rygdata{15:20--16:55}{08:57--10:33}{12:08--13:44}}
\caldata{JUNE}{13}{\sunmonth{वृषभः}{30}{}{ज्यैष्ठः-(अधिकः)}{ग्रीष्मऋतुः}{बुधः}{विलम्बः}{उत्तरायणम्}{वसन्तऋतुः}}
{\sunmoonsrdata{05:46}{18:31}{---}{18:13}{12:09}
{\kalas{04:16 05:01 09:10 08:19 10:01 16:49 10:52 13:25 15:58 17:40 19:16 21:20 22:44 01:33*}}}
{\tnykdata{\anga{\tithi{30}{अमावास्या}}{\time{48-38}{01:13*}}\hspace{1ex}}%
{\anga{रोहिणी}{\time{27-19}{16:41}}\hspace{1ex}}{चन्द्रराशिः—\mbox{वृषभः\RIGHTarrow{03:23*}}}%
{\anga{धृतिः}{\time{8-30}{09:10}}\hspace{1ex}\anga{शूलः}{\time{58-50}{05:18*}}\hspace{1ex}\uanga{गण्डः}}%
{\anga{चतुष्पात्}{\time{22-54}{14:55}}\hspace{1ex}\anga{नाग}{\time{48-38}{01:13*}}\hspace{1ex}\uanga{किंस्तुघ्नः}}{}
}
{चिदम्बरे स्वर्ण-सूर्यप्रभ वाहनम्\eventsep पार्वणव्रतम् अमावास्यायाम्\eventsep पञ्च-पर्व-पूजा (अमावास्या)\eventsep सर्व-ज्यैष्ठ-(अधिक)-अमावास्या}
{Wed} 
\cfoot{\rygdata{12:08--13:44}{07:21--08:57}{10:33--12:08}}
\caldata{JUNE}{14}{\sunmonth{वृषभः}{31}{}{ज्यैष्ठः}{ग्रीष्मऋतुः}{गुरुः}{विलम्बः}{उत्तरायणम्}{वसन्तऋतुः}}
{\sunmoonrsdata{05:46}{18:32}{05:58}{19:18}{12:09}
{\kalas{04:16 05:01 09:10 08:19 10:01 16:50 10:52 13:25 15:59 17:41 19:17 21:20 22:44 01:33*}}}
{\tnykdata{\anga{\tithi{1}{शुक्ल-प्रथमा}}{\time{39-49}{21:41}}\hspace{1ex}}%
{\anga{मृगशीर्षम्}{\time{20-42}{14:02}}\hspace{1ex}}{चन्द्रराशिः—\mbox{मिथुनम्}}%
{\anga{गण्डः}{\time{48-52}{01:19*}}\hspace{1ex}\uanga{वृद्धिः}}%
{\anga{किंस्तुघ्नः}{\time{14-15}{11:28}}\hspace{1ex}\anga{बवः}{\time{39-49}{21:41}}\hspace{1ex}\uanga{बालवः}}{}
}
{भद्र-चतुष्टय-व्रतम्\eventsep चिदम्बरे रजत भूत वाहनम्\eventsep दर्शेष्टिः\eventsep करवीर-व्रतम्\eventsep पार्वण-प्रायश्चित्तावकाशः पौर्णमास्याम्\eventsep पुन्नाग-गौरी-व्रतम्\eventsep स्थालीपाकः}
{Thu} 
\cfoot{\rygdata{13:44--15:20}{05:46--07:21}{08:57--10:33}}
\caldata{JUNE}{15}{\sunmonth{मिथुनम्}{1}{\mbox{वृषभः{\tiny\RIGHTarrow}{11:16}}}{ज्यैष्ठः}{ग्रीष्मऋतुः}{शुक्रः}{विलम्बः}{उत्तरायणम्}{ग्रीष्मऋतुः}}
{\sunmoonrsdata{05:46}{18:32}{07:02}{20:20}{12:09}
{\kalas{04:16 05:01 09:10 08:19 10:01 16:50 10:52 13:25 15:59 17:41 19:17 21:21 22:45 01:33*}}}
{\tnykdata{\anga{\tithi{2}{शुक्ल-द्वितीया}}{\time{30-58}{18:09}}\hspace{1ex}}%
{\anga{आर्द्रा}{\time{13-53}{11:19}}\hspace{1ex}}{चन्द्रराशिः—\mbox{मिथुनम्\RIGHTarrow{03:20*}}}%
{\anga{वृद्धिः}{\time{38-57}{21:21}}\hspace{1ex}\uanga{ध्रुवः}}%
{\anga{बालवः}{\time{5-22}{07:55}}\hspace{1ex}\anga{कौलवः}{\time{30-58}{18:09}}\hspace{1ex}\anga{तैतिलः}{\time{56-40}{04:26*}}\hspace{1ex}\uanga{गरः}}{}
}
{चन्द्र-दर्शनम्~18:32\RIGHTarrow{}20:20\eventsep चिदम्बरे रजत ऋषभ वाहनम्\eventsep मिथुन-रवि-सङ्क्रमण-षडशीति-पुण्यकालः~11:16\RIGHTarrow{}18:32\eventsep रवि-सङ्क्रमण-पुण्यकालः~05:46\RIGHTarrow{}17:40\eventsep सङ्क्रमण-दिन-पूर्वाह्ण-पुण्यकालः~05:46\RIGHTarrow{}12:09\eventsep शृङ्गेरी ३२ जगद्गुरु श्री-नृसिंह भारती आराधना}
{Fri} 
\cfoot{\rygdata{10:33--12:09}{15:20--16:56}{07:22--08:57}}
\caldata{JUNE}{16}{\sunmonth{मिथुनम्}{2}{}{ज्यैष्ठः}{ग्रीष्मऋतुः}{शनिः}{विलम्बः}{उत्तरायणम्}{ग्रीष्मऋतुः}}
{\sunmoonrsdata{05:46}{18:32}{08:08}{21:18}{12:09}
{\kalas{04:16 05:01 09:10 08:19 10:01 16:50 10:53 13:26 15:59 17:41 19:17 21:21 22:45 01:34*}}}
{\tnykdata{\anga{\tithi{3}{शुक्ल-तृतीया}}{\time{22-28}{14:46}}\hspace{1ex}}%
{\anga{पुनर्वसुः}{\time{7-19}{08:42}}\hspace{1ex}}{चन्द्रराशिः—\mbox{कर्कटः}}%
{\anga{ध्रुवः}{\time{29-21}{17:31}}\hspace{1ex}\uanga{व्याघातः}}%
{\anga{गरः}{\time{22-28}{14:46}}\hspace{1ex}\anga{वणिजः}{\time{48-28}{01:09*}}\hspace{1ex}\uanga{विष्टिः}}{}
}
{चिदम्बरे रजत-गजवाहनम्\eventsep रम्भा-तृतीया}
{Sat} 
\cfoot{\rygdata{08:58--10:33}{13:45--15:21}{05:46--07:22}}
\caldata{JUNE}{17}{\sunmonth{मिथुनम्}{3}{}{ज्यैष्ठः}{ग्रीष्मऋतुः}{भानुः}{विलम्बः}{उत्तरायणम्}{ग्रीष्मऋतुः}}
{\sunmoonrsdata{05:46}{18:32}{09:11}{22:10}{12:09}
{\kalas{04:16 05:01 09:11 08:19 10:02 16:50 10:53 13:26 15:59 17:41 19:17 21:21 22:45 01:34*}}}
{\tnykdata{\anga{\tithi{4}{शुक्ल-चतुर्थी}}{\time{14-40}{11:38}}\hspace{1ex}}%
{\anga{पुष्यः}{\time{1-20}{06:19}}\hspace{1ex}\anga{आश्रेषा}{\time{56-17}{04:17*}}\hspace{1ex}}{चन्द्रराशिः—\mbox{कर्कटः\RIGHTarrow{04:17*}}}%
{\anga{व्याघातः}{\time{20-23}{13:56}}\hspace{1ex}\uanga{हर्षणः}}%
{\anga{विष्टिः}{\time{14-40}{11:38}}\hspace{1ex}\anga{बवः}{\time{41-8}{22:14}}\hspace{1ex}\uanga{बालवः}}{}
}
{चिदम्बरे कैलास वाहनम्\eventsep कदली-गौरी-व्रतम्/पूजा\eventsep रविपुष्ययोग-पुण्यकालः\RIGHTarrow{}06:18\eventsep उमा-अवतारः\eventsep शुक्ल-चतुर्थी-व्रतम्}
{Sun} 
\cfoot{\rygdata{16:57--18:32}{12:09--13:45}{15:21--16:57}}
\caldata{JUNE}{18}{\sunmonth{मिथुनम्}{4}{}{ज्यैष्ठः}{ग्रीष्मऋतुः}{सोमः}{विलम्बः}{उत्तरायणम्}{ग्रीष्मऋतुः}}
{\sunmoonrsdata{05:46}{18:33}{10:11}{22:58}{12:10}
{\kalas{04:17 05:01 09:11 08:20 10:02 16:51 10:53 13:26 15:59 17:42 19:18 21:21 22:45 01:34*}}}
{\tnykdata{\anga{\tithi{5}{शुक्ल-पञ्चमी}}{\time{7-51}{08:55}}\hspace{1ex}}%
{\anga{मघा}{\time{52-25}{02:45*}}\hspace{1ex}}{चन्द्रराशिः—\mbox{सिंहः}}%
{\anga{हर्षणः}{\time{12-15}{10:41}}\hspace{1ex}\uanga{वज्रम्}}%
{\anga{बालवः}{\time{7-51}{08:55}}\hspace{1ex}\anga{कौलवः}{\time{34-53}{19:44}}\hspace{1ex}\uanga{तैतिलः}}{}
}
{षष्ठी-व्रतम्\eventsep चिदम्बरे भिक्षाटन स्वर्णरथः\eventsep काञ्ची ५० जगद्गुरु श्री-चन्द्रचूडेन्द्र सरस्वती २ आराधना~\#{७२२}\eventsep काञ्ची ६ जगद्गुरु श्री-शुद्धानन्देन्द्र सरस्वती आराधना~\#{२१४२}\eventsep श्रीनिवासमङ्गापुरे साक्षात्कार-वैभवोत्सव-आरम्भः}
{Mon} 
\cfoot{\rygdata{07:22--08:58}{10:34--12:10}{13:45--15:21}}
\caldata{JUNE}{19}{\sunmonth{मिथुनम्}{5}{}{ज्यैष्ठः}{ग्रीष्मऋतुः}{मङ्गलः}{विलम्बः}{उत्तरायणम्}{ग्रीष्मऋतुः}}
{\sunmoonrsdata{05:47}{18:33}{11:07}{23:42}{12:10}
{\kalas{04:17 05:02 09:11 08:20 10:02 16:51 10:53 13:26 16:00 17:42 19:18 21:21 22:46 01:34*}}}
{\tnykdata{\anga{\tithi{6}{शुक्ल-षष्ठी}}{\time{2-14}{06:40}}\hspace{1ex}\anga{\tithi{7}{शुक्ल-सप्तमी}}{\time{57-59}{04:58*}}\hspace{1ex}\avamA{}}%
{\anga{पूर्वफल्गुनी}{\time{49-52}{01:44*}}\hspace{1ex}}{चन्द्रराशिः—\mbox{सिंहः}}%
{\anga{वज्रम्}{\time{5-8}{07:50}}\hspace{1ex}\anga{सिद्धिः}{\time{59-9}{05:26*}}\hspace{1ex}\uanga{व्यतीपातः}}%
{\anga{तैतिलः}{\time{2-14}{06:40}}\hspace{1ex}\anga{गरः}{\time{29-56}{17:45}}\hspace{1ex}\anga{वणिजः}{\time{57-59}{04:58*}}\hspace{1ex}\uanga{विष्टिः}}{}
}
{आरण्य-गौरी-व्रतम्\eventsep \tamil{அமரநீதி நாயனார் (6) குருபூஜை}\eventsep चिदम्बरे रथोत्सवः\eventsep काञ्ची २३ जगद्गुरु श्री-सच्चित्सुखेन्द्र सरस्वती आराधना~\#{१५०७}\eventsep वरुण-पूजा\eventsep विन्ध्यावासिनी-देवी-पूजा\eventsep श्रीनिवासमङ्गापुरे साक्षात्कार-वैभवोत्सवः}
{Tue} 
\cfoot{\rygdata{15:21--16:57}{08:58--10:34}{12:10--13:46}}
\caldata{JUNE}{20}{\sunmonth{मिथुनम्}{6}{}{ज्यैष्ठः}{ग्रीष्मऋतुः}{बुधः}{विलम्बः}{उत्तरायणम्}{ग्रीष्मऋतुः}}
{\sunmoonrsdata{05:47}{18:33}{12:01}{00:25*}{12:10}
{\kalas{04:17 05:02 09:11 08:20 10:02 16:51 10:53 13:27 16:00 17:42 19:18 21:22 22:46 01:34*}}}
{\tnykdata{\anga{\tithi{8}{शुक्ल-अष्टमी}}{\time{55-10}{03:51*}}\hspace{1ex}}%
{\anga{उत्तरफल्गुनी}{\time{48-45}{01:17*}}\hspace{1ex}}{चन्द्रराशिः—\mbox{सिंहः\RIGHTarrow{07:34}}}%
{\anga{व्यतीपातः}{\time{54-19}{03:31*}}\hspace{1ex}\uanga{वरीयान्}}%
{\anga{विष्टिः}{\time{26-24}{16:20}}\hspace{1ex}\anga{बवः}{\time{55-10}{03:51*}}\hspace{1ex}\uanga{बालवः}}{}
}
{बुधाष्टमी\eventsep चिदम्बरे नटराजस्य राजसभायां महाभिषेकः\eventsep धूमावती-जयन्ती\eventsep ज्येष्ठाष्टमी\eventsep काञ्ची १६ जगद्गुरु श्री-उज्ज्वल शङ्करेन्द्र सरस्वती आराधना~\#{१६५२}\eventsep \tamil{நடராஜர் ஆனி திருமஞ்சனம்}\eventsep व्यतीपात-श्राद्धम्\eventsep श्रीनिवासमङ्गापुरे साक्षात्कार-वैभवोत्सव-समापनम्}
{Wed} 
\cfoot{\rygdata{12:10--13:46}{07:23--08:58}{10:34--12:10}}
\caldata{JUNE}{21}{\sunmonth{मिथुनम्}{7}{}{ज्यैष्ठः}{ग्रीष्मऋतुः}{गुरुः}{विलम्बः}{उत्तरायणम्}{ग्रीष्मऋतुः}}
{\sunmoonrsdata{05:47}{18:33}{12:53}{01:06*}{12:10}
{\kalas{04:17 05:02 09:11 08:20 10:02 16:51 10:54 13:27 16:00 17:42 19:18 21:22 22:46 01:35*}}}
{\tnykdata{\anga{\tithi{9}{शुक्ल-नवमी}}{\time{53-48}{03:18*}}\hspace{1ex}}%
{\anga{हस्तः}{\time{49-4}{01:25*}}\hspace{1ex}}{चन्द्रराशिः—\mbox{कन्या}}%
{\anga{वरीयान्}{\time{50-40}{02:03*}}\hspace{1ex}\uanga{परिघः}}%
{\anga{बालवः}{\time{24-18}{15:30}}\hspace{1ex}\anga{कौलवः}{\time{53-48}{03:18*}}\hspace{1ex}\uanga{तैतिलः}}{}
}
{(सायन) दक्षिणायन-पुण्यकालः~05:47\RIGHTarrow{}15:37\eventsep ब्रह्माणी-पूजा\eventsep चिदम्बरे मुत्तुप्पल्लक्कु\eventsep दक्षिणायनारम्भः\eventsep काञ्ची ६१ जगद्गुरु श्री-महादेवेन्द्र सरस्वती ४ आराधना~\#{२७३}\eventsep महेश-नवमी\eventsep रवि-सङ्क्रमण-पुण्यकालः~09:13\RIGHTarrow{}18:33\eventsep सायन-सङ्क्रमण-दिन-अपराह्ण-पुण्यकालः~12:10\RIGHTarrow{}18:33\eventsep शुचि-मासः/दक्षिणायनम्~15:37\RIGHTarrow{}\eventsep शुक्ल-देवी-पूजा}
{Thu} 
\cfoot{\rygdata{13:46--15:22}{05:47--07:23}{08:59--10:34}}
\caldata{JUNE}{22}{\sunmonth{मिथुनम्}{8}{}{ज्यैष्ठः}{ग्रीष्मऋतुः}{शुक्रः}{विलम्बः}{उत्तरायणम्}{ग्रीष्मऋतुः}}
{\sunmoonrsdata{05:47}{18:34}{13:45}{01:48*}{12:10}
{\kalas{04:17 05:02 09:12 08:20 10:03 16:51 10:54 13:27 16:00 17:43 19:19 21:22 22:46 01:35*}}}
{\tnykdata{\anga{\tithi{10}{शुक्ल-दशमी}}{\time{53-50}{03:20*}}\hspace{1ex}}%
{\anga{चित्रा}{\time{50-46}{02:06*}}\hspace{1ex}}{चन्द्रराशिः—\mbox{कन्या\RIGHTarrow{13:41}}}%
{\anga{परिघः}{\time{48-8}{01:03*}}\hspace{1ex}\uanga{शिवः}}%
{\anga{तैतिलः}{\time{23-39}{15:15}}\hspace{1ex}\anga{गरः}{\time{53-50}{03:20*}}\hspace{1ex}\uanga{वणिजः}}{}
}
{दशहरा/गङ्गावतरणम्/दशपापहरा-दशमी\eventsep देवी-पर्व-३\eventsep काञ्ची १७ जगद्गुरु श्री-सदाशिवेन्द्र सरस्वती आराधना~\#{१६४४}\eventsep काञ्ची ५३ जगद्गुरु श्री-पूर्णानन्द सदाशिवेन्द्र सरस्वती आराधना~\#{५२१}\eventsep सुदर्शन-जयन्ती}
{Fri} 
\cfoot{\rygdata{10:35--12:10}{15:22--16:58}{07:23--08:59}}
\caldata{JUNE}{23}{\sunmonth{मिथुनम्}{9}{}{ज्यैष्ठः}{ग्रीष्मऋतुः}{शनिः}{विलम्बः}{उत्तरायणम्}{ग्रीष्मऋतुः}}
{\sunmoonrsdata{05:47}{18:34}{14:36}{02:30*}{12:11}
{\kalas{04:18 05:02 09:12 08:21 10:03 16:52 10:54 13:27 16:01 17:43 19:19 21:22 22:46 01:35*}}}
{\tnykdata{\anga{\tithi{11}{शुक्ल-एकादशी}}{\time{55-12}{03:52*}}\hspace{1ex}}%
{\anga{स्वाती}{\time{53-46}{03:18*}}\hspace{1ex}}{चन्द्रराशिः—\mbox{तुला}}%
{\anga{शिवः}{\time{46-39}{00:27*}}\hspace{1ex}\uanga{सिद्धः}}%
{\anga{वणिजः}{\time{24-21}{15:32}}\hspace{1ex}\anga{विष्टिः}{\time{55-12}{03:52*}}\hspace{1ex}\uanga{बवः}}{}
}
{\tamil{பெரியாழ்வார் திருநக்ஷத்திரம்}\eventsep सर्व-पाण्डव-निर्जला-एकादशी}
{Sat} 
\cfoot{\rygdata{08:59--10:35}{13:46--15:22}{05:47--07:23}}
\caldata{JUNE}{24}{\sunmonth{मिथुनम्}{10}{}{ज्यैष्ठः}{ग्रीष्मऋतुः}{भानुः}{विलम्बः}{उत्तरायणम्}{ग्रीष्मऋतुः}}
{\sunmoonrsdata{05:48}{18:34}{15:28}{03:15*}{12:11}
{\kalas{04:18 05:03 09:12 08:21 10:03 16:52 10:54 13:27 16:01 17:43 19:19 21:23 22:47 01:35*}}}
{\tnykdata{\anga{\tithi{12}{शुक्ल-द्वादशी}}{\time{57-45}{04:54*}}\hspace{1ex}}%
{\anga{विशाखा}{\time{57-57}{04:59*}}\hspace{1ex}}{चन्द्रराशिः—\mbox{तुला\RIGHTarrow{22:31}}}%
{\anga{सिद्धः}{\time{46-7}{00:15*}}\hspace{1ex}\uanga{साध्यः}}%
{\anga{बवः}{\time{26-20}{16:20}}\hspace{1ex}\anga{बालवः}{\time{57-45}{04:54*}}\hspace{1ex}\uanga{कौलवः}}{}
}
{अलर्मेल्मङ्गापुरे प्लवोत्सव-प्रारम्भः\eventsep चम्पक-द्वादशी\eventsep गवामयन-द्वादशी\eventsep हरिवासरः\RIGHTarrow{}10:05\eventsep काञ्ची २ जगद्गुरु श्री-सुरेश्वराचार्य आराधना~\#{२४२४}\eventsep रामलक्ष्मण-द्वादशी}
{Sun} 
\cfoot{\rygdata{16:58--18:34}{12:11--13:47}{15:22--16:58}}
\caldata{JUNE}{25}{\sunmonth{मिथुनम्}{11}{}{ज्यैष्ठः}{ग्रीष्मऋतुः}{सोमः}{विलम्बः}{उत्तरायणम्}{ग्रीष्मऋतुः}}
{\sunmoonrsdata{05:48}{18:34}{16:20}{04:01*}{12:11}
{\kalas{04:18 05:03 09:12 08:21 10:03 16:52 10:54 13:28 16:01 17:43 19:19 21:23 22:47 01:35*}}}
{\tnykdata{\fulltithi{\tithi{13}{शुक्ल-त्रयोदशी}}}%
{\fullanga{अनूराधा}}{चन्द्रराशिः—\mbox{वृश्चिकः}}%
{\anga{साध्यः}{\time{46-28}{00:23*}}\hspace{1ex}\uanga{शुभः}}%
{\anga{कौलवः}{\time{29-27}{17:35}}\hspace{1ex}\uanga{तैतिलः}}{}
}
{अलर्मेल्मङ्गापुरे प्लवोत्सवः\eventsep छत्रपति-शिवाजी-राज्याभिषेकः~\#{३४५}\eventsep दुर्गन्ध-दौर्भाग्य-नाशक-त्रयोदशी\eventsep सोम-प्रदोष-व्रतम्~18:34\RIGHTarrow{}19:58\eventsep विद्यारण्य-स्वामि-आराधना~\#{६२७}}
{Mon} 
\cfoot{\rygdata{07:24--09:00}{10:35--12:11}{13:47--15:23}}
\caldata{JUNE}{26}{\sunmonth{मिथुनम्}{12}{}{ज्यैष्ठः}{ग्रीष्मऋतुः}{मङ्गलः}{विलम्बः}{उत्तरायणम्}{ग्रीष्मऋतुः}}
{\sunmoonrsdata{05:48}{18:34}{17:11}{04:50*}{12:11}
{\kalas{04:18 05:03 09:12 08:21 10:04 16:52 10:55 13:28 16:01 17:43 19:19 21:23 22:47 01:36*}}}
{\tnykdata{\anga{\tithi{13}{शुक्ल-त्रयोदशी}}{\time{1-24}{06:22}}\hspace{1ex}}%
{\anga{अनूराधा}{\time{3-11}{07:05}}\hspace{1ex}}{चन्द्रराशिः—\mbox{वृश्चिकः}}%
{\anga{शुभः}{\time{47-34}{00:50*}}\hspace{1ex}\uanga{शुक्लः}}%
{\anga{तैतिलः}{\time{1-24}{06:22}}\hspace{1ex}\anga{गरः}{\time{33-36}{19:15}}\hspace{1ex}\uanga{वणिजः}}{}
}
{अलर्मेल्मङ्गापुरे प्लवोत्सवः\eventsep वेङ्कटाचले ज्येष्ठ-अभिद्येयकाभिषेकः (वज्र-कवचम्)}
{Tue} 
\cfoot{\rygdata{15:23--16:59}{09:00--10:36}{12:11--13:47}}
\caldata{JUNE}{27}{\sunmonth{मिथुनम्}{13}{}{ज्यैष्ठः}{ग्रीष्मऋतुः}{बुधः}{विलम्बः}{उत्तरायणम्}{ग्रीष्मऋतुः}}
{\sunmoonrsdata{05:48}{18:34}{18:02}{05:39*}{12:11}
{\kalas{04:19 05:03 09:13 08:22 10:04 16:52 10:55 13:28 16:01 17:44 19:19 21:23 22:47 01:36*}}}
{\tnykdata{\anga{\tithi{14}{शुक्ल-चतुर्दशी}}{\time{6-0}{08:13}}\hspace{1ex}}%
{\anga{ज्येष्ठा}{\time{9-20}{09:33}}\hspace{1ex}}{चन्द्रराशिः—\mbox{वृश्चिकः\RIGHTarrow{09:33}}}%
{\anga{शुक्लः}{\time{49-19}{01:32*}}\hspace{1ex}\uanga{ब्रह्म}}%
{\anga{वणिजः}{\time{6-0}{08:13}}\hspace{1ex}\anga{विष्टिः}{\time{38-37}{21:15}}\hspace{1ex}\uanga{बवः}}{}
}
{अलर्मेल्मङ्गापुरे प्लवोत्सवः\eventsep ज्येष्ठाभिषेकः\eventsep मन्वादिः-(भौत्यः-[१४])\eventsep पार्वणव्रतम् पूर्णिमायाम्\eventsep पञ्च-पर्व-पूजा (पूर्णिमा)\eventsep वेङ्कटाचले ज्येष्ठ-अभिद्येयकाभिषेकः (मुत्यल-कवचम्)\eventsep वेङ्कटाचले पूर्णिमा-गरुड-सेवा}
{Wed} 
\cfoot{\rygdata{12:11--13:47}{07:24--09:00}{10:36--12:11}}
\caldata{JUNE}{28}{\sunmonth{मिथुनम्}{14}{}{ज्यैष्ठः}{ग्रीष्मऋतुः}{गुरुः}{विलम्बः}{उत्तरायणम्}{ग्रीष्मऋतुः}}
{\sunmoonrsdata{05:49}{18:35}{18:51}{---}{12:12}
{\kalas{04:19 05:04 09:13 08:22 10:04 16:53 10:55 13:28 16:01 17:44 19:20 21:23 22:48 01:36*}}}
{\tnykdata{\anga{\tithi{15}{पौर्णमासी}}{\time{11-24}{10:22}}\hspace{1ex}}%
{\anga{मूला}{\time{16-15}{12:19}}\hspace{1ex}}{चन्द्रराशिः—\mbox{धनुः}}%
{\anga{ब्रह्म}{\time{51-36}{02:27*}}\hspace{1ex}\uanga{इन्द्रः}}%
{\anga{बवः}{\time{11-24}{10:22}}\hspace{1ex}\anga{बालवः}{\time{44-21}{23:33}}\hspace{1ex}\uanga{कौलवः}}{}
}
{अलर्मेल्मङ्गापुरे प्लवोत्सव-समापनम्\eventsep ऎरुवक-पूर्णिमा\eventsep कबीरदास-जयन्ती\eventsep पार्वण-प्रायश्चित्तावकाशः दर्शे\eventsep पूर्णमासेष्टिः\eventsep पूर्णिमा-व्रतम्\eventsep स्थालीपाकः\eventsep वेङ्कटाचले ज्येष्ठ-अभिद्येयकाभिषेकः (स्वर्ण-कवचम्)\eventsep वट-पूर्णिमा/वट-सावित्री-व्रतम्}
{Thu} 
\cfoot{\rygdata{13:48--15:23}{05:49--07:24}{09:00--10:36}}
\caldata{JUNE}{29}{\sunmonth{मिथुनम्}{15}{}{ज्यैष्ठः}{ग्रीष्मऋतुः}{शुक्रः}{विलम्बः}{उत्तरायणम्}{ग्रीष्मऋतुः}}
{\sunmoonsrdata{05:49}{18:35}{19:38}{06:29}{12:12}
{\kalas{04:19 05:04 09:13 08:22 10:04 16:53 10:55 13:28 16:02 17:44 19:20 21:23 22:48 01:36*}}}
{\tnykdata{\anga{\tithi{16}{कृष्ण-प्रथमा}}{\time{17-25}{12:47}}\hspace{1ex}}%
{\anga{पूर्वाषाढा}{\time{23-45}{15:19}}\hspace{1ex}}{चन्द्रराशिः—\mbox{धनुः\RIGHTarrow{22:05}}}%
{\anga{इन्द्रः}{\time{54-13}{03:30*}}\hspace{1ex}\uanga{वैधृतिः}}%
{\anga{कौलवः}{\time{17-25}{12:47}}\hspace{1ex}\anga{तैतिलः}{\time{50-35}{02:03*}}\hspace{1ex}\uanga{गरः}}{}
}
{}
{Fri} 
\cfoot{\rygdata{10:36--12:12}{15:23--16:59}{07:25--09:00}}
\caldata{JUNE}{30}{\sunmonth{मिथुनम्}{16}{}{ज्यैष्ठः}{ग्रीष्मऋतुः}{शनिः}{विलम्बः}{उत्तरायणम्}{ग्रीष्मऋतुः}}
{\sunmoonsrdata{05:49}{18:35}{20:22}{07:19}{12:12}
{\kalas{04:19 05:04 09:13 08:22 10:04 16:53 10:56 13:29 16:02 17:44 19:20 21:24 22:48 01:37*}}}
{\tnykdata{\anga{\tithi{17}{कृष्ण-द्वितीया}}{\time{23-47}{15:20}}\hspace{1ex}}%
{\anga{उत्तराषाढा}{\time{31-33}{18:27}}\hspace{1ex}}{चन्द्रराशिः—\mbox{मकरः}}%
{\anga{वैधृतिः}{\time{57-0}{04:37*}}\hspace{1ex}\uanga{विष्कम्भः}}%
{\anga{गरः}{\time{23-47}{15:20}}\hspace{1ex}\anga{वणिजः}{\time{57-1}{04:38*}}\hspace{1ex}\uanga{विष्टिः}}{}
}
{वैधृति-श्राद्धम्}
{Sat} 
\cfoot{\rygdata{09:01--10:36}{13:48--15:24}{05:49--07:25}}
\caldata{JULY}{1}{\sunmonth{मिथुनम्}{17}{}{ज्यैष्ठः}{ग्रीष्मऋतुः}{भानुः}{विलम्बः}{उत्तरायणम्}{ग्रीष्मऋतुः}}
{\sunmoonsrdata{05:49}{18:35}{21:03}{08:08}{12:12}
{\kalas{04:20 05:04 09:14 08:23 10:05 16:53 10:56 13:29 16:02 17:44 19:20 21:24 22:48 01:37*}}}
{\tnykdata{\anga{\tithi{18}{कृष्ण-तृतीया}}{\time{30-12}{17:54}}\hspace{1ex}}%
{\anga{श्रवणः}{\time{39-22}{21:34}}\hspace{1ex}}{चन्द्रराशिः—\mbox{मकरः}}%
{\anga{विष्कम्भः}{\time{59-40}{05:42*}}\hspace{1ex}\uanga{प्रीतिः}}%
{\anga{विष्टिः}{\time{30-12}{17:54}}\hspace{1ex}\uanga{बवः}}{}
}
{कृष्णपिङ्गल-महागणपति सङ्कटहर-चतुर्थी-व्रतम्\eventsep श्रवण-व्रतम्}
{Sun} 
\cfoot{\rygdata{16:59--18:35}{12:12--13:48}{15:24--16:59}}
\caldata{JULY}{2}{\sunmonth{मिथुनम्}{18}{}{ज्यैष्ठः}{ग्रीष्मऋतुः}{सोमः}{विलम्बः}{उत्तरायणम्}{ग्रीष्मऋतुः}}
{\sunmoonsrdata{05:50}{18:35}{21:43}{08:56}{12:12}
{\kalas{04:20 05:05 09:14 08:23 10:05 16:53 10:56 13:29 16:02 17:44 19:20 21:24 22:48 01:37*}}}
{\tnykdata{\anga{\tithi{19}{कृष्ण-चतुर्थी}}{\time{36-16}{20:20}}\hspace{1ex}}%
{\anga{श्रविष्ठा}{\time{46-48}{00:33*}}\hspace{1ex}}{चन्द्रराशिः—\mbox{मकरः\RIGHTarrow{11:05}}}%
{\fullanga{प्रीतिः}}%
{\anga{बवः}{\time{3-18}{07:09}}\hspace{1ex}\anga{बालवः}{\time{36-16}{20:20}}\hspace{1ex}\uanga{कौलवः}}{}
}
{}
{Mon} 
\cfoot{\rygdata{07:25--09:01}{10:37--12:12}{13:48--15:24}}
\caldata{JULY}{3}{\sunmonth{मिथुनम्}{19}{}{ज्यैष्ठः}{ग्रीष्मऋतुः}{मङ्गलः}{विलम्बः}{उत्तरायणम्}{ग्रीष्मऋतुः}}
{\sunmoonsrdata{05:50}{18:35}{22:21}{09:44}{12:13}
{\kalas{04:20 05:05 09:14 08:23 10:05 16:53 10:56 13:29 16:02 17:44 19:20 21:24 22:48 01:37*}}}
{\tnykdata{\anga{\tithi{20}{कृष्ण-पञ्चमी}}{\time{41-34}{22:28}}\hspace{1ex}}%
{\anga{शतभिषक्}{\time{53-24}{03:12*}}\hspace{1ex}}{चन्द्रराशिः—\mbox{कुम्भः}}%
{\anga{प्रीतिः}{\time{1-55}{06:36}}\hspace{1ex}\uanga{आयुष्मान्}}%
{\anga{कौलवः}{\time{9-2}{09:27}}\hspace{1ex}\anga{तैतिलः}{\time{41-34}{22:28}}\hspace{1ex}\uanga{गरः}}{}
}
{}
{Tue} 
\cfoot{\rygdata{15:24--17:00}{09:01--10:37}{12:13--13:48}}
\caldata{JULY}{4}{\sunmonth{मिथुनम्}{20}{}{ज्यैष्ठः}{ग्रीष्मऋतुः}{बुधः}{विलम्बः}{उत्तरायणम्}{ग्रीष्मऋतुः}}
{\sunmoonsrdata{05:50}{18:35}{22:58}{10:30}{12:13}
{\kalas{04:20 05:05 09:14 08:23 10:05 16:53 10:56 13:29 16:02 17:44 19:20 21:24 22:49 01:37*}}}
{\tnykdata{\anga{\tithi{21}{कृष्ण-षष्ठी}}{\time{45-39}{00:06*}}\hspace{1ex}}%
{\anga{पूर्वप्रोष्ठपदा}{\time{58-46}{05:21*}}\hspace{1ex}}{चन्द्रराशिः—\mbox{कुम्भः\RIGHTarrow{22:52}}}%
{\anga{आयुष्मान्}{\time{3-28}{07:14}}\hspace{1ex}\uanga{सौभाग्यः}}%
{\anga{गरः}{\time{13-47}{11:21}}\hspace{1ex}\anga{वणिजः}{\time{45-39}{00:06*}}\hspace{1ex}\uanga{विष्टिः}}{}
}
{}
{Wed} 
\cfoot{\rygdata{12:13--13:49}{07:26--09:02}{10:37--12:13}}
\caldata{JULY}{5}{\sunmonth{मिथुनम्}{21}{}{ज्यैष्ठः}{ग्रीष्मऋतुः}{गुरुः}{विलम्बः}{उत्तरायणम्}{ग्रीष्मऋतुः}}
{\sunmoonsrdata{05:51}{18:36}{23:37}{11:18}{12:13}
{\kalas{04:21 05:06 09:14 08:24 10:05 16:53 10:56 13:30 16:02 17:45 19:20 21:24 22:49 01:38*}}}
{\tnykdata{\anga{\tithi{22}{कृष्ण-सप्तमी}}{\time{48-10}{01:07*}}\hspace{1ex}}%
{\fullanga{उत्तरप्रोष्ठपदा}}{चन्द्रराशिः—\mbox{मीनः}}%
{\anga{सौभाग्यः}{\time{3-57}{07:26}}\hspace{1ex}\uanga{शोभनः}}%
{\anga{विष्टिः}{\time{17-7}{12:42}}\hspace{1ex}\anga{बवः}{\time{48-10}{01:07*}}\hspace{1ex}\uanga{बालवः}}{}
}
{}
{Thu} 
\cfoot{\rygdata{13:49--15:24}{05:51--07:26}{09:02--10:37}}
\caldata{JULY}{6}{\sunmonth{मिथुनम्}{22}{}{ज्यैष्ठः}{ग्रीष्मऋतुः}{शुक्रः}{विलम्बः}{उत्तरायणम्}{ग्रीष्मऋतुः}}
{\sunmoonsrdata{05:51}{18:36}{00:16*}{12:07}{12:13}
{\kalas{04:21 05:06 09:15 08:24 10:06 16:54 10:57 13:30 16:03 17:45 19:21 21:24 22:49 01:38*}}}
{\tnykdata{\anga{\tithi{23}{कृष्ण-अष्टमी}}{\time{48-48}{01:22*}}\hspace{1ex}}%
{\anga{उत्तरप्रोष्ठपदा}{\time{2-31}{06:51}}\hspace{1ex}}{चन्द्रराशिः—\mbox{मीनः}}%
{\anga{शोभनः}{\time{3-9}{07:07}}\hspace{1ex}\uanga{अतिगण्डः}}%
{\anga{बालवः}{\time{18-43}{13:20}}\hspace{1ex}\anga{कौलवः}{\time{48-48}{01:22*}}\hspace{1ex}\uanga{तैतिलः}}{}
}
{\tamil{ஏயர்கோன் கலிக்காம நாயனார் (28) குருபூஜை}\eventsep भृगुरेवती-पुण्यकालः~06:51\RIGHTarrow{}\eventsep पञ्च-पर्व-पूजा (अष्टमी)\eventsep तिन्दुकाष्टमी\eventsep त्रिलोचनाष्टमी\eventsep विनायकाष्टमी\eventsep शीतलाष्टमी}
{Fri} 
\cfoot{\rygdata{10:38--12:13}{15:24--17:00}{07:26--09:02}}
\caldata{JULY}{7}{\sunmonth{मिथुनम्}{23}{}{ज्यैष्ठः}{ग्रीष्मऋतुः}{शनिः}{विलम्बः}{उत्तरायणम्}{ग्रीष्मऋतुः}}
{\sunmoonsrdata{05:51}{18:36}{00:59*}{12:58}{12:13}
{\kalas{04:21 05:06 09:15 08:24 10:06 16:54 10:57 13:30 16:03 17:45 19:21 21:25 22:49 01:38*}}}
{\tnykdata{\anga{\tithi{24}{कृष्ण-नवमी}}{\time{47-27}{00:50*}}\hspace{1ex}}%
{\anga{रेवती}{\time{4-26}{07:38}}\hspace{1ex}}{चन्द्रराशिः—\mbox{मीनः\RIGHTarrow{07:38}}}%
{\anga{अतिगण्डः}{\time{0-50}{06:11}}\hspace{1ex}\anga{सुकर्म}{\time{56-57}{04:38*}}\hspace{1ex}\uanga{धृतिः}}%
{\anga{तैतिलः}{\time{18-22}{13:12}}\hspace{1ex}\anga{गरः}{\time{47-27}{00:50*}}\hspace{1ex}\uanga{वणिजः}}{}
}
{दुर्गा-स्वापनम्}
{Sat} 
\cfoot{\rygdata{09:02--10:38}{13:49--15:24}{05:51--07:27}}
\caldata{JULY}{8}{\sunmonth{मिथुनम्}{24}{}{ज्यैष्ठः}{ग्रीष्मऋतुः}{भानुः}{विलम्बः}{उत्तरायणम्}{ग्रीष्मऋतुः}}
{\sunmoonsrdata{05:51}{18:36}{01:47*}{13:52}{12:14}
{\kalas{04:21 05:06 09:15 08:24 10:06 16:54 10:57 13:30 16:03 17:45 19:21 21:25 22:49 01:38*}}}
{\tnykdata{\anga{\tithi{25}{कृष्ण-दशमी}}{\time{44-7}{23:30}}\hspace{1ex}}%
{\anga{अश्विनी}{\time{4-22}{07:36}}\hspace{1ex}}{चन्द्रराशिः—\mbox{मेषः}}%
{\anga{धृतिः}{\time{51-27}{02:26*}}\hspace{1ex}\uanga{शूलः}}%
{\anga{वणिजः}{\time{16-1}{12:16}}\hspace{1ex}\anga{विष्टिः}{\time{44-7}{23:30}}\hspace{1ex}\uanga{बवः}}{}
}
{}
{Sun} 
\cfoot{\rygdata{17:00--18:36}{12:13--13:49}{15:25--17:00}}
\caldata{JULY}{9}{\sunmonth{मिथुनम्}{25}{}{ज्यैष्ठः}{ग्रीष्मऋतुः}{सोमः}{विलम्बः}{उत्तरायणम्}{ग्रीष्मऋतुः}}
{\sunmoonsrdata{05:52}{18:36}{02:39*}{14:50}{12:14}
{\kalas{04:21 05:07 09:15 08:25 10:06 16:54 10:57 13:30 16:03 17:45 19:21 21:25 22:49 01:38*}}}
{\tnykdata{\anga{\tithi{26}{कृष्ण-एकादशी}}{\time{38-58}{21:27}}\hspace{1ex}}%
{\anga{अपभरणी}{\time{2-22}{06:48}}\hspace{1ex}\anga{कृत्तिका}{\time{58-37}{05:19*}}\hspace{1ex}}{चन्द्रराशिः—\mbox{मेषः\RIGHTarrow{12:30}}}%
{\anga{शूलः}{\time{44-29}{23:39}}\hspace{1ex}\uanga{गण्डः}}%
{\anga{बवः}{\time{11-45}{10:34}}\hspace{1ex}\anga{बालवः}{\time{38-58}{21:27}}\hspace{1ex}\uanga{कौलवः}}{}
}
{कृत्तिका-व्रतम्\eventsep सर्व-योगिनी-एकादशी}
{Mon} 
\cfoot{\rygdata{07:27--09:03}{10:38--12:14}{13:49--15:25}}
\caldata{JULY}{10}{\sunmonth{मिथुनम्}{26}{}{ज्यैष्ठः}{ग्रीष्मऋतुः}{मङ्गलः}{विलम्बः}{उत्तरायणम्}{ग्रीष्मऋतुः}}
{\sunmoonsrdata{05:52}{18:36}{03:37*}{15:52}{12:14}
{\kalas{04:22 05:07 09:16 08:25 10:06 16:54 10:57 13:30 16:03 17:45 19:21 21:25 22:49 01:38*}}}
{\tnykdata{\anga{\tithi{27}{कृष्ण-द्वादशी}}{\time{32-13}{18:45}}\hspace{1ex}}%
{\anga{रोहिणी}{\time{53-23}{03:13*}}\hspace{1ex}}{चन्द्रराशिः—\mbox{वृषभः}}%
{\anga{गण्डः}{\time{36-14}{20:22}}\hspace{1ex}\uanga{वृद्धिः}}%
{\anga{कौलवः}{\time{5-46}{08:10}}\hspace{1ex}\anga{तैतिलः}{\time{32-13}{18:45}}\hspace{1ex}\anga{गरः}{\time{58-22}{05:13*}}\hspace{1ex}\uanga{वणिजः}}{}
}
{कूर्म-जयन्ती\eventsep पापनाशिनी-महाद्वादशी\eventsep प्रदोष-व्रतम्~18:36\RIGHTarrow{}20:00}
{Tue} 
\cfoot{\rygdata{15:25--17:00}{09:03--10:38}{12:14--13:49}}
\caldata{JULY}{11}{\sunmonth{मिथुनम्}{27}{}{ज्यैष्ठः}{ग्रीष्मऋतुः}{बुधः}{विलम्बः}{उत्तरायणम्}{ग्रीष्मऋतुः}}
{\sunmoonsrdata{05:52}{18:36}{04:40*}{16:56}{12:14}
{\kalas{04:22 05:07 09:16 08:25 10:07 16:54 10:58 13:30 16:03 17:45 19:21 21:25 22:49 01:39*}}}
{\tnykdata{\anga{\tithi{28}{कृष्ण-त्रयोदशी}}{\time{24-14}{15:34}}\hspace{1ex}}%
{\anga{मृगशीर्षम्}{\time{47-2}{00:41*}}\hspace{1ex}}{चन्द्रराशिः—\mbox{वृषभः\RIGHTarrow{14:00}}}%
{\anga{वृद्धिः}{\time{26-58}{16:39}}\hspace{1ex}\uanga{ध्रुवः}}%
{\anga{वणिजः}{\time{24-14}{15:34}}\hspace{1ex}\anga{विष्टिः}{\time{49-53}{01:50*}}\hspace{1ex}\uanga{शकुनिः}}{}
}
{मासशिवरात्रिः\eventsep पञ्च-पर्व-पूजा (चतुर्दशी)}
{Wed} 
\cfoot{\rygdata{12:14--13:49}{07:28--09:03}{10:38--12:14}}
\caldata{JULY}{12}{\sunmonth{मिथुनम्}{28}{}{ज्यैष्ठः}{ग्रीष्मऋतुः}{गुरुः}{विलम्बः}{उत्तरायणम्}{ग्रीष्मऋतुः}}
{\sunmoonsrdata{05:52}{18:36}{05:46*}{18:00}{12:14}
{\kalas{04:22 05:07 09:16 08:25 10:07 16:54 10:58 13:30 16:03 17:45 19:21 21:25 22:50 01:39*}}}
{\tnykdata{\anga{\tithi{29}{कृष्ण-चतुर्दशी}}{\time{15-21}{12:01}}\hspace{1ex}}%
{\anga{आर्द्रा}{\time{39-59}{21:52}}\hspace{1ex}}{चन्द्रराशिः—\mbox{मिथुनम्}}%
{\anga{ध्रुवः}{\time{16-58}{12:40}}\hspace{1ex}\uanga{व्याघातः}}%
{\anga{शकुनिः}{\time{15-21}{12:01}}\hspace{1ex}\anga{चतुष्पात्}{\time{40-44}{22:10}}\hspace{1ex}\uanga{नाग}}{}
}
{पार्वणव्रतम् अमावास्यायाम्\eventsep पञ्च-पर्व-पूजा (अमावास्या)\eventsep सर्व-ज्यैष्ठ-अमावास्या (अलभ्यम्–आर्द्रा, पुष्कला)}
{Thu} 
\cfoot{\rygdata{13:49--15:25}{05:52--07:28}{09:03--10:39}}
\caldata{JULY}{13}{\sunmonth{मिथुनम्}{29}{}{ज्यैष्ठः}{ग्रीष्मऋतुः}{शुक्रः}{विलम्बः}{उत्तरायणम्}{ग्रीष्मऋतुः}}
{\sunmoonsrdata{05:53}{18:36}{---}{19:01}{12:14}
{\kalas{04:22 05:08 09:16 08:25 10:07 16:54 10:58 13:31 16:03 17:45 19:21 21:25 22:50 01:39*}}}
{\tnykdata{\anga{\tithi{30}{अमावास्या}}{\time{6-1}{08:17}}\hspace{1ex}\anga{\tithi{1}{शुक्ल-प्रथमा}}{\time{56-38}{04:32*}}\hspace{1ex}\avamA{}}%
{\anga{पुनर्वसुः}{\time{32-39}{18:57}}\hspace{1ex}}{चन्द्रराशिः—\mbox{मिथुनम्\RIGHTarrow{13:41}}}%
{\anga{व्याघातः}{\time{6-35}{08:31}}\hspace{1ex}\anga{हर्षणः}{\time{56-10}{04:21*}}\hspace{1ex}\uanga{वज्रम्}}%
{\anga{नाग}{\time{6-1}{08:17}}\hspace{1ex}\anga{किंस्तुघ्नः}{\time{31-18}{18:24}}\hspace{1ex}\anga{बवः}{\time{56-38}{04:32*}}\hspace{1ex}\uanga{बालवः}}{}
}
{भोगशायि-पूजा\eventsep दर्शेष्टिः\eventsep काञ्ची २५ जगद्गुरु श्री-सच्चिदानन्दघनेन्द्र सरस्वती आराधना~\#{१४७१}\eventsep पार्वण-प्रायश्चित्तावकाशः पौर्णमास्याम्\eventsep स्थालीपाकः\eventsep वाराही-नवरात्र-आरम्भः}
{Fri} 
\cfoot{\rygdata{10:39--12:14}{15:25--17:00}{07:28--09:04}}
\caldata{JULY}{14}{\sunmonth{मिथुनम्}{30}{}{आषाढः}{ग्रीष्मऋतुः}{शनिः}{विलम्बः}{उत्तरायणम्}{ग्रीष्मऋतुः}}
{\sunmoonrsdata{05:53}{18:36}{06:52}{19:58}{12:14}
{\kalas{04:23 05:08 09:16 08:26 10:07 16:54 10:58 13:31 16:03 17:45 19:21 21:25 22:50 01:39*}}}
{\tnykdata{\anga{\tithi{2}{शुक्ल-द्वितीया}}{\time{47-34}{00:55*}}\hspace{1ex}}%
{\anga{पुष्यः}{\time{25-28}{16:05}}\hspace{1ex}}{चन्द्रराशिः—\mbox{कर्कटः}}%
{\anga{वज्रम्}{\time{46-1}{00:18*}}\hspace{1ex}\uanga{सिद्धिः}}%
{\anga{बालवः}{\time{22-1}{14:42}}\hspace{1ex}\anga{कौलवः}{\time{47-34}{00:55*}}\hspace{1ex}\uanga{तैतिलः}}{}
}
{अमृतलक्ष्मी-व्रतम्\eventsep चन्द्र-दर्शनम्~18:35\RIGHTarrow{}19:58\eventsep जगन्नाथ-रथ-यात्रा}
{Sat} 
\cfoot{\rygdata{09:04--10:39}{13:50--15:25}{05:53--07:28}}
\caldata{JULY}{15}{\sunmonth{मिथुनम्}{31}{}{आषाढः}{ग्रीष्मऋतुः}{भानुः}{विलम्बः}{उत्तरायणम्}{ग्रीष्मऋतुः}}
{\sunmoonrsdata{05:53}{18:36}{07:55}{20:49}{12:14}
{\kalas{04:23 05:08 09:16 08:26 10:07 16:54 10:58 13:31 16:03 17:45 19:21 21:25 22:50 01:39*}}}
{\tnykdata{\anga{\tithi{3}{शुक्ल-तृतीया}}{\time{39-13}{21:35}}\hspace{1ex}}%
{\anga{आश्रेषा}{\time{18-51}{13:26}}\hspace{1ex}}{चन्द्रराशिः—\mbox{कर्कटः\RIGHTarrow{13:26}}}%
{\anga{सिद्धिः}{\time{36-28}{20:29}}\hspace{1ex}\uanga{व्यतीपातः}}%
{\anga{तैतिलः}{\time{13-16}{11:12}}\hspace{1ex}\anga{गरः}{\time{39-13}{21:35}}\hspace{1ex}\uanga{वणिजः}}{}
}
{\tamil{புகழ்த்துணை நாயனார் (54) குருபூஜை}}
{Sun} 
\cfoot{\rygdata{17:00--18:36}{12:14--13:50}{15:25--17:00}}
\caldata{JULY}{16}{\sunmonth{मिथुनम्}{32}{\mbox{मिथुनम्{\tiny\RIGHTarrow}{22:02}}}{आषाढः}{ग्रीष्मऋतुः}{सोमः}{विलम्बः}{उत्तरायणम्}{ग्रीष्मऋतुः}}
{\sunmoonrsdata{05:54}{18:35}{08:55}{21:37}{12:14}
{\kalas{04:23 05:08 09:17 08:26 10:07 16:54 10:58 13:31 16:03 17:45 19:21 21:25 22:50 01:39*}}}
{\tnykdata{\anga{\tithi{4}{शुक्ल-चतुर्थी}}{\time{31-57}{18:40}}\hspace{1ex}}%
{\anga{मघा}{\time{13-12}{11:10}}\hspace{1ex}}{चन्द्रराशिः—\mbox{सिंहः}}%
{\anga{व्यतीपातः}{\time{27-48}{17:01}}\hspace{1ex}\uanga{वरीयान्}}%
{\anga{वणिजः}{\time{5-25}{08:04}}\hspace{1ex}\anga{विष्टिः}{\time{31-57}{18:40}}\hspace{1ex}\anga{बवः}{\time{58-49}{05:25*}}\hspace{1ex}\uanga{बालवः}}{}
}
{कर्कट-सङ्क्रमण-पुण्यकालः~10:02\RIGHTarrow{}18:35\eventsep \tamil{மாணிக்கவாசகர் குருபூஜை}\eventsep रवि-सङ्क्रमण-पुण्यकालः~15:38\RIGHTarrow{}18:35\eventsep सङ्क्रमण-दिन-अपराह्ण-पुण्यकालः~12:14\RIGHTarrow{}18:35\eventsep व्यतीपात-श्राद्धम्\eventsep शुक्ल-चतुर्थी-व्रतम्}
{Mon} 
\cfoot{\rygdata{07:29--09:04}{10:39--12:14}{13:50--15:25}}
\caldata{JULY}{17}{\sunmonth{कर्कटः}{1}{}{आषाढः}{ग्रीष्मऋतुः}{मङ्गलः}{विलम्बः}{दक्षिणायनम्}{ग्रीष्मऋतुः}}
{\sunmoonrsdata{05:54}{18:35}{09:52}{22:21}{12:15}
{\kalas{04:23 05:09 09:17 08:26 10:08 16:54 10:58 13:31 16:03 17:45 19:21 21:25 22:50 01:40*}}}
{\tnykdata{\anga{\tithi{5}{शुक्ल-पञ्चमी}}{\time{26-2}{16:19}}\hspace{1ex}}%
{\anga{पूर्वफल्गुनी}{\time{8-48}{09:25}}\hspace{1ex}}{चन्द्रराशिः—\mbox{सिंहः\RIGHTarrow{15:05}}}%
{\anga{वरीयान्}{\time{20-15}{14:00}}\hspace{1ex}\uanga{परिघः}}%
{\anga{बालवः}{\time{26-2}{16:19}}\hspace{1ex}\anga{कौलवः}{\time{53-41}{03:23*}}\hspace{1ex}\uanga{तैतिलः}}{}
}
{सर्वनदी-रजस्वला\eventsep स्कन्द-पञ्चमी\eventsep शमी-गौरी-व्रतम्}
{Tue} 
\cfoot{\rygdata{15:25--17:00}{09:04--10:39}{12:15--13:50}}
\caldata{JULY}{18}{\sunmonth{कर्कटः}{2}{}{आषाढः}{ग्रीष्मऋतुः}{बुधः}{विलम्बः}{दक्षिणायनम्}{ग्रीष्मऋतुः}}
{\sunmoonrsdata{05:54}{18:35}{10:47}{23:04}{12:15}
{\kalas{04:24 05:09 09:17 08:26 10:08 16:54 10:59 13:31 16:03 17:44 19:20 21:25 22:50 01:40*}}}
{\tnykdata{\anga{\tithi{6}{शुक्ल-षष्ठी}}{\time{21-45}{14:36}}\hspace{1ex}}%
{\anga{उत्तरफल्गुनी}{\time{5-58}{08:17}}\hspace{1ex}}{चन्द्रराशिः—\mbox{कन्या}}%
{\anga{परिघः}{\time{14-0}{11:30}}\hspace{1ex}\uanga{शिवः}}%
{\anga{तैतिलः}{\time{21-45}{14:36}}\hspace{1ex}\anga{गरः}{\time{50-16}{02:01*}}\hspace{1ex}\uanga{वणिजः}}{}
}
{काञ्ची ३५ जगद्गुरु श्री-चित्सुखेन्द्र सरस्वती आराधना~\#{१२८२}\eventsep कुमार-षष्ठी-व्रतम्\eventsep सर्वनदी-रजस्वला}
{Wed} 
\cfoot{\rygdata{12:15--13:50}{07:29--09:04}{10:40--12:15}}
\caldata{JULY}{19}{\sunmonth{कर्कटः}{3}{}{आषाढः}{ग्रीष्मऋतुः}{गुरुः}{विलम्बः}{दक्षिणायनम्}{ग्रीष्मऋतुः}}
{\sunmoonrsdata{05:54}{18:35}{11:40}{23:46}{12:15}
{\kalas{04:24 05:09 09:17 08:27 10:08 16:54 10:59 13:31 16:03 17:44 19:20 21:25 22:50 01:40*}}}
{\tnykdata{\anga{\tithi{7}{शुक्ल-सप्तमी}}{\time{19-13}{13:36}}\hspace{1ex}}%
{\anga{हस्तः}{\time{4-50}{07:51}}\hspace{1ex}}{चन्द्रराशिः—\mbox{कन्या\RIGHTarrow{19:54}}}%
{\anga{शिवः}{\time{9-10}{09:35}}\hspace{1ex}\uanga{सिद्धः}}%
{\anga{वणिजः}{\time{19-13}{13:36}}\hspace{1ex}\anga{विष्टिः}{\time{48-38}{01:22*}}\hspace{1ex}\uanga{बवः}}{}
}
{सर्वनदी-रजस्वला\eventsep वैवस्वत-सप्तमी}
{Thu} 
\cfoot{\rygdata{13:50--15:25}{05:54--07:30}{09:05--10:40}}
\caldata{JULY}{20}{\sunmonth{कर्कटः}{4}{}{आषाढः}{ग्रीष्मऋतुः}{शुक्रः}{विलम्बः}{दक्षिणायनम्}{ग्रीष्मऋतुः}}
{\sunmoonrsdata{05:55}{18:35}{12:32}{00:29*}{12:15}
{\kalas{04:24 05:09 09:17 08:27 10:08 16:54 10:59 13:31 16:03 17:44 19:20 21:25 22:50 01:40*}}}
{\tnykdata{\anga{\tithi{8}{शुक्ल-अष्टमी}}{\time{18-30}{13:19}}\hspace{1ex}}%
{\anga{चित्रा}{\time{5-31}{08:07}}\hspace{1ex}}{चन्द्रराशिः—\mbox{तुला}}%
{\anga{सिद्धः}{\time{5-48}{08:14}}\hspace{1ex}\uanga{साध्यः}}%
{\anga{बवः}{\time{18-30}{13:19}}\hspace{1ex}\anga{बालवः}{\time{48-48}{01:26*}}\hspace{1ex}\uanga{कौलवः}}{}
}
{\tamil{ஆடி~வெள்ளிக்கிழமை}\eventsep काञ्ची १२ जगद्गुरु श्री-चन्द्रशेखरेन्द्र सरस्वती आराधना~\#{१७८४}\eventsep महिषघ्नी-पूजा}
{Fri} 
\cfoot{\rygdata{10:40--12:15}{15:25--17:00}{07:30--09:05}}
\caldata{JULY}{21}{\sunmonth{कर्कटः}{5}{}{आषाढः}{ग्रीष्मऋतुः}{शनिः}{विलम्बः}{दक्षिणायनम्}{ग्रीष्मऋतुः}}
{\sunmoonrsdata{05:55}{18:35}{13:24}{01:13*}{12:15}
{\kalas{04:24 05:10 09:18 08:27 10:08 16:53 10:59 13:31 16:03 17:44 19:20 21:25 22:50 01:40*}}}
{\tnykdata{\anga{\tithi{9}{शुक्ल-नवमी}}{\time{19-32}{13:44}}\hspace{1ex}}%
{\anga{स्वाती}{\time{7-55}{09:05}}\hspace{1ex}}{चन्द्रराशिः—\mbox{तुला\RIGHTarrow{04:14*}}}%
{\anga{साध्यः}{\time{3-51}{07:27}}\hspace{1ex}\uanga{शुभः}}%
{\anga{कौलवः}{\time{19-32}{13:44}}\hspace{1ex}\anga{तैतिलः}{\time{50-40}{02:11*}}\hspace{1ex}\uanga{गरः}}{}
}
{ऐन्द्री-दुर्गा-पूजा\eventsep काञ्ची ४८ जगद्गुरु श्री-अद्वैतानन्दबोधेन्द्र सरस्वती आराधना~\#{८१९}\eventsep \tamil{கழறிற்றறிவார்/சேரமான் பெருமாள் நாயனார் (36) குருபூஜை}\eventsep मन्वादिः-(वैवस्वतः-[७])\eventsep \tamil{ஸுந்தரமூர்த்தி நாயனார் (63) குருபூஜை/திருவாடி ஸ்வாதி}\eventsep उपेन्द्र-नवमी\eventsep वाराही-नवरात्र-समापनम्}
{Sat} 
\cfoot{\rygdata{09:05--10:40}{13:50--15:25}{05:55--07:30}}
\caldata{JULY}{22}{\sunmonth{कर्कटः}{6}{}{आषाढः}{ग्रीष्मऋतुः}{भानुः}{विलम्बः}{दक्षिणायनम्}{ग्रीष्मऋतुः}}
{\sunmoonrsdata{05:55}{18:35}{14:16}{01:59*}{12:15}
{\kalas{04:24 05:10 09:18 08:27 10:08 16:53 10:59 13:31 16:03 17:44 19:20 21:25 22:50 01:40*}}}
{\tnykdata{\anga{\tithi{10}{शुक्ल-दशमी}}{\time{22-10}{14:47}}\hspace{1ex}}%
{\anga{विशाखा}{\time{11-55}{10:41}}\hspace{1ex}}{चन्द्रराशिः—\mbox{वृश्चिकः}}%
{\anga{शुभः}{\time{3-11}{07:12}}\hspace{1ex}\uanga{शुक्लः}}%
{\anga{गरः}{\time{22-10}{14:47}}\hspace{1ex}\anga{वणिजः}{\time{54-1}{03:32*}}\hspace{1ex}\uanga{विष्टिः}}{}
}
{आशा-दशमी\eventsep चातुर्मास्यव्रत-आरम्भः\eventsep नभो-मासः/वर्षऋतुः~02:30*\RIGHTarrow{}\eventsep सायन-सङ्क्रमण-दिन-अपराह्ण-पुण्यकालः~12:15\RIGHTarrow{}18:35}
{Sun} 
\cfoot{\rygdata{17:00--18:35}{12:15--13:50}{15:25--17:00}}
\caldata{JULY}{23}{\sunmonth{कर्कटः}{7}{}{आषाढः}{ग्रीष्मऋतुः}{सोमः}{विलम्बः}{दक्षिणायनम्}{ग्रीष्मऋतुः}}
{\sunmoonrsdata{05:55}{18:34}{15:08}{02:47*}{12:15}
{\kalas{04:25 05:10 09:18 08:27 10:08 16:53 10:59 13:31 16:03 17:44 19:20 21:25 22:50 01:40*}}}
{\tnykdata{\anga{\tithi{11}{शुक्ल-एकादशी}}{\time{26-9}{16:23}}\hspace{1ex}}%
{\anga{अनूराधा}{\time{17-17}{12:51}}\hspace{1ex}}{चन्द्रराशिः—\mbox{वृश्चिकः}}%
{\anga{शुक्लः}{\time{3-38}{07:23}}\hspace{1ex}\uanga{ब्रह्म}}%
{\anga{विष्टिः}{\time{26-9}{16:23}}\hspace{1ex}\anga{बवः}{\time{58-35}{05:22*}}\hspace{1ex}\uanga{बालवः}}{}
}
{गोपद्म-व्रत-आरम्भः\eventsep सेङ्गालिपुरम् अनन्तराम-दीक्षित-जयन्ती~\#{११६}\eventsep सर्व-शयन-एकादशी\eventsep विष्णु-शयनोत्सवः}
{Mon} 
\cfoot{\rygdata{07:30--09:05}{10:40--12:15}{13:50--15:25}}
\caldata{JULY}{24}{\sunmonth{कर्कटः}{8}{}{आषाढः}{ग्रीष्मऋतुः}{मङ्गलः}{विलम्बः}{दक्षिणायनम्}{ग्रीष्मऋतुः}}
{\sunmoonrsdata{05:56}{18:34}{15:59}{03:36*}{12:15}
{\kalas{04:25 05:10 09:18 08:27 10:08 16:53 10:59 13:31 16:02 17:44 19:20 21:25 22:50 01:40*}}}
{\tnykdata{\anga{\tithi{12}{शुक्ल-द्वादशी}}{\time{31-14}{18:25}}\hspace{1ex}}%
{\anga{ज्येष्ठा}{\time{23-44}{15:26}}\hspace{1ex}}{चन्द्रराशिः—\mbox{वृश्चिकः\RIGHTarrow{15:26}}}%
{\anga{ब्रह्म}{\time{5-0}{07:56}}\hspace{1ex}\uanga{इन्द्रः}}%
{\anga{बालवः}{\time{31-14}{18:25}}\hspace{1ex}\uanga{कौलवः}}{}
}
{काञ्ची ३१ जगद्गुरु श्री-ब्रह्मानन्दघनेन्द्र सरस्वती आराधना~\#{१३५१}\eventsep काञ्ची ६३ जगद्गुरु श्री-महादेवेन्द्र सरस्वती ५ आराधना~\#{२०५}\eventsep \tamil{கோட்புலி நாயனார் (55) குருபூஜை}\eventsep \tamil{கலிய நாயனார் (43) குருபூஜை}\eventsep वासुदेव-द्वादशी\eventsep शाकव्रत-आरम्भः}
{Tue} 
\cfoot{\rygdata{15:25--16:59}{09:05--10:40}{12:15--13:50}}
\caldata{JULY}{25}{\sunmonth{कर्कटः}{9}{}{आषाढः}{ग्रीष्मऋतुः}{बुधः}{विलम्बः}{दक्षिणायनम्}{ग्रीष्मऋतुः}}
{\sunmoonrsdata{05:56}{18:34}{16:48}{04:26*}{12:15}
{\kalas{04:25 05:10 09:18 08:28 10:09 16:53 10:59 13:31 16:02 17:44 19:19 21:25 22:50 01:40*}}}
{\tnykdata{\anga{\tithi{13}{शुक्ल-त्रयोदशी}}{\time{37-3}{20:45}}\hspace{1ex}}%
{\anga{मूला}{\time{30-57}{18:19}}\hspace{1ex}}{चन्द्रराशिः—\mbox{धनुः}}%
{\anga{इन्द्रः}{\time{7-2}{08:45}}\hspace{1ex}\uanga{वैधृतिः}}%
{\anga{कौलवः}{\time{4-4}{07:34}}\hspace{1ex}\anga{तैतिलः}{\time{37-3}{20:45}}\hspace{1ex}\uanga{गरः}}{}
}
{प्रदोष-व्रतम्~18:34\RIGHTarrow{}19:59\eventsep वैधृति-श्राद्धम्}
{Wed} 
\cfoot{\rygdata{12:15--13:50}{07:31--09:05}{10:40--12:15}}
\caldata{JULY}{26}{\sunmonth{कर्कटः}{10}{}{आषाढः}{ग्रीष्मऋतुः}{गुरुः}{विलम्बः}{दक्षिणायनम्}{ग्रीष्मऋतुः}}
{\sunmoonrsdata{05:56}{18:34}{17:35}{05:16*}{12:15}
{\kalas{04:25 05:11 09:18 08:28 10:09 16:53 10:59 13:31 16:02 17:43 19:19 21:24 22:50 01:40*}}}
{\tnykdata{\anga{\tithi{14}{शुक्ल-चतुर्दशी}}{\time{43-19}{23:16}}\hspace{1ex}}%
{\anga{पूर्वाषाढा}{\time{38-36}{21:23}}\hspace{1ex}}{चन्द्रराशिः—\mbox{धनुः\RIGHTarrow{04:10*}}}%
{\anga{वैधृतिः}{\time{9-31}{09:45}}\hspace{1ex}\uanga{विष्कम्भः}}%
{\anga{गरः}{\time{10-9}{10:00}}\hspace{1ex}\anga{वणिजः}{\time{43-19}{23:16}}\hspace{1ex}\uanga{विष्टिः}}{}
}
{पवित्र-चतुर्दशी}
{Thu} 
\cfoot{\rygdata{13:50--15:24}{05:56--07:31}{09:06--10:40}}
\caldata{JULY}{27}{\sunmonth{कर्कटः}{11}{}{आषाढः}{ग्रीष्मऋतुः}{शुक्रः}{विलम्बः}{दक्षिणायनम्}{ग्रीष्मऋतुः}}
{\sunmoonrsdata{05:56}{18:33}{18:20}{---}{12:15}
{\kalas{04:25 05:11 09:18 08:28 10:09 16:53 10:59 13:31 16:02 17:43 19:19 21:24 22:50 01:40*}}}
{\tnykdata{\anga{\tithi{15}{पौर्णमासी}}{\time{49-44}{01:50*}}\hspace{1ex}}%
{\anga{उत्तराषाढा}{\time{46-24}{00:30*}}\hspace{1ex}}{चन्द्रराशिः—\mbox{मकरः}}%
{\anga{विष्कम्भः}{\time{12-13}{10:50}}\hspace{1ex}\uanga{प्रीतिः}}%
{\anga{विष्टिः}{\time{16-31}{12:33}}\hspace{1ex}\anga{बवः}{\time{49-44}{01:50*}}\hspace{1ex}\uanga{बालवः}}{}
}
{\tamil{ஆடி~வெள்ளிக்கிழமை}\eventsep आषाढ-पूर्णिमा-स्नानम्\eventsep चन्द्र-ग्रहणम्-(केतुग्रस्त)~23:54\RIGHTarrow{}03:49*\eventsep गुरु-पूर्णिमा/व्यास-पूजा\eventsep काञ्ची १० जगद्गुरु श्री-सुरेश्वरेन्द्र सरस्वती आराधना~\#{१८९२}\eventsep कोकिल-व्रतम्\eventsep मन्वादिः-(ब्रह्मः-[१०])\eventsep पार्वणव्रतम् पूर्णिमायाम्\eventsep पूर्णिमा-व्रतम्\eventsep पञ्च-पर्व-पूजा (पूर्णिमा)\eventsep वेङ्कटाचले पूर्णिमा-गरुड-सेवा\eventsep यतिचातुर्मास्यव्रत-आरम्भः\eventsep शिव-शयनोत्सवः}
{Fri} 
\cfoot{\rygdata{10:40--12:15}{15:24--16:59}{07:31--09:06}}
\caldata{JULY}{28}{\sunmonth{कर्कटः}{12}{}{आषाढः}{ग्रीष्मऋतुः}{शनिः}{विलम्बः}{दक्षिणायनम्}{ग्रीष्मऋतुः}}
{\sunmoonsrdata{05:57}{18:33}{19:03}{06:05}{12:15}
{\kalas{04:25 05:11 09:18 08:28 10:09 16:52 10:59 13:31 16:02 17:43 19:19 21:24 22:50 01:40*}}}
{\tnykdata{\anga{\tithi{16}{कृष्ण-प्रथमा}}{\time{55-58}{04:20*}}\hspace{1ex}}%
{\anga{श्रवणः}{\time{54-5}{03:35*}}\hspace{1ex}}{चन्द्रराशिः—\mbox{मकरः}}%
{\anga{प्रीतिः}{\time{14-55}{11:55}}\hspace{1ex}\uanga{आयुष्मान्}}%
{\anga{बालवः}{\time{22-53}{15:06}}\hspace{1ex}\anga{कौलवः}{\time{55-58}{04:20*}}\hspace{1ex}\uanga{तैतिलः}}{}
}
{काञ्ची ५४ जगद्गुरु श्री-व्यासाचल महादेवेन्द्र सरस्वती आराधना~\#{५१२}\eventsep पार्वण-प्रायश्चित्तावकाशः दर्शे\eventsep पूर्णमासेष्टिः\eventsep स्थालीपाकः\eventsep श्रवण-व्रतम्}
{Sat} 
\cfoot{\rygdata{09:06--10:40}{13:50--15:24}{05:57--07:31}}
\caldata{JULY}{29}{\sunmonth{कर्कटः}{13}{}{आषाढः}{ग्रीष्मऋतुः}{भानुः}{विलम्बः}{दक्षिणायनम्}{ग्रीष्मऋतुः}}
{\sunmoonsrdata{05:57}{18:33}{19:43}{06:53}{12:15}
{\kalas{04:26 05:11 09:19 08:28 10:09 16:52 10:59 13:31 16:02 17:43 19:19 21:24 22:50 01:40*}}}
{\tnykdata{\fulltithi{\tithi{17}{कृष्ण-द्वितीया}}}%
{\fullanga{श्रविष्ठा}}{चन्द्रराशिः—\mbox{मकरः\RIGHTarrow{17:04}}}%
{\anga{आयुष्मान्}{\time{17-26}{12:55}}\hspace{1ex}\uanga{सौभाग्यः}}%
{\anga{तैतिलः}{\time{28-57}{17:32}}\hspace{1ex}\uanga{गरः}}{}
}
{अष्टनाग-पूजा\eventsep अशून्यशयन-व्रतम्\eventsep काञ्ची ६९ जगद्गुरु श्री-जयेन्द्र सरस्वती जयन्ती~\#{८४}}
{Sun} 
\cfoot{\rygdata{16:58--18:33}{12:15--13:49}{15:24--16:58}}
\caldata{JULY}{30}{\sunmonth{कर्कटः}{14}{}{आषाढः}{ग्रीष्मऋतुः}{सोमः}{विलम्बः}{दक्षिणायनम्}{ग्रीष्मऋतुः}}
{\sunmoonsrdata{05:57}{18:33}{20:21}{07:41}{12:15}
{\kalas{04:26 05:11 09:19 08:28 10:09 16:52 10:59 13:30 16:02 17:42 19:18 21:24 22:49 01:41*}}}
{\tnykdata{\anga{\tithi{17}{कृष्ण-द्वितीया}}{\time{1-47}{06:40}}\hspace{1ex}}%
{\anga{श्रविष्ठा}{\time{1-20}{06:29}}\hspace{1ex}}{चन्द्रराशिः—\mbox{कुम्भः}}%
{\anga{सौभाग्यः}{\time{19-32}{13:46}}\hspace{1ex}\uanga{शोभनः}}%
{\anga{गरः}{\time{1-47}{06:40}}\hspace{1ex}\anga{वणिजः}{\time{34-27}{19:44}}\hspace{1ex}\uanga{विष्टिः}}{}
}
{}
{Mon} 
\cfoot{\rygdata{07:32--09:06}{10:40--12:15}{13:49--15:24}}
\caldata{JULY}{31}{\sunmonth{कर्कटः}{15}{}{आषाढः}{ग्रीष्मऋतुः}{मङ्गलः}{विलम्बः}{दक्षिणायनम्}{ग्रीष्मऋतुः}}
{\sunmoonsrdata{05:57}{18:32}{20:59}{08:28}{12:15}
{\kalas{04:26 05:12 09:19 08:28 10:09 16:52 10:59 13:30 16:01 17:42 19:18 21:24 22:49 01:41*}}}
{\tnykdata{\anga{\tithi{18}{कृष्ण-तृतीया}}{\time{6-54}{08:43}}\hspace{1ex}}%
{\anga{शतभिषक्}{\time{7-55}{09:08}}\hspace{1ex}}{चन्द्रराशिः—\mbox{कुम्भः\RIGHTarrow{04:52*}}}%
{\anga{शोभनः}{\time{21-3}{14:22}}\hspace{1ex}\uanga{अतिगण्डः}}%
{\anga{विष्टिः}{\time{6-54}{08:43}}\hspace{1ex}\anga{बवः}{\time{39-7}{21:36}}\hspace{1ex}\uanga{बालवः}}{}
}
{अङ्गारकी गजानन-महागणपति सङ्कटहर-चतुर्थी-व्रतम्}
{Tue} 
\cfoot{\rygdata{15:24--16:58}{09:06--10:40}{12:15--13:49}}
\caldata{AUGUST}{1}{\sunmonth{कर्कटः}{16}{}{आषाढः}{ग्रीष्मऋतुः}{बुधः}{विलम्बः}{दक्षिणायनम्}{ग्रीष्मऋतुः}}
{\sunmoonsrdata{05:57}{18:32}{21:37}{09:15}{12:15}
{\kalas{04:26 05:12 09:19 08:28 10:09 16:51 10:59 13:30 16:01 17:42 19:18 21:23 22:49 01:41*}}}
{\tnykdata{\anga{\tithi{19}{कृष्ण-चतुर्थी}}{\time{11-2}{10:23}}\hspace{1ex}}%
{\anga{पूर्वप्रोष्ठपदा}{\time{13-34}{11:23}}\hspace{1ex}}{चन्द्रराशिः—\mbox{मीनः}}%
{\anga{अतिगण्डः}{\time{21-45}{14:40}}\hspace{1ex}\uanga{सुकर्म}}%
{\anga{बालवः}{\time{11-2}{10:23}}\hspace{1ex}\anga{कौलवः}{\time{42-40}{23:02}}\hspace{1ex}\uanga{तैतिलः}}{}
}
{}
{Wed} 
\cfoot{\rygdata{12:15--13:49}{07:32--09:06}{10:40--12:15}}
\caldata{AUGUST}{2}{\sunmonth{कर्कटः}{17}{}{आषाढः}{ग्रीष्मऋतुः}{गुरुः}{विलम्बः}{दक्षिणायनम्}{ग्रीष्मऋतुः}}
{\sunmoonsrdata{05:58}{18:32}{22:15}{10:02}{12:15}
{\kalas{04:26 05:12 09:19 08:28 10:09 16:51 10:59 13:30 16:01 17:42 19:17 21:23 22:49 01:41*}}}
{\tnykdata{\anga{\tithi{20}{कृष्ण-पञ्चमी}}{\time{13-57}{11:33}}\hspace{1ex}}%
{\anga{उत्तरप्रोष्ठपदा}{\time{18-1}{13:10}}\hspace{1ex}}{चन्द्रराशिः—\mbox{मीनः}}%
{\anga{सुकर्म}{\time{21-29}{14:33}}\hspace{1ex}\uanga{धृतिः}}%
{\anga{तैतिलः}{\time{13-57}{11:33}}\hspace{1ex}\anga{गरः}{\time{44-53}{23:55}}\hspace{1ex}\uanga{वणिजः}}{}
}
{}
{Thu} 
\cfoot{\rygdata{13:49--15:23}{05:58--07:32}{09:06--10:40}}
\caldata{AUGUST}{3}{\sunmonth{कर्कटः}{18}{}{आषाढः}{ग्रीष्मऋतुः}{शुक्रः}{विलम्बः}{दक्षिणायनम्}{ग्रीष्मऋतुः}}
{\sunmoonsrdata{05:58}{18:31}{22:56}{10:52}{12:15}
{\kalas{04:26 05:12 09:19 08:29 10:09 16:51 10:59 13:30 16:01 17:41 19:17 21:23 22:49 01:41*}}}
{\tnykdata{\anga{\tithi{21}{कृष्ण-षष्ठी}}{\time{15-25}{12:08}}\hspace{1ex}}%
{\anga{रेवती}{\time{21-2}{14:23}}\hspace{1ex}}{चन्द्रराशिः—\mbox{मीनः\RIGHTarrow{14:23}}}%
{\anga{धृतिः}{\time{20-4}{13:59}}\hspace{1ex}\uanga{शूलः}}%
{\anga{वणिजः}{\time{15-25}{12:08}}\hspace{1ex}\anga{विष्टिः}{\time{45-33}{00:11*}}\hspace{1ex}\uanga{बवः}}{}
}
{\tamil{ஆடிப்~பெருக்கு}\eventsep \tamil{ஆடி~வெள்ளிக்கிழமை}\eventsep भृगुरेवती-पुण्यकालः\RIGHTarrow{}14:23}
{Fri} 
\cfoot{\rygdata{10:40--12:15}{15:23--16:57}{07:32--09:06}}
\caldata{AUGUST}{4}{\sunmonth{कर्कटः}{19}{}{आषाढः}{ग्रीष्मऋतुः}{शनिः}{विलम्बः}{दक्षिणायनम्}{ग्रीष्मऋतुः}}
{\sunmoonsrdata{05:58}{18:31}{23:40}{11:44}{12:15}
{\kalas{04:26 05:12 09:19 08:29 10:09 16:51 10:59 13:30 16:00 17:41 19:17 21:23 22:49 01:40*}}}
{\tnykdata{\anga{\tithi{22}{कृष्ण-सप्तमी}}{\time{15-15}{12:04}}\hspace{1ex}}%
{\anga{अश्विनी}{\time{22-27}{14:57}}\hspace{1ex}}{चन्द्रराशिः—\mbox{मेषः}}%
{\anga{शूलः}{\time{17-21}{12:55}}\hspace{1ex}\uanga{गण्डः}}%
{\anga{बवः}{\time{15-15}{12:04}}\hspace{1ex}\anga{बालवः}{\time{44-33}{23:48}}\hspace{1ex}\uanga{कौलवः}}{}
}
{चामुण्डेश्वरी-जयन्ती\eventsep पञ्च-पर्व-पूजा (अष्टमी)}
{Sat} 
\cfoot{\rygdata{09:06--10:40}{13:49--15:23}{05:58--07:32}}
\caldata{AUGUST}{5}{\sunmonth{कर्कटः}{20}{}{आषाढः}{ग्रीष्मऋतुः}{भानुः}{विलम्बः}{दक्षिणायनम्}{ग्रीष्मऋतुः}}
{\sunmoonsrdata{05:58}{18:31}{00:28*}{12:39}{12:14}
{\kalas{04:27 05:12 09:19 08:29 10:09 16:50 10:59 13:30 16:00 17:40 19:17 21:23 22:49 01:40*}}}
{\tnykdata{\anga{\tithi{23}{कृष्ण-अष्टमी}}{\time{13-25}{11:20}}\hspace{1ex}}%
{\anga{अपभरणी}{\time{22-12}{14:51}}\hspace{1ex}}{चन्द्रराशिः—\mbox{मेषः\RIGHTarrow{20:43}}}%
{\anga{गण्डः}{\time{13-17}{11:17}}\hspace{1ex}\uanga{वृद्धिः}}%
{\anga{कौलवः}{\time{13-25}{11:20}}\hspace{1ex}\anga{तैतिलः}{\time{41-51}{22:43}}\hspace{1ex}\uanga{गरः}}{}
}
{कृत्तिका-व्रतम्}
{Sun} 
\cfoot{\rygdata{16:57--18:31}{12:14--13:49}{15:23--16:57}}
\caldata{AUGUST}{6}{\sunmonth{कर्कटः}{21}{}{आषाढः}{ग्रीष्मऋतुः}{सोमः}{विलम्बः}{दक्षिणायनम्}{ग्रीष्मऋतुः}}
{\sunmoonsrdata{05:58}{18:30}{01:22*}{13:37}{12:14}
{\kalas{04:27 05:13 09:19 08:29 10:09 16:50 10:59 13:30 16:00 17:40 19:16 21:22 22:48 01:40*}}}
{\tnykdata{\anga{\tithi{24}{कृष्ण-नवमी}}{\time{9-52}{09:56}}\hspace{1ex}}%
{\anga{कृत्तिका}{\time{20-17}{14:05}}\hspace{1ex}}{चन्द्रराशिः—\mbox{वृषभः}}%
{\anga{वृद्धिः}{\time{7-51}{09:07}}\hspace{1ex}\uanga{ध्रुवः}}%
{\anga{गरः}{\time{9-52}{09:56}}\hspace{1ex}\anga{वणिजः}{\time{37-30}{20:58}}\hspace{1ex}\uanga{विष्टिः}}{}
}
{कर्कट-कार्त्तिक-पूजा\eventsep \tamil{மூர்த்தி நாயனார் (15) குருபூஜை}\eventsep \tamil{புகழ்ச்சோழ நாயனார் (39) குருபூஜை}}
{Mon} 
\cfoot{\rygdata{07:32--09:06}{10:40--12:14}{13:48--15:22}}
\caldata{AUGUST}{7}{\sunmonth{कर्कटः}{22}{}{आषाढः}{ग्रीष्मऋतुः}{मङ्गलः}{विलम्बः}{दक्षिणायनम्}{ग्रीष्मऋतुः}}
{\sunmoonsrdata{05:58}{18:30}{02:21*}{14:38}{12:14}
{\kalas{04:27 05:13 09:19 08:29 10:09 16:50 10:59 13:29 16:00 17:40 19:16 21:22 22:48 01:40*}}}
{\tnykdata{\anga{\tithi{25}{कृष्ण-दशमी}}{\time{4-44}{07:52}}\hspace{1ex}\anga{\tithi{26}{कृष्ण-एकादशी}}{\time{58-11}{05:15*}}\hspace{1ex}\avamA{}}%
{\anga{रोहिणी}{\time{16-48}{12:42}}\hspace{1ex}}{चन्द्रराशिः—\mbox{वृषभः\RIGHTarrow{23:47}}}%
{\anga{ध्रुवः}{\time{1-6}{06:25}}\hspace{1ex}\anga{व्याघातः}{\time{53-11}{03:15*}}\hspace{1ex}\uanga{हर्षणः}}%
{\anga{विष्टिः}{\time{4-44}{07:52}}\hspace{1ex}\anga{बवः}{\time{31-37}{18:38}}\hspace{1ex}\anga{बालवः}{\time{58-11}{05:15*}}\hspace{1ex}\uanga{कौलवः}}{}
}
{स्मार्त-कामिका-एकादशी\eventsep \tamil{திருப்பாணாழ்வார் திருநக்ஷத்திரம்}}
{Tue} 
\cfoot{\rygdata{15:22--16:56}{09:06--10:40}{12:14--13:48}}
\caldata{AUGUST}{8}{\sunmonth{कर्कटः}{23}{}{आषाढः}{ग्रीष्मऋतुः}{बुधः}{विलम्बः}{दक्षिणायनम्}{ग्रीष्मऋतुः}}
{\sunmoonsrdata{05:59}{18:29}{03:24*}{15:41}{12:14}
{\kalas{04:27 05:13 09:19 08:29 10:09 16:49 10:59 13:29 15:59 17:39 19:15 21:22 22:48 01:40*}}}
{\tnykdata{\anga{\tithi{27}{कृष्ण-द्वादशी}}{\time{50-29}{02:10*}}\hspace{1ex}}%
{\anga{मृगशीर्षम्}{\time{11-56}{10:45}}\hspace{1ex}}{चन्द्रराशिः—\mbox{मिथुनम्}}%
{\anga{हर्षणः}{\time{44-16}{23:41}}\hspace{1ex}\uanga{वज्रम्}}%
{\anga{कौलवः}{\time{24-27}{15:46}}\hspace{1ex}\anga{तैतिलः}{\time{50-29}{02:10*}}\hspace{1ex}\uanga{गरः}}{}
}
{देवी-पर्व-४\eventsep हरिवासरः\RIGHTarrow{}10:31\eventsep \tamil{கூற்றுவ நாயனார் (38) குருபூஜை}\eventsep वैष्णव-कामिका-एकादशी}
{Wed} 
\cfoot{\rygdata{12:14--13:48}{07:33--09:06}{10:40--12:14}}
\caldata{AUGUST}{9}{\sunmonth{कर्कटः}{24}{}{आषाढः}{ग्रीष्मऋतुः}{गुरुः}{विलम्बः}{दक्षिणायनम्}{ग्रीष्मऋतुः}}
{\sunmoonsrdata{05:59}{18:29}{04:30*}{16:42}{12:14}
{\kalas{04:27 05:13 09:19 08:29 10:09 16:49 10:59 13:29 15:59 17:39 19:15 21:22 22:48 01:40*}}}
{\tnykdata{\anga{\tithi{28}{कृष्ण-त्रयोदशी}}{\time{41-54}{22:45}}\hspace{1ex}}%
{\anga{आर्द्रा}{\time{5-59}{08:23}}\hspace{1ex}\anga{पुनर्वसुः}{\time{59-17}{05:42*}}\hspace{1ex}}{चन्द्रराशिः—\mbox{मिथुनम्\RIGHTarrow{00:23*}}}%
{\anga{वज्रम्}{\time{34-37}{19:50}}\hspace{1ex}\uanga{सिद्धिः}}%
{\anga{गरः}{\time{16-17}{12:30}}\hspace{1ex}\anga{वणिजः}{\time{41-54}{22:45}}\hspace{1ex}\uanga{विष्टिः}}{}
}
{मासशिवरात्रिः\eventsep पञ्च-पर्व-पूजा (चतुर्दशी)\eventsep प्रदोष-व्रतम्~18:29\RIGHTarrow{}19:55}
{Thu} 
\cfoot{\rygdata{13:48--15:22}{05:59--07:33}{09:06--10:40}}
\caldata{AUGUST}{10}{\sunmonth{कर्कटः}{25}{}{आषाढः}{ग्रीष्मऋतुः}{शुक्रः}{विलम्बः}{दक्षिणायनम्}{ग्रीष्मऋतुः}}
{\sunmoonsrdata{05:59}{18:29}{05:34*}{17:40}{12:14}
{\kalas{04:27 05:13 09:19 08:29 10:09 16:49 10:59 13:29 15:59 17:39 19:15 21:21 22:48 01:40*}}}
{\tnykdata{\anga{\tithi{29}{कृष्ण-चतुर्दशी}}{\time{32-51}{19:07}}\hspace{1ex}}%
{\anga{पुष्यः}{\time{52-12}{02:52*}}\hspace{1ex}}{चन्द्रराशिः—\mbox{कर्कटः}}%
{\anga{सिद्धिः}{\time{24-29}{15:47}}\hspace{1ex}\uanga{व्यतीपातः}}%
{\anga{विष्टिः}{\time{7-25}{08:57}}\hspace{1ex}\anga{शकुनिः}{\time{32-51}{19:07}}\hspace{1ex}\anga{चतुष्पात्}{\time{58-15}{05:17*}}\hspace{1ex}\uanga{नाग}}{}
}
{\tamil{ஆடி~வெள்ளிக்கிழமை}\eventsep बोधायन आषाढ (कर्कट) अमावास्या\eventsep पञ्च-पर्व-पूजा (अमावास्या)\eventsep व्यतीपात-श्राद्धम्}
{Fri} 
\cfoot{\rygdata{10:40--12:14}{15:21--16:55}{07:33--09:06}}
\caldata{AUGUST}{11}{\sunmonth{कर्कटः}{26}{}{आषाढः}{ग्रीष्मऋतुः}{शनिः}{विलम्बः}{दक्षिणायनम्}{ग्रीष्मऋतुः}}
{\sunmoonsrdata{05:59}{18:28}{---}{18:35}{12:14}
{\kalas{04:27 05:13 09:19 08:29 10:09 16:48 10:59 13:28 15:58 17:38 19:14 21:21 22:47 01:40*}}}
{\tnykdata{\anga{\tithi{30}{अमावास्या}}{\time{23-40}{15:27}}\hspace{1ex}}%
{\anga{आश्रेषा}{\time{45-9}{00:03*}}\hspace{1ex}}{चन्द्रराशिः—\mbox{कर्कटः\RIGHTarrow{00:03*}}}%
{\anga{व्यतीपातः}{\time{14-13}{11:40}}\hspace{1ex}\uanga{वरीयान्}}%
{\anga{नाग}{\time{23-40}{15:27}}\hspace{1ex}\anga{किंस्तुघ्नः}{\time{49-10}{01:39*}}\hspace{1ex}\uanga{बवः}}{}
}
{आषाढ-स्नानपूर्तिः\eventsep आषाढ (कर्कट) अमावास्या\eventsep बोधायन-इष्टिः\eventsep दीप-पूजा\eventsep काञ्ची ३८ जगद्गुरु श्री-अभिनवशङ्करेन्द्र सरस्वती आराधना~\#{११७९}\eventsep काञ्ची ४१ जगद्गुरु श्री-गङ्गाधरेन्द्र सरस्वती २ आराधना~\#{१०६९}\eventsep काञ्ची ४६ जगद्गुरु श्री-सान्द्रानन्दबोधेन्द्र सरस्वती आराधना~\#{९२१}\eventsep पार्वणव्रतम् अमावास्यायाम्\eventsep पति-सञ्जीवनी-व्रतम्}
{Sat} 
\cfoot{\rygdata{09:06--10:40}{13:47--15:21}{05:59--07:33}}
\caldata{AUGUST}{12}{\sunmonth{कर्कटः}{27}{}{श्रावणः}{वर्षऋतुः}{भानुः}{विलम्बः}{दक्षिणायनम्}{ग्रीष्मऋतुः}}
{\sunmoonrsdata{05:59}{18:28}{06:37}{19:25}{12:13}
{\kalas{04:27 05:13 09:19 08:29 10:09 16:48 10:59 13:28 15:58 17:38 19:14 21:21 22:47 01:40*}}}
{\tnykdata{\anga{\tithi{1}{शुक्ल-प्रथमा}}{\time{14-46}{11:54}}\hspace{1ex}}%
{\anga{मघा}{\time{38-33}{21:24}}\hspace{1ex}}{चन्द्रराशिः—\mbox{सिंहः}}%
{\anga{वरीयान्}{\time{4-6}{07:38}}\hspace{1ex}\anga{परिघः}{\time{54-30}{03:47*}}\hspace{1ex}\uanga{शिवः}}%
{\anga{बवः}{\time{14-46}{11:54}}\hspace{1ex}\anga{बालवः}{\time{40-33}{22:13}}\hspace{1ex}\uanga{कौलवः}}{}
}
{चन्द्र-दर्शनम्~18:28\RIGHTarrow{}19:25\eventsep दर्शेष्टिः\eventsep पार्वण-प्रायश्चित्तावकाशः पौर्णमास्याम्\eventsep स्थालीपाकः}
{Sun} 
\cfoot{\rygdata{16:54--18:28}{12:13--13:47}{15:21--16:54}}
\caldata{AUGUST}{13}{\sunmonth{कर्कटः}{28}{}{श्रावणः}{वर्षऋतुः}{सोमः}{विलम्बः}{दक्षिणायनम्}{ग्रीष्मऋतुः}}
{\sunmoonrsdata{05:59}{18:27}{07:37}{20:12}{12:13}
{\kalas{04:27 05:13 09:19 08:29 10:09 16:48 10:59 13:28 15:58 17:37 19:13 21:20 22:47 01:40*}}}
{\tnykdata{\anga{\tithi{2}{शुक्ल-द्वितीया}}{\time{6-32}{08:36}}\hspace{1ex}\anga{\tithi{3}{शुक्ल-तृतीया}}{\time{59-23}{05:45*}}\hspace{1ex}\avamA{}}%
{\anga{पूर्वफल्गुनी}{\time{32-48}{19:07}}\hspace{1ex}}{चन्द्रराशिः—\mbox{सिंहः\RIGHTarrow{00:37*}}}%
{\anga{शिवः}{\time{45-42}{00:16*}}\hspace{1ex}\uanga{सिद्धः}}%
{\anga{कौलवः}{\time{6-32}{08:36}}\hspace{1ex}\anga{तैतिलः}{\time{32-48}{19:07}}\hspace{1ex}\anga{गरः}{\time{59-23}{05:45*}}\hspace{1ex}\uanga{वणिजः}}{}
}
{हरियाली-तृतीया\eventsep मधुश्रावणि-व्रतम्\eventsep मनोरथ-द्वितीया\eventsep पार्वती-पवित्रारोपणम्\eventsep सत्यनारायण-जयन्ती\eventsep स्वर्ण-गौरी-व्रतम्\eventsep \tamil{திருவாடிப்பூரம்}}
{Mon} 
\cfoot{\rygdata{07:33--09:06}{10:40--12:13}{13:47--15:20}}
\caldata{AUGUST}{14}{\sunmonth{कर्कटः}{29}{}{श्रावणः}{वर्षऋतुः}{मङ्गलः}{विलम्बः}{दक्षिणायनम्}{ग्रीष्मऋतुः}}
{\sunmoonrsdata{06:00}{18:27}{08:34}{20:57}{12:13}
{\kalas{04:27 05:13 09:19 08:29 10:09 16:47 10:58 13:28 15:57 17:37 19:13 21:20 22:46 01:40*}}}
{\tnykdata{\anga{\tithi{4}{शुक्ल-चतुर्थी}}{\time{53-39}{03:27*}}\hspace{1ex}}%
{\anga{उत्तरफल्गुनी}{\time{28-20}{17:20}}\hspace{1ex}}{चन्द्रराशिः—\mbox{कन्या}}%
{\anga{सिद्धः}{\time{38-1}{21:12}}\hspace{1ex}\uanga{साध्यः}}%
{\anga{वणिजः}{\time{26-19}{16:31}}\hspace{1ex}\anga{विष्टिः}{\time{53-39}{03:27*}}\hspace{1ex}\uanga{बवः}}{}
}
{एकविंशति-दिवस-गणपति-व्रत-आरम्भः\eventsep दूर्वा-गणपति-व्रतम्\eventsep सुखा-अङ्गारक-चतुर्थी\eventsep शुक्ल-चतुर्थी-व्रतम्}
{Tue} 
\cfoot{\rygdata{15:20--16:53}{09:06--10:40}{12:13--13:46}}
\caldata{AUGUST}{15}{\sunmonth{कर्कटः}{30}{}{श्रावणः}{वर्षऋतुः}{बुधः}{विलम्बः}{दक्षिणायनम्}{ग्रीष्मऋतुः}}
{\sunmoonrsdata{06:00}{18:26}{09:29}{21:41}{12:13}
{\kalas{04:27 05:13 09:19 08:29 10:08 16:47 10:58 13:28 15:57 17:36 19:13 21:20 22:46 01:40*}}}
{\tnykdata{\anga{\tithi{5}{शुक्ल-पञ्चमी}}{\time{49-39}{01:51*}}\hspace{1ex}}%
{\anga{हस्तः}{\time{25-27}{16:10}}\hspace{1ex}}{चन्द्रराशिः—\mbox{कन्या\RIGHTarrow{03:52*}}}%
{\anga{साध्यः}{\time{31-41}{18:40}}\hspace{1ex}\uanga{शुभः}}%
{\anga{बवः}{\time{21-25}{14:34}}\hspace{1ex}\anga{बालवः}{\time{49-39}{01:51*}}\hspace{1ex}\uanga{कौलवः}}{}
}
{गरुड-पञ्चमी\eventsep नाग-पञ्चमी\eventsep तैत्तिरीय-उपाकर्म हस्ते}
{Wed} 
\cfoot{\rygdata{12:13--13:46}{07:33--09:06}{10:40--12:13}}
\caldata{AUGUST}{16}{\sunmonth{कर्कटः}{31}{}{श्रावणः}{वर्षऋतुः}{गुरुः}{विलम्बः}{दक्षिणायनम्}{ग्रीष्मऋतुः}}
{\sunmoonrsdata{06:00}{18:26}{10:23}{22:25}{12:13}
{\kalas{04:27 05:14 09:19 08:29 10:08 16:46 10:58 13:27 15:57 17:36 19:12 21:19 22:46 01:40*}}}
{\tnykdata{\anga{\tithi{6}{शुक्ल-षष्ठी}}{\time{47-34}{01:02*}}\hspace{1ex}}%
{\anga{चित्रा}{\time{24-24}{15:46}}\hspace{1ex}}{चन्द्रराशिः—\mbox{तुला}}%
{\anga{शुभः}{\time{26-52}{16:45}}\hspace{1ex}\uanga{शुक्लः}}%
{\anga{कौलवः}{\time{18-21}{13:21}}\hspace{1ex}\anga{तैतिलः}{\time{47-34}{01:02*}}\hspace{1ex}\uanga{गरः}}{}
}
{षष्ठी-व्रतम्\eventsep \tamil{பெருமிழலைக் குறும்ப நாயனார் (22) குருபூஜை}\eventsep सूपौदन-व्रतम्}
{Thu} 
\cfoot{\rygdata{13:46--15:19}{06:00--07:33}{09:06--10:39}}
\caldata{AUGUST}{17}{\sunmonth{सिंहः}{1}{\mbox{कर्कटः{\tiny\RIGHTarrow}{06:22}}}{श्रावणः}{वर्षऋतुः}{शुक्रः}{विलम्बः}{दक्षिणायनम्}{वर्षऋतुः}}
{\sunmoonrsdata{06:00}{18:25}{11:17}{23:09}{12:12}
{\kalas{04:27 05:14 09:19 08:29 10:08 16:46 10:58 13:27 15:56 17:35 19:12 21:19 22:46 01:39*}}}
{\tnykdata{\anga{\tithi{7}{शुक्ल-सप्तमी}}{\time{47-32}{01:01*}}\hspace{1ex}}%
{\anga{स्वाती}{\time{25-21}{16:08}}\hspace{1ex}}{चन्द्रराशिः—\mbox{तुला}}%
{\anga{शुक्लः}{\time{23-40}{15:28}}\hspace{1ex}\uanga{ब्रह्म}}%
{\anga{गरः}{\time{17-18}{12:55}}\hspace{1ex}\anga{वणिजः}{\time{47-32}{01:01*}}\hspace{1ex}\uanga{विष्टिः}}{}
}
{अव्यङ्ग-सप्तमी\eventsep द्वादश-सप्तमी\eventsep महाजया-सप्तमी\eventsep पापनाशनी-सप्तमी\eventsep सङ्क्रमण-दिन-पूर्वाह्ण-पुण्यकालः~06:00\RIGHTarrow{}12:12\eventsep सिंह-रवि-सङ्क्रमण-विष्णुपदी-पुण्यकालः~06:00\RIGHTarrow{}12:46\eventsep तुलसीदास-जयन्ती\eventsep शीतला-सप्तमी}
{Fri} 
\cfoot{\rygdata{10:39--12:12}{15:19--16:52}{07:33--09:06}}
\caldata{AUGUST}{18}{\sunmonth{सिंहः}{2}{}{श्रावणः}{वर्षऋतुः}{शनिः}{विलम्बः}{दक्षिणायनम्}{वर्षऋतुः}}
{\sunmoonrsdata{06:00}{18:25}{12:10}{23:55}{12:12}
{\kalas{04:27 05:14 09:19 08:29 10:08 16:45 10:58 13:27 15:56 17:35 19:11 21:19 22:45 01:39*}}}
{\tnykdata{\anga{\tithi{8}{शुक्ल-अष्टमी}}{\time{49-27}{01:47*}}\hspace{1ex}}%
{\anga{विशाखा}{\time{28-15}{17:18}}\hspace{1ex}}{चन्द्रराशिः—\mbox{तुला\RIGHTarrow{10:56}}}%
{\anga{ब्रह्म}{\time{22-2}{14:49}}\hspace{1ex}\uanga{इन्द्रः}}%
{\anga{विष्टिः}{\time{18-15}{13:18}}\hspace{1ex}\anga{बवः}{\time{49-27}{01:47*}}\hspace{1ex}\uanga{बालवः}}{}
}
{दुर्गा-व्रत-आरम्भः}
{Sat} 
\cfoot{\rygdata{09:06--10:39}{13:45--15:19}{06:00--07:33}}
\caldata{AUGUST}{19}{\sunmonth{सिंहः}{3}{}{श्रावणः}{वर्षऋतुः}{भानुः}{विलम्बः}{दक्षिणायनम्}{वर्षऋतुः}}
{\sunmoonrsdata{06:00}{18:24}{13:03}{00:43*}{12:12}
{\kalas{04:27 05:14 09:19 08:29 10:08 16:45 10:58 13:26 15:55 17:34 19:10 21:18 22:45 01:39*}}}
{\tnykdata{\anga{\tithi{9}{शुक्ल-नवमी}}{\time{53-6}{03:15*}}\hspace{1ex}}%
{\anga{अनूराधा}{\time{32-56}{19:11}}\hspace{1ex}}{चन्द्रराशिः—\mbox{वृश्चिकः}}%
{\anga{इन्द्रः}{\time{21-51}{14:45}}\hspace{1ex}\uanga{वैधृतिः}}%
{\anga{बालवः}{\time{21-4}{14:26}}\hspace{1ex}\anga{कौलवः}{\time{53-6}{03:15*}}\hspace{1ex}\uanga{तैतिलः}}{}
}
{\tamil{ஆவணி~ஞாயிற்றுக்கிழமை}\eventsep कौमारी-पूजा}
{Sun} 
\cfoot{\rygdata{16:51--18:24}{12:12--13:45}{15:18--16:51}}
\caldata{AUGUST}{20}{\sunmonth{सिंहः}{4}{}{श्रावणः}{वर्षऋतुः}{सोमः}{विलम्बः}{दक्षिणायनम्}{वर्षऋतुः}}
{\sunmoonrsdata{06:00}{18:24}{13:54}{01:32*}{12:12}
{\kalas{04:27 05:14 09:18 08:29 10:08 16:44 10:57 13:26 15:55 17:34 19:10 21:18 22:45 01:39*}}}
{\tnykdata{\anga{\tithi{10}{शुक्ल-दशमी}}{\time{58-10}{05:16*}}\hspace{1ex}}%
{\anga{ज्येष्ठा}{\time{39-5}{21:38}}\hspace{1ex}}{चन्द्रराशिः—\mbox{वृश्चिकः\RIGHTarrow{21:38}}}%
{\anga{वैधृतिः}{\time{22-54}{15:10}}\hspace{1ex}\uanga{विष्कम्भः}}%
{\anga{तैतिलः}{\time{25-29}{16:12}}\hspace{1ex}\anga{गरः}{\time{58-10}{05:16*}}\hspace{1ex}\uanga{वणिजः}}{}
}
{काञ्ची ५७ जगद्गुरु श्री-परमशिवेन्द्र सरस्वती २ आराधना~\#{४३३}\eventsep वेद/दधि-व्रत-आरम्भः\eventsep वैधृति-श्राद्धम्}
{Mon} 
\cfoot{\rygdata{07:33--09:06}{10:39--12:12}{13:45--15:18}}
\caldata{AUGUST}{21}{\sunmonth{सिंहः}{5}{}{श्रावणः}{वर्षऋतुः}{मङ्गलः}{विलम्बः}{दक्षिणायनम्}{वर्षऋतुः}}
{\sunmoonrsdata{06:00}{18:23}{14:44}{02:21*}{12:12}
{\kalas{04:27 05:14 09:18 08:29 10:08 16:44 10:57 13:26 15:54 17:33 19:09 21:17 22:44 01:39*}}}
{\tnykdata{\fulltithi{\tithi{11}{शुक्ल-एकादशी}}}%
{\anga{मूला}{\time{46-15}{00:31*}}\hspace{1ex}}{चन्द्रराशिः—\mbox{धनुः}}%
{\anga{विष्कम्भः}{\time{24-51}{15:57}}\hspace{1ex}\uanga{प्रीतिः}}%
{\anga{वणिजः}{\time{31-4}{18:26}}\hspace{1ex}\uanga{विष्टिः}}{}
}
{गजेन्द्र-मोक्षः\eventsep काञ्ची २९ जगद्गुरु श्री-पूर्णबोधेन्द्र सरस्वती आराधना~\#{१४०१}\eventsep \tamil{குங்கிலியக்கலய நாயனார் (10) குருபூஜை}\eventsep \tamil{பிட்டுக்கு மண் சுமந்த லீலை}}
{Tue} 
\cfoot{\rygdata{15:17--16:50}{09:06--10:39}{12:12--13:44}}
\caldata{AUGUST}{22}{\sunmonth{सिंहः}{6}{}{श्रावणः}{वर्षऋतुः}{बुधः}{विलम्बः}{दक्षिणायनम्}{वर्षऋतुः}}
{\sunmoonrsdata{06:00}{18:22}{15:32}{03:11*}{12:11}
{\kalas{04:27 05:14 09:18 08:29 10:08 16:43 10:57 13:26 15:54 17:33 19:09 21:17 22:44 01:39*}}}
{\tnykdata{\anga{\tithi{11}{शुक्ल-एकादशी}}{\time{4-9}{07:40}}\hspace{1ex}}%
{\anga{पूर्वाषाढा}{\time{54-0}{03:37*}}\hspace{1ex}}{चन्द्रराशिः—\mbox{धनुः}}%
{\anga{प्रीतिः}{\time{27-21}{16:57}}\hspace{1ex}\uanga{आयुष्मान्}}%
{\anga{विष्टिः}{\time{4-9}{07:40}}\hspace{1ex}\anga{बवः}{\time{37-21}{20:57}}\hspace{1ex}\uanga{बालवः}}{}
}
{आयुष्मद्-बव-सौम्य-संयॊगः\eventsep सर्व-पवित्रोपान-एकादशी\eventsep उन्मीलनी महाद्वादशी}
{Wed} 
\cfoot{\rygdata{12:11--13:44}{07:33--09:06}{10:39--12:11}}
\caldata{AUGUST}{23}{\sunmonth{सिंहः}{7}{}{श्रावणः}{वर्षऋतुः}{गुरुः}{विलम्बः}{दक्षिणायनम्}{वर्षऋतुः}}
{\sunmoonrsdata{06:00}{18:22}{16:18}{04:01*}{12:11}
{\kalas{04:27 05:14 09:18 08:29 10:07 16:43 10:57 13:25 15:54 17:32 19:08 21:16 22:44 01:38*}}}
{\tnykdata{\anga{\tithi{12}{शुक्ल-द्वादशी}}{\time{10-37}{10:15}}\hspace{1ex}}%
{\fullanga{उत्तराषाढा}}{चन्द्रराशिः—\mbox{धनुः\RIGHTarrow{10:24}}}%
{\anga{आयुष्मान्}{\time{30-5}{18:03}}\hspace{1ex}\uanga{सौभाग्यः}}%
{\anga{बालवः}{\time{10-37}{10:15}}\hspace{1ex}\anga{कौलवः}{\time{43-52}{23:33}}\hspace{1ex}\uanga{तैतिलः}}{}
}
{(सायन) षडशीति-पुण्यकालः~09:38\RIGHTarrow{}18:22\eventsep दामोदर-द्वादशी\eventsep दधि-व्रत-आरम्भः\eventsep नभस्य-मासः~09:38\RIGHTarrow{}\eventsep प्रदोष-व्रतम्~18:22\RIGHTarrow{}19:49\eventsep रवि-सङ्क्रमण-पुण्यकालः~06:00\RIGHTarrow{}16:02\eventsep सायन-सङ्क्रमण-दिन-पूर्वाह्ण-पुण्यकालः~06:00\RIGHTarrow{}12:11\eventsep शाकव्रत-समापनम्}
{Thu} 
\cfoot{\rygdata{13:44--15:16}{06:00--07:33}{09:06--10:38}}
\caldata{AUGUST}{24}{\sunmonth{सिंहः}{8}{}{श्रावणः}{वर्षऋतुः}{शुक्रः}{विलम्बः}{दक्षिणायनम्}{वर्षऋतुः}}
{\sunmoonrsdata{06:00}{18:21}{17:01}{04:50*}{12:11}
{\kalas{04:27 05:14 09:18 08:29 10:07 16:42 10:57 13:25 15:53 17:32 19:08 21:16 22:43 01:38*}}}
{\tnykdata{\anga{\tithi{13}{शुक्ल-त्रयोदशी}}{\time{17-4}{12:50}}\hspace{1ex}}%
{\anga{उत्तराषाढा}{\time{1-50}{06:45}}\hspace{1ex}}{चन्द्रराशिः—\mbox{मकरः}}%
{\anga{सौभाग्यः}{\time{32-43}{19:06}}\hspace{1ex}\uanga{शोभनः}}%
{\anga{तैतिलः}{\time{17-4}{12:50}}\hspace{1ex}\anga{गरः}{\time{50-10}{02:05*}}\hspace{1ex}\uanga{वणिजः}}{}
}
{अनङ्ग-त्रयोदशी\eventsep \tamil{நடராஜர் மஹாபிஷேகம்}\eventsep वरलक्ष्मी-व्रतम्\eventsep श्रवण-व्रतम्}
{Fri} 
\cfoot{\rygdata{10:38--12:11}{15:16--16:49}{07:33--09:06}}
\caldata{AUGUST}{25}{\sunmonth{सिंहः}{9}{}{श्रावणः}{वर्षऋतुः}{शनिः}{विलम्बः}{दक्षिणायनम्}{वर्षऋतुः}}
{\sunmoonrsdata{06:01}{18:21}{17:42}{05:38*}{12:11}
{\kalas{04:27 05:14 09:18 08:29 10:07 16:42 10:57 13:25 15:53 17:31 19:07 21:16 22:43 01:38*}}}
{\tnykdata{\anga{\tithi{14}{शुक्ल-चतुर्दशी}}{\time{23-8}{15:16}}\hspace{1ex}}%
{\anga{श्रवणः}{\time{9-23}{09:46}}\hspace{1ex}}{चन्द्रराशिः—\mbox{मकरः\RIGHTarrow{23:12}}}%
{\anga{शोभनः}{\time{35-0}{20:01}}\hspace{1ex}\uanga{अतिगण्डः}}%
{\anga{वणिजः}{\time{23-8}{15:16}}\hspace{1ex}\anga{विष्टिः}{\time{55-56}{04:23*}}\hspace{1ex}\uanga{बवः}}{}
}
{\tamil{ஓணம்}\eventsep ऋग्वेद-उपाकर्म\eventsep पञ्च-पर्व-पूजा (पूर्णिमा)\eventsep वेङ्कटाचले पूर्णिमा-गरुड-सेवा}
{Sat} 
\cfoot{\rygdata{09:06--10:38}{13:43--15:16}{06:01--07:33}}
\caldata{AUGUST}{26}{\sunmonth{सिंहः}{10}{}{श्रावणः}{वर्षऋतुः}{भानुः}{विलम्बः}{दक्षिणायनम्}{वर्षऋतुः}}
{\sunmoonrsdata{06:01}{18:20}{18:21}{---}{12:10}
{\kalas{04:27 05:14 09:18 08:28 10:07 16:41 10:56 13:24 15:52 17:31 19:07 21:15 22:43 01:38*}}}
{\tnykdata{\anga{\tithi{15}{पौर्णमासी}}{\time{28-33}{17:26}}\hspace{1ex}}%
{\anga{श्रविष्ठा}{\time{16-21}{12:33}}\hspace{1ex}}{चन्द्रराशिः—\mbox{कुम्भः}}%
{\anga{अतिगण्डः}{\time{36-45}{20:43}}\hspace{1ex}\uanga{सुकर्म}}%
{\anga{बवः}{\time{28-33}{17:26}}\hspace{1ex}\uanga{बालवः}}{}
}
{\tamil{ஆவணி~ஞாயிற்றுக்கிழமை}\eventsep गायत्री-जयन्ती\eventsep हयग्रीव-जयन्ती\eventsep काञ्ची २० जगद्गुरु श्री-मूकशङ्करेन्द्र सरस्वती आराधना~\#{१५८२}\eventsep नारिकेल-पूर्णिमा\eventsep पार्वणव्रतम् पूर्णिमायाम्\eventsep पूर्णिमा-व्रतम्\eventsep रक्षाबन्धनम्\eventsep संस्कृत-दिवसः\eventsep सर्प-बलि-प्रारम्भः\eventsep श्रावण्युपवासः प्रायश्चित्तार्थः\eventsep वैखानस-महर्षि-जयन्ती\eventsep यजुर्वेद-उपाकर्म}
{Sun} 
\cfoot{\rygdata{16:48--18:20}{12:10--13:43}{15:15--16:48}}
\caldata{AUGUST}{27}{\sunmonth{सिंहः}{11}{}{श्रावणः}{वर्षऋतुः}{सोमः}{विलम्बः}{दक्षिणायनम्}{वर्षऋतुः}}
{\sunmoonsrdata{06:01}{18:19}{19:00}{06:25}{12:10}
{\kalas{04:27 05:14 09:18 08:28 10:07 16:41 10:56 13:24 15:52 17:30 19:06 21:15 22:42 01:38*}}}
{\tnykdata{\anga{\tithi{16}{कृष्ण-प्रथमा}}{\time{33-5}{19:15}}\hspace{1ex}}%
{\anga{शतभिषक्}{\time{22-30}{15:01}}\hspace{1ex}}{चन्द्रराशिः—\mbox{कुम्भः}}%
{\anga{सुकर्म}{\time{37-48}{21:08}}\hspace{1ex}\uanga{धृतिः}}%
{\anga{बालवः}{\time{0-55}{06:23}}\hspace{1ex}\anga{कौलवः}{\time{33-5}{19:15}}\hspace{1ex}\uanga{तैतिलः}}{}
}
{पार्वण-प्रायश्चित्तावकाशः दर्शे\eventsep पूर्णमासेष्टिः\eventsep सहस्रगायत्रीजपः प्रायश्चित्तार्थः\eventsep स्थालीपाकः}
{Mon} 
\cfoot{\rygdata{07:33--09:05}{10:38--12:10}{13:42--15:15}}
\caldata{AUGUST}{28}{\sunmonth{सिंहः}{12}{}{श्रावणः}{वर्षऋतुः}{मङ्गलः}{विलम्बः}{दक्षिणायनम्}{वर्षऋतुः}}
{\sunmoonsrdata{06:01}{18:19}{19:37}{07:13}{12:10}
{\kalas{04:27 05:14 09:18 08:28 10:07 16:40 10:56 13:23 15:51 17:30 19:05 21:14 22:42 01:37*}}}
{\tnykdata{\anga{\tithi{17}{कृष्ण-द्वितीया}}{\time{36-36}{20:39}}\hspace{1ex}}%
{\anga{पूर्वप्रोष्ठपदा}{\time{27-42}{17:06}}\hspace{1ex}}{चन्द्रराशिः—\mbox{कुम्भः\RIGHTarrow{10:37}}}%
{\anga{धृतिः}{\time{38-4}{21:14}}\hspace{1ex}\uanga{शूलः}}%
{\anga{तैतिलः}{\time{4-58}{08:00}}\hspace{1ex}\anga{गरः}{\time{36-36}{20:39}}\hspace{1ex}\uanga{वणिजः}}{}
}
{अशून्यशयन-व्रतम्\eventsep बृहती-वृक्षक-पूजा\eventsep भीम-चण्डी-जयन्ती\eventsep श्री-राघवेन्द्र-स्वामि-आराधना~\#{३४७}}
{Tue} 
\cfoot{\rygdata{15:14--16:46}{09:05--10:37}{12:10--13:42}}
\caldata{AUGUST}{29}{\sunmonth{सिंहः}{13}{}{श्रावणः}{वर्षऋतुः}{बुधः}{विलम्बः}{दक्षिणायनम्}{वर्षऋतुः}}
{\sunmoonsrdata{06:01}{18:18}{20:16}{08:00}{12:09}
{\kalas{04:27 05:14 09:17 08:28 10:06 16:40 10:56 13:23 15:51 17:29 19:05 21:14 22:42 01:37*}}}
{\tnykdata{\anga{\tithi{18}{कृष्ण-तृतीया}}{\time{39-3}{21:38}}\hspace{1ex}}%
{\anga{उत्तरप्रोष्ठपदा}{\time{31-51}{18:45}}\hspace{1ex}}{चन्द्रराशिः—\mbox{मीनः}}%
{\anga{शूलः}{\time{37-30}{21:01}}\hspace{1ex}\uanga{गण्डः}}%
{\anga{वणिजः}{\time{7-58}{09:12}}\hspace{1ex}\anga{विष्टिः}{\time{39-3}{21:38}}\hspace{1ex}\uanga{बवः}}{}
}
{कज्जली-तृतीया\eventsep तुष्टि-प्राप्ति-तृतीया}
{Wed} 
\cfoot{\rygdata{12:09--13:42}{07:33--09:05}{10:37--12:09}}
\caldata{AUGUST}{30}{\sunmonth{सिंहः}{14}{}{श्रावणः}{वर्षऋतुः}{गुरुः}{विलम्बः}{दक्षिणायनम्}{वर्षऋतुः}}
{\sunmoonsrdata{06:01}{18:18}{20:56}{08:49}{12:09}
{\kalas{04:27 05:14 09:17 08:28 10:06 16:39 10:55 13:23 15:50 17:28 19:04 21:13 22:41 01:37*}}}
{\tnykdata{\anga{\tithi{19}{कृष्ण-चतुर्थी}}{\time{40-21}{22:09}}\hspace{1ex}}%
{\anga{रेवती}{\time{34-54}{19:58}}\hspace{1ex}}{चन्द्रराशिः—\mbox{मीनः\RIGHTarrow{19:58}}}%
{\anga{गण्डः}{\time{36-1}{20:25}}\hspace{1ex}\uanga{वृद्धिः}}%
{\anga{बवः}{\time{9-51}{09:57}}\hspace{1ex}\anga{बालवः}{\time{40-21}{22:09}}\hspace{1ex}\uanga{कौलवः}}{}
}
{बहुला-चतुर्थी\eventsep हेरम्ब-महागणपति महासङ्कटहर-चतुर्थी-व्रतम्}
{Thu} 
\cfoot{\rygdata{13:41--15:13}{06:01--07:33}{09:05--10:37}}
\caldata{AUGUST}{31}{\sunmonth{सिंहः}{15}{}{श्रावणः}{वर्षऋतुः}{शुक्रः}{विलम्बः}{दक्षिणायनम्}{वर्षऋतुः}}
{\sunmoonsrdata{06:01}{18:17}{21:38}{09:40}{12:09}
{\kalas{04:27 05:14 09:17 08:28 10:06 16:39 10:55 13:22 15:50 17:28 19:04 21:13 22:41 01:37*}}}
{\tnykdata{\anga{\tithi{20}{कृष्ण-पञ्चमी}}{\time{40-26}{22:12}}\hspace{1ex}}%
{\anga{अश्विनी}{\time{36-46}{20:44}}\hspace{1ex}}{चन्द्रराशिः—\mbox{मेषः}}%
{\anga{वृद्धिः}{\time{33-37}{19:28}}\hspace{1ex}\uanga{ध्रुवः}}%
{\anga{कौलवः}{\time{10-32}{10:14}}\hspace{1ex}\anga{तैतिलः}{\time{40-26}{22:12}}\hspace{1ex}\uanga{गरः}}{}
}
{रक्षा-पञ्चमी}
{Fri} 
\cfoot{\rygdata{10:37--12:09}{15:13--16:45}{07:33--09:05}}
\caldata{SEPTEMBER}{1}{\sunmonth{सिंहः}{16}{}{श्रावणः}{वर्षऋतुः}{शनिः}{विलम्बः}{दक्षिणायनम्}{वर्षऋतुः}}
{\sunmoonsrdata{06:01}{18:16}{22:24}{10:33}{12:08}
{\kalas{04:27 05:14 09:17 08:28 10:06 16:38 10:55 13:22 15:49 17:27 19:03 21:12 22:40 01:37*}}}
{\tnykdata{\anga{\tithi{21}{कृष्ण-षष्ठी}}{\time{39-18}{21:44}}\hspace{1ex}}%
{\anga{अपभरणी}{\time{37-26}{21:00}}\hspace{1ex}}{चन्द्रराशिः—\mbox{मेषः\RIGHTarrow{02:59*}}}%
{\anga{ध्रुवः}{\time{30-13}{18:06}}\hspace{1ex}\uanga{व्याघातः}}%
{\anga{गरः}{\time{10-1}{10:02}}\hspace{1ex}\anga{वणिजः}{\time{39-18}{21:44}}\hspace{1ex}\uanga{विष्टिः}}{}
}
{हल-षष्ठी}
{Sat} 
\cfoot{\rygdata{09:05--10:37}{13:40--15:12}{06:01--07:33}}
\caldata{SEPTEMBER}{2}{\sunmonth{सिंहः}{17}{}{श्रावणः}{वर्षऋतुः}{भानुः}{विलम्बः}{दक्षिणायनम्}{वर्षऋतुः}}
{\sunmoonsrdata{06:01}{18:15}{23:15}{11:29}{12:08}
{\kalas{04:27 05:14 09:17 08:28 10:06 16:38 10:55 13:22 15:48 17:27 19:02 21:12 22:40 01:36*}}}
{\tnykdata{\anga{\tithi{22}{कृष्ण-सप्तमी}}{\time{36-55}{20:47}}\hspace{1ex}}%
{\anga{कृत्तिका}{\time{36-52}{20:46}}\hspace{1ex}}{चन्द्रराशिः—\mbox{वृषभः}}%
{\anga{व्याघातः}{\time{25-50}{16:21}}\hspace{1ex}\uanga{हर्षणः}}%
{\anga{विष्टिः}{\time{8-16}{09:19}}\hspace{1ex}\anga{बवः}{\time{36-55}{20:47}}\hspace{1ex}\uanga{बालवः}}{}
}
{\tamil{ஆவணி~ஞாயிற்றுக்கிழமை}\eventsep भानुसप्तमी\eventsep कृत्तिका-व्रतम्\eventsep पञ्च-पर्व-पूजा (अष्टमी)\eventsep श्री-जयन्ती\eventsep श्रीकृष्णजन्माष्टमी}
{Sun} 
\cfoot{\rygdata{16:44--18:15}{12:08--13:40}{15:12--16:44}}
\caldata{SEPTEMBER}{3}{\sunmonth{सिंहः}{18}{}{श्रावणः}{वर्षऋतुः}{सोमः}{विलम्बः}{दक्षिणायनम्}{वर्षऋतुः}}
{\sunmoonsrdata{06:01}{18:15}{00:10*}{12:27}{12:08}
{\kalas{04:27 05:14 09:16 08:28 10:05 16:37 10:54 13:21 15:48 17:26 19:02 21:11 22:40 01:36*}}}
{\tnykdata{\anga{\tithi{23}{कृष्ण-अष्टमी}}{\time{33-16}{19:20}}\hspace{1ex}}%
{\anga{रोहिणी}{\time{35-4}{20:02}}\hspace{1ex}}{चन्द्रराशिः—\mbox{वृषभः}}%
{\anga{हर्षणः}{\time{20-26}{14:11}}\hspace{1ex}\uanga{वज्रम्}}%
{\anga{बालवः}{\time{5-15}{08:07}}\hspace{1ex}\anga{कौलवः}{\time{33-16}{19:20}}\hspace{1ex}\uanga{तैतिलः}}{}
}
{एकविंशति-दिवस-गणपति-व्रत-समापनम्\eventsep काञ्ची २१ जगद्गुरु श्री-सार्वभौमगुरुः चन्द्रचूडेन्द्र सरस्वती आराधना~\#{१५७२}\eventsep मन्वादिः-(दक्षः-[९])\eventsep नन्दोत्सवः\eventsep \tamil{திருச்செந்தூர் முருகன் ஆவணித் திருவிழா தொடக்கம்/கொடியேற்றம்}\eventsep \tamil{வரகூர் உறியடி உத்ஸவம்}\eventsep श्रीकृष्णदेवराय-राज्याभिषेकः}
{Mon} 
\cfoot{\rygdata{07:33--09:04}{10:36--12:08}{13:40--15:11}}
\caldata{SEPTEMBER}{4}{\sunmonth{सिंहः}{19}{}{श्रावणः}{वर्षऋतुः}{मङ्गलः}{विलम्बः}{दक्षिणायनम्}{वर्षऋतुः}}
{\sunmoonsrdata{06:01}{18:14}{01:09*}{13:27}{12:07}
{\kalas{04:27 05:14 09:16 08:28 10:05 16:36 10:54 13:21 15:47 17:25 19:01 21:11 22:39 01:36*}}}
{\tnykdata{\anga{\tithi{24}{कृष्ण-नवमी}}{\time{28-26}{17:23}}\hspace{1ex}}%
{\anga{मृगशीर्षम्}{\time{32-3}{18:50}}\hspace{1ex}}{चन्द्रराशिः—\mbox{वृषभः\RIGHTarrow{07:30}}}%
{\anga{वज्रम्}{\time{14-2}{11:38}}\hspace{1ex}\uanga{सिद्धिः}}%
{\anga{तैतिलः}{\time{1-0}{06:25}}\hspace{1ex}\anga{गरः}{\time{28-26}{17:23}}\hspace{1ex}\anga{वणिजः}{\time{55-35}{04:15*}}\hspace{1ex}\uanga{विष्टिः}}{}
}
{अरविन्द-जयन्ती\eventsep चण्डिका-पूजा\eventsep काञ्ची २४ जगद्गुरु श्री-चित्सुखेन्द्र सरस्वती आराधना~\#{१४९२}\eventsep कौमार-पूजा\eventsep \tamil{திருச்செந்தூர் முருகன் ஆவணித் திருவிழா 2ம் நாள்}}
{Tue} 
\cfoot{\rygdata{15:11--16:42}{09:04--10:36}{12:07--13:39}}
\caldata{SEPTEMBER}{5}{\sunmonth{सिंहः}{20}{}{श्रावणः}{वर्षऋतुः}{बुधः}{विलम्बः}{दक्षिणायनम्}{वर्षऋतुः}}
{\sunmoonsrdata{06:01}{18:13}{02:12*}{14:27}{12:07}
{\kalas{04:27 05:14 09:16 08:27 10:05 16:36 10:54 13:20 15:47 17:25 19:01 21:10 22:39 01:36*}}}
{\tnykdata{\anga{\tithi{25}{कृष्ण-दशमी}}{\time{22-28}{15:01}}\hspace{1ex}}%
{\anga{आर्द्रा}{\time{27-57}{17:12}}\hspace{1ex}}{चन्द्रराशिः—\mbox{मिथुनम्}}%
{\anga{सिद्धिः}{\time{6-42}{08:42}}\hspace{1ex}\anga{व्यतीपातः}{\time{58-33}{05:27*}}\hspace{1ex}\uanga{वरीयान्}}%
{\anga{विष्टिः}{\time{22-28}{15:01}}\hspace{1ex}\anga{बवः}{\time{49-8}{01:40*}}\hspace{1ex}\uanga{बालवः}}{}
}
{\tamil{திருச்செந்தூர் முருகன் ஆவணித் திருவிழா 3ம் நாள்—முருகன் பவனி}\eventsep व्यतीपात-श्राद्धम्}
{Wed} 
\cfoot{\rygdata{12:07--13:39}{07:32--09:04}{10:36--12:07}}
\caldata{SEPTEMBER}{6}{\sunmonth{सिंहः}{21}{}{श्रावणः}{वर्षऋतुः}{गुरुः}{विलम्बः}{दक्षिणायनम्}{वर्षऋतुः}}
{\sunmoonsrdata{06:01}{18:13}{03:15*}{15:25}{12:07}
{\kalas{04:27 05:14 09:16 08:27 10:05 16:35 10:54 13:20 15:46 17:24 19:00 21:10 22:38 01:35*}}}
{\tnykdata{\anga{\tithi{26}{कृष्ण-एकादशी}}{\time{15-35}{12:15}}\hspace{1ex}}%
{\anga{पुनर्वसुः}{\time{22-55}{15:11}}\hspace{1ex}}{चन्द्रराशिः—\mbox{मिथुनम्\RIGHTarrow{09:43}}}%
{\anga{वरीयान्}{\time{49-45}{01:55*}}\hspace{1ex}\uanga{परिघः}}%
{\anga{बालवः}{\time{15-35}{12:15}}\hspace{1ex}\anga{कौलवः}{\time{41-51}{22:45}}\hspace{1ex}\uanga{तैतिलः}}{}
}
{गुरुपुष्य-पुण्यकालः~15:11\RIGHTarrow{}\eventsep सर्व-अजा-एकादशी\eventsep \tamil{திருச்செந்தூர் முருகன் ஆவணித் திருவிழா 4ம் நாள்—யானை வாஹநத்தில் முருகன்-அம்பாள் பவனி}}
{Thu} 
\cfoot{\rygdata{13:38--15:10}{06:01--07:32}{09:04--10:35}}
\caldata{SEPTEMBER}{7}{\sunmonth{सिंहः}{22}{}{श्रावणः}{वर्षऋतुः}{शुक्रः}{विलम्बः}{दक्षिणायनम्}{वर्षऋतुः}}
{\sunmoonsrdata{06:01}{18:12}{04:17*}{16:20}{12:06}
{\kalas{04:26 05:14 09:16 08:27 10:05 16:35 10:53 13:20 15:46 17:23 18:59 21:09 22:38 01:35*}}}
{\tnykdata{\anga{\tithi{27}{कृष्ण-द्वादशी}}{\time{7-58}{09:12}}\hspace{1ex}\anga{\tithi{28}{कृष्ण-त्रयोदशी}}{\time{59-53}{05:58*}}\hspace{1ex}\avamA{}}%
{\anga{पुष्यः}{\time{17-12}{12:54}}\hspace{1ex}}{चन्द्रराशिः—\mbox{कर्कटः}}%
{\anga{परिघः}{\time{40-28}{22:12}}\hspace{1ex}\uanga{शिवः}}%
{\anga{तैतिलः}{\time{7-58}{09:12}}\hspace{1ex}\anga{गरः}{\time{33-58}{19:36}}\hspace{1ex}\anga{वणिजः}{\time{59-53}{05:58*}}\hspace{1ex}\uanga{विष्टिः}}{}
}
{\tamil{செருத்துணை நாயனார் (53) குருபூஜை}\eventsep जया-महाद्वादशी\eventsep प्रदोष-व्रतम्~18:12\RIGHTarrow{}19:41\eventsep रोहिणी-द्वादशी\eventsep \tamil{திருச்செந்தூர் முருகன் ஆவணித் திருவிழா 5ம் நாள்}}
{Fri} 
\cfoot{\rygdata{10:35--12:06}{15:09--16:41}{07:32--09:04}}
\caldata{SEPTEMBER}{8}{\sunmonth{सिंहः}{23}{}{श्रावणः}{वर्षऋतुः}{शनिः}{विलम्बः}{दक्षिणायनम्}{वर्षऋतुः}}
{\sunmoonsrdata{06:01}{18:11}{05:18*}{17:12}{12:06}
{\kalas{04:26 05:14 09:16 08:27 10:04 16:34 10:53 13:19 15:45 17:23 18:59 21:09 22:37 01:35*}}}
{\tnykdata{\anga{\tithi{29}{कृष्ण-चतुर्दशी}}{\time{51-42}{02:42*}}\hspace{1ex}}%
{\anga{आश्रेषा}{\time{11-4}{10:27}}\hspace{1ex}}{चन्द्रराशिः—\mbox{कर्कटः\RIGHTarrow{10:27}}}%
{\anga{शिवः}{\time{30-59}{18:25}}\hspace{1ex}\uanga{सिद्धः}}%
{\anga{विष्टिः}{\time{25-47}{16:20}}\hspace{1ex}\anga{शकुनिः}{\time{51-42}{02:42*}}\hspace{1ex}\uanga{चतुष्पात्}}{}
}
{अघोर-चतुर्दशी\eventsep \tamil{அதிபத்த நாயனார் (41) குருபூஜை}\eventsep \tamil{இளையான்குடி மாற நாயனார் (3) குருபூஜை}\eventsep मासशिवरात्रिः\eventsep पञ्च-पर्व-पूजा (चतुर्दशी)\eventsep \tamil{திருச்செந்தூர் முருகன் ஆவணித் திருவிழா 6ம் நாள்—வெள்ளித் தேர் பவனி}}
{Sat} 
\cfoot{\rygdata{09:04--10:35}{13:37--15:09}{06:01--07:32}}
\caldata{SEPTEMBER}{9}{\sunmonth{सिंहः}{24}{}{श्रावणः}{वर्षऋतुः}{भानुः}{विलम्बः}{दक्षिणायनम्}{वर्षऋतुः}}
{\sunmoonsrdata{06:01}{18:11}{---}{18:00}{12:06}
{\kalas{04:26 05:14 09:15 08:27 10:04 16:33 10:53 13:19 15:45 17:22 18:58 21:08 22:37 01:35*}}}
{\tnykdata{\anga{\tithi{30}{अमावास्या}}{\time{43-45}{23:31}}\hspace{1ex}}%
{\anga{मघा}{\time{4-54}{07:59}}\hspace{1ex}\anga{पूर्वफल्गुनी}{\time{59-4}{05:39*}}\hspace{1ex}}{चन्द्रराशिः—\mbox{सिंहः}}%
{\anga{सिद्धः}{\time{21-33}{14:38}}\hspace{1ex}\uanga{साध्यः}}%
{\anga{चतुष्पात्}{\time{17-40}{13:05}}\hspace{1ex}\anga{नाग}{\time{43-45}{23:31}}\hspace{1ex}\uanga{किंस्तुघ्नः}}{}
}
{६४ योगिनी-पूजा\eventsep \tamil{ஆவணி~ஞாயிற்றுக்கிழமை}\eventsep दर्भ-सङ्ग्रहः\eventsep देवी-पर्व-५\eventsep पार्वणव्रतम् अमावास्यायाम्\eventsep पञ्च-पर्व-पूजा (अमावास्या)\eventsep सर्व-श्रावण-अमावास्या\eventsep \tamil{திருச்செந்தூர் முருகன் ஆவணித் திருவிழா 7ம் நாள்—சிகப்பு சாத்தி அலங்காரம்}\eventsep वृषभ-पूजा}
{Sun} 
\cfoot{\rygdata{16:39--18:11}{12:06--13:37}{15:08--16:39}}
\caldata{SEPTEMBER}{10}{\sunmonth{सिंहः}{25}{}{भाद्रपदः}{वर्षऋतुः}{सोमः}{विलम्बः}{दक्षिणायनम्}{वर्षऋतुः}}
{\sunmoonrsdata{06:01}{18:10}{06:17}{18:46}{12:05}
{\kalas{04:26 05:14 09:15 08:27 10:04 16:33 10:53 13:18 15:44 17:21 18:57 21:08 22:37 01:34*}}}
{\tnykdata{\anga{\tithi{1}{शुक्ल-प्रथमा}}{\time{36-25}{20:35}}\hspace{1ex}}%
{\anga{उत्तरफल्गुनी}{\time{53-58}{03:37*}}\hspace{1ex}}{चन्द्रराशिः—\mbox{सिंहः\RIGHTarrow{11:06}}}%
{\anga{साध्यः}{\time{12-32}{11:02}}\hspace{1ex}\uanga{शुभः}}%
{\anga{किंस्तुघ्नः}{\time{9-59}{10:01}}\hspace{1ex}\anga{बवः}{\time{36-25}{20:35}}\hspace{1ex}\uanga{बालवः}}{}
}
{दर्शेष्टिः\eventsep मृगशीर्ष-व्रतम्\eventsep पार्वण-प्रायश्चित्तावकाशः पौर्णमास्याम्\eventsep स्थालीपाकः\eventsep \tamil{திருச்செந்தூர் முருகன் ஆவணித் திருவிழா 8ம் நாள்—பச்சை சாத்தி அலங்காரம்}}
{Mon} 
\cfoot{\rygdata{07:32--09:03}{10:34--12:05}{13:37--15:08}}
\caldata{SEPTEMBER}{11}{\sunmonth{सिंहः}{26}{}{भाद्रपदः}{वर्षऋतुः}{मङ्गलः}{विलम्बः}{दक्षिणायनम्}{वर्षऋतुः}}
{\sunmoonrsdata{06:01}{18:09}{07:14}{19:31}{12:05}
{\kalas{04:26 05:14 09:15 08:27 10:04 16:32 10:52 13:18 15:44 17:21 18:57 21:07 22:36 01:34*}}}
{\tnykdata{\anga{\tithi{2}{शुक्ल-द्वितीया}}{\time{30-8}{18:04}}\hspace{1ex}}%
{\anga{हस्तः}{\time{50-3}{02:02*}}\hspace{1ex}}{चन्द्रराशिः—\mbox{कन्या}}%
{\anga{शुभः}{\time{4-11}{07:42}}\hspace{1ex}\anga{शुक्लः}{\time{56-52}{04:46*}}\hspace{1ex}\uanga{ब्रह्म}}%
{\anga{बालवः}{\time{3-7}{07:16}}\hspace{1ex}\anga{कौलवः}{\time{30-8}{18:04}}\hspace{1ex}\anga{तैतिलः}{\time{57-29}{05:01*}}\hspace{1ex}\uanga{गरः}}{}
}
{अङ्गारक-जयन्ती\eventsep चन्द्र-दर्शनम्~18:09\RIGHTarrow{}19:31\eventsep कल्कि-जयन्ती\eventsep सामवेद-उपाकर्म\eventsep \tamil{திருச்செந்தூர் முருகன் ஆவணித் திருவிழா 9ம் நாள்}\eventsep विवेकानन्द-भाषणं चिकागोनगरे~\#{१२५}}
{Tue} 
\cfoot{\rygdata{15:07--16:38}{09:03--10:34}{12:05--13:36}}
\caldata{SEPTEMBER}{12}{\sunmonth{सिंहः}{27}{}{भाद्रपदः}{वर्षऋतुः}{बुधः}{विलम्बः}{दक्षिणायनम्}{वर्षऋतुः}}
{\sunmoonrsdata{06:01}{18:09}{08:10}{20:16}{12:05}
{\kalas{04:26 05:14 09:15 08:27 10:03 16:32 10:52 13:18 15:43 17:20 18:56 21:07 22:36 01:34*}}}
{\tnykdata{\anga{\tithi{3}{शुक्ल-तृतीया}}{\time{25-15}{16:07}}\hspace{1ex}}%
{\anga{चित्रा}{\time{47-39}{01:05*}}\hspace{1ex}}{चन्द्रराशिः—\mbox{कन्या\RIGHTarrow{13:28}}}%
{\anga{ब्रह्म}{\time{50-49}{02:21*}}\hspace{1ex}\uanga{इन्द्रः}}%
{\anga{गरः}{\time{25-15}{16:07}}\hspace{1ex}\anga{वणिजः}{\time{53-26}{03:24*}}\hspace{1ex}\uanga{विष्टिः}}{}
}
{हरितालिका-व्रतम्\eventsep मन्वादिः-(तामसः-[४])\eventsep \tamil{திருச்செந்தூர் முருகன் ஆவணித் திருவிழா 10ம் நாள்—தேர்}\eventsep विपत्तार-गौरी-व्रतम्}
{Wed} 
\cfoot{\rygdata{12:05--13:36}{07:32--09:03}{10:34--12:05}}
\caldata{SEPTEMBER}{13}{\sunmonth{सिंहः}{28}{}{भाद्रपदः}{वर्षऋतुः}{गुरुः}{विलम्बः}{दक्षिणायनम्}{वर्षऋतुः}}
{\sunmoonrsdata{06:01}{18:08}{09:05}{21:02}{12:04}
{\kalas{04:26 05:13 09:15 08:26 10:03 16:31 10:52 13:17 15:42 17:19 18:55 21:06 22:35 01:34*}}}
{\tnykdata{\anga{\tithi{4}{शुक्ल-चतुर्थी}}{\time{22-6}{14:51}}\hspace{1ex}}%
{\anga{स्वाती}{\time{47-4}{00:51*}}\hspace{1ex}}{चन्द्रराशिः—\mbox{तुला}}%
{\anga{इन्द्रः}{\time{46-15}{00:31*}}\hspace{1ex}\uanga{वैधृतिः}}%
{\anga{विष्टिः}{\time{22-6}{14:51}}\hspace{1ex}\anga{बवः}{\time{51-15}{02:31*}}\hspace{1ex}\uanga{बालवः}}{}
}
{\tamil{திருச்செந்தூர் முருகன் ஆவணித் திருவிழா 11ம் நாள்}\eventsep श्रीविनायक-चतुर्थी\eventsep शुक्ल-चतुर्थी-व्रतम्}
{Thu} 
\cfoot{\rygdata{13:35--15:06}{06:01--07:32}{09:03--10:34}}
\caldata{SEPTEMBER}{14}{\sunmonth{सिंहः}{29}{}{भाद्रपदः}{वर्षऋतुः}{शुक्रः}{विलम्बः}{दक्षिणायनम्}{वर्षऋतुः}}
{\sunmoonrsdata{06:01}{18:07}{10:00}{21:48}{12:04}
{\kalas{04:26 05:13 09:15 08:26 10:03 16:30 10:51 13:17 15:42 17:19 18:55 21:06 22:35 01:33*}}}
{\tnykdata{\anga{\tithi{5}{शुक्ल-पञ्चमी}}{\time{20-55}{14:23}}\hspace{1ex}}%
{\anga{विशाखा}{\time{48-28}{01:24*}}\hspace{1ex}}{चन्द्रराशिः—\mbox{तुला\RIGHTarrow{19:12}}}%
{\anga{वैधृतिः}{\time{43-19}{23:21}}\hspace{1ex}\uanga{विष्कम्भः}}%
{\anga{बालवः}{\time{20-55}{14:23}}\hspace{1ex}\anga{कौलवः}{\time{51-7}{02:28*}}\hspace{1ex}\uanga{तैतिलः}}{}
}
{ऋषि-पञ्चमी-व्रतम्\eventsep भाद्रपदिक-नाग-पञ्चमी\eventsep \tamil{திருச்செந்தூர் ஆவணித் திருவிழா நிறைவு}\eventsep वैधृति-श्राद्धम्}
{Fri} 
\cfoot{\rygdata{10:33--12:04}{15:06--16:36}{07:32--09:03}}
\caldata{SEPTEMBER}{15}{\sunmonth{सिंहः}{30}{}{भाद्रपदः}{वर्षऋतुः}{शनिः}{विलम्बः}{दक्षिणायनम्}{वर्षऋतुः}}
{\sunmoonrsdata{06:01}{18:06}{10:54}{22:36}{12:04}
{\kalas{04:26 05:13 09:14 08:26 10:03 16:30 10:51 13:16 15:41 17:18 18:54 21:05 22:34 01:33*}}}
{\tnykdata{\anga{\tithi{6}{शुक्ल-षष्ठी}}{\time{21-49}{14:45}}\hspace{1ex}}%
{\anga{अनूराधा}{\time{51-53}{02:46*}}\hspace{1ex}}{चन्द्रराशिः—\mbox{वृश्चिकः}}%
{\anga{विष्कम्भः}{\time{41-59}{22:49}}\hspace{1ex}\uanga{प्रीतिः}}%
{\anga{तैतिलः}{\time{21-49}{14:45}}\hspace{1ex}\anga{गरः}{\time{53-2}{03:14*}}\hspace{1ex}\uanga{वणिजः}}{}
}
{षष्ठीदेवी-षष्ठी-व्रतम्\eventsep \tamil{குலச்சிரை நாயனார் (21) குருபூஜை}\eventsep कुमारिका-स्वपनम्\eventsep ललिता-षष्ठी\eventsep मन्थन-षष्ठी\eventsep सूर्य-षष्ठी}
{Sat} 
\cfoot{\rygdata{09:02--10:33}{13:34--15:05}{06:01--07:32}}
\caldata{SEPTEMBER}{16}{\sunmonth{सिंहः}{31}{}{भाद्रपदः}{वर्षऋतुः}{भानुः}{विलम्बः}{दक्षिणायनम्}{वर्षऋतुः}}
{\sunmoonrsdata{06:01}{18:06}{11:47}{23:25}{12:03}
{\kalas{04:26 05:13 09:14 08:26 10:03 16:29 10:51 13:16 15:41 17:17 18:53 21:05 22:34 01:33*}}}
{\tnykdata{\anga{\tithi{7}{शुक्ल-सप्तमी}}{\time{24-42}{15:54}}\hspace{1ex}}%
{\anga{ज्येष्ठा}{\time{57-7}{04:52*}}\hspace{1ex}}{चन्द्रराशिः—\mbox{वृश्चिकः\RIGHTarrow{04:52*}}}%
{\anga{प्रीतिः}{\time{42-8}{22:52}}\hspace{1ex}\uanga{आयुष्मान्}}%
{\anga{वणिजः}{\time{24-42}{15:54}}\hspace{1ex}\anga{विष्टिः}{\time{56-48}{04:44*}}\hspace{1ex}\uanga{बवः}}{}
}
{\tamil{ஆவணி~ஞாயிற்றுக்கிழமை}\eventsep अमुक्ताभरण-सप्तमी\eventsep अनन्तफल-सप्तमी\eventsep देवी-पर्व-६\eventsep कुक्कुटी-व्रतम्\eventsep महाकाली-जयन्ती\eventsep विजया-भानुसप्तमी}
{Sun} 
\cfoot{\rygdata{16:35--18:06}{12:03--13:34}{15:05--16:35}}
\caldata{SEPTEMBER}{17}{\sunmonth{कन्या}{1}{\mbox{सिंहः{\tiny\RIGHTarrow}{06:15}}}{भाद्रपदः}{वर्षऋतुः}{सोमः}{विलम्बः}{दक्षिणायनम्}{वर्षऋतुः}}
{\sunmoonrsdata{06:01}{18:05}{12:38}{00:15*}{12:03}
{\kalas{04:26 05:13 09:14 08:26 10:02 16:28 10:51 13:15 15:40 17:17 18:53 21:04 22:34 01:33*}}}
{\tnykdata{\anga{\tithi{8}{शुक्ल-अष्टमी}}{\time{29-17}{17:44}}\hspace{1ex}}%
{\fullanga{मूला}}{चन्द्रराशिः—\mbox{धनुः}}%
{\anga{आयुष्मान्}{\time{43-30}{23:25}}\hspace{1ex}\uanga{सौभाग्यः}}%
{\anga{बवः}{\time{29-17}{17:44}}\hspace{1ex}\uanga{बालवः}}{}
}
{दूर्वाष्टमी\eventsep दधीचि-महर्षि-जयन्ति\eventsep कन्या-रवि-सङ्क्रमण-षडशीति-पुण्यकालः~06:15\RIGHTarrow{}18:05\eventsep राधाष्टमी\eventsep रवि-सङ्क्रमण-पुण्यकालः~06:01\RIGHTarrow{}12:39\eventsep सङ्क्रमण-दिन-पूर्वाह्ण-पुण्यकालः~06:01\RIGHTarrow{}12:03\eventsep विश्वकर्म-जयन्ती}
{Mon} 
\cfoot{\rygdata{07:32--09:02}{10:33--12:03}{13:34--15:04}}
\caldata{SEPTEMBER}{18}{\sunmonth{कन्या}{2}{}{भाद्रपदः}{वर्षऋतुः}{मङ्गलः}{विलम्बः}{दक्षिणायनम्}{वर्षऋतुः}}
{\sunmoonrsdata{06:01}{18:04}{13:27}{01:06*}{12:03}
{\kalas{04:25 05:13 09:14 08:26 10:02 16:28 10:50 13:15 15:40 17:16 18:52 21:03 22:33 01:32*}}}
{\tnykdata{\anga{\tithi{9}{शुक्ल-नवमी}}{\time{35-7}{20:04}}\hspace{1ex}}%
{\anga{मूला}{\time{3-46}{07:32}}\hspace{1ex}}{चन्द्रराशिः—\mbox{धनुः}}%
{\anga{सौभाग्यः}{\time{45-42}{00:18*}}\hspace{1ex}\uanga{शोभनः}}%
{\anga{बालवः}{\time{2-5}{06:51}}\hspace{1ex}\anga{कौलवः}{\time{35-7}{20:04}}\hspace{1ex}\uanga{तैतिलः}}{}
}
{अदुःखनवमी\eventsep गोधूमा-नवमी\eventsep महालक्ष्मी-व्रत-आरम्भः\eventsep नन्दा-नवमी\eventsep तालनवमी}
{Tue} 
\cfoot{\rygdata{15:03--16:34}{09:02--10:32}{12:03--13:33}}
\caldata{SEPTEMBER}{19}{\sunmonth{कन्या}{3}{}{भाद्रपदः}{वर्षऋतुः}{बुधः}{विलम्बः}{दक्षिणायनम्}{वर्षऋतुः}}
{\sunmoonrsdata{06:01}{18:04}{14:14}{01:55*}{12:02}
{\kalas{04:25 05:13 09:14 08:26 10:02 16:27 10:50 13:15 15:39 17:15 18:51 21:03 22:33 01:32*}}}
{\tnykdata{\anga{\tithi{10}{शुक्ल-दशमी}}{\time{41-36}{22:40}}\hspace{1ex}}%
{\anga{पूर्वाषाढा}{\time{11-19}{10:33}}\hspace{1ex}}{चन्द्रराशिः—\mbox{धनुः\RIGHTarrow{17:20}}}%
{\anga{शोभनः}{\time{48-19}{01:21*}}\hspace{1ex}\uanga{अतिगण्डः}}%
{\anga{तैतिलः}{\time{8-19}{09:21}}\hspace{1ex}\anga{गरः}{\time{41-36}{22:40}}\hspace{1ex}\uanga{वणिजः}}{}
}
{दशावतार-व्रतम्\eventsep वितस्तोत्सवः}
{Wed} 
\cfoot{\rygdata{12:02--13:33}{07:31--09:02}{10:32--12:02}}
\caldata{SEPTEMBER}{20}{\sunmonth{कन्या}{4}{}{भाद्रपदः}{वर्षऋतुः}{गुरुः}{विलम्बः}{दक्षिणायनम्}{वर्षऋतुः}}
{\sunmoonrsdata{06:01}{18:03}{14:58}{02:45*}{12:02}
{\kalas{04:25 05:13 09:13 08:25 10:02 16:27 10:50 13:14 15:38 17:15 18:51 21:02 22:32 01:32*}}}
{\tnykdata{\anga{\tithi{11}{शुक्ल-एकादशी}}{\time{48-7}{01:16*}}\hspace{1ex}}%
{\anga{उत्तराषाढा}{\time{19-10}{13:41}}\hspace{1ex}}{चन्द्रराशिः—\mbox{मकरः}}%
{\anga{अतिगण्डः}{\time{50-55}{02:23*}}\hspace{1ex}\uanga{सुकर्म}}%
{\anga{वणिजः}{\time{14-53}{11:58}}\hspace{1ex}\anga{विष्टिः}{\time{48-7}{01:16*}}\hspace{1ex}\uanga{बवः}}{}
}
{कटदानोत्सवः\eventsep सर्व-परिवर्तिनी-एकादशी}
{Thu} 
\cfoot{\rygdata{13:32--15:02}{06:01--07:31}{09:01--10:32}}
\caldata{SEPTEMBER}{21}{\sunmonth{कन्या}{5}{}{भाद्रपदः}{वर्षऋतुः}{शुक्रः}{विलम्बः}{दक्षिणायनम्}{वर्षऋतुः}}
{\sunmoonrsdata{06:01}{18:02}{15:40}{03:33*}{12:02}
{\kalas{04:25 05:13 09:13 08:25 10:01 16:26 10:49 13:14 15:38 17:14 18:50 21:02 22:32 01:31*}}}
{\tnykdata{\anga{\tithi{12}{शुक्ल-द्वादशी}}{\time{54-9}{03:41*}}\hspace{1ex}}%
{\anga{श्रवणः}{\time{26-44}{16:43}}\hspace{1ex}}{चन्द्रराशिः—\mbox{मकरः}}%
{\anga{सुकर्म}{\time{53-7}{03:16*}}\hspace{1ex}\uanga{धृतिः}}%
{\anga{बवः}{\time{21-14}{14:31}}\hspace{1ex}\anga{बालवः}{\time{54-9}{03:41*}}\hspace{1ex}\uanga{कौलवः}}{}
}
{अनन्त-द्वादशी\eventsep भुवनेश्वरी-जयन्ती\eventsep दधि-व्रत-समापनम्\eventsep हरिवासरः\RIGHTarrow{}07:54\eventsep पयोव्रत-आरम्भः\eventsep वामन-जयन्ती\eventsep विजया/श्रवण-महाद्वादशी\eventsep श्रवण-व्रतम्}
{Fri} 
\cfoot{\rygdata{10:31--12:02}{15:02--16:32}{07:31--09:01}}
\caldata{SEPTEMBER}{22}{\sunmonth{कन्या}{6}{}{भाद्रपदः}{वर्षऋतुः}{शनिः}{विलम्बः}{दक्षिणायनम्}{वर्षऋतुः}}
{\sunmoonrsdata{06:01}{18:01}{16:20}{04:21*}{12:01}
{\kalas{04:25 05:13 09:13 08:25 10:01 16:25 10:49 13:13 15:37 17:13 18:49 21:01 22:31 01:31*}}}
{\tnykdata{\anga{\tithi{13}{शुक्ल-त्रयोदशी}}{\time{59-15}{05:43*}}\hspace{1ex}}%
{\anga{श्रविष्ठा}{\time{33-35}{19:27}}\hspace{1ex}}{चन्द्रराशिः—\mbox{मकरः\RIGHTarrow{06:08}}}%
{\anga{धृतिः}{\time{54-37}{03:52*}}\hspace{1ex}\uanga{शूलः}}%
{\anga{कौलवः}{\time{26-50}{16:45}}\hspace{1ex}\anga{तैतिलः}{\time{59-15}{05:43*}}\hspace{1ex}\uanga{गरः}}{}
}
{दूर्व-त्रि-व्रतम्\eventsep गो-त्रिरात्रि-व्रतम्\eventsep \tamil{புரட்டாசி~சனிக்கிழமை}\eventsep वन-रक्षक-वैष्णव-हत्या~\#{२८८}\eventsep शनि-प्रदोष-व्रतम्~18:01\RIGHTarrow{}19:31}
{Sat} 
\cfoot{\rygdata{09:01--10:31}{13:31--15:01}{06:01--07:31}}
\caldata{SEPTEMBER}{23}{\sunmonth{कन्या}{7}{}{भाद्रपदः}{वर्षऋतुः}{भानुः}{विलम्बः}{दक्षिणायनम्}{वर्षऋतुः}}
{\sunmoonrsdata{06:01}{18:01}{16:58}{05:08*}{12:01}
{\kalas{04:25 05:13 09:13 08:25 10:01 16:25 10:49 13:13 15:37 17:13 18:49 21:01 22:31 01:31*}}}
{\tnykdata{\fulltithi{\tithi{14}{शुक्ल-चतुर्दशी}}}%
{\anga{शतभिषक्}{\time{39-24}{21:47}}\hspace{1ex}}{चन्द्रराशिः—\mbox{कुम्भः}}%
{\anga{शूलः}{\time{55-15}{04:07*}}\hspace{1ex}\uanga{गण्डः}}%
{\anga{गरः}{\time{31-22}{18:34}}\hspace{1ex}\uanga{वणिजः}}{}
}
{(सायन) विषु-पुण्यकालः~06:01\RIGHTarrow{}11:24\eventsep अनन्त-चतुर्दशी\eventsep अनन्त-पद्मनाभ-व्रतम्\eventsep दक्षिण-विषुव-दिनम्\eventsep इष-मासः/शरदृतुः~07:24\RIGHTarrow{}\eventsep \tamil{நடராஜர் மஹாபிஷேகம்}\eventsep \tamil{நரசிங்கமுனையரைய நாயனார் (40) குருபூஜை}\eventsep रवि-सङ्क्रमण-पुण्यकालः~06:01\RIGHTarrow{}13:48\eventsep सायन-सङ्क्रमण-दिन-पूर्वाह्ण-पुण्यकालः~06:01\RIGHTarrow{}12:01}
{Sun} 
\cfoot{\rygdata{16:31--18:01}{12:01--13:31}{15:01--16:31}}
\caldata{SEPTEMBER}{24}{\sunmonth{कन्या}{8}{}{भाद्रपदः}{वर्षऋतुः}{सोमः}{विलम्बः}{दक्षिणायनम्}{वर्षऋतुः}}
{\sunmoonrsdata{06:01}{18:00}{17:37}{05:56*}{12:00}
{\kalas{04:25 05:13 09:13 08:25 10:01 16:24 10:49 13:12 15:36 17:12 18:48 21:00 22:30 01:31*}}}
{\tnykdata{\anga{\tithi{14}{शुक्ल-चतुर्दशी}}{\time{3-11}{07:18}}\hspace{1ex}}%
{\anga{पूर्वप्रोष्ठपदा}{\time{44-0}{23:37}}\hspace{1ex}}{चन्द्रराशिः—\mbox{कुम्भः\RIGHTarrow{17:12}}}%
{\anga{गण्डः}{\time{54-54}{03:59*}}\hspace{1ex}\uanga{वृद्धिः}}%
{\anga{वणिजः}{\time{3-11}{07:18}}\hspace{1ex}\anga{विष्टिः}{\time{34-41}{19:54}}\hspace{1ex}\uanga{बवः}}{}
}
{काञ्ची ५९ जगद्गुरु श्री-भगवन्नाम बोधेन्द्र सरस्वती आराधना~\#{३२७}\eventsep पार्वणव्रतम् पूर्णिमायाम्\eventsep पञ्च-पर्व-पूजा (पूर्णिमा)\eventsep वेङ्कटाचले पूर्णिमा-गरुड-सेवा}
{Mon} 
\cfoot{\rygdata{07:31--09:01}{10:31--12:01}{13:30--15:00}}
\caldata{SEPTEMBER}{25}{\sunmonth{कन्या}{9}{}{भाद्रपदः}{वर्षऋतुः}{मङ्गलः}{विलम्बः}{दक्षिणायनम्}{वर्षऋतुः}}
{\sunmoonrsdata{06:01}{17:59}{18:15}{---}{12:00}
{\kalas{04:25 05:13 09:13 08:25 10:00 16:23 10:48 13:12 15:36 17:11 18:47 21:00 22:30 01:30*}}}
{\tnykdata{\anga{\tithi{15}{पौर्णमासी}}{\time{5-52}{08:22}}\hspace{1ex}}%
{\anga{उत्तरप्रोष्ठपदा}{\time{47-24}{00:59*}}\hspace{1ex}}{चन्द्रराशिः—\mbox{मीनः}}%
{\anga{वृद्धिः}{\time{53-35}{03:27*}}\hspace{1ex}\uanga{ध्रुवः}}%
{\anga{बवः}{\time{5-52}{08:22}}\hspace{1ex}\anga{बालवः}{\time{36-44}{20:43}}\hspace{1ex}\uanga{कौलवः}}{}
}
{दिक्पाल-पूजा\eventsep महालय-पक्ष-आरम्भः\eventsep पार्वण-प्रायश्चित्तावकाशः दर्शे\eventsep पूर्णमासेष्टिः\eventsep पूर्णिमा-व्रतम्\eventsep स्थालीपाकः\eventsep उमा-महेश्वर-व्रतम्\eventsep उपाङ्ग-ललिता-गौरी-व्रतम्\eventsep विश्वरूप-यात्रा\eventsep यतिचातुर्मास्यव्रत-समापनम्}
{Tue} 
\cfoot{\rygdata{15:00--16:30}{09:01--10:30}{12:00--13:30}}
\caldata{SEPTEMBER}{26}{\sunmonth{कन्या}{10}{}{भाद्रपदः}{वर्षऋतुः}{बुधः}{विलम्बः}{दक्षिणायनम्}{वर्षऋतुः}}
{\sunmoonsrdata{06:01}{17:58}{18:55}{06:45}{12:00}
{\kalas{04:25 05:13 09:12 08:25 10:00 16:23 10:48 13:12 15:35 17:11 18:47 20:59 22:30 01:30*}}}
{\tnykdata{\anga{\tithi{16}{कृष्ण-प्रथमा}}{\time{7-18}{08:56}}\hspace{1ex}}%
{\anga{रेवती}{\time{49-36}{01:52*}}\hspace{1ex}}{चन्द्रराशिः—\mbox{मीनः\RIGHTarrow{01:52*}}}%
{\anga{ध्रुवः}{\time{51-20}{02:33*}}\hspace{1ex}\uanga{व्याघातः}}%
{\anga{कौलवः}{\time{7-18}{08:56}}\hspace{1ex}\anga{तैतिलः}{\time{37-34}{21:03}}\hspace{1ex}\uanga{गरः}}{}
}
{अप्पय्य-दीक्षित-जयन्ती~\#{५००}\eventsep अशून्यशयन-व्रतम्}
{Wed} 
\cfoot{\rygdata{12:00--13:30}{07:31--09:00}{10:30--12:00}}
\caldata{SEPTEMBER}{27}{\sunmonth{कन्या}{11}{}{भाद्रपदः}{वर्षऋतुः}{गुरुः}{विलम्बः}{दक्षिणायनम्}{वर्षऋतुः}}
{\sunmoonsrdata{06:01}{17:58}{19:37}{07:36}{11:59}
{\kalas{04:25 05:13 09:12 08:24 10:00 16:22 10:48 13:11 15:34 17:10 18:46 20:59 22:29 01:30*}}}
{\tnykdata{\anga{\tithi{17}{कृष्ण-द्वितीया}}{\time{7-33}{09:03}}\hspace{1ex}}%
{\anga{अश्विनी}{\time{50-47}{02:20*}}\hspace{1ex}}{चन्द्रराशिः—\mbox{मेषः}}%
{\anga{व्याघातः}{\time{48-15}{01:19*}}\hspace{1ex}\uanga{हर्षणः}}%
{\anga{गरः}{\time{7-33}{09:03}}\hspace{1ex}\anga{वणिजः}{\time{37-18}{20:56}}\hspace{1ex}\uanga{विष्टिः}}{}
}
{\tamil{ருத்ர~பஶுபதி நாயனார் (16) குருபூஜை}}
{Thu} 
\cfoot{\rygdata{13:29--14:59}{06:01--07:31}{09:00--10:30}}
\caldata{SEPTEMBER}{28}{\sunmonth{कन्या}{12}{}{भाद्रपदः}{वर्षऋतुः}{शुक्रः}{विलम्बः}{दक्षिणायनम्}{वर्षऋतुः}}
{\sunmoonsrdata{06:01}{17:57}{20:22}{08:29}{11:59}
{\kalas{04:25 05:13 09:12 08:24 10:00 16:22 10:48 13:11 15:34 17:09 18:45 20:58 22:29 01:30*}}}
{\tnykdata{\anga{\tithi{18}{कृष्ण-तृतीया}}{\time{6-47}{08:44}}\hspace{1ex}}%
{\anga{अपभरणी}{\time{51-1}{02:26*}}\hspace{1ex}}{चन्द्रराशिः—\mbox{मेषः}}%
{\anga{हर्षणः}{\time{44-24}{23:47}}\hspace{1ex}\uanga{वज्रम्}}%
{\anga{विष्टिः}{\time{6-47}{08:44}}\hspace{1ex}\anga{बवः}{\time{36-3}{20:27}}\hspace{1ex}\uanga{बालवः}}{}
}
{गौरी-व्रतम्\eventsep कजरी-तृतीया\eventsep महाभरणी\eventsep विघ्नराज-महागणपति सङ्कटहर-चतुर्थी-व्रतम्}
{Fri} 
\cfoot{\rygdata{10:30--11:59}{14:58--16:28}{07:31--09:00}}
\caldata{SEPTEMBER}{29}{\sunmonth{कन्या}{13}{}{भाद्रपदः}{वर्षऋतुः}{शनिः}{विलम्बः}{दक्षिणायनम्}{वर्षऋतुः}}
{\sunmoonsrdata{06:01}{17:56}{21:12}{09:25}{11:59}
{\kalas{04:25 05:13 09:12 08:24 10:00 16:21 10:47 13:10 15:33 17:09 18:45 20:58 22:28 01:29*}}}
{\tnykdata{\anga{\tithi{19}{कृष्ण-चतुर्थी}}{\time{5-6}{08:04}}\hspace{1ex}}%
{\anga{कृत्तिका}{\time{50-25}{02:11*}}\hspace{1ex}}{चन्द्रराशिः—\mbox{मेषः\RIGHTarrow{08:24}}}%
{\anga{वज्रम्}{\time{39-51}{21:58}}\hspace{1ex}\uanga{सिद्धिः}}%
{\anga{बालवः}{\time{5-6}{08:04}}\hspace{1ex}\anga{कौलवः}{\time{33-56}{19:36}}\hspace{1ex}\uanga{तैतिलः}}{}
}
{दिक्पाल-पूजा\eventsep कृत्तिका-व्रतम्\eventsep \tamil{புரட்டாசி~சனிக்கிழமை}}
{Sat} 
\cfoot{\rygdata{09:00--10:29}{13:28--14:58}{06:01--07:31}}
\caldata{SEPTEMBER}{30}{\sunmonth{कन्या}{14}{}{भाद्रपदः}{वर्षऋतुः}{भानुः}{विलम्बः}{दक्षिणायनम्}{वर्षऋतुः}}
{\sunmoonsrdata{06:01}{17:56}{22:05}{10:22}{11:58}
{\kalas{04:24 05:13 09:12 08:24 09:59 16:20 10:47 13:10 15:33 17:08 18:44 20:57 22:28 01:29*}}}
{\tnykdata{\anga{\tithi{20}{कृष्ण-पञ्चमी}}{\time{2-35}{07:03}}\hspace{1ex}\anga{\tithi{21}{कृष्ण-षष्ठी}}{\time{59-18}{05:45*}}\hspace{1ex}\avamA{}}%
{\anga{रोहिणी}{\time{49-3}{01:38*}}\hspace{1ex}}{चन्द्रराशिः—\mbox{वृषभः}}%
{\anga{सिद्धिः}{\time{34-42}{19:54}}\hspace{1ex}\uanga{व्यतीपातः}}%
{\anga{तैतिलः}{\time{2-35}{07:03}}\hspace{1ex}\anga{गरः}{\time{31-2}{18:26}}\hspace{1ex}\anga{वणिजः}{\time{59-18}{05:45*}}\hspace{1ex}\uanga{विष्टिः}}{}
}
{चन्द्र-षष्ठी\eventsep काञ्ची ३३ जगद्गुरु श्री-सच्चिदानन्दघनेन्द्र सरस्वती २ आराधना~\#{१३२७}\eventsep कावेरी-अन्त्य-पुष्कर-आरम्भः\eventsep कपिल-षष्ठी\eventsep नाग-पूजा\eventsep सप्तर्षि-पूजा/अर्घ्यम्\eventsep \tamil{திருநாளைப்போவார் நாயனார் (17) குருபூஜை}}
{Sun} 
\cfoot{\rygdata{16:26--17:56}{11:58--13:28}{14:57--16:26}}
\caldata{OCTOBER}{1}{\sunmonth{कन्या}{15}{}{भाद्रपदः}{वर्षऋतुः}{सोमः}{विलम्बः}{दक्षिणायनम्}{वर्षऋतुः}}
{\sunmoonsrdata{06:01}{17:55}{23:03}{11:21}{11:58}
{\kalas{04:24 05:13 09:12 08:24 09:59 16:20 10:47 13:09 15:32 17:08 18:43 20:57 22:27 01:29*}}}
{\tnykdata{\anga{\tithi{22}{कृष्ण-सप्तमी}}{\time{55-19}{04:09*}}\hspace{1ex}}%
{\anga{मृगशीर्षम्}{\time{46-59}{00:49*}}\hspace{1ex}}{चन्द्रराशिः—\mbox{वृषभः\RIGHTarrow{13:16}}}%
{\anga{व्यतीपातः}{\time{28-57}{17:36}}\hspace{1ex}\uanga{वरीयान्}}%
{\anga{विष्टिः}{\time{27-24}{16:59}}\hspace{1ex}\anga{बवः}{\time{55-19}{04:09*}}\hspace{1ex}\uanga{बालवः}}{}
}
{महालक्ष्मी-व्रत-समापनम्\eventsep महाव्यतीपात-श्राद्धम्\eventsep सोममृगशीर्ष-पुण्यकालः\eventsep शृङ्गेरी ३५ जगद्गुरु श्री-अभिनव विद्यातीर्थ महास्वामी आराधना}
{Mon} 
\cfoot{\rygdata{07:30--09:00}{10:29--11:58}{13:27--14:57}}
\caldata{OCTOBER}{2}{\sunmonth{कन्या}{16}{}{भाद्रपदः}{वर्षऋतुः}{मङ्गलः}{विलम्बः}{दक्षिणायनम्}{वर्षऋतुः}}
{\sunmoonsrdata{06:01}{17:54}{00:03*}{12:20}{11:58}
{\kalas{04:24 05:13 09:11 08:24 09:59 16:19 10:46 13:09 15:32 17:07 18:43 20:56 22:27 01:29*}}}
{\tnykdata{\anga{\tithi{23}{कृष्ण-अष्टमी}}{\time{50-39}{02:17*}}\hspace{1ex}}%
{\anga{आर्द्रा}{\time{44-13}{23:43}}\hspace{1ex}}{चन्द्रराशिः—\mbox{मिथुनम्}}%
{\anga{वरीयान्}{\time{22-37}{15:04}}\hspace{1ex}\uanga{परिघः}}%
{\anga{बालवः}{\time{23-4}{15:15}}\hspace{1ex}\anga{कौलवः}{\time{50-39}{02:17*}}\hspace{1ex}\uanga{तैतिलः}}{}
}
{अशोकाष्टमी-व्रत-आरम्भः\eventsep जीमूतवाहन-पूजा\eventsep मध्याष्टमी\eventsep पञ्च-पर्व-पूजा (अष्टमी)}
{Tue} 
\cfoot{\rygdata{14:56--16:25}{09:00--10:29}{11:58--13:27}}
\caldata{OCTOBER}{3}{\sunmonth{कन्या}{17}{}{भाद्रपदः}{वर्षऋतुः}{बुधः}{विलम्बः}{दक्षिणायनम्}{वर्षऋतुः}}
{\sunmoonsrdata{06:01}{17:54}{01:04*}{13:17}{11:57}
{\kalas{04:24 05:13 09:11 08:24 09:59 16:19 10:46 13:09 15:31 17:06 18:42 20:56 22:27 01:28*}}}
{\tnykdata{\anga{\tithi{24}{कृष्ण-नवमी}}{\time{45-20}{00:10*}}\hspace{1ex}}%
{\anga{पुनर्वसुः}{\time{40-49}{22:21}}\hspace{1ex}}{चन्द्रराशिः—\mbox{मिथुनम्\RIGHTarrow{16:43}}}%
{\anga{परिघः}{\time{15-46}{12:20}}\hspace{1ex}\uanga{शिवः}}%
{\anga{तैतिलः}{\time{18-4}{13:15}}\hspace{1ex}\anga{गरः}{\time{45-20}{00:10*}}\hspace{1ex}\uanga{वणिजः}}{}
}
{अविधवा-नवमी\eventsep दुर्गा/गौरी-पूजा}
{Wed} 
\cfoot{\rygdata{11:57--13:26}{07:30--08:59}{10:28--11:57}}
\caldata{OCTOBER}{4}{\sunmonth{कन्या}{18}{}{भाद्रपदः}{वर्षऋतुः}{गुरुः}{विलम्बः}{दक्षिणायनम्}{वर्षऋतुः}}
{\sunmoonsrdata{06:01}{17:53}{02:04*}{14:11}{11:57}
{\kalas{04:24 05:13 09:11 08:24 09:59 16:18 10:46 13:08 15:31 17:06 18:42 20:55 22:26 01:28*}}}
{\tnykdata{\anga{\tithi{25}{कृष्ण-दशमी}}{\time{39-29}{21:49}}\hspace{1ex}}%
{\anga{पुष्यः}{\time{36-51}{20:46}}\hspace{1ex}}{चन्द्रराशिः—\mbox{कर्कटः}}%
{\anga{शिवः}{\time{8-25}{09:23}}\hspace{1ex}\uanga{सिद्धः}}%
{\anga{वणिजः}{\time{12-28}{11:01}}\hspace{1ex}\anga{विष्टिः}{\time{39-29}{21:49}}\hspace{1ex}\uanga{बवः}}{}
}
{गुरुपुष्य-पुण्यकालः}
{Thu} 
\cfoot{\rygdata{13:26--14:55}{06:01--07:30}{08:59--10:28}}
\caldata{OCTOBER}{5}{\sunmonth{कन्या}{19}{}{भाद्रपदः}{वर्षऋतुः}{शुक्रः}{विलम्बः}{दक्षिणायनम्}{वर्षऋतुः}}
{\sunmoonsrdata{06:01}{17:52}{03:04*}{15:02}{11:57}
{\kalas{04:24 05:13 09:11 08:24 09:58 16:18 10:46 13:08 15:30 17:05 18:41 20:55 22:26 01:28*}}}
{\tnykdata{\anga{\tithi{26}{कृष्ण-एकादशी}}{\time{33-10}{19:17}}\hspace{1ex}}%
{\anga{आश्रेषा}{\time{32-26}{19:00}}\hspace{1ex}}{चन्द्रराशिः—\mbox{कर्कटः\RIGHTarrow{19:00}}}%
{\anga{सिद्धः}{\time{0-38}{06:17}}\hspace{1ex}\anga{साध्यः}{\time{52-32}{03:02*}}\hspace{1ex}\uanga{शुभः}}%
{\anga{बवः}{\time{6-21}{08:34}}\hspace{1ex}\anga{बालवः}{\time{33-10}{19:17}}\hspace{1ex}\anga{कौलवः}{\time{59-54}{05:59*}}\hspace{1ex}\uanga{तैतिलः}}{}
}
{सर्व-इन्दिरा-एकादशी}
{Fri} 
\cfoot{\rygdata{10:28--11:57}{14:55--16:23}{07:30--08:59}}
\caldata{OCTOBER}{6}{\sunmonth{कन्या}{20}{}{भाद्रपदः}{वर्षऋतुः}{शनिः}{विलम्बः}{दक्षिणायनम्}{वर्षऋतुः}}
{\sunmoonsrdata{06:01}{17:52}{04:01*}{15:50}{11:57}
{\kalas{04:24 05:13 09:11 08:24 09:58 16:17 10:45 13:08 15:30 17:04 18:40 20:54 22:25 01:28*}}}
{\tnykdata{\anga{\tithi{27}{कृष्ण-द्वादशी}}{\time{26-36}{16:40}}\hspace{1ex}}%
{\anga{मघा}{\time{27-46}{17:08}}\hspace{1ex}}{चन्द्रराशिः—\mbox{सिंहः}}%
{\anga{शुभः}{\time{44-18}{23:45}}\hspace{1ex}\uanga{शुक्लः}}%
{\anga{तैतिलः}{\time{26-36}{16:40}}\hspace{1ex}\anga{गरः}{\time{53-18}{03:21*}}\hspace{1ex}\uanga{वणिजः}}{}
}
{गजच्छाया-योगः~16:40\RIGHTarrow{}17:08\eventsep \tamil{புரட்டாசி~சனிக்கிழமை}\eventsep यति-महालयम्\eventsep शनि-प्रदोष-व्रतम्~17:52\RIGHTarrow{}19:23}
{Sat} 
\cfoot{\rygdata{08:59--10:28}{13:25--14:54}{06:01--07:30}}
\caldata{OCTOBER}{7}{\sunmonth{कन्या}{21}{}{भाद्रपदः}{वर्षऋतुः}{भानुः}{विलम्बः}{दक्षिणायनम्}{वर्षऋतुः}}
{\sunmoonsrdata{06:02}{17:51}{04:58*}{16:36}{11:56}
{\kalas{04:24 05:13 09:11 08:23 09:58 16:16 10:45 13:07 15:29 17:04 18:40 20:54 22:25 01:28*}}}
{\tnykdata{\anga{\tithi{28}{कृष्ण-त्रयोदशी}}{\time{20-2}{14:02}}\hspace{1ex}}%
{\anga{पूर्वफल्गुनी}{\time{23-5}{15:16}}\hspace{1ex}}{चन्द्रराशिः—\mbox{सिंहः\RIGHTarrow{20:48}}}%
{\anga{शुक्लः}{\time{36-7}{20:29}}\hspace{1ex}\uanga{ब्रह्म}}%
{\anga{वणिजः}{\time{20-2}{14:02}}\hspace{1ex}\anga{विष्टिः}{\time{46-50}{00:46*}}\hspace{1ex}\uanga{शकुनिः}}{}
}
{द्वापरयुगादिः\eventsep काञ्ची ४४ जगद्गुरु श्री-पूर्णबोधेन्द्र सरस्वती २ आराधना~\#{९७९}\eventsep मासशिवरात्रिः\eventsep पञ्च-पर्व-पूजा (चतुर्दशी)\eventsep शस्त्रहतचतुर्दशी}
{Sun} 
\cfoot{\rygdata{16:22--17:51}{11:56--13:25}{14:54--16:22}}
\caldata{OCTOBER}{8}{\sunmonth{कन्या}{22}{}{भाद्रपदः}{वर्षऋतुः}{सोमः}{विलम्बः}{दक्षिणायनम्}{वर्षऋतुः}}
{\sunmoonsrdata{06:02}{17:50}{05:54*}{17:21}{11:56}
{\kalas{04:24 05:13 09:11 08:23 09:58 16:16 10:45 13:07 15:29 17:03 18:39 20:53 22:25 01:27*}}}
{\tnykdata{\anga{\tithi{29}{कृष्ण-चतुर्दशी}}{\time{13-45}{11:32}}\hspace{1ex}}%
{\anga{उत्तरफल्गुनी}{\time{18-43}{13:31}}\hspace{1ex}}{चन्द्रराशिः—\mbox{कन्या}}%
{\anga{ब्रह्म}{\time{28-17}{17:20}}\hspace{1ex}\uanga{इन्द्रः}}%
{\anga{शकुनिः}{\time{13-45}{11:32}}\hspace{1ex}\anga{चतुष्पात्}{\time{40-50}{22:22}}\hspace{1ex}\uanga{नाग}}{}
}
{गजच्छाया-योगः~13:31\RIGHTarrow{}09:16*\eventsep कात्यायनी-जयन्ती\eventsep महालय-पक्ष-समापनम्\eventsep पार्वणव्रतम् अमावास्यायाम्\eventsep पञ्च-पर्व-पूजा (अमावास्या)\eventsep सर्व-(भाद्रपद) महालय अमावास्या (अलभ्यम्–पुष्कला)\eventsep शृङ्गेरी ३४ जगद्गुरु श्री-चन्द्रशेखर भारती आराधना}
{Mon} 
\cfoot{\rygdata{07:30--08:59}{10:27--11:56}{13:25--14:53}}
\caldata{OCTOBER}{9}{\sunmonth{कन्या}{23}{}{भाद्रपदः}{वर्षऋतुः}{मङ्गलः}{विलम्बः}{दक्षिणायनम्}{वर्षऋतुः}}
{\sunmoonsrdata{06:02}{17:50}{---}{18:06}{11:56}
{\kalas{04:24 05:13 09:10 08:23 09:58 16:15 10:45 13:06 15:28 17:03 18:39 20:53 22:24 01:27*}}}
{\tnykdata{\anga{\tithi{30}{अमावास्या}}{\time{8-6}{09:16}}\hspace{1ex}}%
{\anga{हस्तः}{\time{15-0}{12:02}}\hspace{1ex}}{चन्द्रराशिः—\mbox{कन्या\RIGHTarrow{23:26}}}%
{\anga{इन्द्रः}{\time{21-1}{14:26}}\hspace{1ex}\uanga{वैधृतिः}}%
{\anga{नाग}{\time{8-6}{09:16}}\hspace{1ex}\anga{किंस्तुघ्नः}{\time{35-39}{20:17}}\hspace{1ex}\uanga{बवः}}{}
}
{अश्वशिरो-देव-पूजा\eventsep दर्शेष्टिः\eventsep दौहित्र-प्रतिपत्\eventsep पार्वण-प्रायश्चित्तावकाशः पौर्णमास्याम्\eventsep स्थालीपाकः\eventsep सुजन्मप्राप्ति-व्रतम्\eventsep वैधृति-श्राद्धम्}
{Tue} 
\cfoot{\rygdata{14:53--16:21}{08:59--10:27}{11:56--13:24}}
\caldata{OCTOBER}{10}{\sunmonth{कन्या}{24}{}{आश्वयुजः}{शरदृतुः}{बुधः}{विलम्बः}{दक्षिणायनम्}{वर्षऋतुः}}
{\sunmoonrsdata{06:02}{17:49}{06:50}{18:51}{11:55}
{\kalas{04:24 05:13 09:10 08:23 09:58 16:15 10:45 13:06 15:28 17:02 18:38 20:52 22:24 01:27*}}}
{\tnykdata{\anga{\tithi{1}{शुक्ल-प्रथमा}}{\time{3-28}{07:25}}\hspace{1ex}}%
{\anga{चित्रा}{\time{12-21}{10:58}}\hspace{1ex}}{चन्द्रराशिः—\mbox{तुला}}%
{\anga{वैधृतिः}{\time{14-38}{11:53}}\hspace{1ex}\uanga{विष्कम्भः}}%
{\anga{बवः}{\time{3-28}{07:25}}\hspace{1ex}\anga{बालवः}{\time{31-39}{18:41}}\hspace{1ex}\uanga{कौलवः}}{}
}
{चन्द्र-दर्शनम्~17:49\RIGHTarrow{}18:51\eventsep गृहदेवी-पूजा\eventsep स्तन्यवृद्धि-गौरी-व्रतम्\eventsep शरन्नवरात्र-आरम्भः}
{Wed} 
\cfoot{\rygdata{11:55--13:24}{07:30--08:59}{10:27--11:55}}
\caldata{OCTOBER}{11}{\sunmonth{कन्या}{25}{}{आश्वयुजः}{शरदृतुः}{गुरुः}{विलम्बः}{दक्षिणायनम्}{वर्षऋतुः}}
{\sunmoonrsdata{06:02}{17:49}{07:45}{19:38}{11:55}
{\kalas{04:24 05:13 09:10 08:23 09:57 16:14 10:44 13:06 15:27 17:01 18:38 20:52 22:24 01:27*}}}
{\tnykdata{\anga{\tithi{2}{शुक्ल-द्वितीया}}{\time{0-11}{06:07}}\hspace{1ex}\anga{\tithi{3}{शुक्ल-तृतीया}}{\time{58-35}{05:28*}}\hspace{1ex}\avamA{}}%
{\anga{स्वाती}{\time{11-4}{10:28}}\hspace{1ex}}{चन्द्रराशिः—\mbox{तुला\RIGHTarrow{04:31*}}}%
{\anga{विष्कम्भः}{\time{9-24}{09:47}}\hspace{1ex}\uanga{प्रीतिः}}%
{\anga{कौलवः}{\time{0-11}{06:07}}\hspace{1ex}\anga{तैतिलः}{\time{29-10}{17:42}}\hspace{1ex}\anga{गरः}{\time{58-35}{05:28*}}\hspace{1ex}\uanga{वणिजः}}{}
}
{गुरु-सङ्क्रान्तिः~(तुला\To{}वृश्चिकः)\eventsep कावेरी-अन्त्य-पुष्कर-समापनम्\eventsep मेघपालीय-तृतीया\eventsep महालय-पक्ष-तर्पण-पूर्तिः}
{Thu} 
\cfoot{\rygdata{13:24--14:52}{06:02--07:30}{08:58--10:27}}
\caldata{OCTOBER}{12}{\sunmonth{कन्या}{26}{}{आश्वयुजः}{शरदृतुः}{शुक्रः}{विलम्बः}{दक्षिणायनम्}{वर्षऋतुः}}
{\sunmoonrsdata{06:02}{17:48}{08:41}{20:27}{11:55}
{\kalas{04:24 05:13 09:10 08:23 09:57 16:14 10:44 13:06 15:27 17:01 18:37 20:51 22:23 01:27*}}}
{\tnykdata{\anga{\tithi{4}{शुक्ल-चतुर्थी}}{\time{58-51}{05:35*}}\hspace{1ex}}%
{\anga{विशाखा}{\time{11-29}{10:38}}\hspace{1ex}}{चन्द्रराशिः—\mbox{वृश्चिकः}}%
{\anga{प्रीतिः}{\time{5-30}{08:14}}\hspace{1ex}\uanga{आयुष्मान्}}%
{\anga{वणिजः}{\time{28-28}{17:25}}\hspace{1ex}\anga{विष्टिः}{\time{58-51}{05:35*}}\hspace{1ex}\uanga{बवः}}{}
}
{देवता-सुवासिनी-पूजा\eventsep ताम्रपर्णी-आदि-पुष्कर-आरम्भः\eventsep शुक्ल-चतुर्थी-व्रतम्}
{Fri} 
\cfoot{\rygdata{10:27--11:55}{14:51--16:20}{07:30--08:58}}
\caldata{OCTOBER}{13}{\sunmonth{कन्या}{27}{}{आश्वयुजः}{शरदृतुः}{शनिः}{विलम्बः}{दक्षिणायनम्}{वर्षऋतुः}}
{\sunmoonrsdata{06:02}{17:47}{09:36}{21:16}{11:55}
{\kalas{04:24 05:13 09:10 08:23 09:57 16:13 10:44 13:05 15:26 17:00 18:36 20:51 22:23 01:27*}}}
{\tnykdata{\fulltithi{\tithi{5}{शुक्ल-पञ्चमी}}}%
{\anga{अनूराधा}{\time{13-45}{11:32}}\hspace{1ex}}{चन्द्रराशिः—\mbox{वृश्चिकः}}%
{\anga{आयुष्मान्}{\time{3-6}{07:17}}\hspace{1ex}\uanga{सौभाग्यः}}%
{\anga{बवः}{\time{29-43}{17:55}}\hspace{1ex}\uanga{बालवः}}{}
}
{आश्विन-नाग-पञ्चमी\eventsep ललिता-पञ्चमी\eventsep \tamil{புரட்டாசி~சனிக்கிழமை}\eventsep उपाङ्ग-ललिता-व्रतम्\eventsep शान्ति-पञ्चमी-व्रतम्}
{Sat} 
\cfoot{\rygdata{08:58--10:26}{13:23--14:51}{06:02--07:30}}
\caldata{OCTOBER}{14}{\sunmonth{कन्या}{28}{}{आश्वयुजः}{शरदृतुः}{भानुः}{विलम्बः}{दक्षिणायनम्}{वर्षऋतुः}}
{\sunmoonrsdata{06:02}{17:47}{10:29}{22:07}{11:54}
{\kalas{04:24 05:13 09:10 08:23 09:57 16:13 10:44 13:05 15:26 17:00 18:36 20:51 22:23 01:26*}}}
{\tnykdata{\anga{\tithi{5}{शुक्ल-पञ्चमी}}{\time{1-4}{06:28}}\hspace{1ex}}%
{\anga{ज्येष्ठा}{\time{17-52}{13:11}}\hspace{1ex}}{चन्द्रराशिः—\mbox{वृश्चिकः\RIGHTarrow{13:11}}}%
{\anga{सौभाग्यः}{\time{2-12}{06:55}}\hspace{1ex}\uanga{शोभनः}}%
{\anga{बालवः}{\time{1-4}{06:28}}\hspace{1ex}\anga{कौलवः}{\time{32-52}{19:11}}\hspace{1ex}\uanga{तैतिलः}}{}
}
{षष्ठी-व्रतम्}
{Sun} 
\cfoot{\rygdata{16:19--17:47}{11:54--13:23}{14:51--16:19}}
\caldata{OCTOBER}{15}{\sunmonth{कन्या}{29}{}{आश्वयुजः}{शरदृतुः}{सोमः}{विलम्बः}{दक्षिणायनम्}{वर्षऋतुः}}
{\sunmoonrsdata{06:02}{17:46}{11:20}{22:58}{11:54}
{\kalas{04:24 05:13 09:10 08:23 09:57 16:12 10:44 13:05 15:25 16:59 18:35 20:50 22:22 01:26*}}}
{\tnykdata{\anga{\tithi{6}{शुक्ल-षष्ठी}}{\time{5-5}{08:04}}\hspace{1ex}}%
{\anga{मूला}{\time{23-40}{15:30}}\hspace{1ex}}{चन्द्रराशिः—\mbox{धनुः}}%
{\anga{शोभनः}{\time{2-40}{07:06}}\hspace{1ex}\uanga{अतिगण्डः}}%
{\anga{तैतिलः}{\time{5-5}{08:04}}\hspace{1ex}\anga{गरः}{\time{37-41}{21:07}}\hspace{1ex}\uanga{वणिजः}}{}
}
{काञ्ची ४५ जगद्गुरु श्री-परमशिवेन्द्र सरस्वती १ आराधना~\#{९५८}\eventsep सरस्वती-आवाहनम्}
{Mon} 
\cfoot{\rygdata{07:30--08:58}{10:26--11:54}{13:22--14:50}}
\caldata{OCTOBER}{16}{\sunmonth{कन्या}{30}{}{आश्वयुजः}{शरदृतुः}{मङ्गलः}{विलम्बः}{दक्षिणायनम्}{वर्षऋतुः}}
{\sunmoonrsdata{06:02}{17:46}{12:08}{23:48}{11:54}
{\kalas{04:24 05:13 09:10 08:23 09:57 16:12 10:44 13:04 15:25 16:59 18:35 20:50 22:22 01:26*}}}
{\tnykdata{\anga{\tithi{7}{शुक्ल-सप्तमी}}{\time{10-34}{10:16}}\hspace{1ex}}%
{\anga{पूर्वाषाढा}{\time{30-43}{18:20}}\hspace{1ex}}{चन्द्रराशिः—\mbox{धनुः\RIGHTarrow{01:05*}}}%
{\anga{अतिगण्डः}{\time{4-14}{07:44}}\hspace{1ex}\uanga{सुकर्म}}%
{\anga{वणिजः}{\time{10-34}{10:16}}\hspace{1ex}\anga{विष्टिः}{\time{43-41}{23:31}}\hspace{1ex}\uanga{बवः}}{}
}
{पत्रिका-प्रवेश-पूजा\eventsep शुभ-सप्तमी}
{Tue} 
\cfoot{\rygdata{14:50--16:18}{08:58--10:26}{11:54--13:22}}
\caldata{OCTOBER}{17}{\sunmonth{कन्या}{31}{\mbox{कन्या{\tiny\RIGHTarrow}{18:12}}}{आश्वयुजः}{शरदृतुः}{बुधः}{विलम्बः}{दक्षिणायनम्}{वर्षऋतुः}}
{\sunmoonrsdata{06:03}{17:45}{12:53}{00:38*}{11:54}
{\kalas{04:24 05:13 09:10 08:23 09:57 16:11 10:43 13:04 15:25 16:58 18:34 20:49 22:22 01:26*}}}
{\tnykdata{\anga{\tithi{8}{शुक्ल-अष्टमी}}{\time{16-57}{12:50}}\hspace{1ex}}%
{\anga{उत्तराषाढा}{\time{38-26}{21:25}}\hspace{1ex}}{चन्द्रराशिः—\mbox{मकरः}}%
{\anga{सुकर्म}{\time{6-32}{08:40}}\hspace{1ex}\uanga{धृतिः}}%
{\anga{बवः}{\time{16-57}{12:50}}\hspace{1ex}\anga{बालवः}{\time{50-17}{02:10*}}\hspace{1ex}\uanga{कौलवः}}{}
}
{\tamil{ஏனாதிநாத நாயனார் (8) குருபூஜை}\eventsep भद्रकाळी-पूजा\eventsep दुर्गाष्टमी\eventsep काञ्ची १९ जगद्गुरु श्री-मार्तण्ड विद्याघनेन्द्र सरस्वती आराधना~\#{१६२१}\eventsep काल-त्रिरात्रि-व्रतम्\eventsep मन्वादिः-(स्वायम्भुवः-[१])\eventsep रवि-सङ्क्रमण-पुण्यकालः~11:48\RIGHTarrow{}17:45\eventsep सङ्क्रमण-दिन-अपराह्ण-पुण्यकालः~11:54\RIGHTarrow{}17:45\eventsep तुला-सङ्क्रमण-पुण्यकालः~14:12\RIGHTarrow{}17:45}
{Wed} 
\cfoot{\rygdata{11:54--13:22}{07:30--08:58}{10:26--11:54}}
\caldata{OCTOBER}{18}{\sunmonth{तुला}{1}{}{आश्वयुजः}{शरदृतुः}{गुरुः}{विलम्बः}{दक्षिणायनम्}{शरदृतुः}}
{\sunmoonrsdata{06:03}{17:45}{13:36}{01:26*}{11:54}
{\kalas{04:24 05:14 09:10 08:23 09:57 16:11 10:43 13:04 15:24 16:58 18:34 20:49 22:21 01:26*}}}
{\tnykdata{\anga{\tithi{9}{शुक्ल-नवमी}}{\time{23-34}{15:29}}\hspace{1ex}}%
{\anga{श्रवणः}{\time{46-9}{00:31*}}\hspace{1ex}}{चन्द्रराशिः—\mbox{मकरः}}%
{\anga{धृतिः}{\time{9-5}{09:41}}\hspace{1ex}\uanga{शूलः}}%
{\anga{कौलवः}{\time{23-34}{15:29}}\hspace{1ex}\anga{तैतिलः}{\time{56-45}{04:45*}}\hspace{1ex}\uanga{गरः}}{}
}
{आकाशदीप-आरम्भः\eventsep भद्रकाळी-व्रतम्\eventsep महानवमी/सरस्वती-पूजा\eventsep तुला-कावेरी-स्नान-आरम्भः\eventsep विजयदशमी\eventsep शरन्नवरात्र-समापनम्\eventsep श्रवण-व्रतम्}
{Thu} 
\cfoot{\rygdata{13:21--14:49}{06:03--07:30}{08:58--10:26}}
\caldata{OCTOBER}{19}{\sunmonth{तुला}{2}{}{आश्वयुजः}{शरदृतुः}{शुक्रः}{विलम्बः}{दक्षिणायनम्}{शरदृतुः}}
{\sunmoonrsdata{06:03}{17:44}{14:16}{02:14*}{11:53}
{\kalas{04:24 05:14 09:10 08:23 09:57 16:10 10:43 13:04 15:24 16:57 18:33 20:49 22:21 01:26*}}}
{\tnykdata{\anga{\tithi{10}{शुक्ल-दशमी}}{\time{29-45}{17:57}}\hspace{1ex}}%
{\anga{श्रविष्ठा}{\time{53-15}{03:21*}}\hspace{1ex}}{चन्द्रराशिः—\mbox{मकरः\RIGHTarrow{13:59}}}%
{\anga{शूलः}{\time{11-24}{10:37}}\hspace{1ex}\uanga{गण्डः}}%
{\anga{गरः}{\time{29-45}{17:57}}\hspace{1ex}\uanga{वणिजः}}{}
}
{आयुध-पूजा\eventsep भृगुवार-सुब्रह्मण्य-व्रतम्\eventsep दशहरा\eventsep दुर्गा-पूजा\eventsep गङ्गावतरणम्\eventsep कूष्माण्ड-दशमी\eventsep मध्वाचार्य-जयन्ती~\#{७८१}\eventsep युद्धदेवता-आराधना\eventsep शमी-पूजा}
{Fri} 
\cfoot{\rygdata{10:26--11:53}{14:49--16:16}{07:31--08:58}}
\caldata{OCTOBER}{20}{\sunmonth{तुला}{3}{}{आश्वयुजः}{शरदृतुः}{शनिः}{विलम्बः}{दक्षिणायनम्}{शरदृतुः}}
{\sunmoonrsdata{06:03}{17:44}{14:55}{03:01*}{11:53}
{\kalas{04:24 05:14 09:10 08:23 09:57 16:10 10:43 13:03 15:23 16:57 18:33 20:48 22:21 01:26*}}}
{\tnykdata{\anga{\tithi{11}{शुक्ल-एकादशी}}{\time{34-54}{20:01}}\hspace{1ex}}%
{\anga{शतभिषक्}{\time{59-13}{05:45*}}\hspace{1ex}}{चन्द्रराशिः—\mbox{कुम्भः}}%
{\anga{गण्डः}{\time{13-3}{11:16}}\hspace{1ex}\uanga{वृद्धिः}}%
{\anga{वणिजः}{\time{2-29}{07:03}}\hspace{1ex}\anga{विष्टिः}{\time{34-54}{20:01}}\hspace{1ex}\uanga{बवः}}{}
}
{\tamil{பேயாழ்வார் திருநக்ஷத்திரம்}\eventsep पून्तानं-जयन्ती~\#{१०३४}\eventsep सर्व-पापाङ्कुशा-एकादशी}
{Sat} 
\cfoot{\rygdata{08:58--10:26}{13:21--14:48}{06:03--07:31}}
\caldata{OCTOBER}{21}{\sunmonth{तुला}{4}{}{आश्वयुजः}{शरदृतुः}{भानुः}{विलम्बः}{दक्षिणायनम्}{शरदृतुः}}
{\sunmoonrsdata{06:03}{17:43}{15:33}{03:49*}{11:53}
{\kalas{04:24 05:14 09:10 08:23 09:56 16:10 10:43 13:03 15:23 16:56 18:32 20:48 22:21 01:26*}}}
{\tnykdata{\anga{\tithi{12}{शुक्ल-द्वादशी}}{\time{38-39}{21:31}}\hspace{1ex}}%
{\fullanga{पूर्वप्रोष्ठपदा}}{चन्द्रराशिः—\mbox{कुम्भः\RIGHTarrow{01:09*}}}%
{\anga{वृद्धिः}{\time{13-43}{11:33}}\hspace{1ex}\uanga{ध्रुवः}}%
{\anga{बवः}{\time{6-58}{08:50}}\hspace{1ex}\anga{बालवः}{\time{38-39}{21:31}}\hspace{1ex}\uanga{कौलवः}}{}
}
{द्विदल-व्रत-आरम्भः\eventsep पयोव्रत-समापनम्}
{Sun} 
\cfoot{\rygdata{16:16--17:43}{11:53--13:21}{14:48--16:16}}
\caldata{OCTOBER}{22}{\sunmonth{तुला}{5}{}{आश्वयुजः}{शरदृतुः}{सोमः}{विलम्बः}{दक्षिणायनम्}{शरदृतुः}}
{\sunmoonrsdata{06:03}{17:43}{16:12}{04:38*}{11:53}
{\kalas{04:25 05:14 09:10 08:23 09:56 16:09 10:43 13:03 15:23 16:56 18:32 20:48 22:20 01:26*}}}
{\tnykdata{\anga{\tithi{13}{शुक्ल-त्रयोदशी}}{\time{40-48}{22:23}}\hspace{1ex}}%
{\anga{पूर्वप्रोष्ठपदा}{\time{3-44}{07:33}}\hspace{1ex}}{चन्द्रराशिः—\mbox{मीनः}}%
{\anga{ध्रुवः}{\time{13-12}{11:20}}\hspace{1ex}\uanga{व्याघातः}}%
{\anga{कौलवः}{\time{9-55}{10:02}}\hspace{1ex}\anga{तैतिलः}{\time{40-48}{22:23}}\hspace{1ex}\uanga{गरः}}{}
}
{सोम-प्रदोष-व्रतम्~17:42\RIGHTarrow{}19:15}
{Mon} 
\cfoot{\rygdata{07:31--08:58}{10:25--11:53}{13:20--14:48}}
\caldata{OCTOBER}{23}{\sunmonth{तुला}{6}{}{आश्वयुजः}{शरदृतुः}{मङ्गलः}{विलम्बः}{दक्षिणायनम्}{शरदृतुः}}
{\sunmoonrsdata{06:04}{17:42}{16:51}{05:28*}{11:53}
{\kalas{04:25 05:14 09:10 08:23 09:56 16:09 10:43 13:03 15:22 16:55 18:31 20:47 22:20 01:26*}}}
{\tnykdata{\anga{\tithi{14}{शुक्ल-चतुर्दशी}}{\time{41-21}{22:36}}\hspace{1ex}}%
{\anga{उत्तरप्रोष्ठपदा}{\time{6-42}{08:45}}\hspace{1ex}}{चन्द्रराशिः—\mbox{मीनः}}%
{\anga{व्याघातः}{\time{11-27}{10:38}}\hspace{1ex}\uanga{हर्षणः}}%
{\anga{गरः}{\time{11-16}{10:34}}\hspace{1ex}\anga{वणिजः}{\time{41-21}{22:36}}\hspace{1ex}\uanga{विष्टिः}}{}
}
{(सायन) विष्णुपदी-पुण्यकालः~10:28\RIGHTarrow{}17:42\eventsep ऊर्ज-मासः~16:52\RIGHTarrow{}\eventsep अपत्य-नीराजनम्\eventsep को-जागर्ति-व्रतम्\eventsep सायन-सङ्क्रमण-दिन-अपराह्ण-पुण्यकालः~11:53\RIGHTarrow{}17:42\eventsep ताम्रपर्णी-आदि-पुष्कर-समापनम्}
{Tue} 
\cfoot{\rygdata{14:47--16:15}{08:58--10:25}{11:53--13:20}}
\caldata{OCTOBER}{24}{\sunmonth{तुला}{7}{}{आश्वयुजः}{शरदृतुः}{बुधः}{विलम्बः}{दक्षिणायनम्}{शरदृतुः}}
{\sunmoonrsdata{06:04}{17:42}{17:33}{---}{11:53}
{\kalas{04:25 05:14 09:10 08:23 09:56 16:08 10:43 13:02 15:22 16:55 18:31 20:47 22:20 01:25*}}}
{\tnykdata{\anga{\tithi{15}{पौर्णमासी}}{\time{40-27}{22:15}}\hspace{1ex}}%
{\anga{रेवती}{\time{8-10}{09:20}}\hspace{1ex}}{चन्द्रराशिः—\mbox{मीनः\RIGHTarrow{09:20}}}%
{\anga{हर्षणः}{\time{8-30}{09:28}}\hspace{1ex}\uanga{वज्रम्}}%
{\anga{विष्टिः}{\time{11-4}{10:30}}\hspace{1ex}\anga{बवः}{\time{40-27}{22:15}}\hspace{1ex}\uanga{बालवः}}{}
}
{आग्रयण-होमः द्राविडेषु\eventsep काञ्ची ३६ जगद्गुरु श्री-चित्सुखानन्देन्द्र सरस्वती आराधना~\#{१२६१}\eventsep कौमुदी-उत्सवः\eventsep कुमार-पूर्णिमा/महा-अश्विनी\eventsep कुन्ती-(पार्वती)-व्रतम्\eventsep लक्ष्मी-इन्द्र-कुबेर-पूजा\eventsep मीराबाई-जयन्ती~\#{५२१}\eventsep महा-अन्नाभिषेकः\eventsep पार्वणव्रतम् पूर्णिमायाम्\eventsep पूर्णिमा-व्रतम्\eventsep पञ्च-पर्व-पूजा (पूर्णिमा)\eventsep वाल्मीकि-महर्षि-जयन्ती\eventsep वेङ्कटाचले पूर्णिमा-गरुड-सेवा\eventsep शरद्-पूर्णिमा}
{Wed} 
\cfoot{\rygdata{11:53--13:20}{07:31--08:58}{10:25--11:53}}
\caldata{OCTOBER}{25}{\sunmonth{तुला}{8}{}{आश्वयुजः}{शरदृतुः}{गुरुः}{विलम्बः}{दक्षिणायनम्}{शरदृतुः}}
{\sunmoonsrdata{06:04}{17:41}{18:18}{06:21}{11:53}
{\kalas{04:25 05:14 09:10 08:23 09:56 16:08 10:43 13:02 15:22 16:55 18:31 20:47 22:20 01:25*}}}
{\tnykdata{\anga{\tithi{16}{कृष्ण-प्रथमा}}{\time{38-19}{21:24}}\hspace{1ex}}%
{\anga{अश्विनी}{\time{8-17}{09:23}}\hspace{1ex}}{चन्द्रराशिः—\mbox{मेषः}}%
{\anga{वज्रम्}{\time{4-28}{07:51}}\hspace{1ex}\anga{सिद्धिः}{\time{59-33}{05:53*}}\hspace{1ex}\uanga{व्यतीपातः}}%
{\anga{बालवः}{\time{9-31}{09:52}}\hspace{1ex}\anga{कौलवः}{\time{38-19}{21:24}}\hspace{1ex}\uanga{तैतिलः}}{}
}
{जयावाप्ति-व्रतम्\eventsep पार्वण-प्रायश्चित्तावकाशः दर्शे\eventsep पूर्णमासेष्टिः\eventsep सप्तम-अपरपक्ष-आरम्भः\eventsep स्थालीपाकः\eventsep \tamil{திருமூல நாயனார் (29) குருபூஜை}}
{Thu} 
\cfoot{\rygdata{13:20--14:47}{06:04--07:31}{08:58--10:25}}
\caldata{OCTOBER}{26}{\sunmonth{तुला}{9}{}{आश्वयुजः}{शरदृतुः}{शुक्रः}{विलम्बः}{दक्षिणायनम्}{शरदृतुः}}
{\sunmoonsrdata{06:04}{17:41}{19:07}{07:17}{11:52}
{\kalas{04:25 05:15 09:10 08:24 09:56 16:08 10:43 13:02 15:21 16:54 18:30 20:47 22:20 01:25*}}}
{\tnykdata{\anga{\tithi{17}{कृष्ण-द्वितीया}}{\time{35-13}{20:09}}\hspace{1ex}}%
{\anga{अपभरणी}{\time{7-20}{09:00}}\hspace{1ex}}{चन्द्रराशिः—\mbox{मेषः\RIGHTarrow{14:51}}}%
{\anga{व्यतीपातः}{\time{53-56}{03:39*}}\hspace{1ex}\uanga{वरीयान्}}%
{\anga{तैतिलः}{\time{6-52}{08:49}}\hspace{1ex}\anga{गरः}{\time{35-13}{20:09}}\hspace{1ex}\uanga{वणिजः}}{}
}
{अशून्यशयन-व्रतम्\eventsep चन्द्रोदय-गौरी-व्रतम्\eventsep \tamil{இடங்கழி நாயனார் (52) குருபூஜை}\eventsep कृत्तिका-व्रतम्\eventsep \tamil{நின்றசீர் நெடுமாற நாயனார் (48) குருபூஜை}\eventsep व्यतीपात-श्राद्धम्}
{Fri} 
\cfoot{\rygdata{10:25--11:52}{14:47--16:14}{07:31--08:58}}
\caldata{OCTOBER}{27}{\sunmonth{तुला}{10}{}{आश्वयुजः}{शरदृतुः}{शनिः}{विलम्बः}{दक्षिणायनम्}{शरदृतुः}}
{\sunmoonsrdata{06:04}{17:40}{20:00}{08:16}{11:52}
{\kalas{04:25 05:15 09:10 08:24 09:56 16:07 10:43 13:02 15:21 16:54 18:30 20:46 22:19 01:25*}}}
{\tnykdata{\anga{\tithi{18}{कृष्ण-तृतीया}}{\time{31-23}{18:38}}\hspace{1ex}}%
{\anga{कृत्तिका}{\time{5-33}{08:18}}\hspace{1ex}}{चन्द्रराशिः—\mbox{वृषभः}}%
{\anga{वरीयान्}{\time{47-47}{01:11*}}\hspace{1ex}\uanga{परिघः}}%
{\anga{वणिजः}{\time{3-22}{07:25}}\hspace{1ex}\anga{विष्टिः}{\time{31-23}{18:38}}\hspace{1ex}\anga{बवः}{\time{59-17}{05:47*}}\hspace{1ex}\uanga{बालवः}}{}
}
{कनक-गणेश-व्रतम्\eventsep करक-चतुर्थी\eventsep ललिता-गौरी-व्रतम्\eventsep वक्रतुण्ड-महागणपति सङ्कटहर-चतुर्थी-व्रतम्\eventsep शनिरोहिणी-पुण्यकालः~08:18\RIGHTarrow{}}
{Sat} 
\cfoot{\rygdata{08:58--10:25}{13:19--14:46}{06:04--07:31}}
\caldata{OCTOBER}{28}{\sunmonth{तुला}{11}{}{आश्वयुजः}{शरदृतुः}{भानुः}{विलम्बः}{दक्षिणायनम्}{शरदृतुः}}
{\sunmoonsrdata{06:05}{17:40}{20:57}{09:15}{11:52}
{\kalas{04:25 05:15 09:10 08:24 09:56 16:07 10:43 13:02 15:21 16:53 18:29 20:46 22:19 01:25*}}}
{\tnykdata{\anga{\tithi{19}{कृष्ण-चतुर्थी}}{\time{27-4}{16:54}}\hspace{1ex}}%
{\anga{रोहिणी}{\time{3-10}{07:21}}\hspace{1ex}}{चन्द्रराशिः—\mbox{वृषभः\RIGHTarrow{18:49}}}%
{\anga{परिघः}{\time{41-17}{22:36}}\hspace{1ex}\uanga{शिवः}}%
{\anga{बालवः}{\time{27-4}{16:54}}\hspace{1ex}\anga{कौलवः}{\time{54-47}{04:00*}}\hspace{1ex}\uanga{तैतिलः}}{}
}
{}
{Sun} 
\cfoot{\rygdata{16:13--17:40}{11:52--13:19}{14:46--16:13}}
\caldata{OCTOBER}{29}{\sunmonth{तुला}{12}{}{आश्वयुजः}{शरदृतुः}{सोमः}{विलम्बः}{दक्षिणायनम्}{शरदृतुः}}
{\sunmoonsrdata{06:05}{17:39}{21:57}{10:15}{11:52}
{\kalas{04:25 05:15 09:10 08:24 09:56 16:07 10:43 13:02 15:21 16:53 18:29 20:46 22:19 01:25*}}}
{\tnykdata{\anga{\tithi{20}{कृष्ण-पञ्चमी}}{\time{22-26}{15:03}}\hspace{1ex}}%
{\anga{मृगशीर्षम्}{\time{0-24}{06:15}}\hspace{1ex}\anga{आर्द्रा}{\time{57-25}{05:03*}}\hspace{1ex}}{चन्द्रराशिः—\mbox{मिथुनम्}}%
{\anga{शिवः}{\time{34-35}{19:55}}\hspace{1ex}\uanga{सिद्धः}}%
{\anga{तैतिलः}{\time{22-26}{15:03}}\hspace{1ex}\anga{गरः}{\time{50-3}{02:06*}}\hspace{1ex}\uanga{वणिजः}}{}
}
{घोटक-पञ्चमी\eventsep सेङ्गालिपुरम् अनन्तराम-दीक्षित-आराधना~\#{४९}\eventsep सोममृगशीर्ष-पुण्यकालः\RIGHTarrow{}06:15}
{Mon} 
\cfoot{\rygdata{07:32--08:58}{10:25--11:52}{13:19--14:46}}
\caldata{OCTOBER}{30}{\sunmonth{तुला}{13}{}{आश्वयुजः}{शरदृतुः}{मङ्गलः}{विलम्बः}{दक्षिणायनम्}{शरदृतुः}}
{\sunmoonsrdata{06:05}{17:39}{22:58}{11:13}{11:52}
{\kalas{04:26 05:15 09:10 08:24 09:56 16:06 10:43 13:01 15:20 16:53 18:29 20:46 22:19 01:25*}}}
{\tnykdata{\anga{\tithi{21}{कृष्ण-षष्ठी}}{\time{17-36}{13:08}}\hspace{1ex}}%
{\anga{पुनर्वसुः}{\time{54-18}{03:48*}}\hspace{1ex}}{चन्द्रराशिः—\mbox{मिथुनम्\RIGHTarrow{22:07}}}%
{\anga{सिद्धः}{\time{27-43}{17:11}}\hspace{1ex}\uanga{साध्यः}}%
{\anga{वणिजः}{\time{17-36}{13:08}}\hspace{1ex}\anga{विष्टिः}{\time{45-9}{00:09*}}\hspace{1ex}\uanga{बवः}}{}
}
{}
{Tue} 
\cfoot{\rygdata{14:46--16:12}{08:59--10:25}{11:52--13:19}}
\caldata{OCTOBER}{31}{\sunmonth{तुला}{14}{}{आश्वयुजः}{शरदृतुः}{बुधः}{विलम्बः}{दक्षिणायनम्}{शरदृतुः}}
{\sunmoonsrdata{06:05}{17:39}{23:58}{12:08}{11:52}
{\kalas{04:26 05:16 09:10 08:24 09:57 16:06 10:43 13:01 15:20 16:52 18:28 20:45 22:19 01:25*}}}
{\tnykdata{\anga{\tithi{22}{कृष्ण-सप्तमी}}{\time{12-40}{11:10}}\hspace{1ex}}%
{\anga{पुष्यः}{\time{51-5}{02:32*}}\hspace{1ex}}{चन्द्रराशिः—\mbox{कर्कटः}}%
{\anga{साध्यः}{\time{20-46}{14:24}}\hspace{1ex}\uanga{शुभः}}%
{\anga{बवः}{\time{12-40}{11:10}}\hspace{1ex}\anga{बालवः}{\time{40-11}{22:10}}\hspace{1ex}\uanga{कौलवः}}{}
}
{पञ्च-पर्व-पूजा (अष्टमी)}
{Wed} 
\cfoot{\rygdata{11:52--13:19}{07:32--08:59}{10:25--11:52}}
\caldata{NOVEMBER}{1}{\sunmonth{तुला}{15}{}{आश्वयुजः}{शरदृतुः}{गुरुः}{विलम्बः}{दक्षिणायनम्}{शरदृतुः}}
{\sunmoonsrdata{06:06}{17:38}{00:57*}{12:59}{11:52}
{\kalas{04:26 05:16 09:10 08:24 09:57 16:06 10:43 13:01 15:20 16:52 18:28 20:45 22:19 01:26*}}}
{\tnykdata{\anga{\tithi{23}{कृष्ण-अष्टमी}}{\time{7-40}{09:10}}\hspace{1ex}}%
{\anga{आश्रेषा}{\time{47-50}{01:14*}}\hspace{1ex}}{चन्द्रराशिः—\mbox{कर्कटः\RIGHTarrow{01:14*}}}%
{\anga{शुभः}{\time{13-46}{11:36}}\hspace{1ex}\uanga{शुक्लः}}%
{\anga{कौलवः}{\time{7-40}{09:10}}\hspace{1ex}\anga{तैतिलः}{\time{35-9}{20:10}}\hspace{1ex}\uanga{गरः}}{}
}
{जीमूतवाहन-पूजा\eventsep जीवपुत्रिकाष्टमी\eventsep कालाष्टमी\eventsep मङ्गल-व्रतम्\eventsep महालक्ष्मी-व्रतम्}
{Thu} 
\cfoot{\rygdata{13:19--14:45}{06:06--07:32}{08:59--10:25}}
\caldata{NOVEMBER}{2}{\sunmonth{तुला}{16}{}{आश्वयुजः}{शरदृतुः}{शुक्रः}{विलम्बः}{दक्षिणायनम्}{शरदृतुः}}
{\sunmoonsrdata{06:06}{17:38}{01:53*}{13:46}{11:52}
{\kalas{04:26 05:16 09:10 08:24 09:57 16:06 10:43 13:01 15:20 16:52 18:28 20:45 22:19 01:26*}}}
{\tnykdata{\anga{\tithi{24}{कृष्ण-नवमी}}{\time{2-38}{07:09}}\hspace{1ex}\anga{\tithi{25}{कृष्ण-दशमी}}{\time{57-39}{05:10*}}\hspace{1ex}\avamA{}}%
{\anga{मघा}{\time{44-35}{23:56}}\hspace{1ex}}{चन्द्रराशिः—\mbox{सिंहः}}%
{\anga{शुक्लः}{\time{6-45}{08:48}}\hspace{1ex}\anga{ब्रह्म}{\time{59-47}{06:01*}}\hspace{1ex}\uanga{इन्द्रः}}%
{\anga{गरः}{\time{2-38}{07:09}}\hspace{1ex}\anga{वणिजः}{\time{30-8}{18:09}}\hspace{1ex}\anga{विष्टिः}{\time{57-39}{05:10*}}\hspace{1ex}\uanga{बवः}}{}
}
{भीमसेन-जयन्ती}
{Fri} 
\cfoot{\rygdata{10:25--11:52}{14:45--16:11}{07:33--08:59}}
\caldata{NOVEMBER}{3}{\sunmonth{तुला}{17}{}{आश्वयुजः}{शरदृतुः}{शनिः}{विलम्बः}{दक्षिणायनम्}{शरदृतुः}}
{\sunmoonsrdata{06:06}{17:38}{02:49*}{14:32}{11:52}
{\kalas{04:26 05:16 09:11 08:25 09:57 16:05 10:43 13:01 15:19 16:52 18:28 20:45 22:18 01:26*}}}
{\tnykdata{\anga{\tithi{26}{कृष्ण-एकादशी}}{\time{52-48}{03:14*}}\hspace{1ex}}%
{\anga{पूर्वफल्गुनी}{\time{41-27}{22:41}}\hspace{1ex}}{चन्द्रराशिः—\mbox{सिंहः\RIGHTarrow{04:23*}}}%
{\anga{इन्द्रः}{\time{52-54}{03:16*}}\hspace{1ex}\uanga{वैधृतिः}}%
{\anga{बवः}{\time{25-12}{16:11}}\hspace{1ex}\anga{बालवः}{\time{52-48}{03:14*}}\hspace{1ex}\uanga{कौलवः}}{}
}
{\tamil{சத்தி நாயனார் (44) குருபூஜை}\eventsep कामाक्षी-जयन्ती\eventsep स्मार्त-रमा-एकादशी\eventsep शृङ्गेरी ३४ जगद्गुरु श्री-चन्द्रशेखर भारती-३ जयन्ती}
{Sat} 
\cfoot{\rygdata{08:59--10:25}{13:18--14:45}{06:06--07:33}}
\caldata{NOVEMBER}{4}{\sunmonth{तुला}{18}{}{आश्वयुजः}{शरदृतुः}{भानुः}{विलम्बः}{दक्षिणायनम्}{शरदृतुः}}
{\sunmoonsrdata{06:07}{17:37}{03:43*}{15:15}{11:52}
{\kalas{04:27 05:17 09:11 08:25 09:57 16:05 10:43 13:01 15:19 16:51 18:27 20:45 22:18 01:26*}}}
{\tnykdata{\anga{\tithi{27}{कृष्ण-द्वादशी}}{\time{48-14}{01:24*}}\hspace{1ex}}%
{\anga{उत्तरफल्गुनी}{\time{38-34}{21:32}}\hspace{1ex}}{चन्द्रराशिः—\mbox{कन्या}}%
{\anga{वैधृतिः}{\time{46-15}{00:37*}}\hspace{1ex}\uanga{विष्कम्भः}}%
{\anga{कौलवः}{\time{20-27}{14:18}}\hspace{1ex}\anga{तैतिलः}{\time{48-14}{01:24*}}\hspace{1ex}\uanga{गरः}}{}
}
{गोवत्स-द्वादशी\eventsep हरिवासरः\RIGHTarrow{}08:45\eventsep वैष्णव-रमा-एकादशी\eventsep वैधृति-श्राद्धम्\eventsep वसुदेव-पूजा\eventsep व्याघ्र-द्वादशी}
{Sun} 
\cfoot{\rygdata{16:11--17:37}{11:52--13:18}{14:45--16:11}}
\caldata{NOVEMBER}{5}{\sunmonth{तुला}{19}{}{आश्वयुजः}{शरदृतुः}{सोमः}{विलम्बः}{दक्षिणायनम्}{शरदृतुः}}
{\sunmoonsrdata{06:07}{17:37}{04:38*}{15:59}{11:52}
{\kalas{04:27 05:17 09:11 08:25 09:57 16:05 10:43 13:01 15:19 16:51 18:27 20:45 22:18 01:26*}}}
{\tnykdata{\anga{\tithi{28}{कृष्ण-त्रयोदशी}}{\time{44-8}{23:47}}\hspace{1ex}}%
{\anga{हस्तः}{\time{36-7}{20:34}}\hspace{1ex}}{चन्द्रराशिः—\mbox{कन्या}}%
{\anga{विष्कम्भः}{\time{39-58}{22:06}}\hspace{1ex}\uanga{प्रीतिः}}%
{\anga{गरः}{\time{16-6}{12:34}}\hspace{1ex}\anga{वणिजः}{\time{44-8}{23:47}}\hspace{1ex}\uanga{विष्टिः}}{}
}
{(यम)-दीप-त्रयोदशी\eventsep धन-त्रयोदशी\eventsep धन्वन्तरि-जयन्ती\eventsep गो-त्रिरात्रि-व्रतम्\eventsep मासशिवरात्रिः\eventsep सोम-प्रदोष-व्रतम्~17:37\RIGHTarrow{}19:11}
{Mon} 
\cfoot{\rygdata{07:33--09:00}{10:26--11:52}{13:18--14:45}}
\caldata{NOVEMBER}{6}{\sunmonth{तुला}{20}{}{आश्वयुजः}{शरदृतुः}{मङ्गलः}{विलम्बः}{दक्षिणायनम्}{शरदृतुः}}
{\sunmoonsrdata{06:07}{17:37}{05:32*}{16:43}{11:52}
{\kalas{04:27 05:17 09:11 08:25 09:57 16:05 10:43 13:01 15:19 16:51 18:27 20:45 22:18 01:26*}}}
{\tnykdata{\anga{\tithi{29}{कृष्ण-चतुर्दशी}}{\time{40-48}{22:27}}\hspace{1ex}}%
{\anga{चित्रा}{\time{34-22}{19:52}}\hspace{1ex}}{चन्द्रराशिः—\mbox{कन्या\RIGHTarrow{08:11}}}%
{\anga{प्रीतिः}{\time{34-17}{19:50}}\hspace{1ex}\uanga{आयुष्मान्}}%
{\anga{विष्टिः}{\time{12-21}{11:04}}\hspace{1ex}\anga{शकुनिः}{\time{40-48}{22:27}}\hspace{1ex}\uanga{चतुष्पात्}}{}
}
{दीपोत्सव-चतुर्दशी/यम-तर्पणम्\eventsep देवी-पर्व-७\eventsep कृष्णाङ्गारक-चतुर्दशी-पुण्यकालः/यम-तर्पणम्\eventsep नरक-चतुर्दशी\eventsep पञ्च-पर्व-पूजा (चतुर्दशी)\eventsep प्रेत-चतुर्दशी\eventsep शृङ्गेरी ३५ जगद्गुरु श्री-अभिनव विद्यातीर्थ महास्वामी जयन्ती}
{Tue} 
\cfoot{\rygdata{14:44--16:11}{09:00--10:26}{11:52--13:18}}
\caldata{NOVEMBER}{7}{\sunmonth{तुला}{21}{}{आश्वयुजः}{शरदृतुः}{बुधः}{विलम्बः}{दक्षिणायनम्}{शरदृतुः}}
{\sunmoonsrdata{06:08}{17:36}{---}{17:29}{11:52}
{\kalas{04:27 05:18 09:11 08:25 09:57 16:05 10:43 13:01 15:19 16:51 18:27 20:44 22:18 01:26*}}}
{\tnykdata{\anga{\tithi{30}{अमावास्या}}{\time{38-30}{21:32}}\hspace{1ex}}%
{\anga{स्वाती}{\time{33-35}{19:34}}\hspace{1ex}}{चन्द्रराशिः—\mbox{तुला}}%
{\anga{आयुष्मान्}{\time{29-22}{17:52}}\hspace{1ex}\uanga{सौभाग्यः}}%
{\anga{चतुष्पात्}{\time{9-30}{09:56}}\hspace{1ex}\anga{नाग}{\time{38-30}{21:32}}\hspace{1ex}\uanga{किंस्तुघ्नः}}{}
}
{आग्रयण-होमः द्राविडेषु\eventsep दीपावली/लक्ष्मी-कुबेर-पूजा\eventsep केदार-गौरी-व्रतम्\eventsep पार्वणव्रतम् अमावास्यायाम्\eventsep पञ्च-पर्व-पूजा (अमावास्या)\eventsep सप्तम-अपरपक्ष-समापनम्\eventsep सर्व-आश्वयुज-अमावास्या (अलभ्यम्–स्वाती)\eventsep विक्रमादित्य-पट्टाभिषेकः\eventsep श्रीराम-पट्टाभिषेकः}
{Wed} 
\cfoot{\rygdata{11:52--13:18}{07:34--09:00}{10:26--11:52}}
\caldata{NOVEMBER}{8}{\sunmonth{तुला}{22}{}{कार्त्तिकः}{शरदृतुः}{गुरुः}{विलम्बः}{दक्षिणायनम्}{शरदृतुः}}
{\sunmoonrsdata{06:08}{17:36}{06:28}{18:16}{11:52}
{\kalas{04:28 05:18 09:11 08:26 09:57 16:04 10:43 13:01 15:19 16:50 18:26 20:44 22:18 01:26*}}}
{\tnykdata{\anga{\tithi{1}{शुक्ल-प्रथमा}}{\time{37-28}{21:07}}\hspace{1ex}}%
{\anga{विशाखा}{\time{34-2}{19:45}}\hspace{1ex}}{चन्द्रराशिः—\mbox{तुला\RIGHTarrow{13:39}}}%
{\anga{सौभाग्यः}{\time{25-26}{16:18}}\hspace{1ex}\uanga{शोभनः}}%
{\anga{किंस्तुघ्नः}{\time{7-48}{09:15}}\hspace{1ex}\anga{बवः}{\time{37-28}{21:07}}\hspace{1ex}\uanga{बालवः}}{}
}
{दर्शेष्टिः\eventsep गोवर्धन-पूजा\eventsep पार्वण-प्रायश्चित्तावकाशः पौर्णमास्याम्\eventsep स्थालीपाकः}
{Thu} 
\cfoot{\rygdata{13:18--14:44}{06:08--07:34}{09:00--10:26}}
\caldata{NOVEMBER}{9}{\sunmonth{तुला}{23}{}{कार्त्तिकः}{शरदृतुः}{शुक्रः}{विलम्बः}{दक्षिणायनम्}{शरदृतुः}}
{\sunmoonrsdata{06:08}{17:36}{07:23}{19:06}{11:52}
{\kalas{04:28 05:18 09:12 08:26 09:58 16:04 10:43 13:01 15:19 16:50 18:26 20:44 22:18 01:26*}}}
{\tnykdata{\anga{\tithi{2}{शुक्ल-द्वितीया}}{\time{37-59}{21:20}}\hspace{1ex}}%
{\anga{अनूराधा}{\time{35-58}{20:32}}\hspace{1ex}}{चन्द्रराशिः—\mbox{वृश्चिकः}}%
{\anga{शोभनः}{\time{22-39}{15:12}}\hspace{1ex}\uanga{अतिगण्डः}}%
{\anga{बालवः}{\time{7-31}{09:09}}\hspace{1ex}\anga{कौलवः}{\time{37-59}{21:20}}\hspace{1ex}\uanga{तैतिलः}}{}
}
{चन्द्र-दर्शनम्~17:36\RIGHTarrow{}19:06\eventsep \tamil{பூசலார் நாயனார் (56) குருபூஜை}\eventsep यम/भ्रातृ-द्वितीया}
{Fri} 
\cfoot{\rygdata{10:26--11:52}{14:44--16:10}{07:34--09:00}}
\caldata{NOVEMBER}{10}{\sunmonth{तुला}{24}{}{कार्त्तिकः}{शरदृतुः}{शनिः}{विलम्बः}{दक्षिणायनम्}{शरदृतुः}}
{\sunmoonrsdata{06:09}{17:36}{08:17}{19:57}{11:52}
{\kalas{04:28 05:18 09:12 08:26 09:58 16:04 10:44 13:01 15:18 16:50 18:26 20:44 22:18 01:27*}}}
{\tnykdata{\anga{\tithi{3}{शुक्ल-तृतीया}}{\time{40-8}{22:12}}\hspace{1ex}}%
{\anga{ज्येष्ठा}{\time{39-29}{21:56}}\hspace{1ex}}{चन्द्रराशिः—\mbox{वृश्चिकः\RIGHTarrow{21:56}}}%
{\anga{अतिगण्डः}{\time{21-7}{14:36}}\hspace{1ex}\uanga{सुकर्म}}%
{\anga{तैतिलः}{\time{8-50}{09:41}}\hspace{1ex}\anga{गरः}{\time{40-8}{22:12}}\hspace{1ex}\uanga{वणिजः}}{}
}
{}
{Sat} 
\cfoot{\rygdata{09:01--10:26}{13:18--14:44}{06:09--07:35}}
\caldata{NOVEMBER}{11}{\sunmonth{तुला}{25}{}{कार्त्तिकः}{शरदृतुः}{भानुः}{विलम्बः}{दक्षिणायनम्}{शरदृतुः}}
{\sunmoonrsdata{06:09}{17:36}{09:10}{20:48}{11:52}
{\kalas{04:29 05:19 09:12 08:26 09:58 16:04 10:44 13:01 15:18 16:50 18:26 20:44 22:18 01:27*}}}
{\tnykdata{\anga{\tithi{4}{शुक्ल-चतुर्थी}}{\time{43-57}{23:44}}\hspace{1ex}}%
{\anga{मूला}{\time{44-35}{23:59}}\hspace{1ex}}{चन्द्रराशिः—\mbox{धनुः}}%
{\anga{सुकर्म}{\time{20-52}{14:30}}\hspace{1ex}\uanga{धृतिः}}%
{\anga{वणिजः}{\time{11-50}{10:53}}\hspace{1ex}\anga{विष्टिः}{\time{43-57}{23:44}}\hspace{1ex}\uanga{बवः}}{}
}
{\tamil{ஐயடிகள் காடவர்கோன் நாயனார் (45) குருபூஜை}\eventsep शुक्ल-चतुर्थी-व्रतम्}
{Sun} 
\cfoot{\rygdata{16:10--17:36}{11:52--13:18}{14:44--16:10}}
\caldata{NOVEMBER}{12}{\sunmonth{तुला}{26}{}{कार्त्तिकः}{शरदृतुः}{सोमः}{विलम्बः}{दक्षिणायनम्}{शरदृतुः}}
{\sunmoonrsdata{06:09}{17:36}{10:00}{21:39}{11:53}
{\kalas{04:29 05:19 09:12 08:27 09:58 16:04 10:44 13:01 15:18 16:50 18:26 20:44 22:18 01:27*}}}
{\tnykdata{\anga{\tithi{5}{शुक्ल-पञ्चमी}}{\time{49-13}{01:51*}}\hspace{1ex}}%
{\anga{पूर्वाषाढा}{\time{51-3}{02:35*}}\hspace{1ex}}{चन्द्रराशिः—\mbox{धनुः}}%
{\anga{धृतिः}{\time{21-47}{14:52}}\hspace{1ex}\uanga{शूलः}}%
{\anga{बवः}{\time{16-25}{12:44}}\hspace{1ex}\anga{बालवः}{\time{49-13}{01:51*}}\hspace{1ex}\uanga{कौलवः}}{}
}
{देवसेना-पञ्चमी\eventsep कृत्तिका-सोमवासरः\eventsep पाण्डव-(लाभ)-पञ्चमी\eventsep सर्प-पूजा\eventsep तिरुवनन्तपुर-देवायतन-प्रवेश-घोषणा~\#{८३}}
{Mon} 
\cfoot{\rygdata{07:35--09:01}{10:27--11:52}{13:18--14:44}}
\caldata{NOVEMBER}{13}{\sunmonth{तुला}{27}{}{कार्त्तिकः}{शरदृतुः}{मङ्गलः}{विलम्बः}{दक्षिणायनम्}{शरदृतुः}}
{\sunmoonrsdata{06:10}{17:35}{10:47}{22:29}{11:53}
{\kalas{04:29 05:20 09:13 08:27 09:58 16:04 10:44 13:01 15:18 16:50 18:26 20:44 22:19 01:27*}}}
{\tnykdata{\anga{\tithi{6}{शुक्ल-षष्ठी}}{\time{55-30}{04:22*}}\hspace{1ex}}%
{\anga{उत्तराषाढा}{\time{58-29}{05:34*}}\hspace{1ex}}{चन्द्रराशिः—\mbox{धनुः\RIGHTarrow{09:18}}}%
{\anga{शूलः}{\time{23-36}{15:37}}\hspace{1ex}\uanga{गण्डः}}%
{\anga{कौलवः}{\time{22-15}{15:04}}\hspace{1ex}\anga{तैतिलः}{\time{55-30}{04:22*}}\hspace{1ex}\uanga{गरः}}{}
}
{स्कन्दषष्ठी-व्रतम्}
{Tue} 
\cfoot{\rygdata{14:44--16:10}{09:01--10:27}{11:53--13:18}}
\caldata{NOVEMBER}{14}{\sunmonth{तुला}{28}{}{कार्त्तिकः}{शरदृतुः}{बुधः}{विलम्बः}{दक्षिणायनम्}{शरदृतुः}}
{\sunmoonrsdata{06:10}{17:35}{11:31}{23:18}{11:53}
{\kalas{04:30 05:20 09:13 08:27 09:59 16:04 10:44 13:01 15:18 16:50 18:26 20:44 22:19 01:27*}}}
{\tnykdata{\fulltithi{\tithi{7}{शुक्ल-सप्तमी}}}%
{\fullanga{श्रवणः}}{चन्द्रराशिः—\mbox{मकरः}}%
{\anga{गण्डः}{\time{25-58}{16:34}}\hspace{1ex}\uanga{वृद्धिः}}%
{\anga{गरः}{\time{28-50}{17:43}}\hspace{1ex}\uanga{वणिजः}}{}
}
{सावित्र्य-कल्पादिः\eventsep श्रवण-व्रतम्}
{Wed} 
\cfoot{\rygdata{11:53--13:18}{07:36--09:02}{10:27--11:53}}
\caldata{NOVEMBER}{15}{\sunmonth{तुला}{29}{}{कार्त्तिकः}{शरदृतुः}{गुरुः}{विलम्बः}{दक्षिणायनम्}{शरदृतुः}}
{\sunmoonrsdata{06:11}{17:35}{12:12}{00:06*}{11:53}
{\kalas{04:30 05:20 09:13 08:28 09:59 16:04 10:44 13:01 15:18 16:50 18:26 20:44 22:19 01:28*}}}
{\tnykdata{\anga{\tithi{7}{शुक्ल-सप्तमी}}{\time{2-12}{07:04}}\hspace{1ex}}%
{\anga{श्रवणः}{\time{6-17}{08:42}}\hspace{1ex}}{चन्द्रराशिः—\mbox{मकरः\RIGHTarrow{22:14}}}%
{\anga{वृद्धिः}{\time{28-26}{17:33}}\hspace{1ex}\uanga{ध्रुवः}}%
{\anga{वणिजः}{\time{2-12}{07:04}}\hspace{1ex}\anga{विष्टिः}{\time{35-32}{20:24}}\hspace{1ex}\uanga{बवः}}{}
}
{गोपाष्टमी\eventsep \tamil{பொய்கையாழ்வார் திருநக்ஷத்திரம்}}
{Thu} 
\cfoot{\rygdata{13:19--14:44}{06:11--07:36}{09:02--10:27}}
\caldata{NOVEMBER}{16}{\sunmonth{तुला}{30}{\mbox{तुला{\tiny\RIGHTarrow}{18:02}}}{कार्त्तिकः}{शरदृतुः}{शुक्रः}{विलम्बः}{दक्षिणायनम्}{शरदृतुः}}
{\sunmoonrsdata{06:11}{17:35}{12:51}{00:53*}{11:53}
{\kalas{04:30 05:21 09:14 08:28 09:59 16:04 10:45 13:02 15:18 16:50 18:26 20:44 22:19 01:28*}}}
{\tnykdata{\anga{\tithi{8}{शुक्ल-अष्टमी}}{\time{8-42}{09:40}}\hspace{1ex}}%
{\anga{श्रविष्ठा}{\time{13-49}{11:43}}\hspace{1ex}}{चन्द्रराशिः—\mbox{कुम्भः}}%
{\anga{ध्रुवः}{\time{30-31}{18:24}}\hspace{1ex}\uanga{व्याघातः}}%
{\anga{बवः}{\time{8-42}{09:40}}\hspace{1ex}\anga{बालवः}{\time{41-38}{22:51}}\hspace{1ex}\uanga{कौलवः}}{}
}
{आकाशदीप-समापनम्\eventsep \tamil{பூதத்தாழ்வார் திருநக்ஷத்திரம்}\eventsep देवी-पर्व-८\eventsep काञ्ची २२ जगद्गुरु श्री-परिपूर्णबोधेन्द्र सरस्वती आराधना~\#{१५३८}\eventsep कार्तवीर्यार्जुन-जयन्ती\eventsep सङ्क्रमण-दिन-अपराह्ण-पुण्यकालः~11:53\RIGHTarrow{}17:35\eventsep तुला-कावेरी-स्नान-समापनम्\eventsep वृश्चिक-रवि-सङ्क्रमण-विष्णुपदी-पुण्यकालः~11:38\RIGHTarrow{}17:35}
{Fri} 
\cfoot{\rygdata{10:28--11:53}{14:44--16:10}{07:37--09:02}}
\caldata{NOVEMBER}{17}{\sunmonth{वृश्चिकः}{1}{}{कार्त्तिकः}{शरदृतुः}{शनिः}{विलम्बः}{दक्षिणायनम्}{शरदृतुः}}
{\sunmoonrsdata{06:12}{17:35}{13:29}{01:40*}{11:53}
{\kalas{04:31 05:21 09:14 08:28 09:59 16:04 10:45 13:02 15:18 16:50 18:25 20:44 22:19 01:28*}}}
{\tnykdata{\anga{\tithi{9}{शुक्ल-नवमी}}{\time{14-16}{11:54}}\hspace{1ex}}%
{\anga{शतभिषक्}{\time{20-27}{14:23}}\hspace{1ex}}{चन्द्रराशिः—\mbox{कुम्भः}}%
{\anga{व्याघातः}{\time{31-46}{18:54}}\hspace{1ex}\uanga{हर्षणः}}%
{\anga{कौलवः}{\time{14-16}{11:54}}\hspace{1ex}\anga{तैतिलः}{\time{46-32}{00:49*}}\hspace{1ex}\uanga{गरः}}{}
}
{अक्षया-नवमी\eventsep जगद्धात्री-पूजा\eventsep कृत्तिका-मण्डल-पारायण-आरम्भः\eventsep \tamil{முடவன் முழுக்கு}\eventsep त्रेतायुगादिः}
{Sat} 
\cfoot{\rygdata{09:03--10:28}{13:19--14:44}{06:12--07:37}}
\caldata{NOVEMBER}{18}{\sunmonth{वृश्चिकः}{2}{}{कार्त्तिकः}{शरदृतुः}{भानुः}{विलम्बः}{दक्षिणायनम्}{शरदृतुः}}
{\sunmoonrsdata{06:12}{17:35}{14:07}{02:28*}{11:54}
{\kalas{04:31 05:22 09:14 08:29 10:00 16:04 10:45 13:02 15:18 16:49 18:25 20:44 22:19 01:28*}}}
{\tnykdata{\anga{\tithi{10}{शुक्ल-दशमी}}{\time{18-23}{13:33}}\hspace{1ex}}%
{\anga{पूर्वप्रोष्ठपदा}{\time{25-40}{16:28}}\hspace{1ex}}{चन्द्रराशिः—\mbox{कुम्भः\RIGHTarrow{10:01}}}%
{\anga{हर्षणः}{\time{31-50}{18:56}}\hspace{1ex}\uanga{वज्रम्}}%
{\anga{गरः}{\time{18-23}{13:33}}\hspace{1ex}\anga{वणिजः}{\time{49-48}{02:07*}}\hspace{1ex}\uanga{विष्टिः}}{}
}
{\tamil{கார்த்திகை~ஞாயிற்றுக்கிழமை}\eventsep कंस-वधः}
{Sun} 
\cfoot{\rygdata{16:10--17:35}{11:53--13:19}{14:44--16:10}}
\caldata{NOVEMBER}{19}{\sunmonth{वृश्चिकः}{3}{}{कार्त्तिकः}{शरदृतुः}{सोमः}{विलम्बः}{दक्षिणायनम्}{शरदृतुः}}
{\sunmoonrsdata{06:12}{17:35}{14:45}{03:17*}{11:54}
{\kalas{04:32 05:22 09:14 08:29 10:00 16:04 10:45 13:02 15:19 16:50 18:25 20:45 22:19 01:29*}}}
{\tnykdata{\anga{\tithi{11}{शुक्ल-एकादशी}}{\time{20-43}{14:30}}\hspace{1ex}}%
{\anga{उत्तरप्रोष्ठपदा}{\time{29-8}{17:52}}\hspace{1ex}}{चन्द्रराशिः—\mbox{मीनः}}%
{\anga{वज्रम्}{\time{30-30}{18:25}}\hspace{1ex}\uanga{सिद्धिः}}%
{\anga{विष्टिः}{\time{20-43}{14:30}}\hspace{1ex}\anga{बवः}{\time{51-10}{02:41*}}\hspace{1ex}\uanga{बालवः}}{}
}
{आदि-शङ्कर मानसिक-सन्न्यास-दिनम्\eventsep भीष्म-पञ्चक-व्रत-आरम्भः\eventsep गुरुवायुपुर-एकादशी\eventsep कृत्तिका-सोमवासरः\eventsep कैशिक-एकादशी\eventsep सर्व-उत्थान-एकादशी}
{Mon} 
\cfoot{\rygdata{07:38--09:03}{10:28--11:54}{13:19--14:44}}
\caldata{NOVEMBER}{20}{\sunmonth{वृश्चिकः}{4}{}{कार्त्तिकः}{शरदृतुः}{मङ्गलः}{विलम्बः}{दक्षिणायनम्}{शरदृतुः}}
{\sunmoonrsdata{06:13}{17:35}{15:26}{04:09*}{11:54}
{\kalas{04:32 05:22 09:15 08:29 10:00 16:04 10:46 13:02 15:19 16:50 18:26 20:45 22:19 01:29*}}}
{\tnykdata{\anga{\tithi{12}{शुक्ल-द्वादशी}}{\time{21-7}{14:40}}\hspace{1ex}}%
{\anga{रेवती}{\time{30-45}{18:31}}\hspace{1ex}}{चन्द्रराशिः—\mbox{मीनः\RIGHTarrow{18:31}}}%
{\anga{सिद्धिः}{\time{27-42}{17:18}}\hspace{1ex}\uanga{व्यतीपातः}}%
{\anga{बालवः}{\time{21-7}{14:40}}\hspace{1ex}\anga{कौलवः}{\time{50-38}{02:28*}}\hspace{1ex}\uanga{तैतिलः}}{}
}
{बृन्दावन-द्वादशी\eventsep द्विदल-व्रत-समापनम्\eventsep गोपद्म-व्रत-समापनम्\eventsep कैशिक-द्वादशी\eventsep मन्वादिः-(स्वारोचिषः-[२])\eventsep प्रबोधोत्सवः\eventsep प्रदोष-व्रतम्~17:35\RIGHTarrow{}19:10\eventsep तुलसी-विवाहः\eventsep याज्ञवल्क्य-जयन्ती}
{Tue} 
\cfoot{\rygdata{14:44--16:10}{09:04--10:29}{11:54--13:19}}
\caldata{NOVEMBER}{21}{\sunmonth{वृश्चिकः}{5}{}{कार्त्तिकः}{शरदृतुः}{बुधः}{विलम्बः}{दक्षिणायनम्}{शरदृतुः}}
{\sunmoonrsdata{06:13}{17:35}{16:09}{05:04*}{11:54}
{\kalas{04:32 05:23 09:15 08:30 10:01 16:04 10:46 13:02 15:19 16:50 18:26 20:45 22:20 01:29*}}}
{\tnykdata{\anga{\tithi{13}{शुक्ल-त्रयोदशी}}{\time{19-41}{14:06}}\hspace{1ex}}%
{\anga{अश्विनी}{\time{30-36}{18:28}}\hspace{1ex}}{चन्द्रराशिः—\mbox{मेषः}}%
{\anga{व्यतीपातः}{\time{23-28}{15:37}}\hspace{1ex}\uanga{वरीयान्}}%
{\anga{तैतिलः}{\time{19-41}{14:06}}\hspace{1ex}\anga{गरः}{\time{48-21}{01:34*}}\hspace{1ex}\uanga{वणिजः}}{}
}
{कार्त्तिक-मास-अन्तिमत्रयतिथि-व्रत-आरम्भः\eventsep व्यतीपात-श्राद्धम्}
{Wed} 
\cfoot{\rygdata{11:54--13:19}{07:39--09:04}{10:29--11:54}}
\caldata{NOVEMBER}{22}{\sunmonth{वृश्चिकः}{6}{}{कार्त्तिकः}{शरदृतुः}{गुरुः}{विलम्बः}{दक्षिणायनम्}{शरदृतुः}}
{\sunmoonrsdata{06:14}{17:35}{16:57}{06:02*}{11:54}
{\kalas{04:33 05:23 09:15 08:30 10:01 16:04 10:46 13:03 15:19 16:50 18:26 20:45 22:20 01:30*}}}
{\tnykdata{\anga{\tithi{14}{शुक्ल-चतुर्दशी}}{\time{16-38}{12:53}}\hspace{1ex}}%
{\anga{अपभरणी}{\time{28-54}{17:48}}\hspace{1ex}}{चन्द्रराशिः—\mbox{मेषः\RIGHTarrow{23:33}}}%
{\anga{वरीयान्}{\time{17-58}{13:25}}\hspace{1ex}\uanga{परिघः}}%
{\anga{वणिजः}{\time{16-38}{12:53}}\hspace{1ex}\anga{विष्टिः}{\time{44-36}{00:05*}}\hspace{1ex}\uanga{बवः}}{}
}
{(सायन) षडशीति-पुण्यकालः~14:31\RIGHTarrow{}17:35\eventsep \tamil{பரணீ~தீபம்}\eventsep \tamil{கார்த்திகை}\eventsep मणिकर्णिका-स्नानम्/वैकुण्ठ-चतुर्दशी\eventsep मन्वादिः-(धर्मः-[११])\eventsep पार्वणव्रतम् पूर्णिमायाम्\eventsep पञ्च-पर्व-पूजा (पूर्णिमा)\eventsep रवि-सङ्क्रमण-पुण्यकालः~08:07\RIGHTarrow{}17:35\eventsep सायन-सङ्क्रमण-दिन-अपराह्ण-पुण्यकालः~11:54\RIGHTarrow{}17:35\eventsep सहो-मासः/हेमन्तऋतुः~14:31\RIGHTarrow{}\eventsep \tamil{ஸர்வாலய~தீபம்}\eventsep \tamil{திருவண்ணாமலை~தீபம்}\eventsep त्रिपुरोत्सवः\eventsep वेङ्कटाचले पूर्णिमा-गरुड-सेवा\eventsep श्री-गोविन्द भगवत्पाद आराधना}
{Thu} 
\cfoot{\rygdata{13:20--14:45}{06:14--07:39}{09:04--10:29}}
\caldata{NOVEMBER}{23}{\sunmonth{वृश्चिकः}{7}{}{कार्त्तिकः}{शरदृतुः}{शुक्रः}{विलम्बः}{दक्षिणायनम्}{शरदृतुः}}
{\sunmoonrsdata{06:14}{17:35}{17:50}{---}{11:55}
{\kalas{04:33 05:24 09:16 08:31 10:01 16:04 10:47 13:03 15:19 16:50 18:26 20:45 22:20 01:30*}}}
{\tnykdata{\anga{\tithi{15}{पौर्णमासी}}{\time{12-15}{11:09}}\hspace{1ex}}%
{\anga{कृत्तिका}{\time{25-59}{16:38}}\hspace{1ex}}{चन्द्रराशिः—\mbox{वृषभः}}%
{\anga{परिघः}{\time{11-25}{10:49}}\hspace{1ex}\uanga{शिवः}}%
{\anga{बवः}{\time{12-15}{11:09}}\hspace{1ex}\anga{बालवः}{\time{39-41}{22:07}}\hspace{1ex}\uanga{कौलवः}}{}
}
{आग्रयण-होमः द्राविडेषु\eventsep भीष्म-पञ्चक-व्रत-समापनम्\eventsep चातुर्मास्यव्रत-समापनम्\eventsep काञ्ची ६४ जगद्गुरु श्री-चन्द्रशेखरेन्द्र सरस्वती ५ आराधना~\#{१६८}\eventsep कार्त्तिक-मास-अन्तिमत्रयतिथि-व्रत-समापनम्\eventsep कार्त्तिक-पूर्णिमा-स्नानम्\eventsep कृत्तिका-व्रतम्\eventsep \tamil{கணம்புல்ல நாயனார் (46) குருபூஜை}\eventsep नवम-अपरपक्ष-आरम्भः\eventsep पार्वण-प्रायश्चित्तावकाशः दर्शे\eventsep पूर्णमासेष्टिः\eventsep पूर्णिमा-व्रतम्\eventsep स्थालीपाकः\eventsep \tamil{திருமங்கையாழ்வார் திருநக்ஷத்திரம்}}
{Fri} 
\cfoot{\rygdata{10:30--11:55}{14:45--16:10}{07:40--09:05}}
\caldata{NOVEMBER}{24}{\sunmonth{वृश्चिकः}{8}{}{कार्त्तिकः}{शरदृतुः}{शनिः}{विलम्बः}{दक्षिणायनम्}{शरदृतुः}}
{\sunmoonsrdata{06:15}{17:35}{18:47}{07:03}{11:55}
{\kalas{04:34 05:24 09:16 08:31 10:02 16:04 10:47 13:03 15:19 16:50 18:26 20:45 22:20 01:30*}}}
{\tnykdata{\anga{\tithi{16}{कृष्ण-प्रथमा}}{\time{6-54}{09:01}}\hspace{1ex}}%
{\anga{रोहिणी}{\time{22-10}{15:07}}\hspace{1ex}}{चन्द्रराशिः—\mbox{वृषभः\RIGHTarrow{02:16*}}}%
{\anga{शिवः}{\time{4-4}{07:53}}\hspace{1ex}\anga{सिद्धः}{\time{56-13}{04:44*}}\hspace{1ex}\uanga{साध्यः}}%
{\anga{कौलवः}{\time{6-54}{09:01}}\hspace{1ex}\anga{तैतिलः}{\time{33-58}{19:50}}\hspace{1ex}\uanga{गरः}}{}
}
{अशून्यशयन-व्रतम्\eventsep शनिरोहिणी-पुण्यकालः\RIGHTarrow{}15:07}
{Sat} 
\cfoot{\rygdata{09:05--10:30}{13:20--14:45}{06:15--07:40}}
\caldata{NOVEMBER}{25}{\sunmonth{वृश्चिकः}{9}{}{कार्त्तिकः}{शरदृतुः}{भानुः}{विलम्बः}{दक्षिणायनम्}{शरदृतुः}}
{\sunmoonsrdata{06:15}{17:35}{19:48}{08:05}{11:55}
{\kalas{04:34 05:25 09:17 08:31 10:02 16:05 10:47 13:03 15:19 16:50 18:26 20:45 22:20 01:31*}}}
{\tnykdata{\anga{\tithi{17}{कृष्ण-द्वितीया}}{\time{0-54}{06:37}}\hspace{1ex}\anga{\tithi{18}{कृष्ण-तृतीया}}{\time{54-36}{04:06*}}\hspace{1ex}\avamA{}}%
{\anga{मृगशीर्षम्}{\time{17-49}{13:23}}\hspace{1ex}}{चन्द्रराशिः—\mbox{मिथुनम्}}%
{\anga{साध्यः}{\time{48-4}{01:29*}}\hspace{1ex}\uanga{शुभः}}%
{\anga{गरः}{\time{0-54}{06:37}}\hspace{1ex}\anga{वणिजः}{\time{27-46}{17:22}}\hspace{1ex}\anga{विष्टिः}{\time{54-36}{04:06*}}\hspace{1ex}\uanga{बवः}}{}
}
{काञ्ची ९ जगद्गुरु श्री-कृपाशङ्करेन्द्र सरस्वती आराधना~\#{१९५०}\eventsep \tamil{கார்த்திகை~ஞாயிற்றுக்கிழமை}\eventsep सौभाग्य-सुन्दरी-व्रतम्}
{Sun} 
\cfoot{\rygdata{16:10--17:35}{11:55--13:20}{14:45--16:10}}
\caldata{NOVEMBER}{26}{\sunmonth{वृश्चिकः}{10}{}{कार्त्तिकः}{शरदृतुः}{सोमः}{विलम्बः}{दक्षिणायनम्}{शरदृतुः}}
{\sunmoonsrdata{06:16}{17:35}{20:50}{09:06}{11:56}
{\kalas{04:35 05:25 09:17 08:32 10:02 16:05 10:48 13:04 15:19 16:50 18:26 20:46 22:21 01:31*}}}
{\tnykdata{\anga{\tithi{19}{कृष्ण-चतुर्थी}}{\time{48-17}{01:35*}}\hspace{1ex}}%
{\anga{आर्द्रा}{\time{13-16}{11:35}}\hspace{1ex}}{चन्द्रराशिः—\mbox{मिथुनम्\RIGHTarrow{04:14*}}}%
{\anga{शुभः}{\time{39-53}{22:13}}\hspace{1ex}\uanga{शुक्लः}}%
{\anga{बवः}{\time{21-25}{14:50}}\hspace{1ex}\anga{बालवः}{\time{48-17}{01:35*}}\hspace{1ex}\uanga{कौलवः}}{}
}
{गणाधिप-महागणपति सङ्कटहर-चतुर्थी-व्रतम्\eventsep कृत्तिका-सोमवासरः}
{Mon} 
\cfoot{\rygdata{07:41--09:06}{10:31--11:56}{13:21--14:45}}
\caldata{NOVEMBER}{27}{\sunmonth{वृश्चिकः}{11}{}{कार्त्तिकः}{शरदृतुः}{मङ्गलः}{विलम्बः}{दक्षिणायनम्}{शरदृतुः}}
{\sunmoonsrdata{06:16}{17:35}{21:52}{10:03}{11:56}
{\kalas{04:35 05:26 09:18 08:32 10:03 16:05 10:48 13:04 15:20 16:50 18:26 20:46 22:21 01:31*}}}
{\tnykdata{\anga{\tithi{20}{कृष्ण-पञ्चमी}}{\time{42-9}{23:08}}\hspace{1ex}}%
{\anga{पुनर्वसुः}{\time{8-46}{09:47}}\hspace{1ex}}{चन्द्रराशिः—\mbox{कर्कटः}}%
{\anga{शुक्लः}{\time{31-49}{19:00}}\hspace{1ex}\uanga{ब्रह्म}}%
{\anga{कौलवः}{\time{15-10}{12:21}}\hspace{1ex}\anga{तैतिलः}{\time{42-9}{23:08}}\hspace{1ex}\uanga{गरः}}{}
}
{तिरुविशलूर् गङ्गाकर्षण-महोत्सव-आरम्भः}
{Tue} 
\cfoot{\rygdata{14:46--16:10}{09:06--10:31}{11:56--13:21}}
\caldata{NOVEMBER}{28}{\sunmonth{वृश्चिकः}{12}{}{कार्त्तिकः}{शरदृतुः}{बुधः}{विलम्बः}{दक्षिणायनम्}{शरदृतुः}}
{\sunmoonsrdata{06:17}{17:36}{22:52}{10:56}{11:56}
{\kalas{04:35 05:26 09:18 08:33 10:03 16:05 10:48 13:04 15:20 16:50 18:26 20:46 22:21 01:32*}}}
{\tnykdata{\anga{\tithi{21}{कृष्ण-षष्ठी}}{\time{36-25}{20:51}}\hspace{1ex}}%
{\anga{पुष्यः}{\time{4-33}{08:06}}\hspace{1ex}}{चन्द्रराशिः—\mbox{कर्कटः}}%
{\anga{ब्रह्म}{\time{24-3}{15:54}}\hspace{1ex}\uanga{इन्द्रः}}%
{\anga{गरः}{\time{9-13}{09:58}}\hspace{1ex}\anga{वणिजः}{\time{36-25}{20:51}}\hspace{1ex}\uanga{विष्टिः}}{}
}
{}
{Wed} 
\cfoot{\rygdata{11:56--13:21}{07:42--09:07}{10:32--11:56}}
\caldata{NOVEMBER}{29}{\sunmonth{वृश्चिकः}{13}{}{कार्त्तिकः}{शरदृतुः}{गुरुः}{विलम्बः}{दक्षिणायनम्}{शरदृतुः}}
{\sunmoonsrdata{06:18}{17:36}{23:49}{11:45}{11:57}
{\kalas{04:36 05:27 09:18 08:33 10:04 16:05 10:49 13:04 15:20 16:51 18:27 20:46 22:22 01:32*}}}
{\tnykdata{\anga{\tithi{22}{कृष्ण-सप्तमी}}{\time{31-10}{18:46}}\hspace{1ex}}%
{\anga{आश्रेषा}{\time{0-45}{06:36}}\hspace{1ex}\anga{मघा}{\time{57-31}{05:18*}}\hspace{1ex}}{चन्द्रराशिः—\mbox{कर्कटः\RIGHTarrow{06:36}}}%
{\anga{इन्द्रः}{\time{16-40}{12:58}}\hspace{1ex}\uanga{वैधृतिः}}%
{\anga{विष्टिः}{\time{3-43}{07:47}}\hspace{1ex}\anga{बवः}{\time{31-10}{18:46}}\hspace{1ex}\anga{बालवः}{\time{58-47}{05:48*}}\hspace{1ex}\uanga{कौलवः}}{}
}
{कालभैरवाष्टमी\eventsep पञ्च-पर्व-पूजा (अष्टमी)\eventsep वैधृति-श्राद्धम्}
{Thu} 
\cfoot{\rygdata{13:21--14:46}{06:18--07:42}{09:07--10:32}}
\caldata{NOVEMBER}{30}{\sunmonth{वृश्चिकः}{14}{}{कार्त्तिकः}{शरदृतुः}{शुक्रः}{विलम्बः}{दक्षिणायनम्}{शरदृतुः}}
{\sunmoonsrdata{06:18}{17:36}{00:44*}{12:31}{11:57}
{\kalas{04:36 05:27 09:19 08:34 10:04 16:05 10:49 13:05 15:20 16:51 18:27 20:47 22:22 01:33*}}}
{\tnykdata{\anga{\tithi{23}{कृष्ण-अष्टमी}}{\time{26-31}{16:55}}\hspace{1ex}}%
{\anga{पूर्वफल्गुनी}{\time{54-52}{04:15*}}\hspace{1ex}}{चन्द्रराशिः—\mbox{सिंहः}}%
{\anga{वैधृतिः}{\time{9-43}{10:12}}\hspace{1ex}\uanga{विष्कम्भः}}%
{\anga{कौलवः}{\time{26-31}{16:55}}\hspace{1ex}\anga{तैतिलः}{\time{54-27}{04:05*}}\hspace{1ex}\uanga{गरः}}{}
}
{काञ्ची ४२ जगद्गुरु श्री-ब्रह्मानन्दघनेन्द्र सरस्वती २ आराधना~\#{१०४१}\eventsep काञ्ची ४९ जगद्गुरु श्री-महादेवेन्द्र सरस्वती ३ आराधना~\#{७७२}\eventsep काञ्ची ५८ जगद्गुरु श्री-आत्मबोधेन्द्र सरस्वती आराधना~\#{३८१}\eventsep महादेवाष्टमी}
{Fri} 
\cfoot{\rygdata{10:32--11:57}{14:46--16:11}{07:43--09:08}}
\caldata{DECEMBER}{1}{\sunmonth{वृश्चिकः}{15}{}{कार्त्तिकः}{शरदृतुः}{शनिः}{विलम्बः}{दक्षिणायनम्}{शरदृतुः}}
{\sunmoonsrdata{06:19}{17:36}{01:38*}{13:15}{11:57}
{\kalas{04:37 05:28 09:19 08:34 10:04 16:06 10:50 13:05 15:21 16:51 18:27 20:47 22:22 01:33*}}}
{\tnykdata{\anga{\tithi{24}{कृष्ण-नवमी}}{\time{22-31}{15:19}}\hspace{1ex}}%
{\anga{उत्तरफल्गुनी}{\time{52-53}{03:28*}}\hspace{1ex}}{चन्द्रराशिः—\mbox{सिंहः\RIGHTarrow{10:02}}}%
{\anga{विष्कम्भः}{\time{3-17}{07:38}}\hspace{1ex}\anga{प्रीतिः}{\time{57-25}{05:17*}}\hspace{1ex}\uanga{आयुष्मान्}}%
{\anga{गरः}{\time{22-31}{15:19}}\hspace{1ex}\anga{वणिजः}{\time{50-47}{02:37*}}\hspace{1ex}\uanga{विष्टिः}}{}
}
{\tamil{மெய்ப்பொருள் நாயனார் (4) குருபூஜை}}
{Sat} 
\cfoot{\rygdata{09:08--10:33}{13:22--14:47}{06:19--07:43}}
\caldata{DECEMBER}{2}{\sunmonth{वृश्चिकः}{16}{}{कार्त्तिकः}{शरदृतुः}{भानुः}{विलम्बः}{दक्षिणायनम्}{शरदृतुः}}
{\sunmoonsrdata{06:19}{17:36}{02:32*}{13:57}{11:58}
{\kalas{04:37 05:28 09:20 08:35 10:05 16:06 10:50 13:05 15:21 16:51 18:27 20:47 22:23 01:33*}}}
{\tnykdata{\anga{\tithi{25}{कृष्ण-दशमी}}{\time{19-13}{14:00}}\hspace{1ex}}%
{\anga{हस्तः}{\time{51-36}{02:58*}}\hspace{1ex}}{चन्द्रराशिः—\mbox{कन्या}}%
{\anga{आयुष्मान्}{\time{52-6}{03:10*}}\hspace{1ex}\uanga{सौभाग्यः}}%
{\anga{विष्टिः}{\time{19-13}{14:00}}\hspace{1ex}\anga{बवः}{\time{47-51}{01:28*}}\hspace{1ex}\uanga{बालवः}}{}
}
{आदित्यहस्त-पुण्यकालः\eventsep \tamil{ஆனாய நாயனார் (13) குருபூஜை}\eventsep काञ्ची २८ जगद्गुरु श्री-महादेवेन्द्र सरस्वती १ आराधना~\#{१४१८}\eventsep \tamil{கார்த்திகை~ஞாயிற்றுக்கிழமை}}
{Sun} 
\cfoot{\rygdata{16:12--17:36}{11:58--13:22}{14:47--16:12}}
\caldata{DECEMBER}{3}{\sunmonth{वृश्चिकः}{17}{}{कार्त्तिकः}{शरदृतुः}{सोमः}{विलम्बः}{दक्षिणायनम्}{शरदृतुः}}
{\sunmoonsrdata{06:20}{17:37}{03:25*}{14:40}{11:58}
{\kalas{04:38 05:29 09:20 08:35 10:05 16:06 10:50 13:06 15:21 16:51 18:27 20:47 22:23 01:34*}}}
{\tnykdata{\anga{\tithi{26}{कृष्ण-एकादशी}}{\time{16-40}{13:00}}\hspace{1ex}}%
{\anga{चित्रा}{\time{51-7}{02:47*}}\hspace{1ex}}{चन्द्रराशिः—\mbox{कन्या\RIGHTarrow{14:50}}}%
{\anga{सौभाग्यः}{\time{47-25}{01:18*}}\hspace{1ex}\uanga{शोभनः}}%
{\anga{बालवः}{\time{16-40}{13:00}}\hspace{1ex}\anga{कौलवः}{\time{45-43}{00:37*}}\hspace{1ex}\uanga{तैतिलः}}{}
}
{कृत्तिका-सोमवासरः\eventsep सर्व-उत्पन्ना-एकादशी}
{Mon} 
\cfoot{\rygdata{07:44--09:09}{10:34--11:58}{13:23--14:47}}
\caldata{DECEMBER}{4}{\sunmonth{वृश्चिकः}{18}{}{कार्त्तिकः}{शरदृतुः}{मङ्गलः}{विलम्बः}{दक्षिणायनम्}{शरदृतुः}}
{\sunmoonsrdata{06:20}{17:37}{04:19*}{15:24}{11:58}
{\kalas{04:38 05:29 09:21 08:36 10:06 16:07 10:51 13:06 15:22 16:52 18:28 20:48 22:23 01:34*}}}
{\tnykdata{\anga{\tithi{27}{कृष्ण-द्वादशी}}{\time{14-58}{12:20}}\hspace{1ex}}%
{\anga{स्वाती}{\time{51-31}{02:57*}}\hspace{1ex}}{चन्द्रराशिः—\mbox{तुला}}%
{\anga{शोभनः}{\time{43-26}{23:43}}\hspace{1ex}\uanga{अतिगण्डः}}%
{\anga{तैतिलः}{\time{14-58}{12:20}}\hspace{1ex}\anga{गरः}{\time{44-29}{00:08*}}\hspace{1ex}\uanga{वणिजः}}{}
}
{प्रदोष-व्रतम्~17:37\RIGHTarrow{}19:12}
{Tue} 
\cfoot{\rygdata{14:48--16:12}{09:09--10:34}{11:58--13:23}}
\caldata{DECEMBER}{5}{\sunmonth{वृश्चिकः}{19}{}{कार्त्तिकः}{शरदृतुः}{बुधः}{विलम्बः}{दक्षिणायनम्}{शरदृतुः}}
{\sunmoonsrdata{06:21}{17:37}{05:13*}{16:10}{11:59}
{\kalas{04:39 05:30 09:21 08:36 10:06 16:07 10:51 13:07 15:22 16:52 18:28 20:48 22:24 01:35*}}}
{\tnykdata{\anga{\tithi{28}{कृष्ण-त्रयोदशी}}{\time{14-14}{12:03}}\hspace{1ex}}%
{\anga{विशाखा}{\time{52-55}{03:31*}}\hspace{1ex}}{चन्द्रराशिः—\mbox{तुला\RIGHTarrow{21:20}}}%
{\anga{अतिगण्डः}{\time{40-14}{22:27}}\hspace{1ex}\uanga{सुकर्म}}%
{\anga{वणिजः}{\time{14-14}{12:03}}\hspace{1ex}\anga{विष्टिः}{\time{44-16}{00:04*}}\hspace{1ex}\uanga{शकुनिः}}{}
}
{मासशिवरात्रिः\eventsep पञ्च-पर्व-पूजा (चतुर्दशी)}
{Wed} 
\cfoot{\rygdata{11:59--13:23}{07:45--09:10}{10:34--11:59}}
\caldata{DECEMBER}{6}{\sunmonth{वृश्चिकः}{20}{}{कार्त्तिकः}{शरदृतुः}{गुरुः}{विलम्बः}{दक्षिणायनम्}{शरदृतुः}}
{\sunmoonsrdata{06:21}{17:37}{06:08*}{16:58}{11:59}
{\kalas{04:39 05:30 09:22 08:37 10:07 16:07 10:52 13:07 15:22 16:52 18:28 20:48 22:24 01:35*}}}
{\tnykdata{\anga{\tithi{29}{कृष्ण-चतुर्दशी}}{\time{14-35}{12:12}}\hspace{1ex}}%
{\anga{अनूराधा}{\time{55-28}{04:33*}}\hspace{1ex}}{चन्द्रराशिः—\mbox{वृश्चिकः}}%
{\anga{सुकर्म}{\time{37-56}{21:32}}\hspace{1ex}\uanga{धृतिः}}%
{\anga{शकुनिः}{\time{14-35}{12:12}}\hspace{1ex}\anga{चतुष्पात्}{\time{45-14}{00:27*}}\hspace{1ex}\uanga{नाग}}{}
}
{नवम-अपरपक्ष-समापनम्\eventsep पञ्च-पर्व-पूजा (अमावास्या)\eventsep सर्व-कार्त्तिक-अमावास्या (अलभ्यम्–अनूराधा, पुष्कला)\eventsep तिरुविशलूर् गङ्गाकर्षण-महोत्सव-समापनम्}
{Thu} 
\cfoot{\rygdata{13:24--14:48}{06:21--07:46}{09:10--10:35}}
\caldata{DECEMBER}{7}{\sunmonth{वृश्चिकः}{21}{}{कार्त्तिकः}{शरदृतुः}{शुक्रः}{विलम्बः}{दक्षिणायनम्}{शरदृतुः}}
{\sunmoonsrdata{06:22}{17:38}{---}{17:48}{12:00}
{\kalas{04:40 05:31 09:22 08:37 10:07 16:08 10:52 13:07 15:23 16:53 18:29 20:49 22:24 01:36*}}}
{\tnykdata{\anga{\tithi{30}{अमावास्या}}{\time{16-10}{12:50}}\hspace{1ex}}%
{\anga{ज्येष्ठा}{\time{59-14}{06:04*}}\hspace{1ex}}{चन्द्रराशिः—\mbox{वृश्चिकः\RIGHTarrow{06:04*}}}%
{\anga{धृतिः}{\time{36-35}{21:00}}\hspace{1ex}\uanga{शूलः}}%
{\anga{नाग}{\time{16-10}{12:50}}\hspace{1ex}\anga{किंस्तुघ्नः}{\time{47-26}{01:21*}}\hspace{1ex}\uanga{बवः}}{}
}
{आग्रयण-होमः द्राविडेषु\eventsep बोधायन-इष्टिः\eventsep काञ्ची १८ जगद्गुरु श्री-योगतिलक सुरेन्द्र सरस्वती आराधना~\#{१६३४}\eventsep कार्त्तिक-स्नानपूर्तिः\eventsep पार्वणव्रतम् अमावास्यायाम्}
{Fri} 
\cfoot{\rygdata{10:35--12:00}{14:49--16:13}{07:46--09:11}}
\caldata{DECEMBER}{8}{\sunmonth{वृश्चिकः}{22}{}{मार्गशीर्षः}{हेमन्तऋतुः}{शनिः}{विलम्बः}{दक्षिणायनम्}{शरदृतुः}}
{\sunmoonrsdata{06:22}{17:38}{07:01}{18:39}{12:00}
{\kalas{04:40 05:31 09:23 08:38 10:08 16:08 10:53 13:08 15:23 16:53 18:29 20:49 22:25 01:36*}}}
{\tnykdata{\anga{\tithi{1}{शुक्ल-प्रथमा}}{\time{19-2}{13:59}}\hspace{1ex}}%
{\fullanga{मूला}}{चन्द्रराशिः—\mbox{धनुः}}%
{\anga{शूलः}{\time{36-14}{20:52}}\hspace{1ex}\uanga{गण्डः}}%
{\anga{बवः}{\time{19-2}{13:59}}\hspace{1ex}\anga{बालवः}{\time{50-58}{02:46*}}\hspace{1ex}\uanga{कौलवः}}{}
}
{चन्द्र-दर्शनम्~17:38\RIGHTarrow{}18:39\eventsep दर्शेष्टिः\eventsep \tamil{மூர்க்க நாயனார் (31) குருபூஜை}\eventsep पार्वण-प्रायश्चित्तावकाशः पौर्णमास्याम्\eventsep स्थालीपाकः\eventsep वनदुर्गानवरात्र-आरम्भः}
{Sat} 
\cfoot{\rygdata{09:11--10:36}{13:25--14:49}{06:22--07:47}}
\caldata{DECEMBER}{9}{\sunmonth{वृश्चिकः}{23}{}{मार्गशीर्षः}{हेमन्तऋतुः}{भानुः}{विलम्बः}{दक्षिणायनम्}{शरदृतुः}}
{\sunmoonrsdata{06:23}{17:38}{07:52}{19:30}{12:01}
{\kalas{04:41 05:32 09:23 08:38 10:08 16:08 10:53 13:08 15:23 16:53 18:29 20:50 22:25 01:37*}}}
{\tnykdata{\anga{\tithi{2}{शुक्ल-द्वितीया}}{\time{23-12}{15:40}}\hspace{1ex}}%
{\anga{मूला}{\time{4-14}{08:05}}\hspace{1ex}}{चन्द्रराशिः—\mbox{धनुः}}%
{\anga{गण्डः}{\time{36-50}{21:07}}\hspace{1ex}\uanga{वृद्धिः}}%
{\anga{कौलवः}{\time{23-12}{15:40}}\hspace{1ex}\anga{तैतिलः}{\time{55-46}{04:42*}}\hspace{1ex}\uanga{गरः}}{}
}
{\tamil{கார்த்திகை~ஞாயிற்றுக்கிழமை}\eventsep तिन्त्रिणी-गौरी-व्रतम्}
{Sun} 
\cfoot{\rygdata{16:14--17:38}{12:01--13:25}{14:49--16:14}}
\caldata{DECEMBER}{10}{\sunmonth{वृश्चिकः}{24}{}{मार्गशीर्षः}{हेमन्तऋतुः}{सोमः}{विलम्बः}{दक्षिणायनम्}{शरदृतुः}}
{\sunmoonrsdata{06:24}{17:39}{08:41}{20:21}{12:01}
{\kalas{04:41 05:33 09:24 08:39 10:09 16:09 10:54 13:09 15:24 16:54 18:30 20:50 22:26 01:37*}}}
{\tnykdata{\anga{\tithi{3}{शुक्ल-तृतीया}}{\time{28-35}{17:50}}\hspace{1ex}}%
{\anga{पूर्वाषाढा}{\time{10-26}{10:34}}\hspace{1ex}}{चन्द्रराशिः—\mbox{धनुः\RIGHTarrow{17:15}}}%
{\anga{वृद्धिः}{\time{38-18}{21:43}}\hspace{1ex}\uanga{ध्रुवः}}%
{\anga{गरः}{\time{28-35}{17:50}}\hspace{1ex}\uanga{वणिजः}}{}
}
{\tamil{சிறப்புலி நாயனார் (34) குருபூஜை}}
{Mon} 
\cfoot{\rygdata{07:48--09:12}{10:37--12:01}{13:25--14:50}}
\caldata{DECEMBER}{11}{\sunmonth{वृश्चिकः}{25}{}{मार्गशीर्षः}{हेमन्तऋतुः}{मङ्गलः}{विलम्बः}{दक्षिणायनम्}{शरदृतुः}}
{\sunmoonrsdata{06:24}{17:39}{09:26}{21:11}{12:02}
{\kalas{04:42 05:33 09:24 08:39 10:09 16:09 10:54 13:09 15:24 16:54 18:30 20:50 22:26 01:38*}}}
{\tnykdata{\anga{\tithi{4}{शुक्ल-चतुर्थी}}{\time{34-53}{20:22}}\hspace{1ex}}%
{\anga{उत्तराषाढा}{\time{17-36}{13:27}}\hspace{1ex}}{चन्द्रराशिः—\mbox{मकरः}}%
{\anga{ध्रुवः}{\time{40-25}{22:34}}\hspace{1ex}\uanga{व्याघातः}}%
{\anga{वणिजः}{\time{1-38}{07:03}}\hspace{1ex}\anga{विष्टिः}{\time{34-53}{20:22}}\hspace{1ex}\uanga{बवः}}{}
}
{बदरी-गौरी-व्रतम्\eventsep सुखा-अङ्गारक-चतुर्थी\eventsep शुक्ल-चतुर्थी-व्रतम्}
{Tue} 
\cfoot{\rygdata{14:50--16:15}{09:13--10:37}{12:02--13:26}}
\caldata{DECEMBER}{12}{\sunmonth{वृश्चिकः}{26}{}{मार्गशीर्षः}{हेमन्तऋतुः}{बुधः}{विलम्बः}{दक्षिणायनम्}{शरदृतुः}}
{\sunmoonrsdata{06:25}{17:39}{10:08}{21:59}{12:02}
{\kalas{04:43 05:34 09:25 08:40 10:10 16:09 10:55 13:09 15:24 16:54 18:30 20:51 22:27 01:38*}}}
{\tnykdata{\anga{\tithi{5}{शुक्ल-पञ्चमी}}{\time{41-42}{23:06}}\hspace{1ex}}%
{\anga{श्रवणः}{\time{25-22}{16:34}}\hspace{1ex}}{चन्द्रराशिः—\mbox{मकरः\RIGHTarrow{06:09*}}}%
{\anga{व्याघातः}{\time{42-52}{23:34}}\hspace{1ex}\uanga{हर्षणः}}%
{\anga{बवः}{\time{8-15}{09:43}}\hspace{1ex}\anga{बालवः}{\time{41-42}{23:06}}\hspace{1ex}\uanga{कौलवः}}{}
}
{देवी-पर्व-९\eventsep श्रवण-व्रतम्}
{Wed} 
\cfoot{\rygdata{12:02--13:26}{07:49--09:13}{10:38--12:02}}
\caldata{DECEMBER}{13}{\sunmonth{वृश्चिकः}{27}{}{मार्गशीर्षः}{हेमन्तऋतुः}{गुरुः}{विलम्बः}{दक्षिणायनम्}{शरदृतुः}}
{\sunmoonrsdata{06:25}{17:40}{10:48}{22:46}{12:02}
{\kalas{04:43 05:34 09:25 08:40 10:10 16:10 10:55 13:10 15:25 16:55 18:31 20:51 22:27 01:39*}}}
{\tnykdata{\anga{\tithi{6}{शुक्ल-षष्ठी}}{\time{48-28}{01:49*}}\hspace{1ex}}%
{\anga{श्रविष्ठा}{\time{33-13}{19:42}}\hspace{1ex}}{चन्द्रराशिः—\mbox{कुम्भः}}%
{\anga{हर्षणः}{\time{45-18}{00:32*}}\hspace{1ex}\uanga{वज्रम्}}%
{\anga{कौलवः}{\time{15-7}{12:28}}\hspace{1ex}\anga{तैतिलः}{\time{48-28}{01:49*}}\hspace{1ex}\uanga{गरः}}{}
}
{काञ्ची ३२ जगद्गुरु श्री-चिदानन्दघनेन्द्र सरस्वती आराधना~\#{१३४७}\eventsep मार्गशीर्ष-शिवलिङ्ग-षष्ठी\eventsep सुब्रह्मण्य-षष्ठी-व्रतम्}
{Thu} 
\cfoot{\rygdata{13:27--14:51}{06:25--07:50}{09:14--10:38}}
\caldata{DECEMBER}{14}{\sunmonth{वृश्चिकः}{28}{}{मार्गशीर्षः}{हेमन्तऋतुः}{शुक्रः}{विलम्बः}{दक्षिणायनम्}{शरदृतुः}}
{\sunmoonrsdata{06:26}{17:40}{11:26}{23:32}{12:03}
{\kalas{04:44 05:35 09:26 08:41 10:11 16:10 10:56 13:10 15:25 16:55 18:31 20:52 22:27 01:39*}}}
{\tnykdata{\anga{\tithi{7}{शुक्ल-सप्तमी}}{\time{54-34}{04:16*}}\hspace{1ex}}%
{\anga{शतभिषक्}{\time{40-34}{22:40}}\hspace{1ex}}{चन्द्रराशिः—\mbox{कुम्भः}}%
{\anga{वज्रम्}{\time{47-15}{01:20*}}\hspace{1ex}\uanga{सिद्धिः}}%
{\anga{गरः}{\time{21-38}{15:05}}\hspace{1ex}\anga{वणिजः}{\time{54-34}{04:16*}}\hspace{1ex}\uanga{विष्टिः}}{}
}
{काञ्ची ५ जगद्गुरु श्री-ज्ञानानन्देन्द्र सरस्वती आराधना~\#{२२२३}\eventsep मित्र-सप्तमी\eventsep नारायणीयं-जयन्ती~\#{४३३}\eventsep नन्दा-सप्तमी}
{Fri} 
\cfoot{\rygdata{10:39--12:03}{14:52--16:16}{07:50--09:14}}
\caldata{DECEMBER}{15}{\sunmonth{वृश्चिकः}{29}{}{मार्गशीर्षः}{हेमन्तऋतुः}{शनिः}{विलम्बः}{दक्षिणायनम्}{शरदृतुः}}
{\sunmoonrsdata{06:26}{17:41}{12:03}{00:19*}{12:03}
{\kalas{04:44 05:35 09:26 08:41 10:11 16:11 10:56 13:11 15:26 16:56 18:32 20:52 22:28 01:39*}}}
{\tnykdata{\anga{\tithi{8}{शुक्ल-अष्टमी}}{\time{59-26}{06:13*}}\hspace{1ex}}%
{\anga{पूर्वप्रोष्ठपदा}{\time{46-51}{01:11*}}\hspace{1ex}}{चन्द्रराशिः—\mbox{कुम्भः\RIGHTarrow{18:36}}}%
{\anga{सिद्धिः}{\time{48-18}{01:46*}}\hspace{1ex}\uanga{व्यतीपातः}}%
{\anga{विष्टिः}{\time{27-11}{17:19}}\hspace{1ex}\anga{बवः}{\time{59-26}{06:13*}}\hspace{1ex}\uanga{बालवः}}{}
}
{}
{Sat} 
\cfoot{\rygdata{09:15--10:39}{13:28--14:52}{06:26--07:51}}
\caldata{DECEMBER}{16}{\sunmonth{धनुः}{1}{\mbox{वृश्चिकः{\tiny\RIGHTarrow}{08:43}}}{मार्गशीर्षः}{हेमन्तऋतुः}{भानुः}{विलम्बः}{दक्षिणायनम्}{हेमन्तऋतुः}}
{\sunmoonrsdata{06:27}{17:41}{12:40}{01:06*}{12:04}
{\kalas{04:45 05:36 09:27 08:42 10:12 16:11 10:57 13:11 15:26 16:56 18:32 20:53 22:28 01:40*}}}
{\tnykdata{\fulltithi{\tithi{9}{शुक्ल-नवमी}}}%
{\anga{उत्तरप्रोष्ठपदा}{\time{51-35}{03:05*}}\hspace{1ex}}{चन्द्रराशिः—\mbox{मीनः}}%
{\anga{व्यतीपातः}{\time{48-8}{01:42*}}\hspace{1ex}\uanga{वरीयान्}}%
{\anga{बालवः}{\time{31-13}{18:56}}\hspace{1ex}\uanga{कौलवः}}{}
}
{धनूरवि-सङ्क्रमण-षडशीति-पुण्यकालः~08:42\RIGHTarrow{}17:41\eventsep महाधनुर्व्यतीपात-श्राद्धम्\eventsep प्रलय-कल्पादिः\eventsep रवि-सङ्क्रमण-पुण्यकालः~06:27\RIGHTarrow{}15:06\eventsep सङ्क्रमण-दिन-पूर्वाह्ण-पुण्यकालः~06:27\RIGHTarrow{}12:04\eventsep वनदुर्गानवरात्र-समापनम्}
{Sun} 
\cfoot{\rygdata{16:17--17:41}{12:04--13:28}{14:52--16:17}}
\caldata{DECEMBER}{17}{\sunmonth{धनुः}{2}{}{मार्गशीर्षः}{हेमन्तऋतुः}{सोमः}{विलम्बः}{दक्षिणायनम्}{हेमन्तऋतुः}}
{\sunmoonrsdata{06:27}{17:42}{13:18}{01:56*}{12:04}
{\kalas{04:45 05:36 09:27 08:42 10:12 16:12 10:57 13:12 15:27 16:57 18:33 20:53 22:29 01:40*}}}
{\tnykdata{\anga{\tithi{9}{शुक्ल-नवमी}}{\time{2-33}{07:29}}\hspace{1ex}}%
{\anga{रेवती}{\time{54-27}{04:14*}}\hspace{1ex}}{चन्द्रराशिः—\mbox{मीनः\RIGHTarrow{04:14*}}}%
{\anga{वरीयान्}{\time{46-30}{01:03*}}\hspace{1ex}\uanga{परिघः}}%
{\anga{कौलवः}{\time{2-33}{07:29}}\hspace{1ex}\anga{तैतिलः}{\time{33-24}{19:49}}\hspace{1ex}\uanga{गरः}}{}
}
{}
{Mon} 
\cfoot{\rygdata{07:52--09:16}{10:40--12:04}{13:29--14:53}}
\caldata{DECEMBER}{18}{\sunmonth{धनुः}{3}{}{मार्गशीर्षः}{हेमन्तऋतुः}{मङ्गलः}{विलम्बः}{दक्षिणायनम्}{हेमन्तऋतुः}}
{\sunmoonrsdata{06:28}{17:42}{14:00}{02:48*}{12:05}
{\kalas{04:46 05:37 09:28 08:43 10:13 16:12 10:58 13:12 15:27 16:57 18:33 20:54 22:29 01:41*}}}
{\tnykdata{\anga{\tithi{10}{शुक्ल-दशमी}}{\time{3-42}{07:57}}\hspace{1ex}}%
{\anga{अश्विनी}{\time{55-18}{04:35*}}\hspace{1ex}}{चन्द्रराशिः—\mbox{मेषः}}%
{\anga{परिघः}{\time{43-15}{23:46}}\hspace{1ex}\uanga{शिवः}}%
{\anga{गरः}{\time{3-42}{07:57}}\hspace{1ex}\anga{वणिजः}{\time{33-30}{19:52}}\hspace{1ex}\uanga{विष्टिः}}{}
}
{भौमाश्विनी-पुण्यकालः\eventsep स्मार्त-मोक्षदा-एकादशी (गृहस्थ)\eventsep स्मार्त-वैकुण्ठ-एकादशी (गृहस्थ)}
{Tue} 
\cfoot{\rygdata{14:53--16:18}{09:16--10:41}{12:05--13:29}}
\caldata{DECEMBER}{19}{\sunmonth{धनुः}{4}{}{मार्गशीर्षः}{हेमन्तऋतुः}{बुधः}{विलम्बः}{दक्षिणायनम्}{हेमन्तऋतुः}}
{\sunmoonrsdata{06:28}{17:42}{14:44}{03:44*}{12:05}
{\kalas{04:46 05:37 09:28 08:43 10:13 16:13 10:58 13:13 15:28 16:57 18:33 20:54 22:30 01:41*}}}
{\tnykdata{\anga{\tithi{11}{शुक्ल-एकादशी}}{\time{2-46}{07:35}}\hspace{1ex}\anga{\tithi{12}{शुक्ल-द्वादशी}}{\time{59-53}{06:26*}}\hspace{1ex}\avamA{}}%
{\anga{अपभरणी}{\time{54-12}{04:09*}}\hspace{1ex}}{चन्द्रराशिः—\mbox{मेषः}}%
{\anga{शिवः}{\time{38-26}{21:51}}\hspace{1ex}\uanga{सिद्धः}}%
{\anga{विष्टिः}{\time{2-46}{07:35}}\hspace{1ex}\anga{बवः}{\time{31-33}{19:06}}\hspace{1ex}\anga{बालवः}{\time{59-53}{06:26*}}\hspace{1ex}\uanga{कौलवः}}{}
}
{गीता-जयन्ती\eventsep हरिवासरः\RIGHTarrow{}13:22\eventsep कुचेल-दिनम्\eventsep स्मार्त-मोक्षदा-एकादशी (सन्न्यस्त)\eventsep स्मार्त-वैकुण्ठ-एकादशी (सन्न्यस्त)\eventsep त्रिस्पर्शा-महाद्वादशी\eventsep वैष्णव-मोक्षदा-एकादशी\eventsep वैष्णव-वैकुण्ठ-एकादशी}
{Wed} 
\cfoot{\rygdata{12:05--13:30}{07:53--09:17}{10:41--12:05}}
\caldata{DECEMBER}{20}{\sunmonth{धनुः}{5}{}{मार्गशीर्षः}{हेमन्तऋतुः}{गुरुः}{विलम्बः}{दक्षिणायनम्}{हेमन्तऋतुः}}
{\sunmoonrsdata{06:29}{17:43}{15:34}{04:44*}{12:06}
{\kalas{04:47 05:38 09:29 08:44 10:14 16:13 10:58 13:13 15:28 16:58 18:34 20:54 22:30 01:42*}}}
{\tnykdata{\anga{\tithi{13}{शुक्ल-त्रयोदशी}}{\time{55-13}{04:34*}}\hspace{1ex}}%
{\anga{कृत्तिका}{\time{51-22}{03:02*}}\hspace{1ex}}{चन्द्रराशिः—\mbox{मेषः\RIGHTarrow{09:56}}}%
{\anga{सिद्धः}{\time{32-9}{19:21}}\hspace{1ex}\uanga{साध्यः}}%
{\anga{कौलवः}{\time{27-45}{17:35}}\hspace{1ex}\anga{तैतिलः}{\time{55-13}{04:34*}}\hspace{1ex}\uanga{गरः}}{}
}
{कृत्तिका-व्रतम्\eventsep प्रदोष-व्रतम्~17:43\RIGHTarrow{}19:19}
{Thu} 
\cfoot{\rygdata{13:30--14:54}{06:29--07:53}{09:17--10:42}}
\caldata{DECEMBER}{21}{\sunmonth{धनुः}{6}{}{मार्गशीर्षः}{हेमन्तऋतुः}{शुक्रः}{विलम्बः}{दक्षिणायनम्}{हेमन्तऋतुः}}
{\sunmoonrsdata{06:29}{17:43}{16:29}{05:46*}{12:06}
{\kalas{04:47 05:38 09:29 08:44 10:14 16:14 10:59 13:14 15:29 16:58 18:34 20:55 22:31 01:42*}}}
{\tnykdata{\anga{\tithi{14}{शुक्ल-चतुर्दशी}}{\time{49-8}{02:09*}}\hspace{1ex}}%
{\anga{रोहिणी}{\time{47-6}{01:20*}}\hspace{1ex}}{चन्द्रराशिः—\mbox{वृषभः}}%
{\anga{साध्यः}{\time{24-37}{16:20}}\hspace{1ex}\uanga{शुभः}}%
{\anga{गरः}{\time{22-19}{15:25}}\hspace{1ex}\anga{वणिजः}{\time{49-8}{02:09*}}\hspace{1ex}\uanga{विष्टिः}}{}
}
{सहस्य-मासः/उत्तरायणम्~03:52*\RIGHTarrow{}}
{Fri} 
\cfoot{\rygdata{10:42--12:06}{14:55--16:19}{07:54--09:18}}
\caldata{DECEMBER}{22}{\sunmonth{धनुः}{7}{}{मार्गशीर्षः}{हेमन्तऋतुः}{शनिः}{विलम्बः}{दक्षिणायनम्}{हेमन्तऋतुः}}
{\sunmoonrsdata{06:30}{17:44}{17:29}{---}{12:07}
{\kalas{04:48 05:39 09:30 08:45 10:15 16:14 10:59 13:14 15:29 16:59 18:35 20:55 22:31 01:43*}}}
{\tnykdata{\anga{\tithi{15}{पौर्णमासी}}{\time{42-0}{23:18}}\hspace{1ex}}%
{\anga{मृगशीर्षम्}{\time{41-46}{23:13}}\hspace{1ex}}{चन्द्रराशिः—\mbox{वृषभः\RIGHTarrow{12:19}}}%
{\anga{शुभः}{\time{16-6}{12:56}}\hspace{1ex}\uanga{शुक्लः}}%
{\anga{विष्टिः}{\time{15-40}{12:46}}\hspace{1ex}\anga{बवः}{\time{42-0}{23:18}}\hspace{1ex}\uanga{बालवः}}{}
}
{(सायन) उत्तरायण-पुण्यकालः~06:30\RIGHTarrow{}11:52\eventsep अन्धकासुर-वधः\eventsep अन्नपूर्णा-जयन्ती\eventsep दत्तात्रेय-जयन्ती\eventsep गणितज्ञ-रामानुज-जन्म~\#{१३१}\eventsep मार्गशीर्ष-पूर्णिमा\eventsep पार्वणव्रतम् पूर्णिमायाम्\eventsep पूर्णिमा-व्रतम्\eventsep पञ्च-पर्व-पूजा (पूर्णिमा)\eventsep रवि-सङ्क्रमण-पुण्यकालः~06:30\RIGHTarrow{}10:16\eventsep सायन-सङ्क्रमण-दिन-पूर्वाह्ण-पुण्यकालः~06:30\RIGHTarrow{}12:07\eventsep सर्प-बल्युत्सर्जनम्\eventsep त्रिपुर-भैरवी-जयन्ती\eventsep उत्तरायणारम्भः\eventsep वेङ्कटाचले पूर्णिमा-गरुड-सेवा}
{Sat} 
\cfoot{\rygdata{09:18--10:43}{13:31--14:55}{06:30--07:54}}
\caldata{DECEMBER}{23}{\sunmonth{धनुः}{8}{}{मार्गशीर्षः}{हेमन्तऋतुः}{भानुः}{विलम्बः}{दक्षिणायनम्}{हेमन्तऋतुः}}
{\sunmoonsrdata{06:30}{17:44}{18:33}{06:49}{12:07}
{\kalas{04:48 05:39 09:30 08:45 10:15 16:15 11:00 13:15 15:30 16:59 18:36 20:56 22:32 01:43*}}}
{\tnykdata{\anga{\tithi{16}{कृष्ण-प्रथमा}}{\time{34-12}{20:12}}\hspace{1ex}}%
{\anga{आर्द्रा}{\time{35-47}{20:49}}\hspace{1ex}}{चन्द्रराशिः—\mbox{मिथुनम्}}%
{\anga{शुक्लः}{\time{6-52}{09:15}}\hspace{1ex}\anga{ब्रह्म}{\time{57-17}{05:25*}}\hspace{1ex}\uanga{इन्द्रः}}%
{\anga{बालवः}{\time{8-9}{09:46}}\hspace{1ex}\anga{कौलवः}{\time{34-12}{20:12}}\hspace{1ex}\uanga{तैतिलः}}{}
}
{आर्द्रादर्शनम्\eventsep \tamil{சடைய நாயனார் (61) குருபூஜை}\eventsep काञ्ची १३ जगद्गुरु श्री-सच्चिद्घनेन्द्र सरस्वती आराधना~\#{१७४७}\eventsep पार्वण-प्रायश्चित्तावकाशः दर्शे\eventsep पूर्णमासेष्टिः\eventsep स्थालीपाकः}
{Sun} 
\cfoot{\rygdata{16:20--17:44}{12:07--13:32}{14:56--16:20}}
\caldata{DECEMBER}{24}{\sunmonth{धनुः}{9}{}{मार्गशीर्षः}{हेमन्तऋतुः}{सोमः}{विलम्बः}{दक्षिणायनम्}{हेमन्तऋतुः}}
{\sunmoonsrdata{06:31}{17:45}{19:38}{07:50}{12:08}
{\kalas{04:49 05:40 09:31 08:46 10:16 16:15 11:00 13:15 15:30 17:00 18:36 20:56 22:32 01:44*}}}
{\tnykdata{\anga{\tithi{17}{कृष्ण-द्वितीया}}{\time{26-8}{16:58}}\hspace{1ex}}%
{\anga{पुनर्वसुः}{\time{29-31}{18:20}}\hspace{1ex}}{चन्द्रराशिः—\mbox{मिथुनम्\RIGHTarrow{12:57}}}%
{\anga{इन्द्रः}{\time{47-35}{01:33*}}\hspace{1ex}\uanga{वैधृतिः}}%
{\anga{तैतिलः}{\time{0-10}{06:35}}\hspace{1ex}\anga{गरः}{\time{26-8}{16:58}}\hspace{1ex}\anga{वणिजः}{\time{52-7}{03:22*}}\hspace{1ex}\uanga{विष्टिः}}{}
}
{परशुराम-जयन्ती\eventsep रमण-महर्षि-जयन्ती~\#{१४०}}
{Mon} 
\cfoot{\rygdata{07:55--09:19}{10:44--12:08}{13:32--14:56}}
\caldata{DECEMBER}{25}{\sunmonth{धनुः}{10}{}{मार्गशीर्षः}{हेमन्तऋतुः}{मङ्गलः}{विलम्बः}{दक्षिणायनम्}{हेमन्तऋतुः}}
{\sunmoonsrdata{06:31}{17:45}{20:41}{08:47}{12:08}
{\kalas{04:49 05:40 09:31 08:46 10:16 16:16 11:01 13:16 15:31 17:00 18:37 20:57 22:33 01:44*}}}
{\tnykdata{\anga{\tithi{18}{कृष्ण-तृतीया}}{\time{18-9}{13:47}}\hspace{1ex}}%
{\anga{पुष्यः}{\time{23-23}{15:53}}\hspace{1ex}}{चन्द्रराशिः—\mbox{कर्कटः}}%
{\anga{वैधृतिः}{\time{38-5}{21:45}}\hspace{1ex}\uanga{विष्कम्भः}}%
{\anga{विष्टिः}{\time{18-9}{13:47}}\hspace{1ex}\anga{बवः}{\time{44-19}{00:15*}}\hspace{1ex}\uanga{बालवः}}{}
}
{अङ्गारकी आखुरथ-महागणपति सङ्कटहर-चतुर्थी-व्रतम्\eventsep वैधृति-श्राद्धम्}
{Tue} 
\cfoot{\rygdata{14:57--16:21}{09:20--10:44}{12:08--13:33}}
\caldata{DECEMBER}{26}{\sunmonth{धनुः}{11}{}{मार्गशीर्षः}{हेमन्तऋतुः}{बुधः}{विलम्बः}{दक्षिणायनम्}{हेमन्तऋतुः}}
{\sunmoonsrdata{06:32}{17:46}{21:41}{09:40}{12:09}
{\kalas{04:50 05:41 09:32 08:47 10:17 16:16 11:01 13:16 15:31 17:01 18:37 20:57 22:33 01:45*}}}
{\tnykdata{\anga{\tithi{19}{कृष्ण-चतुर्थी}}{\time{10-36}{10:46}}\hspace{1ex}}%
{\anga{आश्रेषा}{\time{17-42}{13:37}}\hspace{1ex}}{चन्द्रराशिः—\mbox{कर्कटः\RIGHTarrow{13:37}}}%
{\anga{विष्कम्भः}{\time{29-0}{18:08}}\hspace{1ex}\uanga{प्रीतिः}}%
{\anga{बालवः}{\time{10-36}{10:46}}\hspace{1ex}\anga{कौलवः}{\time{37-5}{21:22}}\hspace{1ex}\uanga{तैतिलः}}{}
}
{}
{Wed} 
\cfoot{\rygdata{12:09--13:33}{07:56--09:20}{10:45--12:09}}
\caldata{DECEMBER}{27}{\sunmonth{धनुः}{12}{}{मार्गशीर्षः}{हेमन्तऋतुः}{गुरुः}{विलम्बः}{दक्षिणायनम्}{हेमन्तऋतुः}}
{\sunmoonsrdata{06:32}{17:47}{22:39}{10:28}{12:09}
{\kalas{04:50 05:41 09:32 08:47 10:17 16:17 11:02 13:17 15:32 17:02 18:38 20:58 22:34 01:45*}}}
{\tnykdata{\anga{\tithi{20}{कृष्ण-पञ्चमी}}{\time{3-46}{08:03}}\hspace{1ex}\anga{\tithi{21}{कृष्ण-षष्ठी}}{\time{57-55}{05:42*}}\hspace{1ex}\avamA{}}%
{\anga{मघा}{\time{12-46}{11:39}}\hspace{1ex}}{चन्द्रराशिः—\mbox{सिंहः}}%
{\anga{प्रीतिः}{\time{20-36}{14:47}}\hspace{1ex}\uanga{आयुष्मान्}}%
{\anga{तैतिलः}{\time{3-46}{08:03}}\hspace{1ex}\anga{गरः}{\time{30-42}{18:49}}\hspace{1ex}\anga{वणिजः}{\time{57-55}{05:42*}}\hspace{1ex}\uanga{विष्टिः}}{}
}
{}
{Thu} 
\cfoot{\rygdata{13:34--14:58}{06:32--07:57}{09:21--10:45}}
\caldata{DECEMBER}{28}{\sunmonth{धनुः}{13}{}{मार्गशीर्षः}{हेमन्तऋतुः}{शुक्रः}{विलम्बः}{दक्षिणायनम्}{हेमन्तऋतुः}}
{\sunmoonsrdata{06:33}{17:47}{23:34}{11:14}{12:10}
{\kalas{04:51 05:42 09:32 08:48 10:18 16:17 11:02 13:17 15:32 17:02 18:38 20:59 22:34 01:46*}}}
{\tnykdata{\anga{\tithi{22}{कृष्ण-सप्तमी}}{\time{53-11}{03:49*}}\hspace{1ex}}%
{\anga{पूर्वफल्गुनी}{\time{8-49}{10:05}}\hspace{1ex}}{चन्द्रराशिः—\mbox{सिंहः\RIGHTarrow{15:45}}}%
{\anga{आयुष्मान्}{\time{13-1}{11:45}}\hspace{1ex}\uanga{सौभाग्यः}}%
{\anga{विष्टिः}{\time{25-23}{16:42}}\hspace{1ex}\anga{बवः}{\time{53-11}{03:49*}}\hspace{1ex}\uanga{बालवः}}{}
}
{मार्गशीर्ष-अष्टका-पूर्वेद्युः}
{Fri} 
\cfoot{\rygdata{10:46--12:10}{14:58--16:23}{07:57--09:21}}
\caldata{DECEMBER}{29}{\sunmonth{धनुः}{14}{}{मार्गशीर्षः}{हेमन्तऋतुः}{शनिः}{विलम्बः}{दक्षिणायनम्}{हेमन्तऋतुः}}
{\sunmoonsrdata{06:33}{17:48}{00:28*}{11:57}{12:10}
{\kalas{04:51 05:42 09:33 08:48 10:18 16:18 11:03 13:18 15:33 17:03 18:39 20:59 22:35 01:46*}}}
{\tnykdata{\anga{\tithi{23}{कृष्ण-अष्टमी}}{\time{49-42}{02:26*}}\hspace{1ex}}%
{\anga{उत्तरफल्गुनी}{\time{6-1}{08:58}}\hspace{1ex}}{चन्द्रराशिः—\mbox{कन्या}}%
{\anga{सौभाग्यः}{\time{6-23}{09:06}}\hspace{1ex}\uanga{शोभनः}}%
{\anga{बालवः}{\time{21-16}{15:04}}\hspace{1ex}\anga{कौलवः}{\time{49-42}{02:26*}}\hspace{1ex}\uanga{तैतिलः}}{}
}
{\tamil{இயற்பகை நாயனார் (2) குருபூஜை}\eventsep काञ्ची ४ जगद्गुरु श्री-सत्यबोधेन्द्र सरस्वती आराधना~\#{२२८६}\eventsep मार्गशीर्ष-अष्टका-श्राद्धम्\eventsep पञ्च-पर्व-पूजा (अष्टमी)}
{Sat} 
\cfoot{\rygdata{09:22--10:46}{13:35--14:59}{06:33--07:57}}
\caldata{DECEMBER}{30}{\sunmonth{धनुः}{15}{}{मार्गशीर्षः}{हेमन्तऋतुः}{भानुः}{विलम्बः}{दक्षिणायनम्}{हेमन्तऋतुः}}
{\sunmoonsrdata{06:34}{17:48}{01:21*}{12:40}{12:11}
{\kalas{04:51 05:43 09:33 08:49 10:18 16:18 11:03 13:18 15:33 17:03 18:39 21:00 22:35 01:47*}}}
{\tnykdata{\anga{\tithi{24}{कृष्ण-नवमी}}{\time{47-33}{01:35*}}\hspace{1ex}}%
{\anga{हस्तः}{\time{4-29}{08:21}}\hspace{1ex}}{चन्द्रराशिः—\mbox{कन्या\RIGHTarrow{20:15}}}%
{\anga{शोभनः}{\time{0-45}{06:52}}\hspace{1ex}\anga{अतिगण्डः}{\time{56-11}{05:02*}}\hspace{1ex}\uanga{सुकर्म}}%
{\anga{तैतिलः}{\time{18-27}{13:57}}\hspace{1ex}\anga{गरः}{\time{47-33}{01:35*}}\hspace{1ex}\uanga{वणिजः}}{}
}
{आदित्यहस्त-पुण्यकालः\RIGHTarrow{}08:21\eventsep मार्गशीर्ष-अन्वष्टका-श्राद्धम्}
{Sun} 
\cfoot{\rygdata{16:24--17:48}{12:11--13:35}{15:00--16:24}}
\caldata{DECEMBER}{31}{\sunmonth{धनुः}{16}{}{मार्गशीर्षः}{हेमन्तऋतुः}{सोमः}{विलम्बः}{दक्षिणायनम्}{हेमन्तऋतुः}}
{\sunmoonsrdata{06:34}{17:49}{02:15*}{13:23}{12:11}
{\kalas{04:52 05:43 09:34 08:49 10:19 16:19 11:04 13:19 15:34 17:04 18:40 21:00 22:36 01:47*}}}
{\tnykdata{\anga{\tithi{25}{कृष्ण-दशमी}}{\time{46-44}{01:16*}}\hspace{1ex}}%
{\anga{चित्रा}{\time{4-14}{08:16}}\hspace{1ex}}{चन्द्रराशिः—\mbox{तुला}}%
{\anga{सुकर्म}{\time{52-38}{03:38*}}\hspace{1ex}\uanga{धृतिः}}%
{\anga{वणिजः}{\time{16-58}{13:22}}\hspace{1ex}\anga{विष्टिः}{\time{46-44}{01:16*}}\hspace{1ex}\uanga{बवः}}{}
}
{\tamil{மானக்கஞ்சாற நாயனார் (11) குருபூஜை}}
{Mon} 
\cfoot{\rygdata{07:58--09:23}{10:47--12:11}{13:36--15:00}}
\end{document}
